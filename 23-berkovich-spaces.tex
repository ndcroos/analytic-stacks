% !TeX root = AnalyticStacks.tex

\section{\ufs Berkovich spaces II (Clausen)}

\url{https://www.youtube.com/watch?v=vXZC3WzKZgo&list=PLx5f8IelFRgGmu6gmL-Kf_Rl_6Mm7juZO}
\renewcommand{\yt}[2]{\href{https://www.youtube.com/watch?v=vXZC3WzKZgo&list=PLx5f8IelFRgGmu6gmL-Kf_Rl_6Mm7juZO&t=#1}{#2}}
\vspace{1em}

\begin{unfinished}{0:00}
e  so  this  uh  this  lecture  is  about
burkovich  spaces  which
I  um  topic  which
I  kind  of  began  discussing  uh  last  time
and
um
so  well  let  me  remind  you  kind  of  the
classical  setup  you  have
a  a  banok
ring  so  it's  a  ring  equipped  with  a  norm
which
is  sub  multiplicative  satisfies  a
triangle
inequality  uh  maybe  I'll  I'll  do  it  do
it  better  this  time  Norm  of  one  equal  to
zero  or
one  unless  R  equals  z  um  and  R  is
complete  and  also
of
okay  yes
uh
yeah
um  okay  uh  and  then  to  this  uh  burkovich
assigns  the  compact  house  door
Space  Mr  Norm  which  is  a  subset  over  the
product  over  all  elements  in  R  of  the
closed  interval  from  uh  Z  to  the  norm  of
F  and  it's  the  set  of
all  uh  the  notation  will
be  uh  that  a  point  in  here  will  be
denoted  X  but  what  it  literally  is  is  a
evaluation  uh  a  map  from  R  to  the  non-
negative  real  numbers  which  is  now
multiplicative
so
um  and  again  satisfies  triangle  in
equality  and  now  one  is  one  yeah  now
it's  it's  kind  of  very  it's  strictly
multiplicative  yeah  so  um  yeah  um  and  we
and  uh  you  know  R  is  not  complete  with
respect  to  this  Norm  X  in  fact  there
could  be  many  elements  of  Norm  zero  but
uh  so
remark  if
you
complete  uh  complete  r  with  respect
to  uh
any  uh  Norm  X
sorry  you  get
a  a  complete  valued
field  KX  hat  I  don't  know  which  is  sort
of  the  residue  field  in  the  sense  of  you
know  since  of  burkovich  theory  and
recall  that  there's  basically  kind  of
three  cases
uh  there's  one  is  a  archimedian
and  that's  the  same  thing  bosski  is
saying  it's  either  r  or  C  so  in  the
archimedian  case  there's  no  variety
really  in  complete  armed  Fields  I  will
see  but  the  norm  could  be  a  power  to
alha  Alpha  One  ex  bigger  than  zero  yes
uh
yes
uh  um  the  second  is  a  it's  non-
archimedian
uh  but
discret  um  and  then  you  have  the  trivial
Norm
so
uh
so  and  then  the  third  case  is  a  it's  an
honest  non-  archimedian
field  non-discrete  I  mean  I  mean  the  the
induced  topology  on  the  field  I  mean
it's  non-  discrete  um  the  Valu  could
come  from  a  discrete
valuation  um  and  then  again  so  then  uh
so  so  you  could  have  some  normal  ized
version  if  you  choose  a  on  discreet
there  is  an  ambiguity  yes  it's  not  yeah
it's  not  about  discret  non-isr  exactly
that  that  was  the  parenthetical  comment
I  was  making  a  non-discrete  topology  it
depends  if  you  call  a  discrete  field  a
non  archimedian  field  I  think  not  but
you  you  wouldn't  you  wouldn't  even  call
this  a  non-  archimedian  field  yeah  okay
so  the  the  norm  is  not  maybe  the
yeah  I  it's  not  the  most  people  call
archimedian  yeah  okay  okay  the  norm  is
non  archimedian
but  but  but  the  but  the  field  is  discret
so  yeah  it's  just  a  discret  just  a
discret  field  um  so  anyway  these  are  the
the  three  kind  of  cases  of  of  possible
residue  field  behaviors  as  you
see
um  uh  right  and  then  so  here  are  the
alpha
alpha
okay  um  there  is  no  normalization  no  I
mean  yeah  you  could  choose  a
normalization  if  you  if  you  fix  a  pseudo
uniformize  you  can  CH  and  a  and  a  real
number  you  can  choose  a
normalization
yeah
yeah  all  right
okay  uh
yes
um  yeah  so  these  are  kinds  of  kind  of
points  and
uh  so  what  we're  going  to  do  now  is
we're  going  to  promote  this  so  the  goal
is
to
promote  uh  Mr  Norm  to  an  analytic
stack  uh  using  the  stack  of
norms  n  so  weall
so  a  map
from  a  map  from  an  analytic  stack  X  to  n
was  the  same  thing  as  a  certain  map  from
P1  of  X  P1  X  to  uh  0  plus
infinity  uh  so  and  uh  satisfying  nor
maxium  uh  the  most  important  one  being
multiplicativity
which  you  have  to  phrase  a  little
carefully  but  in  the  end  it's  just  a
it's  just
multiplicativity
um  okay
so  so  and  we  also  we  kind  of
investigated  the  geometry  of  n  last  time
so  we  saw  that  um  n  Lies  over  this
extended  burkovich  spectrum  of  the
integers
so  um  so  that  was  this  this  space  for  um
so  I  mean  we  here  we  have  a  a  triangle
inequality  you  know  the  usual  triangle
inequality  and  that  that's  what  made
that  the  difference  between  like  the
reason  you  couldn't  take  an  arbitrary
power  Alpha  here  but
I  mean  if  you  take  an  arbitrary  power
Alpha  you  get  something  that  basically
basically  behaves  like  a  norm  but  it's
really  only  a  quasi  Norm  so  you  have  to
put  some  constant  in  front  here
um  plus  the  limit  you  have  also  the  kind
of  the  limit  at  in  yes  there's  also  the
limit  at  Infinity  yeah  so  so  that  that
so  that  makes  uh  yeah  so  and  in  this
stack  of  norms  uh  in  this  stack  of  norms
there's  no  triangle  inequality  imposed
and  it  turns  out  what  you  get  is  this
non-strict  triangle  inequality  where  all
powers  of  the  usual  archimedian  normal
are  allowed  and  then  also  there's  a  some
kind  of  very  strange  limit  point  um
where  your  Norm  takes  infinite  values  on
natural
numbers  uh  and  then  you  have  the  uh  the
archimedian  ones  so  two  Tic  absolute
values  which  end  and  here  in  a  point
which  has  both  characteristic  zero  and
characteristic  P  Behavior  but  where  the
norm  of  two  equals  zero
um  uh  and  then  then  for  the  other  primes
P  um
and  we  also  kind  of  saw  what  the  on  the
interior  of  these  line  segments  you  you
were  getting  the
some  uh  the  gaseous  R  theory  on  the
interior  here  you're  getting  the  gaseous
two  EIC  numbers  and  then  as  you  move
along  the  normed  the  norm  is  changing
but  the  kind  of  so  to  speak  the  analytic
ring  is  not  um  and  Q3  and  so  on  and  then
here  you  have  things  living  you  have  F2
living  for  example  um  you  have  things
living  in  characteristic  two  there
living  in  characteristic  three  here  and
here  you  have  things  uh  living  in
characteristic  zero  so  in  some  sense
uh
um  you  can  imagine  that  the  points  of
this  stack  correspond  to  something  like
these  complete  valued  fields  or  or  the
minimal  choices  of  complete  valued
Fields  like  you  have  real  numbers  piic
numbers  you  have  discrete  FP  you  have
discret  uh  discrete  Q
um  uh  so  that's  kind  of  a  substitute  for
the  notion  of  multiplicative  valuation
but  then  we  still  have  to  input  our  um
our  Bon  ring  R  into  the  construction  in
order  to  get  something  non-trivial  so
here's  the  definition  um  and  I  don't
know  what  good  notation  is  I  I'll  write
I'll  write  it  like  this
so
um  so  this  will  be  an  analytic  stack
which  will  be  a  substack  sub  subset  of
uh  stack  of  of  norms  cross  uh  and  then
some  apine  analytic  stack  which  is  just
we  take  r  with  a  trivial  analytic  ring
topology  uh  I  mean  sorry  the  trivial
analytic  ring  structure  So  What  by  this
what  I  mean  is  that  you  take  a  r  r  is  a
banak  ring  so  it  has  a  topology  but  you
can  also  view  it  as  a  light  condensed
ring  so  the
um  so  using  the  topology  you  consider
those
condensate
yes  um  and  then  trivial  analytic  ring
structure  that
is  all  modules  are  allowed  yes  even  not
okay  it's  the  full  you  know  condens
derived  category  of  this  condensed  ring
so  that  means  that  uh  if  you  want  to
give  a  map  from  X  to  this  spec
R  Tri  that's  exactly  the  same  thing  as
giving  a  map  of  condensed  rings  from  R
to  the  the  value  of  the  structure  sheath
on  on
X  uh  right  yeah  so  uh  right  so  this  U
this  is  going  to  be  a  subset  of  this
product  consisting  of  those  so  such  that
M  well  let's  say  maps  from  X  to  here
these  are  going  to  be  by  definition  in
injection
with  um  so  well  it's  going  to  be  a  par
consist  ing  of  a
norm  uh  and  then  a  map  from  R  to  the
Ring  of  functions  on
X  uh  such
that  and  then  we  impose  a  condition
oh  forgot  to  say  the  the  thing  that  ties
this  this  to  that  so
uh  you  require
the  the  uh  multiplicative  valuation  to
be  bounded  by  the  given  norm  and  that's
exactly  what  we're  going  to  do  here  so
such  that  for  all  F  and
R  um  so  when  you  have  the  norm  here  and
you  have  an  element  in  O  of  x  uh  you  get
a  section  of  A1  and  you  have  a  section
from  an  element  of  ox  you  get  a  section
of  A1  and  in  particular  a  section  of  P1
and  then  you  can  compose  with  the  norm
and  you  get  a  map  Norm  F  from  from
X  X  to  0  plus  infinity  uh  such  that  this
map  lands
in  uh  Z  and  then  Norm  of
f
okay
um  so  um  well  I  guess  the  theorem  first
theorem  would  be  that
um  uh  so  this  is  an  analytic
stack  uh  which
localizes
uh
along  uh  the  burkovich  Spectrum
Mr  so  in  the  sense  that
I  discussed  in  the  last  lecture  so  that
if  you  look  at  this  local  of  all  open
substacks  of  this  spec
Burke  R  Norm  uh  this  Maps  naturally
to
um  this  compact  house  door  space  which
is  the  Burk  of  a  Spectrum  so  uh  you  have
the  local  of  open  in  the  usual  topology
and  the  local  wait  usual
topology  oh  oh  on  the  right  yeah  on  the
right  yeah  but  the  on  the  left  open
substock  means  what  it  means
monomorphism  it  means  a  monomorphism
which  is
shable  and  uh  and  chromologic
smooth  and  this  was  disc  discuss  in  one
of  the  talks
okay  and  you  you  could  even  ask  for  a
map  of  antic  step  right  that  well  when
when  this  is  finite  dimensional  and
metrizable  yes  which  is  basically  all
cases  but  I  mean  I
yeah
yeah  yes
yeah  so  again  um  so  in  in  the  case  when
this  is  metable  and  finite
dimensional  to  verify  that  you  have  a  I
mean  it's  really  just  a  condition  to  say
that  You'  get  a  map  of  analytic  Stacks
to  that  thing  and  it's  just  you  have  to
check  some  connectivity  properties  of  of
the  item  potent  algebras  assigned  to
close  subsets  that  are  kind  of  implicit
in  this  in  this  map  so  it's  a  condition
that  you  can  check  in  practice  and  so  is
it  the  case  that  you
have  since  R  is  cool  you  can  think  of  it
as  a  limit  of  C
sub  Rings  which  are  account  many  element
yes  so  you  can  probably  reduce
to  no  but  it  maybe  it's  not  Dimension
well  I'm  not  sure  but  no  no  but  you  can
look  at  as  a  CO  liit  over  finally
generated  sub  rings  and  then  and  that's
always  embedded  in  a  finite  I  mean  okay
okay
okay  uh  yes  yeah  so  that's  one  way  so
there  is  indeed  a  canonical  way  to  write
this  as  an  inverse  limit  of  finite
dimensional  metrizable  spaces  but  um
let's  not  do  it  let's  just  let's  just
work  with  this  uh  this  setup  here  and  is
it  true  that  you  have  inverse  limits  in
analytic  stat  yes  you  have  you  have  all
limits  you  have  all  limits  in  analytic
Stacks  yeah  ah  okay  so  you  can  syn  of  it
as  a  limit  over  the  stack  is  a  limit
over  the  well  this  isn't  a  stack  now
this  is  just  I'm  just  viewing  this  as
yeah  when  you  have  those  nice  sings  you
can
construct
uh  okay  so  it's  not
uh  no  but  then  you  can  take  for  the  nice
subing  you  can  take  the  stock  Associated
to  m  in  those  limit  you  get  something
which  is  yes  you  get  something  yeah  then
then  you  get  something  the  only  problem
I  have  with  that  it's  not  a  real  problem
but  just  is  that  if  you  started  with
something  which  happened  to  be  already
be  finite  dimensional  and  metrizable
then  you'd  be  non-trivially  writing  it
as  an  inverse  limit  of  other  finite
dimensional  metrizable  things  and  I  mean
you  know  so  let  me  just  not  let  me  just
not  get  into  it
okay  yeah  one  other  question  when  when  R
is  State  then  we  we  also  show  that  we
can  localize  over  the  spa  and  then
there's  a  map  from  the  spa  to  the  I  mean
that's  right  that's  right  yeah  there's  a
commutative  di  there's  a  yes  there's  a
commutative  diagram  that  you  can  write
down  where  here  you  have  map  from  you
know  the  Huber  space  mapping  to  this  and
then  you  have  a  some  this  thing  and  then
you  have  the  the  solid  guy  here  mapping
to  that  and  this  this  last  Arrow  I  I
mean  no  no  yeah  when  I  I  I'll  discuss  in
more  detail  what  this  looks  like  in  the
Tate  case  and  then  um  and  then  you'll
see
yeah  uh  okay  so  in  particular  just  I
want  to  uh  highlight
that  uh  we  get  a  structure
chief  on  on  this  topological  space  and
even  a  structure  chief  of  you  know  of
infinity  categories  so  a  theory  of  Quasi
cerent  shes  on  on  the  usual  burkovich
space  um
so
um  I'll  explain  in  basically  the  Tate
case  how  it's  pretty  easy  to  calculate
this  structure  chath  and  see  what  it's
doing  in  the  case  of
uh  Rings  like  the  integers  well  the
integers  you  can  kind  of  do  it  by  hand
but  it  would  be  interesting  to  compare
so
uh  uh  uh  so  it's  we  I'll  explain
basically  how  you  compare  to  the  cases
burkovich  discussed  but  it  would  be
interesting  to
uh  uh  to
compare  uh  to
puu  who  kind
of  more  or  less  by  hand  described
structure  sheaves  in  certain  cases  uh
over  the
integers  um  so  this  is  a  different
different  approach  where  you  define
something  which  is  a  prioria  structure
sheet  and  then  you  have  to  calculate  it
which  can  be  done  in  principle  but  you
know  takes  a  while  um  in  Poo's  case  he
explicitly  assigns  the  value  and  then  he
has  to  maybe  prove  some  yeah  prove  some
descent  results  so  here  we  automatically
get  some  sort  of  infinity  descent  but
then  you  have  to  calculate  the
value  um
okay  to  like  I  know  this  and  should  be
easy  to  say  what  I  on  a  disc  the
structure  sheet
yeah  it  does  reduce  to  seeing  what  goes
on  on  a  dis  but
um  to  see  what  goes  on  on  a  dis  um  yeah
so  I  I  mean  I  I  I  do  believe  that  it
yeah
so  I  mean  I  I  I'll  explain  what  you  need
to  do  to  do  these  calculations  and  I
think  you'll  see  that  it  is  like  but
probably  there  are  like  in  the  case  of
ho  Rings  there  are  probably  some  derived
phenomena  because  when  you  want  to  look
at  the  things  like  the  the  algebra
functions  on  close  or  open  this  anyway
you  you  quau  something  by  certain  I  mean
you  have  non  closed  idea  so  probably  you
need  to  to  work  in  some  derived  sense  to
get  the  right  to  get  what  you  get  from
your  fan  Theory  you  should  probably  have
the  drinks  which  are  complete  we  do  we
do  we  do  okay  yeah  I  mean  that's  what
yeah  um
yeah  so  I  do  believe  that  in  in  cases
like  uh  what  poo  considers  where  it's
you're  starting  with  a  a  discrete  ring
like  say  the  integers  um  that  that  the
calculations  are  quite  feasible  um  but
if  you  started  with  maybe  a  a  more
arbitrary  Bond  offering  it's  it's  maybe
not  so  obvious  how  to  do  the
calculations  um
okay  so  where  are  we  ah
okay  uh  I  want  to  explain  okay  so  why
this  is  true  so  I  guess  proof
um  so  well  that  it's  an  analytic  stack  I
mean  we  defined  it  as  a  a  funter  a
functor  of  points  so  I  the  only  thing  to
make  it  an  analytic  stack  is  some
accessibility  but  let  me  let  me  explain
how  to  give  charts  for  it  so  let's  let's
give
charts  um  well  so  recall  that  we  had
charts  for  the  Norms  the  space  of  norms
so  we  have  uh  we  had  this  uh  spec  of  Z  q
hat  plus  or  minus  one
um  uh  this  gases
Theory  and  this  we  can  view  as  the
universal  um  sort  of
uh  so  it  it  sits  inside  P1  is  kind  of
the  universal  Locus  where  I  say  absolute
value  of  T  is  equal
to2  um
uh  right
so
uh  so  if  we  pull  back  along  this  uh  if
we  pull  back  along
this  uh
then  um
so
so  so  I  mean  we  look  at  uh  so  we  look  at
this  spec  Burke
R  Norm  Norm  and  then
spec  uh  z  q  plus  or  minus  one  gas  and
then  we  get  this  uh  the  UN  some
Universal  disc  over  here  some  Universal
well  I  don't  know  anulus  over  here  uh
let  me  just  call  it  um  let  me  just  call
it
um  uh
y
um
uh
then  so  for  this  y  thing  um  we've
already  got  the  norm  we've  kind  of
artificially  adjoined  an  element  of  Norm
1/2  um  but  the  only  thing  we  need  to
ensure  to  to  go  from  uh  this  to  this  is
we  need  to  give  the  second  map  we  need
to  give  the  map  to  um  to  spec  of  r  with
a  trivial  thing  and  we  need  to  see  that
they  agree  and  we  need  to  enforce  the
condition  that  um  this  Norm  condition
that  NF  lands  in  inside  that  that  part
there  um  so  we  just  so  to  get
y  you  just  take  uh  spec  of  Z  q  hat  plus
or  minus  one  gas  cross  Spec  R
TR  and
then  pass  to  the
closed
subsets  uh  uh  given  by  the  item  potent
algebra
um  obtained  by
uh
uh  taking  the  norm  F  inverse  of  of  these
closed  subsets
here  so  uh  in  other  words  this  Y  is  our
is
aine
so  this  uh  this  burkovich  spectrum  here
has  a  a  fairly  simple  cover
by  by  an  apine  and  to  calculate  this
apine  the  most  difficult  part  for  like  a
completely  General  R  the  most  difficult
part  is  already  in  the  first  step  that
is  making  this  product  and  uh
calculating  what  analytic  ring  this  is
and  in  particular  calculating  what  the
underlying  condensed  ring  is  because
what  this  tensor  product  involves  is  um
you  have  to  calc  calate  the  the  gaseous
localization  of  of  like  R
yeah
so  you  have  to  then  apply  this  gaseous
localization  again  and  recall  from
Peter's  lecture  where  he  described  the
gaseous  localization  that  it's  given  by
some  sequential  Co  limit  there  are  some
coo  complexes  and  it's  a  little  uh  yeah
then
a  um  but  once  you  have  that  I'm  going  to
I'm  going  to  explain  that  calculating
these  item  potent  algebras  is  not  that
hard  so  once  you  know  the  the  the  ring
you  have  here  then  you're  passing  to
certain  item  potent  algebras  over  it  and
that's  that's  not  the  hard  part  of  the
calculation  but  so  in  particular  if
you're  in  a  situation  where  you  can  do
this  calculation  then  it's  it's  quite
feasible  to  to  calculate  Y  which  is
giving  a  presentation  for  this  stack  and
um  and  we  also  saw  that  like  The  Descent
for  this  cover  um  uh  is  quite  simple  it
it  happens  at  the  the  zero  stage  so  like
the  structure  sheath  here  will  just  be  a
retract  of  the  the  structure  sheath  here
so  the  Ring  of  functions  on  this  y  so
what  did  you  say
The
Descent  oh  yeah  when  we  argued  that  uh
that  this  map  was  a
cover  we  did  it  by  saying  that  it  was  uh
proper  and
descendible  and  we  even  showed  that  like
the  unit  object  the  unit  the  the
structure  sheeve  here  is  just  a  retract
of  the  push  forward  of  the  structure  she
there  um  so  that  makes  The  Descent  uh
quite  straightforward  basically
when  you're  passing  from  here  to  Y
you're  adjoining  some  extra  variable  q
and  it  has  some  analytic  properties  but
you  calculate  some  ring  up  there  and  you
just  look  at  the  zeroth  possible  zeroth
coefficients  of  that  ring  and  that  um
that  tells  you  the  value  of  the
structure  Chief  here
um  look  at  ring  is  a  direct  Factor  one
yes  yeah  exactly  not  as  a  ring  but  I
mean  it's  yeah  it's  a  it  it  maps  and
then  there's  a  linear  retraction  yeah
just  like  if  you  had  laurant  series  like
constant  coefficients  inside  lant  Series
yeah  yeah  mod  yes  exactly
yeah
um  right  okay  so  uh  and  the  yeah  so  the
the  the  map  to  the  burkovich  Spectrum  is
it's  kind  of  uh
well
so
um  so  on  this  spec  Burke  so  the  map
to
uh  so
on  spec
Burke  you  have  the  universal
Norm  um  let's  call  it
n  and  and  then  we  also  have  a  for  every
element  in  R  we  have  by  by
Fiat  a  map  to  the  structure  sheath  here
so  uh  then  we  get  a  map  from  spec  uh
spec  Burke  R  to  product  overall  F  and
R  zero  Norm  of
f
um
but  since  we  enforce  that  every  element
of  R  for  every  element  of  R  the  norm  of
that  element  is  bounded  by  the
prescribed  Norm  on  our  boning  r  that  in
particular  implies  that  the  norm  of  two
is  uh  less  than  or  equal  to  two  which
means  that  by  last
time  the  triangle  inequality
holds  for  n  and  that  means  that  this
this  map  lands  in
uh  inside  the  burkovich  Spectrum
indeed  I'm  sorry  the  image  the  image
mind  yeah  Mr  is  by  definition  a  sub
space  of  the  product
yeah
this  construction  we  could  also  keep
some  of  the  which  does  not  satisfy  the
triangle
inity  right  yeah  I  mean  I  um  yeah  it's
it's  natural  from  the  perspective  of
this  stack  of  norms  that  we've  been
discussing  as  we'  seen  to  relax  the
triangle  inequality  and  it's  and  you  can
do  that  I  mean  you  could  I  mean  the  the
formalism  is  quite  General  there's  I  the
reason  I  I  stuck  to  the  classical  thing
is  just  because  it's  the  classical  thing
you  have  for  the  triang  in
quality  you  assume  that  yeah  you  could
require  that  there  exists  a  constant
such  that  uh  you  know  this  that's  that's
one  thing  you  could  do  okay  but  then  if
you  want
to  uh  ah  so  this  still  defines  a  uniform
structure  so  you  can  say  it's
complete  and  uh  and  it  is  not  equivalent
by  fix  changing  things  slightly  to  a
actual  Norm  if  you  have  this  so  well  I
don't  know  because  for
Fields  just  by
power  H  is  that  true  this  wouldn't  be  I
think  this
is  something  z  uhhuh  may  maybe  it's  true
yeah  yeah  but  you  could  also  conceivably
allow  the  norm  to  take  infinite  values
and  try
to  um  tried  to  build  that
into  into  things  as  well  yeah  I  just  I
wanted  just  wanted  to  stick  with  the
classical  thing
um  so  there  is  a  a  a  theorem
about  the
scaling  no  if  you  have  a  norm  with  the
kind  of  I  don't  know  how  it's  called
with  the  constant  yeah  so  so  so  the
theorem  that  for  Fields  I
think  you  only  get  the  Pu  classification
with  an  alpha  I  think  yeah  yeah  I  think
you're  right  yeah  yeah  yeah  but  for
Rings  you  don't  know  that
because  it's  it's  delicate  because  if  Y
is  a  small  this  doesn't  imply  this
condition  doesn't  imply  that  the  nor  of
X  and  of  X  Plus  Clos  yes  yes  yes  I  I
agree  it  could  be  subtle  for  a  general
ring  I  I  I  don't  want  to  make  any  claims
I  actually  want  to  uh  stick  to  the
classical
setting  um  okay  so  this  was  me
explaining  the  general  case  um  but  there
is  um  and  in  the  general  case  uh  why  oh
sorry  this  this  this  thing  despite  the
notation  with  the  spec  this  um  is  not
going  to  be  apine  so  for
General  um  this
spec
Burke  R  is  not  apine  so  it's  not  the
spec  of  a  of  an  analytic  ring  um  so  for
example  well  if  we  look  at  spec  Bur
uh  Z  and  then  the  usual  archimedian
absolute  value  which  is  the  um  kind  of
the
maximal
the  every  every  Norm  you  could  put  on  Z
will  have  to  be  less  than  or  equal  to
this  one  so  this  is  kind  of  the  the
choice  that  gives  you  the  biggest
possible  burkovich
Spectrum  um  uh  this  is  a  this  is  just
this  uh  Locus  where  yeah  two  absolute
value  of  two  is  less  than  or  equal  to
two  inside  this  stack  of  norms  and  it
really  is  a  stack  as  you  can  kind  of  see
at  the  at  the  points  that  live  at  the
boundaries
um  uh  okay  so
however  uh
suppose  so  let  me  make  a  assumption  star
um  that  uh  so  there  exists  a  let's  say
say  a  pi  in
R  uh  such  that  Norm  of  Pi  is  less  than
one  uh  Pi  is  a
unit  in  the  ring  and  I  want  it  to  be
that  it  strictly  multi  like  the  norm
strictly  multiplies  when  you  multiply  by
the  norm  is  multiplicative  with  respect
to  multiplying  by
pi  um
so
uh  so  this  uh  well  this  condition  is
obviously  not  satisfied  here  but  it's
satisfied  quite  broadly  so  so  example  so
any  non  any  non-discrete
yeah  I  apologize  again  non-discrete  I'll
put  it  way  over  here  so  you  don't  think
I'm  saying  non-  discreetly  valued  field
but
non-isr  valued  field
uh  uh  yeah  admit  such  a
norm  sometimes  they  say  nontrivially
valued  field  okay  yeah  yeah  um  so  you
there  you  have  multiplicativity  for  all
elements  and  then  uh  if  it's  not
discreetly  valued  then  there's  something
with  Norm  between  zero  and  one  and  it
that'll  be  a  unit  and  yeah  of  course
there  is  a  slight  ambiguity  in  valued
field  because  sometimes  it  refers  to
absolute  value  sometimes  to  cruel
valuations  it  could  be  higher  rank  and
then  you  have  to  say  I  mean  it's  not
it's  here  is  valued  in  the  sense  of  of
uh  of  yes  real  value  yeah  exactly
exactly  yeah
um  and  then  so  if  if  um  if
R
satisfies  star  and  uh  Fe  from  R  to  R
Prime  with  which  is  a  map  in  the
appropriate  sense
so
so
um  so  contractive
homomorphism  uh  then  RP  Prime  also
satisfies  uh  so  with  respect  to  F  of
Pi  um
so  uh  well  the  reason  is
uh  yeah  the
uh  yeah  you  take  the  same  Pi  its  image
will  be  a  unit  and  it'll  be  the  norm
will
be  uh  less  than  one  but  also  the  norm
will  have  to  be  the  same  because  uh  yeah
this  condition  is  equivalent  to  uh
unless  unless  maybe  R  is  zero  I  don't
know  um  is  equivalent  to  saying  that  the
norm  of  Pi  inverse  is  actually  the
inverse  of  the  norm  of  Pi
uh  as  you  can  see  by  applying  the
multiplicativity  prop  the  sub
multiplicativity  property  of  the
norm  um  so  then  uh  and  that  condition  is
kind  of  more  obviously  preserved  by  okay
so  actually  in  the  belovich  the  I  think
some  is  natural  to  consider  morphisms
which  are  not  the  norm  is  on  Fe  effect
is  less  say  a  constant  nor  effect  this
defin  still  upap  on  ver  spaces  I  suppose
that  your  definition  does  not  depend
that  is  if  let  say  you  have  two  noes
which  are  equivalent  in  this  sense  by
constants  then  all  you  all  of  this  will
be  the  same  for  the  two  NOS  I  suppose
but  well  is  to  check  something  not  quite
so  I  mean  let  let  me  let  me  say  what  I
can  say  and  then  yeah  sorry  I  just  ah
okay  yeah  but  in  in  under  this
assumption  you  mean  or  in
general
I
think  yeah  yeah  yeah  I  I  agree  with  that
so  that's  why  I  want  to  postpone  the
discussion  yeah  uh  a
bit  okay  so  in  particular  I  mean  uh  the
classical  settings  in  burkovich  Geometry
you  work  over  some  fixed  field  which  is
not  well  often  non-discrete  um  and  uh
and  you're  working  with  Bono  algebras
over  that  field  and  they  will  certainly
uh  satisfy  this  condition  star
but  if  you're  working  over  a  discret
ring  you  know  you  won't  have  this  uh
condition
star  um  so  so  then  I  then  I  claim  uh
so  ah  well  let  me  give  yeah  we're  also
uh  yeah  so  also  any  any  uh  any  Kate
Huber
ring  uh  has  a
norm  defining  the  topology  uh  satisfying
star  with  pi  a  pseudo
uniformer  so  also  some  mixed
characteristic  examples  um
exist
um  uh  right  so
um  where  am  I  oh  yeah  so  then
claim  uh  so  if  star
holds  uh  then
spec
burkovich  uh  R  Norm  is
apine  and  corresponds  to  an  analytic
ring
structure  on  the  condensed  string
r
um  so
moreover  uh  this  analytic  ring
structure  only
depends  on  the  condensed  ring
R  uh  not  on  the
norm  satisfying
star
yeah  so  I  hope  in  some  sense  I'll  be
addressing  your  question  I  mean  no  but  I
see  I  think  that  just  for  your
definition  if  if  you  have  two  equivalent
Norms  then  you  you  by  constants  that
simplest  case  then  you  just  look  at  the
condition  you  impose  apply  to  oh  yes  yes
yeah  I  think  you're  right  I  think  you're
right  I  think  you're  right  yes  yes  yes
so  similarly  for  a  morphism  brings  if  if
it  is  only  with  a  constant  you  could
still  apply  the  same  thing  getm  of  St  I
think  you're  right  ofer  yeah  so  this
because  in  some  I  remember  that  in  some
text  I  don't  know  if  the  book  of  ver  in
some  place  they  they  consider  such
things  which  is  more  yeah  apparently
more  natural  because  I  don't  I  don't
know  that  the  why  but  it's  it's  it's
probably  you  can't  I  think  you're  right
that  because  our  Norms  are  by  definition
multiplicative  I  mean  these  geometric
Norms  that  you  can  uh  you  can  argue
exactly  as  you  suggested  yeah
um  yes  that's  a  good  point
thanks  um
okay  right  um  so  so  so  what  doesn't
depend
on  uh  what  what  depends  on  more  than  R
well  on  this  thing  uh  you  have  this
Universal  Norm  n  um  so  the  the  universal
norm  and  on  spec
b  r  uh  does
depend  on  the  norm  you  choose  on  R  um
but  any  two
choices  are  uh
equivalent  under  uh  so  Norm  passes  to
Norm  to  the  alpha  so  for
some  map  Alpha  from  spec  oh
boy  sorry  let  me  continue  over
here  and  I  could  I  could  remove  the  this
the  second  factor  from  the  notation  here
because  as  I  said  that  the  stack  itself
the  stack  itself  only  depends  on  r  but
okay  I'll  keep  it
um  so  there's  a  what  I'm  referring  to
here  is  there's  a  scaling  action  uh  so
I'm  referring  to  the  fact  that  there's
a
um  there's  a  scaling  action  which  sends
a  norm  uh  and  a  continuous  function
Alpha  to  uh  the  norm  you  take  the  norm
and  you  compose  with  the  alpha  map  on
exponentiation  map  on  on  zero  Infinity
exponentiation  by  Alpha  on  Zer
Infinity  so  what  the  here  you
wrote  uh  spec  no  no  I  I'm  sry  the  what
was  the  the  the  notation  for  the
classical  B
spectrum  is  M  of  r  r  m  yeah  M  okay  so
the  alpha  is  a  map  to  positive  realse
but  does  it  factorize  through  M  of  r  or
is  what  is  it  is
it
no  just
right  um
okay  no  that  doesn't  wouldn't  surprise
me  if  it  did  it's  just  the  argument  I
doesn't  doesn't  quite  show  that  but  U
probably  if  you  look  more  closely  it
will
I  let's  give  the  argument  and  then  uh
and  then  maybe  try  to  address  the
question  I'm  going  to  give  the  proof  of
this  and  then  maybe  it'll  be  easier  for
us  to  answer  the
question  after  having  seen  the
proof  I'm  not  claiming  it  right  now  but
uh  I'll  give  the  proof  for  this  claim
and  then  maybe  at  that  point  we'll  be
able  of  course  you  can  just  change  the
the  norm  on  Itself  by  taking  a  power  and
then  this  will  be  the  alpha  if  you  just
change  the  nor  by  a  power  yeah  if  you
change  the  N  by  a  constant  then  what  I
said  before  it  does  this  doesn't  change
but  if  you  change  it  by  raising  to  a
power  then  you  you  probably  prob  be
constant  yes  but  I'm  not  it  doesn't  no  I
mean  so  the  thing  is  like  uh  there's  two
choices  we  need  to  take  account  for  one
is  like  a  choice  well  let  me  let  me  give
the  let  me  give  the  proof  and  then  and
then  we'll  be  better  equipped  to  try  to
answer
questions  such  as
these
um
okay  okay
so  uh  so  the  proof
um  um
so  I'm  going  to  explain  how  to
produce  uh  this  analytic  ring  structure
on  the  condens  ring
R  um  so  take  take
Pi  uh  as  in
Star
um  then  we  get  a  map
from  uh
zq  to  the  condens  spring  R  which  sends  Q
to  Pi
um  but  uh  Pi  is  of  Norm  less  than  one
which  implies  that  it's  topologically
nil  potent  it's  sequence  of  powers  tend
to  zero  that  implies  that  it  factors
through  this  this  ring  here  and  it's
also  a
unit  um  by  assumption  so  we  get  a  a
factoring  through  this  ring  here  um  and
then  uh
so  uh  we  need  to  check  or  we  want  to
check  um  that  uh  R  is
gaseous  Q
gaseous  as
uh  so  recall  that  this  gases  theory  was
a  non-trivial  analytic  ring
structure  which  was  produced  by  taking
uh  by  realizing  that  that  the  the
category  as  a  full  subcategory  of  over
this  ring  um  and  then  there's  this
completion  procedure  which  changes  the
underlying  ring  to  this  gaseous  thing
but  the  the  category  of  modules  was  just
described  at  this
level
um  so  uh  so  and  and  this  is  something
very  uh  this  is  something  very
straightforward  because  the  definition
of  gases  was  that  some  some  map  from  p  p
to  p  p  being  the  universal  null  sequence
so  to  speak  namely  1
minus  t  *  *  Q  uh  should  be  an
isomorphism  on  on  on  map  uh  on  maps  to
R  but  um  when  you  map  out  to  R  from  this
null  sequence  space  to  your  Bono  space
um  you're  just  getting  the  space  of  null
sequences  in  the  Bono  space  so  that's
equivalent  to  saying  that  if  you  look  at
the  space  of  null  sequences  in
R
uh  and  then  you  have  some  1  -  Q  *  shift
this  should  be  an  isomorphism  of
condensed  uh  of
Bono  Bono  ailan  groups  say  um  but  this
is  uh  but  it's  easy  to  see  what  the
inverse  is  supposed  to  be  and  to  write
it  down  you  need  to  just  you  need  that
that  if  uh  sort
of  FN  is  a  null
sequence  uh  then  then  you  can
sum
uh  uh  FN  pi  to  the  n  and  still  get  an
element  in
R
um  and  uh  the  condition  on  Pi  is  and  the
usual  triangle  inequality  stuff
uh  lets  you  write  this  down  it's  just
the  the  the  limit  of  the  you  Koshi
sequence  um  so  it's  a  it's  quite
straightforward  to  check  that  R
is  R  is  liquid  um  and  then  we  can  just
take  the  induced  analytic  ring  structure
so  what's  that
what  oh  I  said  I  meant  to  I  keep  saying
liquid  instead  of  gases  yeah  it  is
liquid  well  anyway  this  Al  it's  gash
this  uh  take  induced  uh  analytic  ring
structure  uh
from  uh  z  q  plus  or  minus  one  gas  just
um
so  now  recall  that  on  so  we're  call  that
on  on  spec  of  uh  Z  q  hat  plus  or  minus
one  gases  we  have  a  universal
Norm  uh  with  the  norm  of  Pi  strictly
equal  to  uh  this  okay  now  maybe  let  me
say  R  is  not  the  zero  ring  it's  the  zero
ring  I  leave  the  claim  as  an  exercise  so
then  if  it's  not  the  zero  ring  then  this
Pi  will  have  to  have  Norm  bigger  than
zero  and  and  less  and  less  than  one
um  and  then  on  uh  this  we  have  the
universal  Norm  where  the  the  norm  of  Pi
lands  inside  this  Singleton  Subspace
maybe  I'll  write  it  like  that  to  remind
you  that  this  is  kind  of  a  subset  and
not  a
value
um  uh  right  and  then  we  can  then  then  by
by  functoriality  of  norms  but  because
Norms  pull  back  we  get  a  get  a  norm
on  uh  on  what  is  Pi  what  ah  Q  uh  Pi  Pi
is  Q  Pi  is  our  we  we're  fixing  one  of
these  guys  which  exists  by
hypothesis
the  Q  Q  oh  thank  you  thank  you  thank  you
I'm  sorry  yes  yes  thank  you
yeah  uh  right  so  we  get  Norm  on
um  r  with
induced  uh  analytic  ring
structure
um  and  the
claim  so  this  normed  analytic
ring  uh  is  uh  is  is
spec
um  spec  Burke
R  respect  to  this
normes  I'm  sorry  represents  yeah  uh
represents  is
yeah
yeah  it's  the  same  thing  right
um  well  from  geometry  okay  uh  spec  of
this
um  okay
so  what  does  one  need  to  do  to
um  to  prove  this
claim  so  so  what  have  we  done  so  well
maybe  I  need  a  name  for  this  normed
analytic  ring  let  me  call  it  r  liquid
our  gaseous  although  in  the  so  so  far  I
haven't  shown  that  it's  independent  of
the  norm  on  R  but  we'll  see  that  in  just
a  sec
uh  but  but  let  me  let  me  hide  that  in
the
notation
um
right
um  uh  where  am  I  oh  yeah  so  so  so  far  so
what  have  we  so
over  spec  uh  our  gas
we  have  the
map  uh  to  Spec  R
triv  and  we  have  the
norm  uh  n  with  uh  Norm  of  Pi  exactly
equal  to  uh
so
um  the  Singleton  value  pi  and  uh  and
it's  easy  to  see  just  by  tracing  through
the  construction  that  this  is  universal
with  respect  to
that
or  with  respect  to  those  that
structure  um  so  what's  the  what's  the
difference  with  the  thing  we're  trying
to  compare  to  uh  it's  that  in  in  instead
of  having  a  condition  on  just  just  the
norm  of
Pi  uh  which  we  have  here  we  have  a
condition  on  the  norm  of  every  element
so
need  so  we
need  that  uh  uh  this  condition  on  a  norm
is  equivalent  to  the  condition  that  Norm
of  f  uh  is  contained  in  zero
f  uh  for  all  F  and
R  so  we  need  this
equivalence  okay  okay  so  One  Direction
is  quite  easy  so  uh  if  you  have  so  so
this  is
easy  so  if  you  have  this  then  you  apply
it  to  Pi  and  to  Pi  inverse  and  you
deduce  that  uh  so
so  so  the  key  is  to  see  that  just
telling  you  what  the  norm  of  Pi  is  then
I  know  I've  constrained  the  norm  of
every  element  in  our  in  our  Bond
offering  um  so  for
uh  uh  and  also  for  for  the  the  other  oh
well  sorry  that's  ambiguous  right  so
um  yeah  so  for  that  direction  and  and
because  it's  good  to  know  uh  we  will
calculate
uh  so  we  calculate  this  Universal
Norm  so  more
precisely  um
so  so  what  is  a  normed  analytic  ring
structure  recall  that  it  amounts  to
specifying  some  item  potent  algebras
over  P1  um  so  so  we'll
calculate  the
algebras  um
so
uh
so  so  for
all  less  than
c  um  so  what  so  a  norm  in  our  Sense  on
an  analytic  ring  is  implicitly  uh  just
telling  you  what  the  overon  convergent
functions  are  on  a  a  disc  of  arbitrary
radius  centered  around  the  origin
um  um  and  uh  yeah  so  we  just  have  to
specify  this  and  my  claim  is  it's  going
to  be  the  usual  thing  from  burkovich
Theory  so  so
claim  so  this  is  equal  to  filtered  Co
limit  of  let's  say  yeah  radius  bigger
than  C  of  um  you  take  the  so  I  I'll
explain  what  this  is  afterwards  but  kind
of  you  can  make  a  universal  Bono  ring
where  the  norm  is  is  less  than  or  equal
to
R
um  and  uh
so  uh  is  this  or  now  I'm  sorry  is  this
loal  or
now  could  you  make  your  question  more
precise  uh  I  don't  know  if  I  have  fin
time  how  much  time  you
need  well  you  take  the  time  you  need  and
when  it's  precise  ask  again
yeah
okay  so  this  is
like  the  formal  series  that  when  you
replace  each  coefficient  by  the  absolute
value  and  replace  T  by  the  ab  by  R  you
it  converges  the  sum  is  is  finite  yeah
so  this  is  the  set  of  uh  set  of  well
it's  just  the  coefficients  but  let's  say
RN  T  TN  such  that  some  absolute  value  RN
oh  boy  I  shouldn't  use  r  let  me  use  F  I
was  using  f  for  elements  in  my  bonering
so
um  and  then  this  is  the  norm  so  this  is
the
on  um  oh  wait  wait  sorry  sorry  yeah  no
yeah  yeah  that's  right  that's
right  uh
okay  so  um  I  was  also  making  claims  last
lecture  about  calculations  of  the  what
these  overon  convergent  functions  were
in  various  cases  so  I'd  like  to  explain
how  to  make  these  calculations  and  it
turns  out  there's  a  trick  where  you
really  don't  have  to  do  anything  it's
just  kind  of  purely  formal  uh  so
uh  so  there's  a  trick  to
calculating  course  this  is  a  in  the
condensed  now  you  have  to  view  this  as  a
condensed
yeah  this  is  bonak  and  then  it's  uh  yeah
so  it's  condensed  yeah  cond  yeah  and
then  this  is  this  filtered  colum  it  is
taking  place  in  the  condensed  category
yeah  um  oh  yeah  so  well  what  what  is  it
we  need  to  calc  what  is  it  that  we're
calculating  here  actually  so  we're
taking  what  we're  doing  is  we're  well
we're  trying  to  calculate  we  have  the
universal  thing  over  this  gaseous
base
uh  which
we  uh  more  or  less  wrote  down  uh
and  then  we  have  to  tensor  it  with  the
gaseous  tensor  product  so  over  this
analytic
ring
um  uh  with  r  where  here  Q  goes  to
Pi  and  we  have  to  I  mean  actually  you
know  a  priori  it's  a  derived  tensor
product  but  this  is  the  kind  of  thing  we
need  to  do  and  if  we're  being  too  naive
about  it  it  it  can  look  kind  of  tricky
because  um  naively  what  You'  do  is  you'd
write  this  as  as  we've  explained  is  some
filtered  Co  limit  over  copies  of  P  so
that's  kind  of  over
convergence  uh  and  then  you'd  take  uh
you'd  first  calculate  P  tensor  R  and
then  you'd  pass  to  the  filtered  Co  limit
but  actually  it's  not  so  easy  to  unwind
what  P  tensor  R  is  in  particular  it's
not  so  easy  to  see  that  it  would  be
concentrated  in  degree  zero  so  let's  use
a  trick  so  so  let's  not  use  this
approach
that  was  how  we  produced  this  thing  in
the  Universal  case  so  recall  the  idea
was  that  the  this  P  was  some  version  of
functions  on  the  open  unit  disc  and  then
when  you  have  this  q  and  maybe  all  of
its  fractional  Powers  you  could  scale
that  open  unit  disc  and  get  some  version
of  uh  functions  on  an  arbitrary  disc  and
it  wasn't  the  correct  one  but  when  you
make  it  over  convergent  it  doesn't
matter  it'll  be  by  some  kind  of
sandwiching
argument  okay  all  right  so  trick  to
calculate  is  some  general  category
Theory  fact  so  so  LMA
is  so
if  so  so  C  symmetric
monoidal  and's  say  infinity  category
it's  not  too  relevant  um  but  then  uh  so
if  you  have  a
tower  so  X1
X2
X3  uh  in  C  where  each
map  is  Trace
class  so  so  X  to  Y  is  Trace
class  means  it  comes
from  a
map  X  so  from  the  unit  to  x  dual  tensor
y  or  X  dual  I'm  not  assum  assuming  X  is
dualizable  this  is  just  the  internal  H
uh  from  X  to
one
uh  sorry  yes  us  closed  thank
you  m  there's  probably  a  way  to  well
never
mind
um
uh  where  are
we  um  ah  then
then  for  all  Y  in  C  uh  we  can  calculate
the  co  limit  Over  N  of  X  and  dual  uh
tensor
y  so  we  pass  to  the  we  have  a  tower  here
we  pass  to  the  Dual  thing  which  gives  a
a  sequence  and  we  take  the  co  limit  over
that  sequence  um  then  this  uh  is  the
same  thing  as  Co  limit  Over  N  of  the
internal  H  from  xn  to
y
so
uh  this  is
Elementary  um  I'll  leave  it  I'll  leave
it  just  like  that  without  giving  the
proof  you  just  have  a  two  systems  and
you  make  two  in  systems  and  you  make
maps  backwards  that  go  up  a  step  using
the  trace  class  hypothesis  okay  so  again
you  how  do  you  know  there  is  a  the  limit
makes
sense
okay  well  this  this  is  actually  an
equality  of  IND  objects  so  uh  I  mean
it's  an  equality  of  end  objects
does  it  doesn't  it  doesn't  matter  okay
as  a  in  object  and  you  have  to  know  what
is  the  end  for  for
uh  this  is  end  for  for  C  in  in  C  which
makes  sense  is  an  Infinity
okay
um  yeah  so  in  particular  if  C  has  co-
limits  you  you  can  just  you  can  remove
the  quotation  marks  um  if  C  has  Co
limits  in  tensor  product  commutes  with
co-  limits  which  is  the  case  in  our
examples  then  you  can  remove  the  I  mean
you
can  yeah
okay  uh  right  um  okay  so  yeah
so  so  what  we're  going  to  do  so  yeah  in
particular  well  yeah  so
well  so  what  we're  going  to  do  is  we're
going  to  recognize  uh  so  if  you  take  y
equals  the  unit  then  um  you  know  and
yeah  that  this  object  here
um  coim  of  the  internal  H  ask  a
technical  question  yes  about  the
definition  it's  nice  what  do  you  mean  by
it  comes  from  a  map  it's  that  the  is
given  by  this  ah  right  so  if  you're
given  a  map  like  this  then  you  can
tensor  it  with  X  you  get  a  map  from  X  to
X  tensor  x  dual  tensor  Y  and  X  tensor  x
dual  has  an  evaluation  map  to  the  unit
so  you  then  get  a  map  from  X  to  y  yeah
yeah
um  right  um  so  so  we're  going  to  we'll
so  we  uh  will
recognize
uh  c  as
a  CO  limit  over  now  P
du
um  and  apply
this  with  Trace  class  transition
Maps  so  once  we  do  that  then  we  reduce
to
uh  reduce  to  um  uh  some  to  looking  at
null  sequences  in  R  again  and  then  just
some
so  some  filtered  co-limit  of  some  space
of  null  sequences  in  R  and
um  you  can  actually  modify  this  to  the
thing  where  you  require  these  to  form  a
null  sequence  and  they  wouldn't  be  the
same  at  each  term  but  it's  quite  easy  to
be  the  they're  the  same  when  you  take
the  filtered  colum  again  any  two
versions  of  the  unit  dis  are  kind  of  the
same  after  you  make  them  overon
convergent  um  so  uh  yeah  and  then  the
calc  the  calculation  is  very  easy  uh
once  you  once  you  do
this  and  for  this
uh  uh  for  that  we  can  use  S
Duality  on
P1
uh  so  over  well  it  happens  to  be  over
the  liquid  base  but  it  I  mean  the
gaseous  base  but  it  doesn't  much  matter
um  so  St  Duality  on  P1  this  if  you  so
and  the  and  the  six  funter
formalism  so  um  sidity  on  P1  implies
that  the  well  so  we  have  this  proper  map
to  the
base  and  algebraic  serid  Duality  uh
implies  that  the  this  map  is  smooth
smooth  and  proper  um  and  that  the
dualizing  object  is  just  this  Omega  1
shifted  by
one
um  and  then  you  can  calculate  the  then
it's  easy  yeah  it's  easy  to  do
calculations  and  I'll  just  tell  you  what
the  conclusion  is  uh  the  conclusion  is
that  if  you  take  the  Dual  of  the
um
uh  this  identifies  with  uh  the  Ring
of  functions  on  the  open  complement
so
um  so  the
DU
of  in  the  sense  of  uh  uh  a  linear  dual
in  the  category  so  this  is  dual  in
uh  in
there  I'm
sorry  so  this  is  an  object  in  this
category  which  implicitly  there's  an
underlying  condensed  ailan  group  and  so
on  so  yeah  yeah  do  do  you  know  that  you
deal  these  object  which  are  sufficiently
f  so  like  the  double  du  right  so  you
need  to  right  so  we  would  want  to  put
the  Dual  on  the  other  side  but  we  can  do
it  with  a  trick  more  or  less  because
um  so  this  gives  also  so  this  this  over
con  this  by  the  over  convergence  um  you
get  that  the
the  the  single  Duel  of  the  end  object
sorry  sorry  we  can  then  we  can  write
this  uh  this  is  a  pro  object  we  can  we
get  so  yeah  we  get  we  can  get  this  we
can  view  this  over  convergent  thing  as
an  end  object  and  it's  dual  will  be  a
pro  object  and  it  will  be  the  pro  object
given  by  this  thing  where  you  increase
the  radius  as  well  um  but  then  now  we
can  view  that  Pro  object  as  an  inverse
limit  of  the  overon  convergent  things
with  the  the  non-strict  inequalities
over  there  and  then  um  and  then  use  The
Duality  result  in  this  direction  on  each
of  those  and
uh  um  so  there's  some  trick  trick  with
overon  convergence
uh  uh  this  implies  that  if  you  if  you
take  the  the  Dual  of  uh  this
so  if  you  view  this  as  a
proobject  um  because  it's  an  inverse
limit  of  the  things  where  you  um  we  have
a  yeah  greater  than  uh  then  the  Dual  of
that  Pro  object  is  the  end  object  uh
that  we're  interested
in
and  that  gives  a
um  that  gives  an  expression  exactly  like
this  that  this  guy  is  a  CO  limit  of  uh
of  du  of
P's  um  we  still  need  the  trace  class
claim  but  I  claim  that  that  also  holds
for  soft  reasons
um
so  the  trace  class
so  there  there's  a  general  topology  fact
that  if  x  is  a  topological  space  and  z
and  uh  Z  Prime  are  closed
subsets  such  that  there
exists  an  open
U  uh  which  lies  in  between  them  so  two
closed  subsets  which  are  separated  by  an
open  subset  uh  then  uh  the  map  from  uh
the  push  forward  of  the  constant  chath
uh  the  Restriction
map
um  is  Trace
class  in  the  drive  category  of
shes  okay  or  really  I  should  say  maybe  I
should  say  she's  on  X
Val
um  Shi  on  X  and  then  oh  with  values  in  D
of  Z  yeah
um  which  is  not  the  category  of  X  in
general  because  it's
right
um  yeah  uh
so  so  then  uh  on  the  level  of  just  this
closed  interval  from  0  to  plus  infinity
then  any  of  these  transition  mths  will
be  Trace  class  for  this  reason  and  then
you  can  pull  back  that's  a  symmetric
monoidal  funter  you  get  that  Trace  class
you  get  a  trace  class  map  in  the  derived
category  of  P1  but  it  lives  on  A1  and
then  there's  another  trick  to  see  that
it's  image  under  the  forgetful
funter  um  to  the  base  is  also  Trace
class  so  there's  so  so  pull
back  to  A1  use  another
trick  and  the  conclusion  is  that  the
Restriction  map  say  from  uh  any  of  these
overon  convergent  guys
uh  is  Trace
class  um  and  then  also  for  pres  yeah
then  again  by  sand  sandwiching  different
versions  of  the  discs  and  changing  the
Radia  you  also  get  the  and  using  the
trace  class  Maps  as  a  two-sided  ideal  in
all  maps  then  you  get  the  presentation
like  this  which  let  you  calculate  do  it
by  hand  do  what  by
hand  you  want  to  write  over  functions  on
this  disc  I  mean  the  de  clance  is  to  use
pH  piece  yeah  you  can  just  system  and
see  that  transition  has  really
dis  yeah  that  there's  um  yeah  I  I
believe  you  can  do  that  so  certainly  I
remember  doing  that  in  the  complex  case
and  I  assume  it  works  over  the  gashes
base  too  but  I  thought  it  was  nice  to  be
able  to  do  it  without  doing  any
calculations  um
yeah
okay  St  you  still  need  to  know  that  the
way  you  presented  is  the  way  you  scene
you
presented  well  yes  that's  true  that's
true
yes  way  what  was  the  remark  the  the
remark  was  I  I  wasn't  being  very  careful
here  about  writing  what  the  filtered
systems  are  and  all  this  okay  you  have
to  show  that  it  is  there
yeah  um  okay  so  what  that  was  a  that  was
a  a  bit  of  a  uh  um  a  bit  of  digression
so  what  were  we  doing  we  were  um  we  were
we  I  said  we  were  calculating  the  normed
analytic  ring  structure  and  the
conclusion  was  that  the  so  so  the  norm
on  uh
Spec  R  gas  is  given  by
usual  over  convergent
functions  on
diss  over
r  so  that  was  the  the
conclusion
um  so  then
um  then  you  need  to  so  to  see  what  did
what  were  we  trying  to  show  we  were
trying  to  show  that  uh  for  this  Norm
here  this  Universal  Norm  that  we
produced  by  fixing  the  norm  of  Pi  that
automatically  the  norm  of  every  other
element  is  is  correctly  bounded  um  so  to
show  Norm  of  f  uh  contained  in  zero  f
for  all  f  it's  it  suffices  to
show  it  trans  it  algebraically
translates  into
um  oh  no  oh  no  now  we  have  bad
notation
uh  it's  not  clear  a  priority  that  it  is
finite  sorry  you  have  to  know  even  the
finess  is  not  is  a  statement  uh  that's
correct  yeah  yeah  um  so  we  have  to  show
that  if  you  take  this  and  you  mod  out  by
T  minus  F  uh  you  just  get  uh  just  get  R
this  okay  this  this  is  elementary  and
indeed  this  is  Elementary  so  the  map  the
map  giving  this  is  of  course  setting  T
equals  to
F  and  um  it's  quite  Elementary  as  ofer
says  to  see  that  you  get  the  correct
short  exact  sequence  of  uh  of  banak
spaces  so  that  the  kernel  of  this  map  of
T  minus
f  is  of  the  what's
that  mere  existence  of  the  m  is  mere
existence  of  the  map
enough  oh  that's  a  really  good  point
that's  a  really  good  point  Peter  yeah
thanks  yeah  so  what  Peter  was  saying
yeah  so  what  Peter  was  saying  is  that
uh  right  we  know  a  priori  that  this
thing  is  an  IDM  poent
algebra  uh  over  A1  so  over  the
polinomial  ring  on  one  generator  um
therefore  when  you  base  change  it  um
along  here  then  you  get  an  itm  potent
algebra  over  R  bracket  T  mod  T  minus  F
which  is  R  so  this  thing  is  an  item
potent  algebra  over  R  and  so  is  R  itself
and  if  you  want  to  show  that  two  item
poent  algebras  over  R  are  equal  it's
enough  to  Just  Produce  an  algebra  map
between  them  thanks  a  lot  that  that
indeed  makes  it  very  me  very  no  just  uh
we  already  have  the  unit  map  yeah  we  a
map  in  both  directions  indeed  but  we
already  have  the  unit  map  uh  so
so  so  uh  let's  see  uh  because  when  you
do  it  analytically  with  this  uh  instead
of  over  converion  just  converion  on  the
Clos  disc  like  before  and  of  course  you
have  a  a  a  map
when  f  is  less  than  or  equal  to  C  you
get  them  up  but  it  is  but  to  prove  V
division  you  still  get  to  some  conver  is
not  okay  in  the  aredian  case  and  so  it
is  but  it's  not  item  potent  it's  not
item  potent  in  that  case  it's  not  item
potent  in  that  case  if  you  don't  do  the
overc  convergent  one  if  you  just  do  the
the  one  without  the  overon  convergence
it's  not  going  to  be  item  potent  as
argument  would  not  work  exactly  so
somehow  the  proof  of  item  potent  is  you
must  have  already  done  work  similar  to
showing  that  you
know  but  in  the  non  archimedian  case  the
the  thing  with  theid  algebra  does  work
and  in  this  case  aoid  algebras  are
important  or  not  they  are  they  are  yeah
yes  yeah  when  you  do  it  with  the  with
the  the  non  archimedian  I  mean  the
everything  yeah  everything  non-
archimedian  yeah  yeah  then  then  why  is
it  it  ah  because  okay  I  mean  we
basically  we  basically  proved  it  when  we
discussed  the  solid  Theory  but  maybe
okay  if  everything  is  not  so  there  you
can  use  the  yeah  and  solid
then  then  it
is  uh  right  so  okay  but  yeah  so  in  any
case  um  yeah  the  map  exists  because  you
can  evaluate  at  tals  F  and  as  Peter
points  out  that's  enough  thanks  Peter  um
so  uh  right  okay  so  that  was  the  that
was  the  first  part  of  the
claim
um  so  that  was  the  claim  that  this  thing
is  um  this  B  burkovich  Spectrum  this
analytic  stack  which  I'm  calling  the
burkovich  spectrum  is  um  is  apine  when
you  have  that  assumption
star  the  next  claim  was  that  the
analytic  ring  structure  is  independent
of  the  the  choice  of  the  norm  um  so  so
so  proof
continued  uh  need  that  uh  it's  spec  our
gas  is
independent  of  norm  and  Pi
say
um  no  and  that  is  if  you  have  two
topologically
isomorphic  does  depend  on  the  on  the  on
the  condens  the  topological  exactly
exactly  and
uh  without  so  you  could  have  one  nor
with  this  one  Pi  which  is  for  one  Pi
another  Norm  is  pi  Prim  yeah  yeah
exactly  exactly  yeah  so  let  but  let  me
just  give  the  descript  the  independent
description  so  uh  so
claim  so  so  our  gas  uh  is  the  initial
analytic
ring  uh  with  a  map
from  R
triv  to  R
gas  such  that  and  then  we  just  have  a
condition  such  that  for  all  uh
topologically  nil  potent
units  pi  and  R  it's  a  condition  that
only  depends  on  the  topological  ring  R
um  uh  so
the  our  gas  is  uh  is  pi
gases
I.E  that  if  you  the  map  from  z  q  hat
plus  or  minus  one  to  R  uh  Q  goes  to  Pi
um  and  R  lies  in
the  gases
modules  what's
that  oh  I'm  sorry  thank  you  thank  you
thank  you  thank  you  thank  you  yes  uh
yeah  yeah
yeah  thank  thank  you  Peter  yeah  not  just
R
but  is  it  enough  to  check  for  R  the
trivial  module  or  no  no  because  the  I
mean  that  then  you  you  wouldn't  have  to
change  the  analytic  ring  structure  at
all
then  yeah  D  of  r  d  of  R  is  meant  is  d  as
a
condensed  D  of  the  of  the  condensed
uh  D  of  R  means  box  no  it's  the  analytic
ring  it's  the  analytic  ring  the  R  so
yeah  oh  this  should  be  yeah  this  should
be  um  yeah  so  I  want  this  to  yeah  so
this  so  this  map  uh  gives  a  map  of
analytic  Rings  the  condition  is  that
this  map  gives  a  map  of  analytic  Rings
uh  to  R  and  what  that  condition  means
instru  on  R  was  what  was  I  oh  sorry  uh
so  sorry  our  I  mean  I  was  I'm  sorry  let
me  say  R  gas  is  the  initial  analytic
ring  a  with  a  map  from  R  Tri  to  a  such
that  for  all  topologically  nil  potent
units  a  is  pi  gases
IE  uh  this  map
here  uh  yields  a  map  of  analytic  Rings
like  that
so  there  is  an  easier  probably  much
easier  question  in  this  context  which  if
you  have  two
Norms  the  same  then  the  bovich  spaces
are  the  same  is  it  true  or  not  exactly
because  if  you  have  two  Norms  so  since
the  the  condition
is  is  less  or  equal  to
some  some  you  don't  if  you're  not  a
topology  you  don't  control  things  far
away  from  zero  but  because  of  the  unit
you  can  scale  so  but  still  I  wonder
about  at  Le  in  the  another  comedian  case
one  can  compare  to  Uber  and  get  that
it's  the  same  do  fication  it  is  theic  of
some  you  can  but  in  the  aredian  case  of
course  there  is  this  in  the  bovich
Spectrum  there  is  this  condition  that  I
mean  I  think  if  you  take  the  um  you
could  also  like  try  to  make  a
modification  of  the  burkovich  spectrum
only  thinking  of  our  is  a  topological
ring  where  you  ask  for  these  seminorms
to  be  continuous  and  then  I  think  if  you
take  that  space  and  then  you  mod  out  by
this  exponentiation  action  by  positive
real
numbers  um  then  that  will  be  the  same  as
the  burkovich  Spectrum  for  any  fixed
Norm  uh  satisfying  condition
star  no  no  no  because  when  aredian  the
the  non  aredian  things  yes  uh  you  don't
because  in  the  b  space  you  don't
identify  no  to  its  power  to  its  powers
so  I  don't  see  how  you  no  but  yeah  the
identification  is  maybe  not  so  obvious
it's  kind  of  well  I  I  mean  I'm  not  I'm
not  sure  I'm  not  sure  but  so  you  because
you  you  you  want  to  claim  that  your  your
your  belov  spectrum  of  course  it  maps  to
the  space  the  Bel  space  as  we  said  yes
and  but  you  don't  claim  here  that  the
map  is  because  if  you  want  to  CL  the  map
is  the  same  you  have  to  compare  the
verage  spaces  and  and  this  looks  like  a
little  bit  tricky  at  least  in  the  away
from  the  non  archimedian  Cas  we  can
understand
it  anywhere  I'm  not  sure  well  yeah  I
don't  I  don't  know  um  right
um  so
um  yeah  so  let  me  uh  okay  let  me  let  me
give  the  the  proof  of  this  claim  which
is  kind  of  giving  a  intrinsic
description  of  this  gases  analytic  ring
structure  so  um  so  note  that  if  Pi  is
topologically  nil
potent  then
there  exists  a  n  such  that
uh  uh  so  we
um  um  and  after  passing  to  some  power
you  have  small  Norm  in  particular  you
have  Norm  less  than  one  and
um  let  me  note  that  this  condition  here
uh  this  is  invariant
under  uh  replacing  Pi  by  any  power  power
this  is  actually  a  remark  that  that
Peter  made  at  some
point  um  some  point  early
on  uh  so  so  we  can  assume  uh  that
Pi  uh  is  Norm  less  than  one
um  but  then  uh  but  then  uh  for  the
universal  Norm  we  built  over  our  gaseous
uh  sorry  well  sorry  I  need  to  fix  Maybe
okay
so  so  my  claim  is  going  to  be  so  so
certainly  uh  this  R  gas  that  we  built  we
built  it  so  that  it  uh  satisfies  a
weaker  version  of  this  property  where
you  only  demand  it  for  a  fixed  Pi  uh
satisfying  this  condition  here  and  what
we  need  to  show  is  that  so  let's  say  so
we  so  let  me  say  so  our  gas  was
built  to
satisfy  star  just  for  uh  some  fixed  uh
Pi
0  uh  and  now  we  have  to  show  that  it's
satisfied  for  all  choices  of  Pi  um  but
uh  so  yeah
so  um  but  then  uh  so  yeah  so  so  can
assume
uh  0  less  than  Pi  less  than  1  and  then
that  implies  uh  that  the  norm  of  Pi  uh
is  in  this  interval  from  0  to  1  um  which
we  already  showed  implies  that  Pi  is
gaseous  just  from  the  axioms  of  a  a
normed  analytic
ring
okay
uh  what  is  pi  z  r  for  Ral  Z  what  is
it  doesn't  make  sense  because  I've  only
defined  this  when  R  is  a  bonering
satisfying  this  condition  star  well
sorry  the  the  other  the  other  condition
star  about  the  existence  of  a  pi  Norm
between  zero  and  one  Etc  Pi  Z  no  fixed
Pi  Z  is  a  is
a  it's  a  fixed  I  the  way  I  built  the  way
I  built  this  was  I  I  took  my  norm  and  I
took  a
fixed  uh  uh  pi  zero  satisfying  condition
star  and  then  I  built  my  analytic  ring
and  I  built  my  Norm  over  it  and  now  I
want  to  check  that  that  thing  satisfies
this  Universal  Property  which  means  that
so  it  was  universally  built  to  satisfy
that  for  just  for  a  fixed  one  um  but
then  um  and  to  have  the  correct  Norm  on
there  but  and  then  I  want  to  argue  that
it
um  or
well  I  want  to  argue  that  automatically
all  of  the  other  possible  pies  are  also
gases  and  we  can  use  the  norm  to  prove
that
because  um  the  norm  is  such  that  it  you
know  yeah  well  such  that  we  have  this
chain  of
implications
okay
um  so  uh  right  and  then  the  last  part
h
so  the  last  part  is
um  uh  right  that  the  norm  if  you  have  so
that  that  was  the  analytic  ring
structure  being  independent  of  the
choice  of  Norm  on  R  but  then  the  um  so
the  last
part  um  so  this  Norm  on  uh
spec  our  gas  which  this  a  priori  depends
on  um  is
independent  of  the  norm  up  to
exponentiation  by  a
map  uh  spec  our
gas  are  greater  than  zero
um  so
um  yeah
right
so  so  let's  say  we  have  two  different
Norms  uh  we  have  n  given  by  this  here
and  we  have  n  Prime  given  by  this
here
um  um  so  and  both  Norms  have  to  satisfy
conditions  star  but  possibly  for
different  choices  of
Pi  and  so  this  one  uh
satisfies  star  for  say  Pi
Prime
um  then  um  okay  what  can  we  look  at
um
so  again  uh  we  can  assume  that  the  norm
of  Pi  Prime  is  less  than  one  um  so  now
Pi  Prime  had  some  nice  fixed  properties
with  respect  to  this  but  now  with  this
one  it's  just  some  arbitrary  map  to  01
but  that  implies  that  with  respect  to
the  norm  n  uh  that  if  you  take  n  of  Pi
Prime  this  is  a  map  from  Spec  R  gasius
uh  to
01  um  and  then  uh  it  follows  that  but
there  exists  an  alpha  such  that  n  of  Pi
Prime  raised  to  the  alpha  is  equal  to
just  a  constant  uh  value  the  norm  of
Pi  because  uh  exponentiation  acts
transitively  so  or  acts  simply
transitively
so  uniquely  transitively
on
zero
um  so
um  yeah  so  this  is  the  this  is  the  thing
we  have  to
um  yeah  let  me  finish  the  argument  then
we'll  I  think  you're  probably  right  um
so  we'll  address  o  first  point  at  the
end  of  the  the  end  of  the  argument  um  so
then  um  but  then  n  and  n  Prime  are  two
Norms  Norms  on
spec  are
gaseous  uh  uh  both  with
uh
uh  um  wait  sorry  uh  normal  denoted  by
double  double  bar  double  bar  Prime  I
think  n  Prime  should  be  double  bar  Prime
n  Prime  should  be  double
bar  I  don't  know
um  sorry  uh  so  n  to  the  alpha  n  Prime
yeah  are  two  Norms
um  with  uh  sorry  sorry  I  sorry  I  want  to
actually  make  this  equal  to  um  uh  wait
sorry
yeah  right  okay  okay  are  two  two  Norms
with
um  uh  same
value  uh  on  Pi  Prime  so  then  by  the
classification  of
norms  uh  they  must  be
equal  on  P  some  value  so  same  fixed
value  the  the  value  is  a  fixed  real
number  between  zero  and  one  and  we  we
show  that  such  a  norm  is  uniquely
determined  uh  when  we  proved  this
classification  of
norms  okay  uh  and  maybe  I  wanted  to  do
this  or  something  I  don't  know  we  can  do
whatever  we
want  I  I  forget  to
yeah  yeah  I  don't  know  I  think  that's
right  I  think  that's  what  I  want  to
write  okay  um  so  now  to  address  ofer's
question  about  whether  this  map  Alpha  um
is  pulled  back  from  the  burkovich  space
so  first
question  is
Alpha  Spec  R  gas  uh  to  R  greater  than
zero  pulled
back
from
uh  so  this  is  the  the  canonical  map  and
and  this  question  is  whether  there's
some  map  there  um  to  answer  this
question  we  need  to  answer  whether  this
map  is  pulled  back  from  the  burkovich
Spectrum  so  true  if  Norm  of  Pi  Prime  is
but  that's  true  by
construction  because  the  mapping  the
Burk  of  dis  Spectrum  was  exactly
recording  the  Norms  of  all  of  the
elements  in  R  and  in  particular  we're
recording  the  norm  of  Pi  Prime  so  so  the
answer  to  your  question  is
yes  it's  okay  okay  now  I  see  I
understand
the
uh  I
understand  and  I  think  it  should  be  also
true  that  any  two  choices  of  NOS  on  your
ring  are  differ  by  exponentiation  to  and
up  to  con  equivalent  in  the  S  of  to
constant  by  any  two  any  one  is
equivalent  to
some  constant  I  mean  equivalent  in  the
sense  of  bounded  ratio  from  two  side  to
the  power  of  the  other  in  the  in  the
under  conditioned  star  you  mean  under  if
you  have  a  r  topologically  important
unit  first  of  all  if  you  have  a
topological  ring  complete  let  me  say  you
don't  need  complete  this  topologically
important  unit  then  you  can  see  the
topology  is  given  by  a  norm  SA  in  what
you  what  you  want  oh  is  that  true  this
is  something  in  general  topology  where
they  for  a  uniform  space  they  construct
metrics  that  give  it  but  if  you  have  an
ailan  group  it's  enough  if  you  have  got
a  a  system  of  neighborhoods  of  zero
fundamental  systems  that  un  plus  one
plus  un+  one  containing  un  and  let  us
say  symmetric  I'm  not  sure  then  you  can
decide  to  make  a  metric  by  deciding
elements  of  un  of  Norm  it's  most  say  one
two  10  you  check  the  metric  condition
here  you  can  do  something  similar  you
because  if  you  have  topologically
important  unit  yeah  and  if  you  have  a
fundamental  I'm  not  sure  about  the
accountability  okay  the  idea  my  idea  is
to  well  I  need  to  know  that  it's  not
only  topologically  important  but  there
is  a  neighborhood  of  zero  such  that  Q  to
the  pi  to  the  N  time  it  is  Convergence
to  zero  so  some  boundedness  some
conditional  terminology  yeah  okay  then
you  take  U  and  Omega  you  rep  Omega  by
power  so  that  well  okay  I  I  don't  do  you
work  with  two  equivalence  or  no  no  he's
not  asking  a  question  he's  telling  me
something  and  I  want  to  hear  so  I  I  am  I
am  afraid  that  I  don't  know  the
conditions  that  I  need  but  maybe  there
are  some  bound  but  some  natural
condition  in  the  sense  of  topological
ring  that  I  have  to  add  I  have  to  take  a
small  neighborhood  open  neighborhood
such  that  the  power  of  the  that  it  times
small  neighborhood  converges  to  zero  so
pi  to  the  end  it  goes  to  zero  then  like
for  some  end  pi  to  the  end  of  it  plus
end  of  it  is  containing  in  and  then  you
can  construct  the  metric  okay  uh  using  a
replacing  p  and  then  you  you  will
construct  at  least
uh  by  construction  I  think  somehow  I
force  the  P  to  to  have  no  one  enough
roughly  I  can  probably  make  it  Global
okay  one  is  to  work  it  out  the  details
and  then  it  will  turn  out  that  if  you
have  a
continuous  map  to  a  valed  field  which  is
continuous  and  the  norm  of  my  Pi  yes  is
less  than  or  equal
to  is  one  half  anyway  then  it  will  be
also  less  than  or  equal  to  this  no
probably  well  it  will  be  equal  to  1  half
yeah  okay  then  it  will  be  less  than  or
equal  to  this  yeah  and
then  okay  this  is  one  idea  but  then  if
you  have  got  a  but  by  the  way  for  with
this  construction  you'd  get  only  the
triangle  inequality  maybe  up  to  some
constant  again  no  no  no  no  no  if  I  have
what  I  claim  is  this  this  is  something
that  I  check  so  if  you  have  a
non-commutative  forgot  now  if  you  have
got  a  in  general  for  uniform  spes  they
they  have  three  three  uh  you  need  to
they  work  with  three  but  if  you  have  an
Community  grou  you  can  do  it  with  two  so
I  just  claim  first  that  if  you  have  if
you  have  a  topologic  cian  group  with
topology  is  defined  by  sequence  of  a
symmetric  neighborhoods  of  the  origin  un
un  plus  one  plus  un  plus  contain  un  then
you  you  just  get  a  metric  by  this  by
imposing  that  the  guys  in  un  know  at
most  one  over  two  to  the  end  so  you  have
a  metric  in  a  generalized  sense  it  could
be  plus  infinity  for  something  then  I
will  change  it  using  the  unit
my  my  uh  I  will  correct  it  using  my  sud
uniformer  but  at  least  I  will
get  uh  okay  maybe  what  I'm  saying  is  a
bit  uh  maybe  I'm  thinking  too  fast  with
some  mistakes  but  in  any  case  I  think  it
will  also  come  out  that  any  two
Norms
are  uh  in  some  sense
uh  yeah  but  this  I  am  not  sure  any
to  okay  maybe  we  can  discuss  later  let
me  let  me  just  finish  I  have  only  one
more  thing  to  say  and  it's  quite  short
so  yeah
um  so  uh  right  um  the  last  thing  is  when
I  talk  a  bit  about  just  say  Global
globalization  only  to  say  that  it's
trivial
um  so  we  could  make  a  definition  I  don't
know  I  mean  a  definition  of  a
burkovich  analytic  space  I  don't
know  is  a  pair
um  X  or
triple  x  s  uh  PI  from  uh  local  of  opens
in  X  to  S  where  X  is  an  analytic
stack  uh  s  is  a  locally  compact  house
door
space  um  and  Pi  is  a  map  of
locals  uh
such
that
um  such  that  uh  locally  on
S  for  the
topology  the  open  section  topology  or
the  the  local  section
topology
um  uh  it  is
isomorphic
to  spec
Burke  are  uh  uh  sorry  R
Norm  Mr  Norm  uh  and  then  this  canonical
map  Pi  for
some
Bing
R
um  okay  so  this
is  completely  trivial  now  uh  to
globalize  and  yeah  so  the  only  point  to
note  is  that  uh  working  locally  on  S
you're  also  automatically  working
locally  on  X  and  that's  because  these  by
by  definition  of  a  map  of  local  is  a  a
cover  and  the  open  cover  topology  gives
a  cover  in  the  sense  of  open  covers  of
analytics  stxs  here  and  those  are  covers
in  our  Gro  and  deque  topology  that  we
use  to  Define  analytic  Stacks  they're
even  open  covers  in  the  sense  of  the  six
functor  formalism  for
the  what  do  mean  local  section  to  I  mean
the  gro  topology  on  locally  compact
house  door  spaces  which  is  generated  by
open  covers  okay  yeah  it  is  isomorphic
to  isomorphic
to  this  basic
object  space  so  you  locally  yeah  so
recall  that  these  these  aine  ones  are
always  compact  house  DWF  so  you're  not
going  to  kind  of  if  you  if  you  say
locally  in  too  naive  a  sense  you're  not
going  to  get  any  examples  because  you
know  open  covers  open  subsets  of  say  R
are  usually  not  compact  right  so  but  if
you  do  this  usual  thing  of  having  a
compact  neighborhood  of  every  point  then
it's  fine  but  I  I  this  looks  it  I  wonder
about  the  derived  nature  of  the  of  the  r
when  you  localize  because  it  seems  to  me
that
of  course  you  can  make  this  definition
but  then  you  can  ask  whether  for  example
what  does  mean  locally  oness  if  it  is
true  if  you  take  D  do  you  have  like  for
sufficiently  fine  for  small  open  it  is
let  us  say  that  is  if  it  is  true  for
some  cover  it  sufficiently  fine  one  then
probably  you  have  to  to  pass  to  to  to
the  ring  Associated  to  some  small
distance  those  are  derived  in  some
that's  it's  we  have  the  same  problem  we
have  in  general  with  the  you  know  the  H
discussion  of  the  Huber  theory  in  the
solid  case  it's  the  exact  same  issue  and
it's  the  exact  same  situation  that  yeah
you  can  fix  it  by  starting  with  some
more  General  class  of  derived  things
here  probably  I  didn't  think  of  the
details  and  but  also  in  practice  in  most
examples  it  doesn't  show  I  mean  in  in
Practical  examples  it  tends  not  to  tends
not  to  matter  I  mean  that  in  the  sense
that  these  local  the  you'll  have  a
neighborhood  basis  that  you  know  when
you  calculate  these  neighborhood  bases
they  will  just  be  in  degree  zero  and  the
Rings  will  just  be  in  degree  zero
and  and  they  will  fit  back  in  the
framework  so  you're  right  that  there's  a
bit  of  a  fly  in  the  ointment  in  terms  of
the  way  we're  presenting  things  and  that
we're  starting  always  with  classical
objects  like  when  we  talked  about
Huber's
Theory  and  when  we're  talking  about  this
Theory  here  we're  starting  with
classical  objects  instead  of  derived
objects  and  a  priori  when  you  work
locally  on  our  class  on  the  Spectrum
attached  to  our  classical  object  you  see
some  derived  objects  instead  but  in
practice  it  doesn't  tend  to  happen  and
in  any  way  you  can  modify  the
definitions  of  it  to  accommodate  them  so
it's  not  such  a  big  deal  it's  slightly
more  sub  here  that  the  basic
building  rings  but  then  she  of  over
converion  so  it's  actually  never  B  she
of  kind  dis  between  the  global  Al  start
with  and  no  but  still  that  doesn't  I
mean  that  doesn't  obstruct  the  claim
that  that  there's  a  neighborhood  base
you
know  yes  I  mean  ofer's  question  was
about  a  neighborhood  base
yeah  yeah  that's  that  is  a  good  point
and  I  mean  you  know  you  can  modify  these
you  could  also  you  know  from  instead  of
instead  of  these  guys  you  could  also
pass  to  inverse  limits  so  you  could
starting  with  these  apine  guys  you  could
pass  to  arbitrary  inverse
limits  for  example  like  so  inverse
limits  in  the  compact  house  door  space
and  just  filtered  Co  I  mean  inverse
limits  in  the  category  of  analytic
Stacks  which  in  the  Aline  case  is  just
you  know  Co  filtered  Co  limits  of
analytic  rings  and  then  then  these
overon  convergent  things  would  also
count  is  apine  and  then  maybe  that's  a
little
nicer  uh  to  work  with  and  there's  no
harm  in  in  doing  that  um  but  say  will
not  be  B  rings  but  they  will  be
condensed  rings  with  certain  yeah
they'll  still  they'll  still  correspond
to  analytics  an  analytic  stack  with  a
structure  map  to  a  compact  house  door
space  and  so  on  okay  so  you  probably
instead  of  B  ring  you  can  have  a
condensed  ring  with  s  proper  yeah  yeah
with  with  some  Norm  satisfying  some
properties  and  so  on  yeah  I  mean  we
didn't  we  didn't  try  to  give  the  best
possible  formulation  was  just  a  just
wanted  to  connect  to  the  classical  thing
yeah  okay  so  that's  all  thank
you  sry  can  you  again  with  it  local
section  oh  yes  so  on  the  on  the  category
of  locally  compact  house  door  spaces  you
can  define  a  Gro
topology  where  um  a  you  know  a  set  of
maps  I  mean
it's  a  set  of  maps  like  x  i  to  S  forms  a
covering  if  uh  for  every  point  of  s
there's  an  open  neighborhood  of  that
point  and  an  index
I  and  a  section  of  the  pullback  uh  uh
you  know  you  pull  back  x  i  to  that  open
neighborhood  you  should  have  a  a  section
there  um  and  the  map  can  be  ar  yeah  the
map  can  be  arbitrary  the  map  doesn't
have  to  be  an  open  inclusion  but  it's
also  the  same  thing  as  the  gro  dig
topology  generated  by  the  the  covering
families  which  are  just  the  usual  open
covers  so  if  you  look  at  just  the  usual
open  covers  and  say  that  you  want  sheath
condition  for  that  you  automatically  get
sheath  condition  for  anything  any  any
map  that  has  local  sections  so  that's  a
so  if  you  want  if  you  yeah  so  the  seeve
will  always  be  the  same  as  the  seeve
generated  by  some  open  cover  of  s  and  so
yeah  but  it's  convenient  when  you  want
to  talk  about  the  sense  in  which  a
locally  compact  housef  space  is  locally
compact  house  DWF  because  it's  not  true
in  some  naive  sense  but  it's  true  in
this
sense  okay  other
questions  another  question  it  seems  to
me  if  you  take  at  least  naively  you  take
another  p  as  the  norm  you're  supposed  to
be  a  constant  one  uh  n  Prime  of  Pi  Prime
was  supposed  to  be  a  constant  but  n  of
Pi  Prime  can  vary  over  the  burkovich
Spectrum  but  uh  I  think
with  supposed  to  be  a  constant  because
is  small  than  the  normal  Prim  Prim
inverse  the  same  was  it  to  be
actually  no  see  the  norm  Pi  Pi  Prime  was
adapted
to  P  Prime  was  adapted  to
um  uh  absolute  value  prime  it  wasn't
adapted  to  absolute  value  so  it's  not
adapted  so  this  this  n  here  wasn't  such
that  it  satisfies  that  property  with
respect
to
I  so  how  did  we  built  this  n  from  this
absolute  value
here  for  which  Pi  had  this  property  that
Norm  so  we  had  in  other  words  we  had  the
norm  of  Pi  inverse  equals  Norm  of  Pi
inverse
here  right  and  then  we  built  this
n  using  this  so  that  the  the  norm  of
every  element  would  be  bounded  by  the
norm  prescribed  here  that  implies  that  n
of  Pi  has  to  be  this  fixed  value  but  it
doesn't  imply  anything  about  n  of  Pi
Prime  because  Pi  Prime  doesn't  NE  Pi
Prime  doesn't  necessarily  satisfy  this
property  for  this  Norm  it  only  satisfies
it  for  this
Norm  always  you
okay  thanks
guys  
\end{unfinished}