% !TeX root = AnalyticStacks.tex

\section{\ufs Analytic stacks and 6 functors (Scholze)}

\url{https://www.youtube.com/watch?v=BV0-dlAuS3U&list=PLx5f8IelFRgGmu6gmL-Kf_Rl_6Mm7juZO}
\renewcommand{\yt}[2]{\href{https://www.youtube.com/watch?v=BV0-dlAuS3U&list=PLx5f8IelFRgGmu6gmL-Kf_Rl_6Mm7juZO&t=#1}{#2}}
\vspace{1em}

\begin{unfinished}{0:00}
e
so  last  was  starting  the  discussion  of
and  was  really  any  of  actual  definition
that  we  used  but  was  trying  to  give  an
overview  of  what  the  definition  should
look  like  and  what  kind  of  examples
should  commodate  um  and  so
today  I  want  go  more  into
like  I  want  to  talk  about  analytic
spe  um  but  something  kind  thinks  how  we
set  up  uh  how  set  up  this  definition  is
actually  a  series  of  six  functions
so
okay
so  okay  so  last
time  uh  result  and
motivation  for  making  a  definition  of
s
um  want  be  some  categor  just
ch  this
definition  mod  issues  of  set  that
explain  um  actually
nowc  and  then  in  that  Ser  of  CH  other
soci  related  to  soal  Hy
she
um  I  will  probably  also  say  something  at
the  end  of  this
lecture
um  uh  so  we  would  like  to  Define  our
geometric  object
by  which  gives  us  our  basic  fing  box  the
spes  and  then  specify
Somey  tell  us  how  we  loser  um  so  key
question  is  like
which  the  sound  transmission  is  is  is  is
not  very  good  here  we  cannot  we  cannot
it's  hard  to  identify  what  you  say  but  I
don't  know  what's  the  problem  but  the
problem  is  I  didn't
micr  so  sorry  I  say  thanks  for  catching
that  sorry
um  sorry  do  you  hear  me  better
now  yes  no
maybe  now  it's  seem
yes  okay
um  sorry  uh  yeah  so  I  said  that  I  want
talk  about  analytics  sex  and  six  funs
and  that  last  time  we  saw  some
motivation  that  we  should  try  maybe  to
Define  our  category  of  geometric  objects
these  analytics  Stacks  by  starting  with
the  fine  analytics  spaces  corresponding
to  an  rinks  as  basic  building  blocks  and
then  specifying  some  drok  topology  which
tells  us  how  we  allowed  to  L  them
um  in  setting  up  the  precise  details  of
this  there  is  some  setic  issues  related
to  the  fact  that  this  is  a  large
category
and  consider  small  sheets  um  and  another
technical  issue  that  will  arise  is  that
we're  really  considering  sheeps  of
infinity  groupids  or  what  we  call  sheeps
of  Anima  here  um  and  then  there  is  an
issue  of  Chiefs  versus  hypers  sheeps  and
we  will  actually  at  the  end  consider
some  something  strictly  in  between  the
okay  let  me  not  get  into  this  uh  but  the
key  question  really  is  which  broque
topologies  are  uh  to  to  to  put  on
this  there  was  some  motivation  for  this
definition  and  uh  lots  of  examples  were
mentioned  um  that  we  will  uh  be  taking
up  again  once  we  have  the  proper
definition  of  the
category  all  right
um  be  before  I  I  get  there  let  me  remark
that  what  we're  doing
here  um  is  kind  of  very  very  Clos  uh  so
at  the  beginning  of  the  course
uh  we  did  something  extremely
similar  we  defined  condens  sets  or  live
condens  sets
you  can  put  or  not  put  light  on  both
sides  of  equation  um  sh  on  the  category
of  profite
sets  or  uh  if  you  want  Prof  sets  I  was
thinking
Alas  and  so  there  we  playing  the  similar
game  that  we  had  some  basic  objects  that
we  start  with  and  profite  sex  and  then
we're  building  a  larger  category  uh  by
some  allowing  ourselves  to  glue  them  in
a  certain  way  and  the  gluing  that  we
allow  was  specified  in  the  gr  topology
that  we  chose  and  there  we  chose  is
rather  General  Gren  topology  allowing
all  in  particular  all  subjective  maps  of
profile  sets  um  and  and  some  we're  now
repeating  we  are  now  repeating  this  game
with  a  much  more  broad  class  of  uh  basic
geometric  objects  corresponding  to  this
antic  Rings  uh  and  these  are  some  able
to  model  all  sorts  of  different
geometries  they  can  model  SS  of
just  algebraic  geometry  I  can
model  uh  some  pic  geometry  some  geometry
some  complex  Geet  whatever  um  and  all
these  different  kinds  of  geometry  are
someh
built  you  can  put  them  together  in  this
world  of  analytics  stxs  and  build  build
spaces  out  of  all  of  these  things
together  right  um
so  uh  so  something  that  done  in  some
areas  of  M  also  I  know  maybe  consider
instead  of  some  nasty  class  topological
space  like  Prof  fin  sensing  rather  some
nice  class  like  Smooth  manifolds  and
then  play  these  games
and  there  different  logical  space  or
something  like  this  and  but  all  of  these
would  would  be  part  of  this  world  of
analytics  Texs
now
um  all  right  so  so  so  the  key  question
is  which  gr
needs  so
here  here  are  some  things  that  we
want  um  so  the  most  important  variant
um  so  if  you  have
a  a  triangle
a
um  antic  GRE  and  let  me  justess  that  but
this  a  really  analytic  animated
rings
again  most  practical  examples  I  will  not
have  this  animated  structure  but  the
general  formalism  works  much  better  uh
if  you  allow  these  things  in  your
formalism  um  so  whenever  you  have  such  a
thing
uh  then  the  primary  inent  that  we're
interested
in  is  uh  it's  dve  category  of  complete
modules  recall  that  this  was  defined  as
full  subcategory  of  the  dve  category  of
the  condens  a  triangle  modules  such  that
all  the  homology  groups  are
complete  and  okay  so  this  is  the  primary
variant  and  we
want  uh  that  some  a  met  the  v
a  uh  is  a
sheet  but  this  is
this  it's  a  drive  the  only  sensible  way
to  phrase  this  is  that  it's  a  cheap  of
infinity  category  so  we  actually  need  to
treat  from  now  on  always  this  as  an
Infinity
category  so  let's  s  with  something  we
want  we  want  it  this  so  this  this
already  means  that  for
any  analytic
stack
x  uh  we  will  be  able  to
Define  uh  the  rough  category  of  POS  curs
on  XY
descent  so  it's  just  limit  over  of  a
over  all  a  that
to  so  that's
good  uh  but  actually  we  want  something
flatly  more  this  is  some  this  business
with  the  six  funs
um  so  we  also
want
X  to
X
as  um  so  let  me  say  a  little  bit  about
what  this
is  so  there  there  are  six  funs
um  there  is  always  a  fender  uh  which  has
some  kind  of  right  on  which  is  a
internal
H  uh  then  there  is  a  pullback  which  has
a  r  on  which  P
forward  um  and  then  and  these  will  be
things  we  already  have  and  then  there
will  be  two  more  that  we  would  like  to
have  you  would  have  to  like  a  FL
funter  uh  and  the  r
join  uh  so  always  this  one  is
left  what  what's  the  structure  here  so
each  d
ofx  uh  should  be  should  have  a
temper  I  a  symetric  model
and  then  whenever  you  have  a  tend  you
can  ask  for  an  internal  home  object
which  has  this  usual  Junction  to  to  the
T  end  if  if  such  internal  H  exists  is
called  a  close  symmetric
structure
um  then  whenever  you  have  a
MTH  F  from  y
tox  uh  there  should  be  a  pullback  fun  it
should  be  able  to  pullback
she  and  this  should  actually  be
compatible  with  a  tender  product  should
be  tender
fun
um  and  this  should  have  a
right  so  also  this  is  something  we  just
get  automatically  because  uh  like  when
you  have  a  Mor  of  rings  you  can  base
change  modules  and  also  then  this  the
same  fun  reality  on  D  ofx
and  it's  the  way  this  form  Works
automatically  if  you  change  the  modules
will  be  compatible  space  change  so  so
this  is  structure  that  we  definitely
already  have
and  yeah  uh  but  then  there  there's  these
lower  Street  fun  not  Street  fun  um  and
so  this  is  a  more  delicate  kind  of
structure  which  only  exists  for  certain
labs  for  certain  F  from  y  to
X  we  want
um
um  want  um  the  function  that  flow
straight
soal  formology  is  compa
support  uh  and  this  should  satisfy
uh  uh  two
properties
one  is  the  base
chain  uh  so  whenever  you  have  your
morphism
Y  and  you  have  any  base  change  of
this  first  of  all  the  base  change  should
be  again  in  the  star  where  this  work  is
Define  uh  but  secondly  there
be  like  taking  first  the  floor
streak  and  then  pulling  back  this  should
be  naturally  the  same  as  first  pulling  X
and
then
three  and  another  thing  that  should
satisfy  is  a  projection  form
isomorphism  that  for
a  and  v  of
x  b  and  of
Y  you're  trying  to  say  how  these  lower
Street  functions  they  interact  with  the
existing  structure  and  so  the  first
thing  is  that  with  a  pullback  they
should  just  commute  in  this  sense  and
then  with  a  tender  you  have  the
following  property  that  when  have  a  and
b  then  take  F  three
ofer  star  a
CER
B  the  same  thing
as  a
this
so  this  is  the  kind  of  structure  um  that
arises  in  a  lot  of  different  contexts  in
mathematics  um  the  most
classical  uh  in  some  sense  if  you  take
some  nice  of  top  space  something  locally
com  Co  of
spaces  uh  find  a  version  of  the  right
category  um  then  this  has  a  structure
actually  for
morphisms
uh  where  and  where  this  is  literally
callous  compa  support  so  you  looking  at
the  fun  of  sections  which  have  comple
support  uh  and  then  you  can  dve  that  F
what  with  setting
um  actually  the  first  time  this  was
developed  not  for  in  this  case  but  it
was  for  a  to  modu  schemes  I  think
forly  um  but  then  it  was  realized  that
there's  I  don't  know  there  you  can  also
do  this  for  D  modules  and  you  can  also
there  lots  of  different  settings  where
you  can  do  this  um  one  setting  where  it
was  however  not  so  much  developed  is  a
setting  Vari  now  is  some  kind  of
coherent  settings  or  quality  coherent
setting  where  usually  not  really  have  a
notion  of  comy  coherent  com  with  complex
support  there  is  this  appendex  of  dele
to  Heart  STS  resid  andity  where  he  kind
of  says  that  he  can  develop  some  s
formalism
but  and  there  there  are  a  few  papers
here  and  there  which  talk
about  complex  support  but  it's  not
really  uh  a  series  that  is  used  much  I
don't  say  um  but  actually  our  our
naturally  will  through  a  Ser  of  uh  C  for
compact  support  that  behaves  rather  well
and  we  we  do  absolutely  want  uh  to  have
it
uh  right  um  but  before  going  on  maybe
let  me  make  a  remark  because  I  think
govern  will  complain  in  second  that
there's  a  completely  imprecise
definition  here  of  what  the  six  fun
formalism  is  because
uh  when  I  write  this  isomorphism  here
there  is  no  natural  comparison  with
between  the  two  these  are  just  two
random
funs  but  they  should  naturally  be
identified  so  you  have  but  you  have  to
supply  this  isomorphism  but  once  you
start  supplying  this  isomorphism  then
you  run  into  trouble  because  now  I  don't
know  you  can  like  base  change  twice  and
then  they're  like  comparison  here
comparison  here  and  one  for  the
composite  well  of  course  you  would  hope
that  they  commute  but  then  you  start  to
wonder  how  many  different  such  things
you  can  write  down  and  which  kind  of
compatibilities  you  have  to  enforce
and
I  I  think  for  a  while  it  was  some  kind
of  open  question  what  what  a  really  good
and  minimal  way  to  encode  all  data  is
that  that  is  present  in  a  six  fun
formalism
um  and  there  has  been  work  by  Liu  Jang
uh  where  they  do  this  in  the  world  of
like  AR  Dex  and  there's  been  work  of  Gat
Ros  and
blo  where  well  they  actually  mon  Cent
although
the  their  fers  have  different  names  and
so  they  don't  really  talk  about  complex
support  chology  at  all
um  but  also  in  C  setting  they  they  set
up  some  kind  of  notion  of  what  such  as
things  fun  formalism  is
um  that  treatment  is  however  infinity2
categorical  and  so  uh  that's  difficult
uh  and
then  like  Lucas  man  maybe  really
isolated  a  key  key  structure  that  you
need  so  say  yeah
so  so  yeah  there  is  a
goal
no  such
anract  and  this  is  something  that  I  gave
a  course  about  last  winter  and  let  me  uh
simply  refer  you  you
there  right  and  so  so  but  something  uh
some  morals  that  I  also  personally  was
taking  away  from  this  course  is  said
um  uh  if  you're  dis  are  that  you  really
just  want  the  six  fun
formalism  uh  then  this  kind  of  dictates
everything  including  the  grth
topology
including  in  quotation  Mar  correct
go
all  right
so
um  right  so  yeah  before  starting  the
discussion
uh  in  our  our  specific
case  let  Rec  call  uh  the  usual
definition  of  such  an  FL  stre
fun
uh  like
usually  so  you  do  this  by  specifying  to
uh  collections  of  morphisms  so  there's  a
class
of  proper
morphisms  and  here  a  Flo  stre  will
actually  be  just  a  floor
star
um  uh  but  if  you  want  to  make  this
definition  you  better  check  that  it
satisfies  the  properties  that  uh  that  an
a  FL  should  satisfy
so  need  proper  Bas
thing  and  the  projection  formula  in
the  to  hold  um  but  this  time  during  like
better  situation
because  when  when  you  take  this  to  be  a
flow  star  then  there's  actually  always  a
natural  base  change
transformation  and  so  then  here  I'm  not
supplying  data  I'm  just  asking
conditions  so  when  when  you  want  to
declare  a  morphism  to  be  proper  and  so  a
FL  streak  to  be  a  FL  star  there's
something  you  can  simply  check  whether
does  the  natural  trans  base  change
transformation  is  an  alph  morphism  the
case  of  this  morphism  F  and  similarly
for  the  projection  form
there  is  a  natural  map  from  here  to  here
if  you  think  about  it  when  when  this  FL
start  um  so  these  are  just
conditions  there's  no  data
Supply  okay  so  there  there's  a  clause  of
prop  morph  uses  and  then  there  is  a
clause  of
open
immersions  where  instead  you  asking  that
of  nor  Street  I  mean  here  he  ask  that
it's  the  right  joint  of  P  for  open
emergency  it's  the  left  joint  of
PB
and  again
uh  um  you  need  to  check  that  this  is
reasonable  definition  so  you  need  to
check  that  satisfies  space  change  and
projection  formula  but  this  time  again
also  if  it's  a  left  ad  joint  there  is  a
natural  uh  uh  comparison  M  between  these
two  things  this  time  going  in  the  other
direction  uh  and  you  ask  that  the  these
things  are
satisfied  again  base  change  and
projection  formula  are  conditions
better
and
uh  then  the  general
Nets  f  it  are  low
or  call  them
streetable  these  these  are  the  ones  that
are  for  which  this  no  streak  fun  is
Define
um  they're  taken  to  be  the
Composites  of  an  open
merer  and  a
proper  uh  so  we  have  open  called
problem  that  that
far
said  these  Ms  that  you  can  choose  are
the  ones  which  you  can  somewh  compactify
like
that  and  then  you  want  to  declare  that
the  street
is  comp
streets  where  the  case  have  no  have  L
that  here  this  left  joint  of  P  get
right  uh  this  definition  comes  up  prior
with  a  really  big  caveat  that  when  you
write  down  this  as  a  definition  it's  not
really  a  definition  because  it  chose  the
compactification  in  general  there  are
many  many  possible  ways  to  comp
vile  um  and  so  you  have  to  show  that
this  definition  is  independent  of  the
choices  canonically  because  I  mean  you
really  want  to  get  inal  structure
so  you  don't  need  just  need  to  show  the
some  unique  up  to  isomorphism  but  even
the  isomorphism  is  unique  up  to  high
isomorphism  and  so  on  and  so  on  and  so
on  and  so  on  um  uh  so  this  sounds  like  a
real  pain  but
fortunately  there  is  a  serum  that  was
essentially  proved  by
and  then  like  slightly  streamlined  in
this  formulation  by  Lucas  man
um  that  under  really  minimalistic
assumptions  on  the  classes
of  proper  Maps
open
Emer  and  some  general  three  Maps  so
essentially  just  what  I  said
except  like  for  example  you  want  that  a
composite  of  fre  Maps  is  still  shootable
so  you  need  that  a  composite  of  two  such
Maps  uh  again  can  be  compactified
U  so  there  are  some  very  small
assumptions  but  they're  extremely
minimalistic
um  uh
the  theem  is  that  Z  can  always  produce
such  such  an  abstra  an
autic  the  preise  theorem  is
in  this  lecture  notes  on  or  in  luas  CES
or
whatever  in  particular  these  assumptions
include  no  condition  whatsoever  onto
unique  of
compactifications  so  do  you  assume
something  like  the  two  compactifications
can  be  dominated  by  a  third  one  no  no  no
uh  is  it  a  question  there  is  a  question
in  the
chat  I  mean  these  are  the  kind  of  things
you  want  to  do  with  this  yeah  so  the  the
question  in  chat  is  using  can  formology
characteristic  classes  index
Antics  yes  that's  the  kind  of  things  we
hope  to  do  with  this
stuff
um  right  so  gaba's  question  was  uh  about
um  whether  I
should  yeah  assume  that  two
complexifications  can  be  dominated  by  a
third
um  well  there  is  something  that  follows
from  the  mere  existence  of  comp  of  morms
um  right  so  you  you  have  two  then  he  can
take  a  fiber  product  which  is  a  third
thing  that's  still  this  m  should  still
be  so  it  might  be  that  a  little  bit
follows  but  it's  it's  not  explicitly
mentioned  that  you  have  to  check
something  like  this
see  you  need  to  know  that  all  maps  which
are  shable  are  such  Composites  but  you
just  ask  for  each  shable  map  you  just
need  to  produce  one
complexification
um  all
right  right  so  uh  so  now  we
apply  these  General  ideas  to  our
setting
so  so  start  we  start  with  the
category  or  infin  category  um  C  which  is
analytic  Rings
up  and  we  call  them  F  analytics
section  if  you  want  analytics
spaces
and  uh  we  will  generally  call  an  object
in  here  the  analytic  spectrum  of  of  some
a  but  the  unpe  does  I  mean  it's  just  a
symbol
yeah  there  is  no  topological  space  now
it's  it's  just  a  symbol  to  to  say  we're
taking  over  geometric  object  passing
through  the  opposite  here  um  all  right
and  so  we  need  to  figure  out  which  morm
here  should  be  proper  which  should  be
open
emergence  uh  and  at  this  point  we  should
forget  any  preconceived  notion  of  what  a
prop  morphism  algebraic  geometry  is  and
really  just  look  at  what  the  formulism
tells  us
okay
uh  and  that  F  un
a  so  in  other  words  map  of  antic  R  from
HB
um  is
proper  so  usually  you  would  expect  this
virtually  never  case  or  it's  only  for
finite  Maps  now  it  will  be  a  rather
General  thing  okay  um  so  this  notion  of
properness  is  different  from  the  usual
one  you  use  in  alge
geometry  if
uh  from  D  of  D  to  D  of
a  right  so  this  is  simply  just  forgetful
fun  right  you  morph  of  rings  from  A  to  B
so  in  particular  any  B  module  is  an  a
module  is  a  sa  projection
formula
a  prob  I  should  ask  for  more  like  for
basent  and  so  on  it  turns  out
that  if  this  happens  then  all  the  other
good  things  you  want  are  also  true
so
you
uh  so  to  analyze  to  analyze  this  not  of
properness  uh  let  me
recall  uh  so  if  you  have  such  a
m  uh  this  will  always  factor  in  the
following  way  so  you  can  always
uh  first
take  this
thing  in  t  analytic  R  let  me  just  second
so  what  is  this  this  is  an  triangle  and
The  Cloud  of  all
those  the  triangle
modules  such  that  if  you  forget  the  B
module  structure  down  to  an  a
modu  structure  so  triangle  module
structure  this  is  uh  complete  in  the
cent  of
a  and  so  some  such  things  we  call  an
induced  analytic  ring
struction  so  do  you  implicitly  work  with
animated  ring  and  take  modules  over  Pi
zero  as  the  category  yes  yeah  yeah  yeah
yeah
um  yes
uh  okay  let  let  me  use  D  notation  so  he
was  always  talking  about  in  The  Animated
context  he  was  talking  about
this  he  was  talking  about  a  pair  of
a  animated  condensing  light  condensing
uh  and  the  subcategory  of  the  like
connective  der  category  satisfying  some
conditions  and
then  I  could  just  as  well
say  and  of  course  working  the  context
the  T  project  incly  derived
too
um  right  so  whenever  you  have  any
analytic  green  and  just  an  animated
algebra  over  the
underlying  animate  cond  algebra  here
then  you  can  always  endow  this  one  with
an  antic  ring  structure  where
completeness  is  just  completeness  where
you  restrict
the  reaction  to  the  suffering
um  right  so  whenever  we  have  any  map  of
analytic
Rings  there's  an  induced  one  and  then
there  is  some  kind  of
localization  where  you're  just  taking
the  same  ring  but  then  doing  a  f
completion  so  you  always  have  the
spectation  um  and  this  it  turns  out
geometrically  precisely  corresponds  to
the  the  process  of  compactifications
basically  except  this  may  may  or  may  not
be  an  open  Emer  but  these  are  these
structures  they  are  pris
prop  okay  but  here  here's  a
proposition  as  as  above
proper  if  only
is
um
B
hased
saan  is  saying  that  the  localization  is
ISS  right  yeah  this  the
second
um  maybe
before
uh
uh  going  there  um  the  only  reason  this
it  all  seem  like  particular  if  you  like
work  with  if  you  usual  algebraic  Rings
commuter  Rings  into  this  condens  setting
then  you  would  like  do  the  probably  by
choosing  the  trivial  antic  ring
structure  and  then  any  map  of  commu
rings  and  usual  algebraic  geometry  will
be  declared  to  be  proper  which  seems
very  strange  the  only  reason  to  seem  at
all  sensible  to  me  is  actually  because
uh  this  did  seem  familiar  from  uh  H's
tic  spaces  so  in  the  Ser  of  H  tic
spaces  so  yeah  so  this
matches
notion  what  apply
to  uh  than  for  the  central  image  of  like
your  which
s
yes  series  there's  always  this  A+  and  if
you  have  a  map  of  AIDS  where  the  A+
upstairs  is  like  the  smallest  possible
thing  it  can  be  determined  by  the  a  plus
downstairs  and  this  is  something  calls
already  proper
um  and  turns  out  that  it  satisfies  also
in  the  T  hology  all  the  good  properties
that  it  proper  map  should  satisfy  like
probation
um  do  you  take  ear  analytic  Huber  pairs
or  or  or  your  addic  Maps  or
or  um  implicitly  I  might  be  thinking
about  the  like  let  me  say  take  just  so
that  I'm  completely  confident
uh
all  right  um
Pro  okay  so  what  does  the  projection
formula  say
um
um  all
a
I  want  to  say  ask  l
f  m  tender  over  B  with  n  it's  the  same
thing  as  mender
8  so  both  sides  here  are
okay  so  what  does  this
really  say  here
um
sorry  what  should
um
so  what  you're  doing  here  is  BAS
changing
a  m  from  A  to
B  this  is  what  F  upper  star  is  and  then
you  change  the  r  over
B  was
say  the  question  was  say
El  and  this  seems  like  it  should
obviously  be  true  because  you  should  be
able  to
cancel  um  but  now  you  have  to  be  a
little  bit  careful  about  meaning  of
symbols
and  then  just  doing  structure  you  want
the  that  tole
be
thanks  um  so  okay  so  let's  analyze  this
so
uh  if  has
instru  then  all  ters  can  be  understood
to  be  CH
in
in  or  or  like  the  relative  to  and  then
just  checkes
out
um  so  this  is  One
Direction  um  but  let's  say  in  the
opposite  case  let's  assume  if  a  triangle
speed
triangle  um  then  base  changing  to  B  is
just  some  sort  of
completion
to  and  let's  even  forget  the  N  like  even
if  n  is  just
B
uh
then  like
take  and  to  be  just  each
triangle  oh  I  see  under  Lin  say
uh  which  is
H  then  then  on  neither  on  the  left  nor
on  the  right  this  tendering  do  does
anything  right  you're  just  tendering
with  a  unit  um  so  then  we
get  that  M  tender  AB  must  be  equal  to  B
to
M  this  is  what  this  formula
says
uh  so  in  other  words  the  further
completion  uh  shouldn't  actually  do
anything
right  was
already  and  so  says  that  if  a  triangle  B
triangle  and  the  map  is  proper
satisfying  the  projection  formula  then
actually  theing  must  be  the
same
all  right  and  so  the  so  in  general  you
just  have  to  do  the  same  argument  uh
more  carefully
um  namely  I  mean  in  general  you  can  just
take  n  equal  to  B  uh  then  the  left  hand
side  becomes  the  base  change  from  M  to
of  M  from  A  to  B  in  the  sense  of
analytic  rings  but  the  right  hand  side
is  a  base  change  from  of  M  from  A  to  B
purely  uh  in  the  sense  of  complete  a
module
and  so  that  these  agree  is  precise  in
the  assertion  that  uh  it's  in
very
for
so  you  get  this  equation  uh  where  the  T
of  these  meanings  and  uh  that  they  agree
that  the  base  change  in  the  S  of
analytic  is  just  the  tensor  of  a  is
precisely  saying  that  this  must  be
induc  because  for  the  IND  ener  structure
is  precisely  how  you  comp  to  Bas
change  in  general  when  you  compute  the
base  change  foric  rinks  and  first  you  do
the  base  change  for  juice  so  you  do  this
but  then  afterwards  you  would  still  have
to  make  it  complete  as  a  b  module  but
here  it's  saying  that  you  don't  actually
have  to  do
it
right  uh
okay  was  a
question  I'm  not  sure  what  the  question
like
like  the  map  from  in  usual  albra
geometry  the  map  from  the  fine  line  to  a
point  is  not  proper  but  if  you  like  pass
through  this  word  it's  counted  as  a  prop
here  because  satisfies  the  projection
will  also  satisfy  prob  base  change  and
so  on  I  mean  actually  a  classical  base
change  the  for  C  geometry  so  you  don't
need  to  ask  for  prop  you  only  need  to
ask  for  qcq
estimate  all  right
so  and  Dustin  would  be  able  to  give  to
really  Sol  you  was  this  this  is
good
um  right  so
so  it's  just  a  cloud  of  proper
map  it's  on  the  base  St
um  uh  and  proper  base  change  and  the
projection  form  proper  base
change  same  base  change  or  it  tells  you
that  after  any  base  change  still  the
projection  formula  holds  uh  but  also
proper  base
CH
and
he  simple
check  if  you  haven't  induced  any  the
structur  then  you  can  just  unravel  what
all  symbols
mean  comes  out
right
all  right  so  let's  get  to  the  class  of
open  emergence  and  again  I  ask  you  to
forget  any  preconceived  notion  of  what
an
open
open
merging  I  will  give  examples  in  a  second
uh  let  me  actually  call  it  J  just  for
psychological  Comfort  um  merge  and
if  the  P  map  B  of  modules  Adit
left  three  uh  set  the  projection
from
so  let  me  give  first  on  example  and  then
an  example  so  non
example  is  any  open  immersion  algebraic
geometry  that's  not  also  a  tost
immersion
open  close  immersions  okay  there  I  don't
know  they  kind  of  stupid  but  uh  like  if
you  take  I  don't
know  the  GM  Z  joint  t  plusus  one  inside
of  some  Z  joint  t  uh  with  like  like  triv
ring  structure  this  is  not  an  open
imersion  because  I  mean  usually  in  Al
geometry  pack  like  tendering  of  modules
virtually  never  commutes  with  any
infinite  products  right  if  you  want  left
joint  should  means  that  the  pullback
should  commute  with  products  it
just
um  uh  all  right  uh  but  here's  but  there
is  a  way  using  some  uh  tic
spaces  and  like  relative  solid  ring
structures  so  there's  a  different  way  to
Su  a  path  from  algebraic  geometry  to
analytic  spaces  using  solid  analytic
ring  structures  and  if  you  do  that  then
actually  it  is  okay  then  open  immersion
go  to  open  immersion  so  here's  a  key
example  if  you
take
um  like  here's  one  example  um
the  UN
spec  of  Z  join  P  plus  minus  one
solid  this  will  be  an
example  or
also
um  and  sometimes  more  primitive  because
this  one  can  be  realized  as  the  base
change  of  this  one  not  quite  but
essentially  like  the  key  thing  to  know
is  that  if  you  take  like  these  joint
solid  and  Advance
this  into  the  induced
one  solid  uh  this  is
open  this  is  something  that
already  came  up  in  a  remark  once
somewhere  in  the  lectures  but  that  I
want  to  spend  a  little  bit  of  time  on
now
again
I  forget  something  the  definition  I
wanted  to  be  an  open  version  not
just  which  for  which  I  should  ask
that  left  the  Johnson  before  he
face
it's  aival  to  the  rer  Joint  which  always
exists  J  start  to  be  fully  fix  basically
some  kind  localization  here
um
um  right  so  why  is  that  two
uh  so  what  is  J  upar  the  second
example  uh  jar
modu  is  the  internal  home
from
uh  all's  let's
say  if  I  if  I  take  a  module  on  the  base
pull  back  and  then  push  down  again  so
that's  some  the  completion
let
notification  of  an  M  which  is  in
the
and  uh  Dustin  show  how  to  compute  this
in  terms  of  this  from  this  funny
object  this  thing  should  receive  a  map
from  M  so  the
r  um  and  indeed  I  mean  this  complex  it
Ms  to
Z
homologus  homological  degus
um  so
this  which  on  our  homes  gives
me  which  see
Junction
um  I  think  yeah  if  you  look  back  at
Dustin's  lecture  on  like  sertification
of  a  z  joint  ke  this  is  Formula  he
gave  and  this  means  that  actually  if  you
tender  with  this  object  then  this  would
become  the  left
joint
um  so  it  is
m  i
in  this  one
Str  and  here  thiss  mean  the
comp  and  so  this  means  that  this  left
the  joint
exists  uh  on  the  image  of  jaapa  but
jasta  is  essentially  subject  so  this
um  It  also  says  that  like  this  left  just
given  by  tender  some  module  which  is
exactly  what  you  need  to  check  that
projection
for  all  right  uh  so  why  is  this
reasonable  why  why  does  this  have
anything  to  do  with  the  intuitive
idea  that  these  lower  Street  funs  there
should  be  some  kind  of  com  with  compact
support  So  in
particular  like  of  the  structure
sheet  of
ECT  uh
the  um  I  the  Upp  speak  of
some  so  this  should  be  like  the
compactly  supported  coherent  chology  in
some  sense  of  the  airine  line  um  well
what  should  it  take  to  someh  give  a  like
a  compactly  supported  thing  should  give
you  yeah  if  you  want  a  function  on  your
thing  which  has  compa  support  should
should  vanish  near  in  so  giving  a  giving
a  function  well  that's  an  element  Z  Jo  T
and  now  you  want  to  say  that  ADV
Infinity  but  like  an  Infinity  is  going
to  get  functions  the  Infinity  of  this
long  Series  in  t  inverse  and  so  you
would  like  to  take  functions  that  vanish
near  Infinity  so  that  L  the
kernel  unfortunately  like  any  po
function  is  detered  by  Infinity  so  map
injective  but  as  a  complex  still  a
reasonable  thing  to  do  and
you  have  a  CO  which  is  non  Tri
nice  it's  actually  just  a  product  of  Z
so  like  a  complex  object  z
um
for
and  so  the  key  thing  making  this  work
here  that  you  get
some  uh
such  uh  localization  with  this
properties  and  like  the  key  thing  you
need  to  check  when  you  want  to  check
that  for  example  this  is  what  the
completion  does  is  that  Z  inverse  is
animportant
algebra  so  it's  an  object  if  you  T  it
with  itself  STS
itself  all  right  and  this  is  in  fact  the
uh  General  description  of  these  open
emergence
um
so
given  some
x  a
um  the  open
imersion  J  from  some
Y  into  X
um
are
equivalent  of
such  is  equivalent
to
uh  either  equivalent  or  anti  equivalent
I'm  not  sure  now  right  now  um
to  uh  item
potent  Comm
Algebra  I
know  let  me  call  C  um
Ina  not  all  of  them  but  those  such
that  the  internal
home  from
C  uh
551
preserves  some  contivity
assumtion  so  here
Jay  maps  to  um  the  following  so  you  can
check  J  freak  of  the  unit
and  this  will  always  M  back  to  the  unit
and  the  cone  of  that
map  this  you  can  check  is  always  an  item
P  algebra  so  it  always  comes  so  what
is  to  find  item  algebra  you  have  to  only
Supply  little  data  you  only  have  to
supply  the  unit  and  then  you  just  have
to  check  that  when  you  take  this  unit  in
tend  we  see  it  becomes  an  ni
morphism  but  if  you  take  this  map  intend
it  with  uh  C  again  uh  then  well  it's
some  simple  matter  of  using  the
projection  formula  for  J  stre  to  see
that  nothing
changes  the  key  here  is  that  if  you  take
jow  streak  of  one  and  10  J  stre  of
one  J  streak  of  one  10  of
one  but  just
J  so  also  J  of  it  would
be  so  it's  also  equivalent  to  item  to
algebras  J  stre  of  the  unit  would  always
be  an  it  coalgebra  but  but  I'd  like  to
take  talk  about  coalgebra  so  much  and  it
turns  out  that  they  anyways  canonically
equivalent  to  algebra  why  this  fun
procedure  taking  the  cone
to
sorry  there  something
else  such  as  this  fun  preserves  all
Po  and
some  come  back
here
H  how  does  the
one  uh  well  it's  a  con  right  so  one  just
maps  The  Cone  right  you  have  con  from  X
to  Y  then  always  y  Maps  into  the  con
okay
whenever  always
from  right  I  there  always  X  to
Y  just
this
um  right
so
uh  right  I
mean  let  me  just  say  a  few
words  the  projection
formula  implies  that
stre
module  is  really  just  the  streak  of  the
unit  and  the
rest  and  this  here
s
um  and  so  this  means
that  this  thing  is  completely  determined
by
uh  right  and  so  right  and  so  this  means
that  uh  the  right
on  this  is  actually  just  given
by  uh  the  right  fun  so  it's  actually
just  the  internal  home  from  day  because
the
unit  right  so  the  left  joint  tendering
with  some  object  is  our  home  from
the  and  so  we  know  that  the  completion
is  really  just  uh  given  like  this  um  and
so  this  means  that  yeah  like  particular
this  this  new  analy  green  structure  is
completely  detered  by  this  object  and
then  you  just  have  to  specify  or
equivalently  by  by  by  this  Co  right
because  then  we  can  recover  it  by  taking
the  fiber  from  one  to
C  um  and  then  then  you  just  need  to
supply  the  conditions  on  this  so  that
this  completion  really  determines  the
analytic  R  structure  and  so  for  this  you
need  to  check  two  things  basically  that
this  commutes  this  completion  commut
with  all  Co
limits  uh  which  is  one  condition  I  just
put  and  the  other  condition  is  that
should  preserve  connectivity  U  and  the
connectivity  comes  down  to  this  other
condition  I  put
um
and  yeah  all  the  other  properties  of
antic  R  structure
aut  it's  just  some
formal  formal  procedure  to  check
that  right  so  basically
uh  when  you  have  an  open  Merion  then
there  is  always  this  item  but  commu  AUD
which  you  think  of  as  some  Comm  of
algebra  describing  a  space  near  Infinity
right  so  in  the  sense  of  this  openion
you  have  this  open  imersion  and  then
there's  some  kind  of  complimentary
closed  subset  determined  by  some  algebra
which  is  functions  infinity  and  these
are  other  important  algebra  so  that's
the  general  idea  that  whenever  you  have
open  emion  and  there's  a
complimentary  th  stop  at  Infinity  which
is  described  by  some  item  po  algebra
this  and  again  as  a  call  of  this  you  can
check
that
uh  the  class  ofion  is  stable  on  a  base
change
and
this
all  right
so  so  we've  isolated  what  we  want  the
proper  me  open  measures  to  be  or  not
what  we  want  them  to  but  the  follow  them
tells  us  what  they  are
um  and  then
uh  this  leads  us  to  a  shable
map
if  well  if  you  fact  open  work  and  the
proper  map  but  we  already  know  which
proper  map  to  take  because  I  already
told  you  about  this  canic
characterization  of  any
moris  structure  and  then  some  kind  of
localization  so  this  F  it  always
spectors  some  canonical  F
Bar  in  this  case  the  compactification  is
canonical  which  is  nice  uh  from
this  so
this
thing  get  let's  call  it
J  this  J  always  exists  this  will  also
always  have  the  property  that  J  does  for
the  faceful  uh  so  the  real  only
condition  is  that  yeah  someone  have  left
joint  the  projection  formula  so
if
and  the  interesting  example  for  example
here
is  if  you  take  the  solid  structure  on  Z
and  that  which
is
solid  this  what  satisfies  in  this  case
is  a  complexification  but  precisely
these  one
over  so  this  far  this
that  more  generally  like  if  you  have  any
M  fire
type  algebras  and  take  the
solid  the  relative  solid  ring  structure
then  these  map  will  all  all
beable  or  if  you  more  General  if  you
have  like  map  of  U
pairs  then  being  shable  is  exactly  this
condition  that  hu  calls  being  of  plus
weekly  finite  type  which  just  means  that
this  the  suffering  of  integral  elements
A+  is  as  a  ring  of  integral  elements
generated  by  just  finding  many  new
elements
so  H  was  defining  these  kind  of  fin
Notions  also  for  six  but  for  talk  these
where
notion  Mor  for  which  you  could  Define
stre  fun  this  uh  it's  hard
spaces  but  now  they  are  also  the  ones
where  you  can  Define  Ro  Street
fun  this  compatibilities  for  proper  open
ersion  out  there  as  well
yes  yes  so  that's  next  yeah  so  there  is
a  little  bit  of  check  like  if  you  take  a
composite  of  two  still  recable  that's
true  um  basically  just  by  comp  on  the
base  chain  and  then  for  the  six  fun
there  one  you  have  to  check  which  is
some  interaction  between  the  low  and  the
low  star  fun  for  open  mer  proper  Maps
again  that's  very
check
so  I  didn't  discuss  in  this  lecture  like
all  the  little  EXs  you  have  to  put  to
make  theism  work  there  are  some  little
little  bits  to  check  but  each  of  them  is
like  a  really  simple  check  uh  and  so  as
a
proposition  basically  a  COR  to  this
General  construction  of  sucess  some  the
fall  re  you  can  someone  else  say  that
yeah  this  data  like  of  proper  Maps  open
maps
satisfies  all  the  required
conditions  to  get  enough
FM  get
XEX  on  on  our  categ  C  which  I  recall
was  or  you  want  f
x  um  and  uh  so  cloud  of  three
Maps
so  in  our  the  first  C  we  gave  on  my
condensed  mathematics  uh  this  was  the
aim  of  this  was  precisely  to  uh  explain
this  in  the  case  of  like  formology  so
there  we  some  specialized  to  EIC  spases
and  in  fact  the  ones  where  the
underlying  ring  is  just  theet  and  we
work  with  the  solid  modules  and
particular  these  examples  and  then  we
discussed  that  there  should  be  such  a
thing  we  didn't  actually  fully  construct
it  then  because  we  didn't  have  the
required  technology  really  so  we  nicely
set  it  up  but  now  it's  really  just  a
consequence  of  these  General  existence
results
fors  and  uh  then  it's  starting  from  this
it's  really  easy  to  prove  and  this  is
what's  done  in  these  Mones  to  prove
titles  of
prology  and  uh  sity  and  so  on  and  so
for  all  right
okay  so  that's  that's
nice  um  you  get  this  nice  Ser  of  in
particular  get  some  nice  Ser  of
complexing
uh
but  uh  I  started  this  lecture  by  wanting
to  define  the  SP  topology
sorry  uh
so  here's  a  general
question  uh  that
I  so  given  the  six
fun  of  X
um  from  some  some  some  category  C
towards  some
categories  yeah  on
some  um  you  may  want  to  pass  to  larger
category  of  geometric  objects  which  are
some  built  by  gluing  objects  and  see  uh
just  like  um  schemes  both  from  fine
schemes  and  so  on  or  more  TT  so  uh  you
want  to  try  to
extend
the  uh  to
Chiefs  Chiefs  on  C  Chiefs  of  C  was
values
in  if  you  want  what  we
call  any  um
uh  for  some  notion  of  for  some
of  and  this  in  particular  was  maybe  the
original  question  that  Leo  and  Jen  were
interested  in  when  they  uh  when  they
wrote  their  their  paper  some  wanted  to
extend  from
the  from  schemes  to  AR  Stacks  or  higher
AR  Stacks  something  like
that  and  I  Pro  some  general  uh  General
results  about  how  one  can  go  about
extend  form  some  from  one  category  to
category
um  and
then  some  of  this  was  also  used  by  Lucas
Min  was  start  rephrasing  it  and  then
when  I  get  to  SC  I  again  slightly
rephrased  what  they  did  and  uh  uh  so
yeah  so  in  this  in  these  notes  on  six
funters  I  tried  to  analyze  this  question
and  try  to  some  pin  down  like  what  the
best  apology  is
um  find  such
discussion  including  the
question  including  the  general
definition  of  growth  need
for  did  I
a  d
there  no  I'm  actually
not  like  I  maybe  not  completely  happy
with  just  a  very  precise  combination  of
this  topology  in  the  most  case  it  works
but  it's  slightly  ugly  at  one  point
um  but  there's  still  takeaway  from  this
so  that  uh  if  you  want  to  do  the  X
tangent  the  take
away
it  said  the
covers  should  basically  be
those  let
satisfy  Universal  Universal  descent  what
I  call  Universal  star
desent  and
Universal  and  I  will  explain  what  I  mean
by  this  one  in  just  a  second
this
for
um  let's  say
mically  to  the
Limit
Delta
of
and  then  there's  a  whole  simpal  diagram
here  I
guess  um  where  this  here  is  a  pullback
and  then  here  is  P1  upper  star  P2  upper
star  and  everywhere  you  put  upper
stars  and
Universal  this  means  the  same  after  in
Bas  Universal
start
and  then  there  is  a  similar  s  where  you
replace  the  oppos  St  by  the  OPP  Treet  so
map  but  then  of
course  the  map  admit  such  soakable  map
now  the  same  thing  holds  but  for  the
upper
three  so  f  it  Street  fun  and  any  P  of  it
should  also  have  Street  fun  so  all  these
s  and  again  Universal  means
auen
it's  actually  small  sub  here  which  is
precisely  why  I'm  not  with  topology  um
in  setting  this  up  abstractly  I  actually
need  to  set  ask  this  in  the  case  of
shriek  descent  I  need  to  ask  this  after
base  change  to  any
preceive  which  is  a  really  awkward
condition  um  I  don't  want  to  do  that  and
I  don't  do  it
now  so  when  I  say  any  base  change  here
here  I  really  just  want
to  B  change  that  staying  in  my
categories
all  right  so  don't  have  much  time  so  let
me  just  St  a  serum  that
actually  uh  in  the  case  we're  in
actually  like  up  you  should  ask  for
Universal  start  and  Universal  which
seems  like  a  lot  to  check  actually  you
need  to  check  much  much  less  uh  to  some
extent  is  always  any  sixun  but  I  think
it  seems  better  here  so  theem  is
that  if  a  streetable
net  to
X  satisfy  St  this
then
then  actually  Universal  to  and
univers  in  fact  it  satisfies  something
even  stronger  it  satisfies
to  the  end  so  if  you're  not  thinking
about  like  usually  we  have  a  ring  and
you  think  about  the  category  of  modules
but  now  we  can  go  one  categoric  level
higher  and  think  about  presentable
stable  Infinity  is  linear  over  D
OFA  and  it  turns  out  that  asking  for
Street  descend  is  equivalent  to  asking
descend  at  a  two  categorical
level  which  is  actually  how  the  serum  is
proved  but  this  is  probably  not  one  I
want  to  do  now  in  the  last  five  minutes
um
one
category
um  so  let  me  just  end  by  giving  the
definition  of  the  gr
project
we  that's
t  uh  was  the  brok  to
fall  well  gener  on  the  one  hand  just  by
fire  just  unions  so  whenever  you  have
a  antic  St  just  Union  of  several  then
it's  covered  by
those  this  includes  the  empty  cover  of
the  empty
set
um  and  these  stream  mtis
F  so  this  is  a  rather  General  class  of
things  it's  actually  uh  very  very  close
to  related  to  this  notion  of  descend
ability  that  AR  defined  um  which  is  some
yeah  nice  notion  nice  notion  of  descent
that's  satisfied  for  virtually  all
faceful  flat  maps  at  least  all  those
that  are  accountably
presented  which  will  be  another
motivation  actually  for  us  to  at  one
point  switch  to  the  lightens  setting
because  there  things  accountably  the
Rings  are  accountably  generated
um
right
yeah  so  particular  if  we  restrict  to
proper  Maps  then  this  condition  of  stre
descend  is  precisely  the  the  condition
that  the  map  of  algebra  is
descend
um  yeah  have  thaty  and  then  okay  they
like  in  the  beginning  of  my  lecture  I
said  that  there  are  two  small  wrinkles
one  about  set  Series  so  again  we  should
look  at  accessible  things  and  for  this
we  should  check  that  this  gr  topology
has  some  approximation  properties  that
means  that  specification  of  something
acceptable  stays  acceptable
the  other  some  hyper  complet  issues  is
that  we  actually  don't  want  the  thing
that's  just  chaps  on  this  because  also
on  the  light  condens  setting  we're
actually  considering  hypers  Chiefs  so  we
actually  want  to  allow  a  certain  class
of  hyper
but  uh  it's  it's  the  same  ideas  that  go
into  the  CL  of  hyper  about  but
okay  basically  analytics  TX  will  be
sheets  on  F  analytics  TX
forsus  all  right  let
me
question  what  do  you  mean  by  fin
joint  well  I  mean  you  can  have  like  f
analytics  STX  they  have  fin  unions  I
mean  du  like  rings  and  fin
products  and  so  whenever  I  take  such  an
F  fin  dint  Union  if  an  F  analytic  Tes  it
should  stay  this  fin  dint  Union
in  analytics  ST  which  like  it's  covered
by  the  individual  wants
right  no
just  this  is  a  cover  whenever  I  is  a
finite  set  and  x  i  SE  then  this  is  also
an  so  like  for  for  Prof  sets  we  were
looking  at  all  subjective  NS  and
fin  so  just  includ  the  EMP  includes  the
empty  cover  of
the  so  for  for  usual  schemes  a  fine
schemes  so  what  are  the  the  the  shable
maps  and  what  are  the  those  which
satisfy  the  scent  for  shrick  or  for  star
and  what  is  the  topology  I  mean
yes  you  could  do  the  same  discussion
just  with  usual  commu  reins  uh  in  that
case  I  would  declare  all  my  maps  to  be
proper  all  my  maps  of  f  seems  to  be
proper
um  seems  bit  awkward  but  it  works  um  and
it's  actually  like  in  this  largic
category  of  analytics  it's  actually
reasonable  um  anyway  so  all  all  maps  of
f  are  shakable  the  condition  of
satisfying  shink  descent  is  precisely
the  condition  that  the  M  of  algebra  is
what  Arc  called
descendible
um  descendible  maps  that  the  ra  General
class  for  example  I  mean  it  includes  all
accountably  presented  Facey  flat
Maps  it  also  includes  all  quots  by  NE
poent
ideals  maybe  more
surprising  it  also  includes  all  H
covers  so  this  is  wot's  H
topology  um  so  you  find  schemes  that
covering  wot's  H  topology  that's  also
allow
here  actually  when  you  restrict  to
theeran  schemes  uh  that's  equivalent  so
when  Y  and  X  are  fine  and  thean  then
this  satisfies  this  condition  of  descend
ability  I  mean  satisfying  STC  if  and
only  if  it's  a  cover  and  what's  H  do  you
mean  V  topology  you  have  the  H  topology
with  present  I  restricted  to  yes  it's
there  are  some  questions  about  POs  to
the  limit  which  are  a  little  bit  subtle
but  when  I  in  a  seran  setting  then  H  and
V  cover  are  the  same  anyways
so  um  so  I  want  sorry  I  wanted  to  fin
presented  map  of  thean  schemes  maybe  and
then  it's  it's  a  cover  here  if  and  only
if  it's  a  h
cover
if  okay  and  what  is  The  Descent  so  when
you  have  Shri  descent  do  you  also  have
star  descent  or  those  are  unrelated
that's  what  I  said  here  right  so  if
satisf  I  mean  the  same
Ser  some  special  case  so  if  it  satisfies
descent  satisfies  Universal  descent  and
Universal  St  descent  ah  this  is  just
didn't  read  it  properly  and  what  is  the
other  statement  what  is  the  meaning  of
in  descent  for  for  infinite  two  of
module
categories
uh  okay  let  me  just  very  briefly  do  it  I
mean  I  think  this  will  take  a  whole
other  lecture
um  so  it's  an  extremely  wonderful
structure  that  Jacob  L  def  find  it  this
category  of  presentable  Infinity
categories  with  call  preserving
functions  uh  and  in  there  you  have  a  sub
category  of  stable  ones  or  also  like
whenever  you  work  over  some  ring  you
have  some  categories  which  are  linear
over  some  base  ceg  D  of
a  this  itself  so  the  objects  are
something  like  DG  categories  and  which
ofas  vely  Speaking  that  there's  a
precise  definition  of  what  this  means  as
the  funes  are  colum  preserving  fun  um
and  this  has  a  well  maybe  me  I  mean  this
also  has  a  Tang
structure  um  and  now  you  can  also  ask
the  question  whether  the  association
that  takes  such  an  A  or  the  unspe  of
it  to  this  Infinity  one  or  Infinity  2
category  actually  it  doesn't  matter
which  one  you  choose  for  this  purpose  of
asking  for  descent  that  this  satisfies
descent  so  by  modular  categories  I  mean
objects  in  this  same
and  Jacob  the  is
formalism  so  the  okay  so  the  so  for
example  a  derived  category  of  something
over  a  is  is  of  some  B  over  a  is  in  this
category  derived  category  of  an  a
algebra  is  right  right  yes  yes
yes  H  okay  so  you  some  of  you  giving
such  things  locally  then  you  are  given
the  you  I  mean  check  the  send  at  list
for  okay  and  what  why  is  it  Infinity  so
the  home  between  two  such  things  is
viewed  as  a  as  a  as
itself  most  naturally  would  be  Infinity
one  category  actually  for
this  I  mean  you  can't  forget  about  the
non  inable  two  morphisms  and  then  you
get  an  autic  one  category  and  actually
for  the  question  of  the  same  turns  out
it's  like  two  morm  the  non  ones  will
automatically  satisfy  the  S  as  well  so
in  some
sense  the  two  categoric  structure  of
this  thing  doesn't  matter  so
much  uh  and  yeah  so
uh  right  for  these  questions  I  would
really  recommend  that  you  read  this
paper  of  AR
mesu  which  some  awward
what
all  right
questions  so  for  for  the  last
statement  like  from  a  to
present  which  type  of  descent  what  does
it  mean  it  set  by  be  set  well  I  mean  is
a  pre  right  I  mean  so  you  can  just  when
map  of  range  you  can  base  change
mod  basically  and  this  pre-  should  be  a
she  so  it's  some  kind  of
St  yeah  so  two  cat  start  desc  is  equival
to  one  category  s  descend  which  a  bit
awkward
all
right
stuff
\end{unfinished}