% !TeX root = AnalyticStacks.tex

\section{\ufs Berkovich spaces (Clausen)}

\url{https://www.youtube.com/watch?v=fnEPiDIF9_k&list=PLx5f8IelFRgGmu6gmL-Kf_Rl_6Mm7juZO}
\renewcommand{\yt}[2]{\href{https://www.youtube.com/watch?v=fnEPiDIF9_k&list=PLx5f8IelFRgGmu6gmL-Kf_Rl_6Mm7juZO&t=#1}{#2}}
\vspace{1em}

\begin{unfinished}{0:00}
  e
so  we're  we're  nearing  the  end  of  the
course  actually  and  um  at  the  very
beginning  in  the  first  lecture  I  I  kind
of  gave  an  introduction  to  what  we
wanted  to  have  out  of  this  theory  of
analytic  stacks  and  in  particular  I  was
putting  some  emphasis  on  the  fact  that
we  want  to  know  that  traditional
Frameworks  for  analytic  geometry  can  fit
into  this  perspective  uh  of  analytic
stacks  and  we've  sort  of  already
discussed  um  attic
spaces  and  little  bit  of  geometry  over
this  um  the  little  bit  of  things  like
like  tape
curves  um  but  what  I  want  to  adct  spaces
there  was  this  issue  of  shiftin  and  then
the  idea  was  that  you  have  to  Define
some  kind  of  derived  adct  spaces  this
will  not  explain  in  full  what  is  Uber
Uber  set  up  and  in  your  formul  yeah  yeah
yeah  maybe  there  are  different  ways  to
I'm  not  sure  it's  possible  there  are
different  ways  to  do  it  yeah  so  this  is
yet  in  in  progress  already  yeah  you  yeah
let's  say
that  or  something
the  it's  in  Juan  estaban's  paper  on  uh
the  analytic  the
ramstack  that's  okay  so  there  you  go  um
all  right  um  okay  so  but  what  I  want  to
start  moving  towards  today  is  uh
relation  with  burkovich
spaces
um
um  but  before  discussing  that  um  I  want
to  just  do  a  little  bit  of  setup  um  so
but
first  I  want  to  talk  about  different
Notions  of  how  to  localize
uh  uh  an  analytic
stack  uh  sort  of  over  at  topological
space  um  so  well  we've  already  actually
seen  one  way  so  if  um  so  let's  say  well
let's  give  ourselves  an  analytic  stack
um  and  if  we  have  uh  s  which  is  maybe
let's  say  a  yeah
metable  finite  dimensional  compact  house
door  space  or  it  could  be  a  maybe  a
locally  compact  house  door  space  which
is
locally  one  of  these  um  then  we've  seen
that  that  can  also  be  viewed  as  a  as  an
analytic  stack  so  we  can  also  view  it  as
an  analytic  stack  um  and  then  we  could
just  ask  for  a  map  uh  F  from  X  to  s  in
the  category  of  analytic
Stacks  um  and  we  discussed  uh  kind  of
what  what  structure  you  get  on  X
basically  you  get  a  bunch  of  uh  item
potent  algebras  in
and  D  of  X  corresponding  to  the  closed
subsets  of  s  um  or  you  can  think  in
terms  of  the  complimentary  opens  as  well
and  you  get  sort  of  some  item  potent
coalgebras  which  would  be  like  the  J
lower  shrieks  of  the  constant  sheath  on
the  open  subset  as  opposed  to  the  like  I
lower  stars  of  the  constant  Chiefs
giving  you  these  item  potent  algebras
but  in  any  case  you  get  a  whole  bunch  of
algebra  objects  in  here  which  kind  of
let  you  localize  this  category  um  over
the  topological  space  s
um  but  um  we're  going  to  want  to  when
discussing  the  relation  with  burkovich
spaces  um  it's  going  to  be  too  annoying
to  require  uh  these  kinds  of  conditions
even  though  they're  satisfied  in
practice  so  I  just  want  to  discuss
something  uh  slightly  more  General  um
some  more  General  notion  of  how  to
localize  an  analytic  stack  along  a
topological  space  and  and  the  relation
with  this  one  in  the  case  where  you  do
have  a  metrizable  finite  dimensional
compact  house  DWF  space
um  so  so  what's  another  what's  another
thing  you  can  do  well  if  you  have  an
analytic  stack  so  we  have  X  um  well  to  X
we  can  assign  a  a
local  I  me  just  call  it  LO
X  which  is  just  uh  so  given  by  the
um  so
monomorphisms
uh  in  analytic
Stacks  so  this  is  kind  of  a  general
thing  you  can  do  whenever  you  have  a
toose  or  or  an  Infinity  topos  it  always
has  an  underlying  local  um  where  you
only  look  at  the
monomorphisms  and  if  you  look  at  the
topos  axioms  then  you  know  the  yeah  you
can  see  the
unions  here  the  POS  unions  and
intersections  exist  yes  but  uh  yeah  so  I
so  I  I  I  I  should  say  that  potentially
there's  some  set  theoretic  difficulty  so
that
this  yeah  potentially  this  is  not  a  set
I  didn't  I  didn't  investigate  seriously
but  it  doesn't  matter  you'll  see  that  it
doesn't  matter  in  a  second  when  I  get  to
the  definition  mean  when  you  allow
larger  and  larger  sizes  you  could  have
more  and  more  sub  objects  in  principle  I
didn't  yeah  I
didn't  I  didn't  I  didn't  spend  much  time
trying  to  prove  that  it's  a  set  because
um  about  the  local  axom  like  of  course
you  have  finite  you  said  that  there  is  a
way  to  you  have  to  produce  finite  binary
intersection  unions  but  also  you  have  to
produce  infin  I  mean  no  but  I  mean  this
is  this  category  has  all  limits  and  Co
limits  this  category  has  all  limits  and
co-  limits  so  I  mean  we  don't  have  to  we
don't  have  to  do  anything  special  here
to  make  this
definition  but  why  the  co  limit  is  a  sub
monomorph  no  Sor  the  co  limit  ah  okay
you  take  the  co  limit  and  then  you  quo
by  the
is  it  any  Z  tox  as  an  image
here  uh  yes  yes  that's  right  yeah  yeah
you  take  the  you  could  that's  a  general
Infinity  topus  thing  for  example  you
could  take  the  check  nerve  of  the  map
and  take  the  co  liit  and  and  this  is
what  Infinity  to  the  analytic  St  yeah
modulo  set  theoretic  technicality  so  it
satisfies  all  the  same  exactness
properties  as  an  Infinity  topos  but  it's
not  presentable  so  it's  okay  so  the
still  could  be
some  okay  up  to  okay  yeah  so  but  we're
going  to  so  so  maybe  it's  not  a  set  I
don't  I  don't  I  don't  care  and  you'll
see  when  I  make  the  definition  that  I'm
interested  in  that  it  doesn't  matter
whether  it's  a  set  or  not  um  sorry  yes
question  the  monom  if  I  only  consider  an
that's  like  taking  connector  component
yes  yes  yes  yes  yes
exactly  yeah  so  yeah  it  just  means  that
if  you  do  the  fiber  product  then  the
diagonal  I  mean  the  diagonal  map
Associated  to  this  inclusion  is  an
isomorphism  um  okay  right  okay  so  and
then  of  course  a  topological  space  also
gives  rise  to  a  local  but  let  me
actually  change  perspectives  and  just
say  s  is  a  local  instead  of  saying  s  is
a  topological  space  um  then  we  can  ask
for  a
map  of
locals  uh  Lo  X  to
S  and  um  whether  or  not  this  is  a  set  uh
this  is  some  data  that  makes  honest
mathematical  sense  because  you're  just
saying  that  for  every  so  a  local  is  is
by  definition  the  collection  of  open
subsets  which  is  a  set  and  you're  just
saying  for  every  open  subset  you  have  to
give  such  a
monomorphism  and  you  know  unions  have  to
go  to  unions  and  finite  intersections
have  to  go  to  finite  intersections  in
the  literature  about  loal  I  don't
remember  I  didn't  don't  use  much  but  so
this  is  like  the  direction  of  M  of  too
but  I  don't  know  when  they  Define  m  of
local  is  it  in  the  other  direction  or  in
this  yeah  yeah  I  don't  know  either  so
I'm  I'm  I'm  doing  the  the  like  geometric
morphism  let's  say  a  geometric  morphism
of  localis  yeah  morph  of  SES  well
morphism  sites  goes  the  other  direction
you  the  morphism  of  SES  goes  the  other
direction  then  I  mean  yes  so  I  mean  I'm
talking  about  a  geometric
morphism  but  also  of  goes  in  the
direction  oh  maybe  oh  I  forgot  yeah  yeah
yeah  okay  yeah  okay  go  in  the  same
direction  okay  not  I
don't  it's  not  used  much
by  okay  well
yeah  uh  anyway  um  okay  um  but  we  can
also  ask  for  a  stronger
property
uh  so  that
each  uh  inclusion  of  open
subsets  so  V  inside  s
um  uh  so  that  this  maps  to  so  it's
supposed  to  give  some  monomorphisms  so
maybe  I  should  give  some  name  for  this
like  f  or  Pi  maybe  so
so  this  goes  to  like  Pi  inverse  U  subset
Pi  inverse  V  subset  X  we  could  ask  that
uh  these  inclusion  maps  are  actually
open
immersions  from  the  perspective  of  the
six  funter
formalism  that  uh  that  Peter  discussed
last  time  the  extended  six  funter
formalism  on  analytic  Stacks  so  we  could
ask  that  each  of  these  inclusions  be
shable  um  and  that  they  be  well
chromologic
smooth  but  in  the  case  of  a  monomorphism
then  it's  quite  easy  to  see  that  the
dualizing  object  is  canonically  just  the
unit  and  it's  the  same  thing  as  an  open
immersion  um
yeah  uh
so  right  so  a  general  monomorphism  I
mean  it  could  look  like  could  look  like
anything  really  it  could  like  look
closed  it  could  look  open  it  could  look
like  some  mix  um  but  um  it's  it's
reasonable  for  to  ask  for  this  stronger
property  that  what  looks  like  an  open
immersion  on  the  level  of  the
topological  space  is  also  is  looks  like
an  open  immersion  from  the  perspective
of  the  six  funter
formalism  um  and  then  uh  in  the
case  where  X  is  a  metri
uh  finite
dimensional  compact  hous  door  space  can
ask  for  as  before  s  s  thank  you  thank
you  can  ask  for  a
map  uh  an  analytics
tax
um  so  um  yeah  sorry  just  a  clarification
So  when  you  say  s  is  a  local  like  what's
the  definition  sorry  oh  the  definition
of  a  local  yeah  so  like  I'm  just  like
what  does  it  mean  for  u  v  to  be  an  open
inside  right  so  right  I'm  using  a  bit  of
loose  terminology  here  but  um  so  about
the  definition  of  a  local  is  it's  a  a
certain  postet  okay  and  the  axioms  that
you  impose  on  this  postet  are  the  axioms
that  are  satisfied  by  the  postet  of  open
subsets  of  a  given  topological
space  but  by  abusive  notation  I'm  kind
of  using  this  symbol  U  subset  s  to  mean
that  U  is  an  element  of  the  postet  which
s  is  secretly  but  it's  because  I'm
thinking  of  it  as  a  topological
equivalent  loal  can  be  sort  of  the
category  of  sub  object  or  the  final
object  in  some  topos  yeah  yeah  and  then
this  defines  a  kind  of  localic
reflection  topos  is  the  about  yeah
there's  another  perspective  is  it's  like
it's  like  a  it's  like  you  take  the
definition  of  a  topos  but  instead  of
sheaves  of  sets  you  look  at  like  sheaves
of  truth  values  I  don't  know  if  this  is
helpful  or  not  but  it's  kind  of  yeah  um
so  it's  it  would  be  so  if  the  usual
topos  is  a  one  topos  then  maybe  a  local
is  a  zero  topos  I  don't  know  or
something
anyway  uh  um  okay  uh  oh  yeah  right  so  um
if  you  have  this  kind  of  structure  then
you  get  this  kind  of  structure  and  if
you  have  this  kind  of  structure  well
kind  of  topologically  you  get  this  kind
of
structure  note  also  that  if  you  fix  X
and  S  then  the  possible  such  this  guy
this  guy  or  this  guy  they're  all  just
sets  um  they're  not  Ana  or  whatever
because  here  well  we're  just  asking  for
a  map  of  locals  it's  just  a  map  of  post
sets  in  the  other  direction  no  but  when
you  say  set  yeah  it  also  means  in  this
sense  of  the  CTIC  size  difficulty  oh
that's  true  that's  true  so  up  to  size
difficult  I  mean  I  mean  there  are  no
automorphisms  of  any  fixed  one  of  these
let  me  say  I  mean  there  no  non-trivial
automorphisms  yeah  yeah  that's  a  that's
a  good  point  okay  so  yeah  so  so  this  is
tological  um  so  yeah  why  so  why  is  this
what  does  this  imply  this
well  basically  you  can  just  you  can  just
check  on  the  level  of  the  analytic  stack
Associated  to  such  a  guy  that  every  open
inclusion  um  actually  is  an  open
inclusion  from  the  perspective  of  the
six  funter  formalism  Peter  more  or  less
discussed  this  last  time  that  our  six
funter  formalism  on  analytic  Stacks  when
we  restricted  to  this  case  recovers  the
usual  six  funter  formalism  on  locally
compact  hous  dorf  spaces  and  then  this
property  of  being  an  open  immersion  is
is  stable  under  base  change  so  that's
um  uh  and  uh  you  get  the  yeah  so  can  you
say  so  you  know  that  open  immersion
holds  for
ah  it  is  stable  under  M  of  analytic
Stacks  or  well  uh  pullback  yeah  it's
stable  under  pullback
yeah
yeah  um  so  yeah  so  this  is  the  strongest
condition  so  if  you  have  this  you  can
ask  whether  this  hold  you  just  check
whether  certain  inclusions  or  open
immersions  if  you  have  this  you  can  ask
whether  this  holds  that's  as  we
discussed  uh  the  last  couple  of  times
that  corresponds  to  some  connectivity
condition  on  the  on  the  itm  potent
algebra  as  you  see  um  okay  um  Let  me
give  a  small
example  the  second  one  is  just  given  off
by  Lo  is  just  one  of  open  opener  yeah
that's  that's  true  yeah  so  yeah  so  right
so  you  could  think  of  this  in  terms  of
there's  Lo  local  X  and  then  there's  some
quotient  local  which  is  like  the  local
of  just  monomorphisms  that  are  opener
verions  with  respect  to  the  six  funter
formalism  and  then  um  the  second  bit  of
data  is  a  map  like  this  um  yeah  thanks
for  that  remark
Peter
uh  and  is  it  the  case  that  the  union
maybe  I  go  confus  Union  of  open
immersions  is  an  open  imersion  or  to
make  it  an  open  yeah  no  no  that  that
that's  true  that  it's  an  open  immersion
yeah  which  is  needed
to  well  it's  needed  to  yeah  it's  needed
to  have  this  loal  I  mean  yeah  so  or  well
remote  I  mean  depends  on  how  your  whe
right  right  right  no  but  it's  true  so
that  if  you  have  a  yeah  a  union  of  open
immersions  then  it's  still  an  open
immersion  and  that's  an  important  point
to  check  when  you're  discussing  these
things  um  and  it  follows  from
this  extension  procedure  for  six  funter
formalisms  that  Peter  discussed
so  if  you  have  a  union  along  open
immersions  then  those  are  shable  maps
and  but  it's  also  a  cover  and  so  you  can
you  can  get  a  a  lower  shriek  map  defined
on  the  union  and  then  you  can  actually
check  locally  that  it's  comically
smooth  um
okay  uh  so  let's  give  an  example
um  so  well  remember  way  back  when  when
we  had  Hub  Pairs  and  stuff  like  that  so
let's
see  um
then  uh  we  assign  to  this  an  analytic
ring  r  r  plus
solid  um  then  can  take  its
Spectrum  um  and  then  we  can  look  at  the
local  Associated  to
this  and  what  we  essentially  already  saw
was  that  this  this  so  This  analytic
stack  localizes  along  the  usual
um  usual  Huber  topological  space  of
continuous  valuations
no  you  you  ah  now  you  consider  the  spec
in  your  in  your  yeah  this  is  this  is
what  this  is  spec  n  so  Peter  yeah  I  I'm
lazy  so
I  yeah  I'm  lazy  so  I  just  say  spec  and
and  then  uh  spies  in  the  old  sense  yes
it's  a  topological  space  yes  and
then  okay  the  local  okay  and  then  you
have  a
map
of  oh
yeah  this  lock  this  means  for  every  open
in  the  add  space  you  give  a  a
monomorphism  of  analytic
stocks  and  uh  this
of  course  this  involves  passing  to  some
derived  pairs  in  some  sense  yeah  so
because  you  are
not  but  your  original  pair  is  is
not  right  so  if  on  the  level  of  rational
opens  this  is  just  going  to  give  another
apine  uh  analytic  stack  which  is  the  one
where  you  enforce  uh  that  F
invertible  just  in  the  world  of  a
analytic  Rings  you  enforce  that  f  is
invertible  and  that  G1  over  F  Etc  GN
over  f  is
solid
solid  um  and  that  defines
some  uh  that  defines  some  analytic  ring
uh  under  this  analytic  ring  here  and
passing  to  spec  it's  actually  a
monomorphism  so  on  the  level  of  uh  the
derived  categories  for  example  it's  a
localization  um  and  that  yeah  that  gives
some  inclusion  of  analytic  Stacks  here
and  um  what  we  argued
um  when  we  were  proving  that  so  we
proved  a  a  less  um  a  less  precise
version  of  this  claim  early  on  when  we
argued  that  the  derived  category  so  this
um  so  this  um  refines  the  statement
that
uh  that  the  drr  plus
solid  uh
localizes  on  Spa
rr+  so  recall  that  we  argue  that  you  you
have  a  sheath  of  infinity  categories  on
this  topological  space  whose  Global
sections  is  equal  to  this  and  whose
sections  on  a  rational  open  is  the
analogous  category  where  you  impose
these  conditions
um  and  we  had  several  discussions  about
when  this  is  the  same  as  when  this  is
also  of  this  form  for  some  non-derived
Huber  pair  and  so  on  so  if  you're  if
you're  a  tate  hu  pair  and  shifi  then
then  in  particular  this
um  this  again  is  just  of  the  same  form
as
before
um  right  and  then  so  it  refines  this
statement  and  the  proof  of  the  proof  uh
of  this  St  statement  that  we  gave  is
actually  shows  this  stronger  claim  so
what  we  did  was  we  argued  that  um  you
can  always  refine  any  open  cover  here  so
that  uh  so  in  fact  you  get  a  an  open
cover  so  in  fact  we
showed  without  having  the  L  without
having  introduced  the  language  for  it  we
showed  that  any  open
cover  or  cover  of  a  rational
open
uh  in  Spa
rr+  uh  pulls
back  to  an  open
cover  uh  after  can  be  refined  to
uh  uh  pulls
back  to  an  open
cover  of  uh  you  know
the
pulls  back
under
Pi  uh  to  an  open
cover  uh  in  the  sense  of  the  six  funter
formalism
no  so  let  me  make  a  a  remark  so  let  me
make
a  open
cover  to
cover  cover  by  monomorphisms  yeah  so  let
me  let  me  make  a  remark  um  so  in
general  uh
this  uh  yeah  it's  not
true  that  a  rational
open  uh  pulls
back  to  an  open
immersion  R  is  there  a  proof  of  this  of
this  oh  yeah  I'll  give  an  I'll  give  an
example  right  now  so  um  so  yeah  I  mean
you  can  do  something  like  I  don't  know
see  yeah
so  I  don't
know  have  two  formal  variables  maybe  p
and  X
um  oh  sorry  I  should
say  just  zp  Q
uh  yeah  that's  yeah  I  don't  need  two
variables  thanks
uh  so  just  one  formal  variable  and  then
invert
it  um  so  what  is  this  correspond  to  uh
so  on  the  level  of  analytic  Rings  we  get
this  and  on  the  level  of  derived
categories
uh  so  this  is  just  solid  zp  modules  and
then  uh  this  is  just  a  um
well  this  is  just  algebraically
inverting  P  so  so  the  left  adjoint  here
is  just  um  inclusion  of  full  subcategory
or  the  right  adjoint  here  is  the
inclusion  of  the  full
subcategory  where  uh  p  is  invertible
so  this  passing  from  zp  to  QP  is  just
algebraically  inverting
P  yeah  p  is  a  prime  yeah  yeah  yeah  um
and  that  actually  uh  from  the
perspective  of  the  six  funter  formalism
is  a  closed  immersion  not  an  open
immersion  that's  a  proper  map  because
the  uh  this  hasn't  the  second  variable
hasn't  changed  so  we're  just  changing
the  underlying  ring  and  not  the  analytic
ring  structure
um  but  it's  also  a  yeah  proper
monomorphism  it's  a  um  it's  a  closed
inclusion  it's
closed  not  immersion  not
open  H  so  so  I  think  we  treated  the  case
of  some  uh  covering  given  by  a  function
like  f  let  F  like  two  Lan  like  the
lauran
cover  yeah  the  one  where  F  or  one  minus
f  is  less  than  equal  to  mean  this  kind
of  basic  cover  and  this  for  those  it  was
they  were  is  it  the  case  for  those  show
that  if  it's  anal  what's  that  I  mean  if
it's  anal  yes  so  then  let  me  make
another  so  however  yeah  uh  if  if  R  is
Tate  um  then  then  indeed  then  this
doesn't
arise  so  in  general  the  problem  is
exactly  these  o  Open  covers  in  hubber
sense  which  are  just  given  by
algebraically  inverting  a  function
without  enforcing  any  any  inequalities
but  in  the  setting  of  a  a  tate  Huber
pair  then  you  know  if  you  want  to  invert
something  it's  you  can  always  write  it
the  the  the  subset  where  you  invert
something  can  always  be  written  as  a
union  of  subsets  obtained  by  forcing
inequalities  so  like  like  the  set
were  some  you  Union  of  like  well  I  I
don't  I  don't  know  yeah  but  it  is  not
quasi  compact  in  general  right  right
right  it's  not  going  to  be  quasi  compact
dealing  with  something  which  is  in  the
sense
of  of  the  adct  space  is  qu  compact
because  it  is  a  rational  domain  I  think
in  the  yes  but  I  mean  it  is  Rush  so  it
is  qu  compact  in  the  yes  in
the  so  this  also  has  to  do  with  the  fact
that  we  didn't  directly
I  was  SL  sliding  one  little  thing  under
the  rug  here  which  is  we  didn't  really
directly  Define  this  as  a  map  we  didn't
really  directly  Define  it  like  this
remember  we  actually  had  this  valuative
spectrum  of  all  of  these  guys  um  and  we
actually  defined  this  map  and  then  we
used  Huber's  retraction  here  to  so  this
was  this  was  not  quite  accurate  um
because  that  was  that  that  would  be
accurate  on  the  level  of  spav  but  that's
not  exactly  how  we  get  it  for  spa  a  so
instead  we
what  just  assume  that
theal  sub  of  SP  ah  yeah  okay  right  right
right  yeah  okay  yeah  theidea  is  open  and
then  it's  okay  yeah  so  then  the  point  is
that  then  um  right
yeah  well  yeah  the  point  is  that  the
rational  opens  in
the  the  rational  opens  in  the  Tate  case
can  always  be  described  by  forcing
inequality
uh  among  the
valuations  um  okay  uh  yeah  let  me  just
try  to  finish  what  I  was  trying  to  say
it's  like  like  I  don't  know  this  is  very
um  I  just  just  to  give  an  idea  of  what's
going  on  um  I  don't  want  to  get  too
precise  about  it  but  it's  this  kind  of
phenomenon  and  where  where  these  guys
correspond  to  um  this  kind  of  thing
which  does  correspond  to  an  open
immersion  so  this  is  open  in  general
this  is  closed  and  in  complete
generality  your  subsets  are  going  to  be
a  mix  of  open  and  closed  but  if  you  only
need  to  refer  to  these  things  and
they're  always  going  to  be
open
so  sheets  on  this  F  is  not  there  in  fact
you  not  algebraically  invert  F  but  but
using  the  right  hand  side  that's  right
that's  right
yep  um
okay  and  also  so
also  uh  there  is  another
topology  on  the  same
space  uh  with  the  same  constructible
subsets  so  where  um  where  uh  open  pulls
back  to
open
so  another  another  fix  even  outside  the
Tate  case  is  to  um  to  choose  a  slightly
different  topology  from  the  one  Huber
described  uh  where  or  like  kind  of  both
of  these  things  are
open  um  that's  that's  not  the
constructible  topology  or  no  it's  not
the  constructible  topology  no  yeah  so  is
it  a  topology  Which  is  less
it's  incomparable  to  the  Huber
topology  so  like  for  this  one  so  like
for  this  subset  here  like  P  not  equal  to
zero  it's  going  to  be  closed  from  one
perspective  and  open  from  the  other
perspective  so  it's  open  from  Huber's
perspective  but  will  be  closed  from  the
perspective  of
this  this  other
topology  is  it  the  spectral  space  it's  a
spectral  space
yeah
to  be  open
now  so  this  is  some  what  he  say  sorry
he's  saying  basically  you  redeclare
zariski  open  subsets  sorry  yeah  you
redeclare  zariski  open  subsets  to  be
closed  which  the  risk  open  they  not
they're  not
open  so  so  for  example  inverting  P
here  inverting
P
okay  is  now  closed  in  the  new
yes  we  haven't  discussed  so  far
that  element  now  thetics  actually  have
compl  that's  true  we  haven't  talked
about  that  yeah  but  let's  leave  that
aside  I  think  there's  already  enough
information  being
um  is  it  the  the  opposite  spectral  space
there  is  a  con  spectral  space  no  it's
not  the  opposite  spectral  space  because
these  ones  are  still  the  ones  where
you're  enforcing  inequalities  like  f
less  than  or  equal  to  one  that's  still
open  that's  open  in  both
cases  it  is  offet  of  of  I  mean  of  of
some  something  a  paper  about  he  def  a
lot  of  topology  and  one  of  them  op  it's
opposite  of  one  of  the  ones  Huber
describes  in  one  of  his  papers  yeah  okay
yeah
yes  can  make  this  more
precise  I'm  going  to  hesitantly  say  yes
it's  obvious  if  you  remember  the
definitions  of  everything  um  maybe  we
need  compact  or  not  because  because
because  I  mean  open  subset  I  mean  I  mean
you  want  you  want  an  item  put  in  Al
correspond  to  it  but  I  I'm  not  sure
whether  if  it's  not  quasy  compact  you
you  have  yeah  let  me  yeah  let  me  say
quasi  compact  uh  yeah
yeah  no  I  mean  I  I  mean  it's  something
we've  actually  discussed  before  in  the
in  the  course  I  drew  some  pictures
trying  to  explain  why  it  makes  sense
that  Z  risky  opens  should  be  closed  um  I
don't  know
yeah  so
um  oh  okay  um  everything  is  defend  all
rational  Subs  are  defend  by  inequality
is  f  i  less  equal  to  g  n  g  not  equal  to
zero  yes  and  this  one  you  replace  by  we
say  that  that  is  a  closed
condition  for  any  G  or  any  yeah  yeah
well  I  mean  you  I  mean  well  let's  say
you  have  a  rational  open  so  the
conditions  are  satisfied  like  the  ideal
generated  by  all  of  them  is
open  um  so  I  mean  so  you  you  can't  just
invert  G  if  that  doesn't  correspond  to  a
a  rational  open  in  Huber  sense  but  in
situations  where  you  can  it's  uh  it
corresponds  to  a  closed  conclusion  I
mean  we  we  actually  discussed  this
topology  in  the  notes  on  um  comp
in  so  I  think  it  was  Gaga  Redux  the
chapter  called  Gaga
Redux  um  we  discussed  this  this
modification  of  Huber's
topology  um  and  yeah  and  apparently  it's
also  in  one  of  Huber's  papers  just
passing  to  the
opposite  uh  okay
uh  so  one  ah  yes
so  one  last
remark
um  about  this  General
setup
uh  so
if  so  if  um  SI  is  some
uh  is  some  inverse
system  of  compact  hous  door
spaces  then  giving  a  map  from  local  X  so
then  giving  compatible
Maps  uh  from  Lal  of  x  to  SI  for  all  I  is
equivalent  to  giving  a  map  um  from  Lal  x
to  the  inverse
limit  so  the  only  claim  I'm  making  here
is  that  like  inverse  limits  in
topological  spaces  are  the  same  as  in
locals  in  this  specific  situation  which
is  something  that  you  can  uh  quite
easily  see  um  and  also  if  if  if  uh  and
it's  the  same  for  Lo
up  so  if  all  of  those  ones  Factor
through  L  up  then  this  one  will  also
Factor  through  l
up
um
look  the  one  we  still  the  one
yeah
um  and  then  um  so  then  there's  a  remark
that  every  compact  house  door
space  is  a
limit  of  metrizable  finite
dimensional
uh  uh  so  potentially  potentially  in  many
different
ways  but  uh  in  many  situations  there's
kind  of  a  natural  way  of  doing  it  we'll
see  this  in  the  setting  of  burkovich
spaces  and  then  this  gives  you  a  way  of
uh  going  from  the  case  of  metrizable
finite  dimensional  compact  house  store
spaces  where  you  can
sometimes  go  from  this  perspective  and
you  can  go  to  this  perspective  and  then
you're  closed  you  know  the  situation
kind  of  passes  to  inverse  limits  in  a
nice  way  and  um  but  now  you  no  longer
have  to  worry  about  things  being
metrizable  or  finite
dimensional  okay  that  was  my  little  um
just
uh  is  a  map  to  s  the  same  thing  as  a  map
from  loal  no  because  you  have  this
connectivity
condition  so  it's
the  you  still  have  to  impose  this
locally  that  locally  all  the  item  poent
algebras  are
connective  so
yeah  I  mean  as  far  I  it's  not  like  I
produced  a  counter  example  but  I  I  I
kind  of  believe  that  it's  not  the
same  uh  okay
okay
so  always  confused  is  there  really  a  t
structure  def  over  D  of  that  object  and
no  no  there's  no  T  structure  so  what  I
didn't  I  don't  have  to  say  what  I  mean
by  connective  I  have  to  say  what  I  mean
by  connective  locally  and  what  I  mean  by
connective  locally  is  that  there  exists
a  cover  by  apine  things  such  that  it's
your  your  object  when  you  pull  back  to
each  of  those  apine  things  is
connective  now  once  you're  in  the  apine
world  connectivity  is  a  pullback  of
something  connective  is  you  have  a  t
structure  and  the  pullback  of  something
connective  is  connective  so  it's  kind  of
and  also  if  the  pullback  is  connective
then  it's  connective  or  not  no  that  that
is  not  necessary  neily  true  okay  okay
okay  so  it's  not  like  so  it's  not
sufficient  condition  for  having
a  so  this  is  like  the  for  schemes  that
you  have  the  the  non  aine
orens  but  still  if  you  use  as  the  flat
or  pqc  still  if  it's  up  upstairs  it's  up
downstairs  so  that's  but  in  your
topology  of  course  you  have  much  more
stuff  yes  yes  so  you  can  have  these
wheel  we  things  where  yeah  you  have  a
map  which  is  apine  locally  but  with  an
with  a  map  with  apine  Target  which  is
apine  locally  but  not
globally  and  this  kind  of  yeah
um  yeah  that's  kind  of  unavoidable  when
you're  doing  analytic  geometry
so  uh  okay  um  where  are  we  okay  so  yeah
so  um  so  burkov
spaces  so  let  me  just  start  with  a
reminder  on  the  definition  so  we  know
where  we're  going  so  so  so  a  this  is  a
bon
ring  so  what  does  that  mean  it  means
that  a  is  a  commutative
ring  um  and  this  Norm  map  from  a  to  the
non-  negative  reals  um
uh
satisfies  so  first  of  all  uh  Norm  of  0
equals  0  second  of  all  uh  the  triangle
inequality  Norm  of  x  +  y  is  less  than  or
equal  to  the  norm  of  X  plus  the  norm  of
Y  and  third  of  all  um  it's  sub
multiplicative
so
um  and  uh  probably  have  to  put  the  norm
of  -  one  is  equal  to  the  norm  of  one
which  is  e  z  yeah  I  should  probably  put
that  yeah  so  uh
either  a  equal  Z  or  Norm  of  1  equal  1
and  Norm  ofus  one  also  do  you  need  that
it  does  not
follow  uh  I'm  not  sure  how  much  it
follows  but  it
it's  usually  they  want  the  triangle
inequality  Al  with  xus  Y  okay  I  think
you  can  you  can  get  it  you  can  give  a
crazy  thing  like  the  negative  multiply
by  two  you  can  do  something  that  still
satisfy  the  a  without  but  of  course  it
would  be  equivalent  something  saf  the  a
this  is  probably  not  how  to  show  oh
okay
uhhuh  yeah
um  yeah  and  then  so  let's  say  that  and
then  a  is  a
complete  uh  with  respect
to  the  nric
uh  um  so  an  example  yeah
there's  so  example  for  say  so  say  s  is  a
compact  house  door
space  uh  you  could  take  for  a  to  be  ous
functions  say  I  don't  know  complex  ah  uh
complex  valued  continuous  functions  and
take  the  norm  to  be  the  soup
Norm  where  you  have  the  usual  absolute
value
um  and  this
maybe  this  is  kind
of  um
uh  right  another  example  would  be  if  uh
if  R  is  a  a  tate  Huber
ring  um  and  if  you  choose  a  pseudo  UT
forer
um  topologically  nilp  poent
unit  uh  then  you  can
Define  uh  Define  a  norm  uh  Peter
actually  wrote  down  the  formula  before
but  uh  basically  it's  like  the  ptic
topology  um
so  yeah  you  can  find
Norm  uh  on
R  uh
with  say  Norm  of  pi  equal  12  and  you
want  and  it  satisfies  also  that  the  norm
of  Pi  *  f  is  always  exactly  the  same  as
the  norm  of  Pi  *  Norm  of
f
um  and  you  need  a  ring  of  definition  for
the
norm  you  have  some  ring  of  definition
where  the  you  declare  the  norm  to  be
less  than  or  equal  to  one
yeah  you  can  take  one  plus
P  1  plus  P  1  plus  P  what's
P
well  uh  I  don't  I  don't
understand  in  zp  you  could  take  P  for
for
yeah  and  also  with  this  there  was  this
again  any  or  no  not  zp  sorry  QP  yeah  QP
take  P  yeah  I  mean  zp  is  not  a  is  not  a
tate  hubber  ring  what  Q  QP  in  QP  bar  you
can  also  take  you  can  also  take  P
actually  so  first  of  you  must  have
complete  because  you  assume  it's
complete  oh  thank  you  yes  yes  I  mean
yeah  thank  you  and  also  if  it  is  zero
then  certainly  Norm  of  Pi  is  not  1  half
so  except  when  it  is  zero  the  norm  of  Pi
is  one  up  and  and  otherwise  you  have  to
to  to  uh  to
say  thank  you  thank
you  Yep  this  have  is  this  nor
multiplicative  it's  say  it's  it's
multiplicative  when  when  one  of  the  guys
is  pi  it's  not  not  multiplicative  in
general
um  okay
uh  so  uh
right
um  so  uh  right  and  then  so  to  this  data
so  so  this  to  a  and  the  norm  Huber
assigned  uh  the  space
um  which  has  a
set  burkovich  thank  you  yeah  we're
switching  over  now  thought  burkovich
um  which  as  a  set  is  given  by  the
um  ah  ah  yeah  okay  I  guess  completeness
includes  that  um  Norm  of  f  equals  z  if
and  only  if  f  equals  z  um  right
so
uh  so  I'm  just  using  this  notation  X  so
kind  of  it's  just  a  decoration  to  uh  so
and  this  is  now  a  multiplicative
seminorm  um
with  which  is  bounded  by  the  given  Norm
you  have  on  the  the  bonac  algebra  a
um  so  um  yeah
so
well  um  yeah  so  for  example
in  and  you  can
take
uh
and  um  if
you  require  that  the  in  this  example  if
you  requir  that  the  norm  restricts  to
the  usual  Norm  on  the  complex  numbers
then  in  fact  um  these  are  all  the
examples  so  that's  maybe  gon's  theorem
so  Bas  so  those  are  basically  the  only
examples  in  this  situation  and
multiplicative  includes  the  of  Zer
element  that  is  the  norm  of  one  is
one  yes  yes  yes
yeah  this  is  otherwise  we  will  get
trivial  thing  out
except  but  this  is
maybe
um  sorry  zero  mul  multiplication  of  zero
numbers  yeah  we  don't  want  that  so  we
want  these  to  be  like  points  so  we  want
them  to  be  non-empty  uh  so  we  want  one
to  be  different  from  zero  I  don't  know  I
mean  yeah  so
yeah  um  okay
but  do  you  allow  a  z  r  no  well  wait  I
allow  okay  I  allow  a  is  the  zero  ring
when  this  will  be  the  empty  set
yeah  um  okay  so  uh  yeah  so  there's  the
so  this  is  actually  a  compact  house  door
space  so  I  didn't  describe  the  topology
but  here  it
is  you  can  view  it  as  a  subset  of  the
product  over  all  F  and  a
uh  um  of  the  interval  from  zero  to  the
norm  of
F  and  it's  actually  a  closed
subset  and  this  map  just  takes  a  a  norm
and  Records  its  value  uh  on
F  uh  and  the  topology  is  just  the
Subspace  topology  so  it's  a  a  compact
house  DWF
space
Okay
so
so  what  we're  going  to  do  is  we're  going
to
um  move  this  definition  we're  going  to
try  try  to  uh  we're  going  to  try  to  make
the  same
definition  to  make  the  same
definition  uh  but  in  the  world  of
analytic  stacks
uh  instead  of  topological
spaces  so  we  want  to  take  the  same  idea
which  is  that  we  want  to  look  at  the  set
of  all  Norms  multiplicative  seminorms  on
a  um  bounded  by  the  given  Norm  on  a  but
we  want  to  say  that  in  the  language  of
analytic  stacks  and  that  we've  discussed
um  so  using  using  the  notion  of
norm
uh  on  an  analytic
ring  uh  we  discussed
earlier  yes  yes  yes  yes  yes  that's
true
um  okay  so  let  me  let  me  remind  you
about  this  notion  of  a  norm  so  so  the
definition  of  so  so  um  well  we  could  say
even  more  generally  in  an  IC
stack  so
uh  um  a  norm  on  X  is  a
map  from  the  algebraic  P1
/x  uh  to  the  closed  interval  from  zero
to  Infinity  including  Infinity
um  uh  which
is  U
multiplicative
uh  so  away  from
Infinity  so  let's  say  on  Norm  inverse  uh
0
Infinity  uh
and  let  me  just  write  it
uh  so  sort  of  suggestively  like  this  you
really  should  write  down  the  commutative
diagram  with  inversion  on  P1  inversion
of  the  coordinate  on  P1  and  inversion  of
the
this  extended  real  positive  non-
negative  real  axis  here
um  um  and  some
condition  on  how  P
sits  p  p  p  uh  yeah  so  this  is  the
P  oh  t  t  is  the  variable  on  P1  yeah  yeah
right
um  um  yes  what  do  you  mean  by  how  P  sits
yeah  so  so  remember  P  was  this
um  uh  this  this  this  ring  you  have  over
any  this  ring  object  you  have  over  any
analytic  ring  which  is  the  free  guy  on  a
topologically  nil  potent  element  so  and
it's  some  version  of  a  unit  dis  um  and
what  we  ask  is  that  it  but  this  Norm
function  gives  you  another  version  of
the  unit  dis  which  is  like  the  inverse
image  of  closed  interval  from  0  to  one
or  you  also  have  the  inverse  image  of
the  open  interval  from  well  half  open
interval  from  0  to  one  and  you  want  to
say  that  P  sits  in  between  the  two  of
those
yeah
uh  okay  uh  where  am  I  yeah  a  nor  okay
and  then  recall  so  so
write  uh  for  the  set  of
norms  on
X  um  so
recall  um  n  is  is  an  analytic
stack  that's  correct  yes  yeah  it's  a
set  because  it's  a  we  I  mean  we  it's
just  given  by  a  map  in  our  category  and
our  category  is  locally
small  a  map  in  the  category  of  analytic
Stacks
yeah  so  these  objects  are  fixed  right
when  you  fix  X  this  is  fixed  and  this  is
fixed  yes  that  you  know  that  there  is
a
uh  why  was  it  uh  well  we  basically  by
definition  every  analytic  stack  was  a
small  co-limit  of  representable  analytic
Stacks  okay  you  have
a
because  there  was  a  notion  of  analytic
ring  where  there  is  the  category  yeah
okay  there  there  you  you  you  handle  the
problem  of  large  sizes  I  mean  because
it's  enough  to  check  the  condition  some
smoke  but  then  you  have  the  analytic
stock  where  you  cover  but  you  don't  know
which  covering  you  need  to  give  aism  so
you  have  need  all  possible  covers  could
be  covers  by  bigger  and  bigger  see  but
you  say  the
category  is  accessible  yeah  so  the  the
the  main  technical  result  you  need  to
prove  is  that  the  the  sheif  ification  of
an  accessible  prief  is  still  an
accessible  prief  that's  that's  the  main
technical  result  you  need  to  prove  we
did  not  discuss  this  at  all  but  that's
what's  what's  underlying  the  the
resolution  to  these
issues  it's  like  in  Waterhouse
uh
yeah
um  okay  so  so  this
um  um  in
fact  so  there's  a
cover  so  this  uh  this  gaseous
base
uh  maps  to
n  um  in  other  words  you  can  write  down  a
norm  on  this  gous  space  Tack  and  it's  uh
and  this  uh  so  this  uh  and  this  is
actually  the  universal
Norm  on  an  analytic
ring  uh  uh  with  a  variable  an  analytic
ring
now  with  a  variable  Q
in  R  such  that  uh  Norm  of  Q  is  precisely
equal  to  1/  12
um  what  did  you  write  NX  is  the  analytic
Stack  n  n  is  an  analytic  stack  yeah  so
yeah  so  the  so  NX  the  set  of  norms  is
actually  just  the  set  of  maps  from  X  to
some  Stack  n
um  right
so  yeah  so  if  you  ask  for  a  norm  on  an
analytic  ring  and  an  element  whose  Norm
is  exactly  equal  to  1/2  which  is  kind  of
somewhat  stringent  condition  because  a
prioria  again  the  norm  map  is  Norm  of  Q
is  a  map  from  Spec  R  uh  to  0  infinity
and  you're  asking  that  it  Factor  through
this  a  prior  its  image  could  be  some
interval  or  something  but  you're  asking
that  its  image  be  exactly  this
Singleton
um  uh  the  universal  example  of  that  is
this  guy  and  moreover  the  map  to  the
base  stack  is  actually  a  cover  in  the
sense  of  our  Gro  de  topology  on  analytic
Stacks  because  every  Norm  on  an  analytic
ring  locally  uh  you  can  find  such  an
element  we  had  this  argument  we  we
discussed  this  two  lectures
ago  um  right  everything  here  was  like  X
was  a  spec  of  an  analytic  rate  um  yeah  I
mean  I  I  made  this  definition  for  a
general  X  but  it  I  mean  it  doesn't
matter  because  the  the  condition  I  mean
the  Norms  on  uh  an  analytic  ring  they
satisfy  descent  basically  I  mean  it's
kind  of  follows  from  General  nonsense
and  so  it  automatically  glues  to  say
what  a  norm  is  on  an  arbitrary  analytic
stack  and  it  just  unwinds  to  the  same
thing  um  so  the  cover  also  exist  because
it  exists  locally  like  you  GRE  it  well
the  map  the  map  no  the  map  exists
because  you  can  write  down  this
Norm  um  and  then  it's  a  cover  because
given  any  Norm  on  an  analytic  ring  uh
after  a  cover  you  can  find  a  q  that
satisfies  this
property  sure  do  you  need  to  do  some
base  tank  I'm  sorry  do  you  need  to  do  a
base  change  falls  back  to  Q  gu  a  base
change  what  do  you
mean
to
is  it  okay  or  you  still  have  a  question
okay
um  so
um  uh  I  want  to  before  before  finishing
the  discussion  of  or  introducing  the
definition  of  this  uh  enhanced  burkovich
Spectrum  as  an  analytic  stack  um  I  want
to
um  I  want  to  explore  a  little  bit  about
uh  so  so  what  does  n  look
like
um  so
uh
so  well  what  do  we  have  on  N  if  you  have
a  norm  on  an  arbitrary  analytic  ring  um
so  if  you  have  a  a
norm  p1r  goes  to  Zer
Infinity  um  so  as  mentioned  for  any
section  here  uh  you  get  a  function  from
Spec  R  to  to  Z  Infinity  but  over  an
arbitrary
analytic  ring  the  only  thing  we  know
exist  are  the
integers  um  so  given  n  in
z  uh  we  get  a
map  which  records  the  value  of  your  Norm
on  uh  on  the  integer  a  little
n
um  so
what  does  this  give  this  gives  a  map
from  n  to  right  so  now  I  should  maybe
pass  to  the  local  perspective  uh  to
product  over  all  integers  of  this
extended  half  real
axis  so  this  um  this  stack  of  norms  it
localizes  on  this  this  product
here  um  and
uh  we  can  ask
for  so  we  want  to  know  what  the  what  the
image  is  so  to  speak
so  and  let  me  say  what  I'm  what  I  mean
by  image  I  mean  the
complement  of  the  uh  largest  open
subset  uh
so  let's  see  with  uh  Pi  inverse  of  U
equals  mty
set
by
definition
yeah  um  and  but  but  we'll  also  see  that
we'll  also  see  that  the  fiber  over  any
point  in  the  image  is  non-empty  so  so
that  at  least  in  this  case  maybe  it's  a
general  fact  I  don't  know  at  least  in
this  case  it's  kind  of  a  theorem  that
the  image  is  closed  so  to  speak  um
okay  uh  right  so  now  note  that  uh  this
so
so  uh  the  more  traditional  thing  is  this
burkovich  spectrum  of  the  integers  so
that  was  also  by  definition  a  subset  of
uh  this
product  uh  going  from  zero  to  infinity
and  it  was  given  by  those  uh
Norms
yeah  um  so
multiplicative  uh  avoiding  Infinity  um
satisfying  triangle
inequality
uh  let  me  remind  you  what  uh  what  this
thing  looks  like  in  in  case  people
haven't  seen  this  before  so
recall  uh  MZ  looks  as
follows
um  you  have  a  point  at  the  center  so  to
speak  um  which  corresponds  to  the
trivial  Norm
um  meaning  uh  zero  Norm  of  Z  is  z  norm
and  Norm  of  everything  else  is  equal  to
one  and  then  you  have  several  branches
so  you  have  um  an  aredian  Branch  so  at
the  end  of  the  archimedian  branch  you
have  the  usual  Norm  uh  the  usual
archimedian
Norm  usual  absolute
value  um  but  then  for  each  prime  P  you
have  uh  and  maybe  uh  yeah  so  for  H  Prime
P  you  have  another  branch  which  also  uh
ends  at  some  point  and  to  get  the
correct  topology  on  embedding  it  into  R2
you  should  probably  make  the  branches
get  shorter  and  shorter  and  shorter  but
okay  that's
um  yeah  but  it's  a  the  situation  with
these  ptic  branches  is  a  little  bit
different  so  what's  going  on  here  here
you  have  the  usual  absolute  value  and
here  at  the  halfway  point  you  have  the
uh  square  root  of  the  usual  absolute
value  and  here  you  know  then  you  know
you  can  put  any  Alpha  between  zero  and
one  and  you  can  kind  of  see  from  the
intuitive  perspective  that  this
interpolates  between  the  usual  absolute
value  and  the  um  the  trivial  absolute
value  here  what  goes  at  the  top  is  not
the  usual  ptic  absolute  value  the  usual
ptic  absolute  value  sits  somewhere  here
so  normalized  to  say  so  that  P  equals  1
over  p  and  now  you  can  actually  scale  it
to  any
uh  any  uh  positive
real
um  um  and  then  there's  also  a  limit  as
the  scaling  goes  to  infinity  and  what
that  gives  is  the  trivial  Norm  or  the
pullback  of  the  trivial  Norm  on  the
residue  field
FP  so  in  other  words  the  norm  of  any
multiple  of  p  is  equal  to  zero  and  the
norm  of  everything  else  is  is
one
um
okay
what  of
course  no  I
yeah  this  is  great  way  I  think  this  was
something  like  this  is  the  first  talk
there  was  a  talk  in  this
course  it  was  in  this  course  or  another
probably  in  this  course  there  was  some
discussion  of  but  maybe  I  could  do  this
another  thing  yeah  so  I'm  I'm  I'm  yeah
I'm  I'm  trying  to  recall  something  which
is  well  known  indeed  um  in  order  to  set
up  the  discussion  of  uh  of  What's
following  here  so  but  but  in  particular
I  want  to  emphasize  this  is  a  really  big
space  but  the  Subspace  uh  MZ  is  quite
small  you  know  onedimensional
um
yeah
um  so  now  what  we're  going  to
see  so  there  there  you  allow  allow  value
to  be  in  but  here  you  don't  that's
correct  so  um  so  here's  going  to  be  the
the
claim  so  the  image  of  Norm  is  a  is  a
larger
subset
uh  so  it  looks  like  this  um  so  you  have
all  the  same  ones  as
before  uh  sorry  I'm  I'm  trying  to  say
that  it  stops  there  yeah  trying  to  draw
like  a  closed  interval
sign  but  then  also  at  the  archimedian
place  um  it  gets  extended  so  here  now
the  usual  archimedian  absolute  value  is
also  in  the  middle  of  of  the
interval  and  you  can  take  arbitrary
powers  of  it  so  for  any  Alpha  in  R
greater  than  zero  you  can  take  powers  of
it
um  so  it'll  go  to  there  in  one  direction
and  to  the  other  direction  you  get  some
really  strange  Point  uh  so  strange
point  where  uh  which  corresponds  to
uh  so  it's  a  subset  of  there  so  it's  a
it's  given  by  some  maps  from  Z  to  the
extended  real  line  there  um  and  it's
given  by  Norm  of  n  equals  infinity  if  uh
if  n  is  not  equal  to  minus  one  0  or
1  z
z  yeah
yeah
so
um  curly  end  here  like  Curly  end  like  in
this  so  it's  like  solid  Z  no  it's  a
different  kind  of
Base  because  I  mean  solid  Z  if  in  over
solid  Z  for  example  the  real  numbers  are
equal  to
zero  so  you
can't
um  I  ask  question  uh  you  can  ask  a
question
yeah  is  the  picture
accurate  in  what  sense  I  mean  I  don't
know  the  price  should  be  closed  or
not  you  mean  this  picture  or  this
picture
this  one
here  I  don't  understand
uh
uh  I  mean  I  I  want  to  say  that  this  is
the  same  as  this  on  all  the  non-
archimedian  branches  and  on  the
archimedian  branch  it  just  gets  extended
past  yeah  so  I  sorry  if  that  wasn't
clear  yeah  it  gets  extended  all  the  way
to  infinity  and  then  compactified  at  the
end  yeah  I  guess  I  guess  what  I  you  can
try  to  write  a  formula  for  it  it's  like
image  Norm  is  like  you  take  the
burkovich  space  of  the  integers  ah  no
let  me  make  a  before  I  say  this  so  in
particular
so  uh  the  triangle  inequality  can
fail  well  that's  quite  clear  here  you
have  Norm  of  one  equal  one  but  Norm  of
two  equals  infinity  okay  that's  pretty
drastic  failure  of  the  triangle
inequality  but  also  like  for  the  square
of  the  usual  absolute  value  which  is  a
new  thing  you  have  then  the  triangle
inequality  fails  as
well  um  so  yeah  I  guess  another  way  of
saying  this  is  like  image  of  Norm  you
can  get  it  from  the  burage  spectrum  of  z
um  on  the  burkovich  spectrum  of  Z  you
have  an  action  of
um  uh  the  real  number  is  bigger  than  or
equal  to  one  wait  did  I  yeah  or
well  yeah
yeah
um  or  maybe  yeah  I  don't  know  but  but  on
the  on  this  other  thing  you  have  an
action  of  the  positive  real  numbers  you
can  kind  of  do  this  and  that  doesn't  do
anything  on  the  non-  archimedian
branches  and  then  on  the  archimedian
branch  it  extends  it  all  the  way  and
then  you  compactify  it  one  point
compactification  so  I  don't  know
um  okay
uh  right  so  let's  explore  this  and  let's
see  what  kind
of  what  what  is  kind  of  going  on  so  uh
let's  so  to  to  explore
this  uh
consider  just  the
map  what's  that  is  the
one  one  point  yeah  the  one  point  yeah
yeah  one  point  Yeah
Yeah
question  picture  you  seem  to  see  this
image  gen  topological  space  but  this  is
supposed  to  be  something  as  a
sub  associated  with  Z  wrong  uh  that's  a
yeah  so  that's  a  good  question  so  by
definition  I  made  it  a  uh  I  was  saying
it's  a  closed  subset  of  uh  of  the
topological
space  um
now  you  can  view  this  as  an  analytic
stack  we  we  saw  how  to  view  this  is  an
analytic  stack  and  you  can  take  the
product  in  the  category  of  analytic
Stacks  that's  perfectly  legitimate  it's
no  longer  metrizable  so  you
know  why  is  it  no  longer  I  mean  no
longer  finite  dimensional  sorry  it  is
metrizable  it's  no  longer  fin  I  said  the
wrong  word  it's  it's  metrizable  it's  no
longer  finite  dimensional  um  so  we
didn't  quite  talk  about  this  thing  but
but  you  can  I  mean  you  can  still  view
this  as  an  analytic  stack  and  you  still
do  get  a  map  of  analytic  Stacks  from  end
to  this  product  and  this  closed  subset
does  correspond  to
a  uh  a  subanalytic  a  monomorphism  of
analytic  Stacks  here  and  the  map  from  n
there  does  Factor  through  that  closed
subset  and  so  you  can  view  it  you  can
view  it  in  several  different  ways  um  as
usual  okay  um  to  explore  this  so  cons
fix  a  prime
p  uh  and  consider  just
uh  uh  as  a  start
um  Norm  of
P
um  so  so  here's  the  first
claim  so  the  claim  is  that  we  can
understand  uh
so
um  so  if  we  look  at  the  um  the  locus  in
this  Universal  space  of  norms  where  the
norm  of  P  lives  between  zero  and  one  we
can  understand  this
um  so  this  is  just  the  same  thing  as  so
maybe  maybe  uh  just  as  notation  I  can
call  this  n  and  then  Z  less  than
absolute  value  of  P  less  than  one  so
it's  the  stack  parametrizing  Norms  on
which  the  variable  P  lives  between  0  and
one  um  this  is  equal  to  uh
spec  QP  the  gashes  version  of  the  pic
numbers  across
this  uh  stack  Associated  to  the  open
interval  from  0  to
one
um
so
yeah  well  in  fact  this  is  quite  easy  to
see  because  um  we  know  that  the
universal  analytic  stack  equipped  with
some  variable  whose  Norm  is  between  zero
and  one  is  uh  so  is  spec  of  zq  hat  plus
or  minus  one  gases  uh  cross  cross
01  um  and  then  you  just  have  to  impose
that  that  variable  becomes  P  so  you  just
set  qals  P  or  you  mod  out  by  Q  minus  P
um  and  then  that  as  Peter  discussed  when
discussing  this  gous  base  stack  gives
you  some  analytic  ring  structure  on  the
pic  numbers  and  then  the  second  variable
doesn't  really
change  uh
so
um
and  so  yeah  and  so  and  then  if  you  look
so  so  I'm  claiming  in  particular  that  if
you  look  at  the  the  the
universal  uh  so  let's  say  we  take  the  so
let's  say  we  take  the  fiber  over  a  point
Lambda  and
01
uh
um  then  the  the  in  particular  you  get  a
normed  analytic  ring  structure
here  um  and  it
is  so  what  does  that  mean  it  means  that
for  every  radius  R  you  have  some  notion
of  overon  convergent  functions  on  a  disc
of  radius  r
uh  and
then  notion
of  on  a  disc  of  radius
R  uh  is  the  usual
one  uh  from  non-  archimedian
geometry  uh
where  uh  you  take  the  normalization  of
the  absolute  value  on  the  peic  numbers
for  which  the  absolute  value  of  p  is
equal  to  this
Lambda
so  um  what  we're  seeing  here  is  the  the
kind  of  the  the  uh  interior  of  the  P
Branch  here  is  kind  of  uh  fairly
straightforward  to
understand
um
so  next  uh  let's  look  at  the  locus  where
another  Locus  that's  fairly  easy  to
understand  um  is  the  locus
where  p  is  between  one  and
infinity  strictly  between  one  and
infinity  because  that's  the  same  thing
as  saying  uh  that  uh  well  first  of  all  P
has  to  be
invertible  um  because  p  is  away  from
zero  the  absolute  value  of  p  is  away
from  zero  um  and  then  it's  the  same
thing  as  saying  that  the  absolute  value
of  1  over  p  is  between  0  and  1
um  and  then  we  again  we  can  use  the
exact  same  argument  to  understand  what
this  is  and  what  you  get  is  you  get  spec
of  uh  the  gaseous  real  numbers  cross
01  and  the  argument  is  the  same  and  that
as  Peter  explained  if  you  take  this  uh
this  gaseous  this  this  ring  here  and  you
mod  out  if  you  set  Q  equal  Al  to  1/  P
then  you  actually  get  the  real  numbers
you  get  a  certain  analytic  ring
structure  on  the  real  numbers  which  is
uh  yeah  this  one  here  um  and  again  if
you  look  at  the  universal  Norm  with  a
fixed  value  of
Lambda
um  uh  fixed  value  of  P  yeah
sorry  sorry  what  for  example  two  P  could
be  two  yeah  yeah
um  the  universal  Norm
here  is  is  the
usual  uh  given  by  usual  overon
convergent
functions  uh  in  archimedian  Geometry
yeah  say  complex
geometry  but  with  respect  to  the
norm  uh  the  which  is  a  power  of  the
usual  um  a  power  of  the  usual  absolute
value  where  Alpha  is  such  that  uh  if  you
take  the  norm  of  1  over  p  uh  uh  you
exactly  get
Lambda
so  so  I'm  not  quite
sure  usual  over  converion  where  does  it
show  in
the
uh  where  where  the  those  uh  over
convergent  so  you  had  some  some  formal
sering  with  Q  yeah  but  and  there  you  can
speak
about  Loy  I  mean  the  the  part  where  the
over  converion  part  I  mean  at  least
there  are  some  algeb  corresponding  to
but  then  you  specify  that  it
is  okay  you  can  walk  over  no  I'm  not  I'm
not  because  the  formal  variable  was
already  replaced  by  P  so  it's  yeah  B
lost  in  this  yeah
usual  yeah  you  have  to  do  a  calculation
so  for  example  yeah  you  could  take  so
the  first  thing  to  understand  of  course
is  why  when  you  take  this  ring  and  you
specialize  to  qal  1  over  P  why  you  get
the  real  numbers  um  so  okay  that  has  to
do  with  some  kind  of  Base  P
expansions  um  and  then  you  have  to
understand  and  say  if  you  take  this
module  P  this  basic  module  P  you  want
you  want  to  know  what  that  base  changes
to  it  should  be  some  module  over  the
real
numbers  so  it  should  be  some  sequence
space  with  some  summability  condition
and  you  can  see  that  the  summability
condition
is  plus  plus  or  minus  Epsilon  maybe  some
little  fuzz  it's  basically  just
um  well  I  mean  Peter  actually  described
what  it  was  it  was  some  kind  of
exponential  decay  thing  thing  but  the
point  is  that  it  that  thing  does  sit
between  the  usual  ring  of  overc
convergent  functions  on  the  unit  dis  uh
and  the  Ring  of  holomorphic  functions  on
the  interior  of  the  unit
dis
um  so  when  you  do  this  over  convergent
business  uh  remember  when  we  did  the
yeah  when  you  do  this  over  convergent
business  it  it  it  it  that  that  subtlety
of  exactly  what  ring  that  is  exactly
what  some  ability  ability  property  you
have  it  doesn't  matter  anymore  and  it
just  Returns  the  usual  ring  of  over
convergent  holomorphic
functions  uh  on  the  dis  and  there's  some
scaling  of  the  usual  absolute  value
which  comes  in  because  when  we  were
building  the  universal  Norm  uh  we  had  to
take  a  fixed  value  of  the  norm  of  Q  and
and  kind
of
yeah  use  that  to  write  down  the  answer
but  it's  such  that  when  you  specialize
to  normal  situations  you  get  the  usual
thing
okay
um  so  in
particular  uh  this  this  Locus
um  this  is  independent  of
P  um
because  in  any  I  just  I  just  told  you
what  the  what  the  what  the  universal  guy
was  and  I  didn't  have  to  make  any  well  I
mean  there's
some  rescaling  property  of  the  of  the
norm  but  um  that  that  kind  of
uh  yeah  it  it  doesn't  affect  this
Subspace  of  the  of  the  of  the  universal
space  of
norms  so  this  is  the  kind  of  thing  that
you  need  to  see  in  order  to  to  see  that
you're  getting  the  Burk  ofage  space  you
have  some  infinite  dimensional  space  but
actually  the  conditions  on  the  various
prime  numbers  are  not  at  all  they're
very  tightly  related  to  each  other  this
is  like  os's  classification  so  if  you
have  a  if  you  have  a  norm  on  the
integers  for  which  the  norm  of  p  is
equal  to  1/2  then  it  you  know  it  kind  of
it  has  to  be  the  pic  norm  and  in
particular  its  values  on  all  the  other
integers  have  to  be  are  determined  by
that  um  and  you  can  see  that  also  in  in
our  situation  as  well  um  that  when  you
force  yourself  to  live  in  this  Locus
which  is  only  a  condition  on  P  that
automatically  tells  you  what  the  norm  of
everything  else  is  because  you  can  just
do  calculations  with  these  usual  uh
rings  of  over  convergent  holomorphic
functions  in  usual  non-  archimedian
geometry  so  on  here  uh  so  the  norm  of
all
other  uh  and  is
determined  each  is  set  of  norms  on  z  uh
yes  uh  not  up  to  equivalence
no  yeah
question  all  the
point  right  yeah
yeah  why  that  would  be  is  there  some
explanation  why  that  would  be  for  me
it's  a  bit  of  a  it's  just  a  calculation
I  mean  um  in  our  in  our  yeah  so  that's  a
good  point  so  in  our  axium  for  normed
analytic  ring  we  had  no  version  of  the
triangle  inequality  whatsoever  we  just
had  this  that  the  norm  should  be
multiplicative  but  then  you  can  look  and
see  well  if  you  believe  what  I'm
claiming  then  kind  of  almost  all  of  it
does  almost  all  of  the  space  does
satisfy  the  triangle  inequality  because
you  can  just  check  on  in  terms  of  the
the  rings  of  functions  that  are  being
assigned  and  you  can  just  verify  that
the  triangle  inequality  holds  so
whenever  you're  in  the  ordinary
burkovich  space  of  Z  you  actually
satisfy  the  the  the  triangle  inequal
your  Norm  actually  satisfies  the
triangle  inequality  and  then  okay  it's  a
you  get  some  sort  of  quasy  Norm  if  you
move  out  in  this  direction  and  this  this
part  is  a  little  funny  but  if  you  throw
that  away  so  there's  always  some  version
of  the  triangle  inequality  that  is
automatically  satisfied  just  as  a
consequence  of  multiplicativity  and
that's  that's  kind  of  funny  yeah  in  the
usual  the  like  in  fact  was  consider
earlier  then  I  mean  it's  I'm  speaking
now  about  Theos  classification  the  or
maybe  so  you  can  put  the  following
condition  I  don't  remember  which  refence
instead  of  inequality  you  can  put  like  X
plusus  Y  Val  less  equal  to  some  constant
time  x  y  and  then  one  can  prove  that
after  a  normalization  by  some  power  you
have  the  triangle  inequality  or  even  the
yeah  and  so  is  it  the  case  that  here  you
can  is  related  somehow  where  yeah  in  the
locus  where  so  if  you  know  that  the
priority  that  the  nor  some  integer  like
or  three  not  Infinity  then  you  can  get
the  triangle  in  quality  by  scaling  it
yes  exactly  exactly  yeah  so  if  um  well  I
have  on  everything  now  I'm  speaking
about  Bigg
like  in  the  big  let  me  make  some  further
claims  so  claim
um  uh  so  yeah  if  um  so  if  you  have  an
normed  analytic
ring  uh  with  the  norm  of  two  say  doesn't
matter  less  than  or  equal  to  one  um
automatically
satisfies  uh  the  non-  archimedian
triangle
inequality
um  a  normed  analytic
ring  uh
with  Norm  of  two  uh  less  than  or  equal
to  two  satisfies  the  usual  triangle
inequality  uh  oh  I  sorry  let  me
move  this  is  the  that  no  of  two  two  is
equivalent  to  no  of  three  three  yes  yes
yes  two  is  an  arbitrary  prime  number
here  two  is  an  arbitrary  prime  number
here  about  to  six  is  it  that's  all
that's  also  equivalent  yeah  yeah  so  I
guess  Prime  is  not  so  important
yeah  two  yeah  two  is  an  arbitrary
integer  bigger  than  one  yeah
um  okay
so
uh  and  then
um  where  am  I  now  oh  yeah  so  a  normed
analytic
ring  uh  with  Norm  of  two  less  than
infinity
always  uh  there  there  always  exists  a
constant  C  greater  than  zero  such  that
the  norm  of  x  +  y  is  less  than  or  equal
to  C  *  the  norm  of
X  the  norm  of
Y  but  this  is  interpreted  not  in  the
sense  of  normal  functions  but  it's  not
yeah  it's  some  Universal  thing  like  uh
so  you  know  you  write  down  you
have  whatever  you
have  yeah  you  have  P1  like  yeah  the
locus  where  the  norm  of  the  T  variable
is  less  than  or  equal  to  a  cross  P1  you
know  Locus  for  the  S  variable  is  less
than  or  equal  to  B  um  yeah  that  this
this  maps
to  so  it's  a  and  b  are  less  than
infinity  here  so
it's  this  maps  to  P1  R  and  then  this
maps  to  Z  Infinity  but  this  should
Factor  through  zero  and  then  C  *  a  plus
b
so  yeah
so
and  this  is
addition
A1  yeah  they  it's  well  defined  because
this  happens  to  live  inside  A1  which  we
already  already  argued  earlier
yeah  yeah  so  there's  always  some  um  some
version  of  the  triangle
inequality  um
yeah
uh  so  how  oh  yeah  so  how  do  you  prove
these  claims  by  the  way  these  claims
imply  the  claim  about  what  the  uh  image
of  uh  of  n  is  because
um  so  yeah  so  and
then
this  uh  this  Extended  burkavage
space  um  the  one  I  mean  the  specific  one
that  I  that  I  wrote  down  uh  the  reason
is  that  you  can  um  you  can  look  at  you
know  the  norm  of
two  uh  which  goes  to  zero  Infinity  ah
yeah  if  you  want  to  if  you  want  to  know
the  the  image  of  something  something  um
you  can  actually  work  uh  you  can
actually  work  on  stratifications  of  your
topologic  you  don't  have  to  work  in
closed  covers  or  open  covers  you  don't
have  to  work  locally  in  the  sense  of
closed  covers  or  open  covers  it's  enough
to  work  locally  in  the  sense  of
stratifications  I  mean  you  can  stratify
the  base  and  look  over  the  different
strata  because  you  you  have  to  see  when
the  pullback  to  an  open  subset  is  empty
and  while  like  a  closed  and  an  open
Don't  form  a  cover  they  still  uh  they
still  detect  ISO  morphisms  in  particular
they  still  detect  whether  an  analytic
stack  is  empty  or  not  so  so  we  can  can
look
over  uh  the  locus  where  two  is  less  than
or  equal  to  one  the  locus  where  one  is
less  than  two  is  less  than  infinity  and
then  the  locus  where  2  is  equal  to
Infinity  um  and  then  we  just  have  to
take  the  union  of  the  the  images  we  see
over
there  and  here  uh  if  you  believe  this
claim  then
uh  where  am  I  uh  yeah  if  you  believe
this  claim  then  there  you  have  the  non-
archimedian  triangle  inequality  so  here
it's  automatic  uh  automatically  a  subset
of  just  by  some  Universal  argument  the
the  non-  archimedian  burkovich  spectrum
of
z  um  so  we're  contained  in  the  in  the
claimed  Locus
there  um  this  thing  we  already
classified  uh  this  is  contained  in  the
archimedian  locus  the  uh  archimedian
Locus  um  and  so  the  last  thing  we  need
to  do  is  to  see  that  uh  this  Locus
consists  of  just  one
point  um  so
need  um  for  all  n  different  from  0  1  and
minus
one  um  and  that  actually  doesn't  follow
from  the  claims  well
yeah  well  I
I'll  what's
that  it  does  follow  from  was
arbitrated  two  was  arated  yeah  that's
right  yeah  yeah  the  claim  that  the  these
conditions  are  independent  of  two  um
does  give  um  does  give  this  as  well  so
um  last  thing  you  say  plus  be  yeah
that's  true  I  mean  you  could  also  yeah
you  can  directly  argue  like  if  three  for
example  wasn't  sent  to  Infinity  then
you'd  be  in  one  of  these  locuses  and
then  two  wouldn't  be  equal  to  Infinity
so  I  mean  it's  kind  of
yeah  um  okay  so  how  do  you  I'm  going  to
finish  up  so  how  do  you  prove  these
kinds  of  claims  well  you  you  just
calculate  so  um  so  to  prove
them
wait  but  for  the  first  part  we  only  know
it's  not  lgers  than  once  and  it's  not
impli  directly  from  yeah  then  you  also
need  to  see  that  uh  if  you  take  any
point  say  here  and  then  take  that  there
exists  some  normed  analytic  ring
which  has  those  values  um  and  we  already
saw  it  for  the  points  in  the  interior  of
the  the  Rays  in  the  Burk  ofage  space  and
it's  also  quite  easy  to
like  hit  the  center  because  you  just  do
some  non-  archimedian  geometry  over  some
lant  Series  ring  with  Q  with  the  trivial
Norm  or  you  can  hit  the  points  at  the
end  for  example  by  doing  some  non-
archimedian  geometry  with  with  FP  so  you
can  you  can  see  that  so  this  gives  a
containment  on  the  image  the  image  is
contained  in  this  this  argument  gives
the  containment  but  you  can  actually  see
the  other  inclusion  by  just  exhibiting
uh  I  just
yeah  um  right  okay
so  uh  where  am  I  oh  yeah  so  to  prove  um
we  um  it's  enough  to  we  we  can
assume  uh  we  have  this  Q  in  addition  we
have  our  arbitrary  Norm  dialytic  ring
satisfying  this  property  but  we  can  add
in  addition  assume  we  have  this  Q  with  a
norm  of  Q  between  0  and
one  um  and  then  we're  working  over
so  uh  so  then  we're  over  this  Gast  to
base
um  and  then  for  the  first  claim  here
this  non-  archimedian
claim  um  what  do  we  need  to  do  so  well
then  the  UN  Universal
case  I  mean  given  that  we  fix  this  data
and  then  then  we  we  we  just  asking  that
Norm  of  two  is  less  than  or  equal  to  one
so  it  just  lives  over  uh  an  item  potent
algebra
so
why  said  pass
the  this  to  it  was  because  of  this  claim
that  you  have  a
cover  okay
yeah  and  you  can  check  things  like  non-
archimedian  triangle  inequality  after
passing  to  a  total  space  of  a  cover  okay
um  and  what  is  this  item  potent
algebra  um  this  is  just  you  take  the  you
know  you  take  the  over  over  this  Q  Plus
orus  One  gas  uh  you  take  the  Ring  of  um
holomorphic  functions  so  to  speak  you
know  this  overon  convergent  version  of
the  unit  dis  um  and  then  you  just  mod
out  by  uh  T  minus
2
and  what  this  gives  um  and  you
calculate  um  and  I  won't  get  into  the
details  um  what  this  gives  is  like  a  on
the  level  of  underlying  Rings  uh  so  you
have
um
um  so  kind  of  yeah  so  you  have  usual  so
you  have  a  power  Ser  or  lant  series  with
integer  coefficients  which  uh
um  over  converge  to  radius  zero  so  which
are  which  are  converge  on  some
unspecified  open  dis  uh  around  the
origin
um  and  yeah  in  this  way  you  see  the  that
this  already  shows  you  that  the
condition  is  independent  of
two  um  because  this  this  description
here  is  independent  of  two  and  then  you
can  just  look  at  the  universal  then  over
this  you  can  also  calculate  these  what
with  the  base  change  of  the  you  know
these  kinds  of  overon  convergent  rings
are  and  you  can  just  check  by  hand  very
brute  force  that  the  non-  archimedian
triangle  inequality  is  satisfied  that
these  it's  just  about  about  a  certain
there  existing  a  certain  map  of  item
potent  algebras  uh  lying  over  the
addition  map  on  on  the  polinomial  ring
um  so  in  particular  this  is  the
formal  the  convergent  formal  Ser
not
well  we  have  we  have  we  have  we  have
negative  powers  of  Q  here  what  does  it
mean
oc0  oh  I'm  sorry  sorry  sorry  sorry  yeah
Q  inverse  yeah  thank  you  thank  you  yeah
miror  morphic  at  the  origin  yeah  thank
you  um  okay  so  uh  that's  that's  how  you
prove  this
claim  um  this  claim  actually  follows
because  uh  the  only  other  Locus  you  need
to  worry  about  is  the  locus  where  you're
between  one  and  two  and  then  you're  in
this  non-  archimedian  branch  and  we  saw
you  get  the  usual  thing  you  have  the
triangle  inequality
there
um  and  um  and  then  um  yeah  so  to  well
actually  yeah  so  then  to  prove  this
claim
um  again  it's  it's  actually  enough  to
it's  I  think  it's  easier  to  prove  well  I
think  the  the  root  is  it's  easiest  to
prove  this  claim  um  and  then  that
implies  that  claim  because  this  means
that  um  again  that  the  only  other
possible  point  to  consider  we're  living
in  the  archimedian  locus  so  we're  just
some  rescaling  of  the  usual  absolute
value  and  then  this  uh  weak  triangle
inequality  this  quasi  Norm  triangle
inequality  is  actually
satisfied  um
so  um  and  to  prove  this  you  again  just
do  a  calculation  it's  sort  of  similar
except  you  um  instead  of  setting  t  equal
to  2  you're  setting  t  equal  to2  um  or
well  or  you're  setting  yeah  oh  there
there's  a  similar  or  t  inverse  equal  to
yeah  what  what  you  end  up  getting  is
some  weird  version  of  functions
convergent  uh  uh  on  the  open  dis  like
this  um  and  again  you  just  observe  that
it's  independent  of
two
um  yeah  so  in  this  Locus  your
convergence  property  is  shrinking  down
to  zero
uh  in  this  Locus  the  the  the  locus
relevant  to  this  claim  um  it's  kind  of
it's  going  in  the  it's  the  laurant  tals
that  are  getting  um  you  know  forcing  the
convergence  out  to  the  boundary  of  the
unit
dis  um  yeah  yes  please  yes  yes  yes  sure
go  ahead  yes  okay  I  don't
know  does  the
a  can  you  uh  uh  explain  a  little  bit
more  by  what  you  mean  by  that  maybe  by
giving  an
example  okay  okay
yeah  um  yeah  okay  so  as  mentioned  uh
this  Friday  there's  no  course  but  please
come  here  and  Peter  will  be  here  in
person  and  so  will  three  other  people
giving  talks  um  some  of  which  are
relevant  to  the  material  here  um  and
then  next  week  I'll  continue  this
discussion  of  Burk  ofit  geometry  next
next  Wednesday  and  then  that  next  Friday
is  actually  going  to  be  the  last  class
so  we're  really  almost  done
here
um  okay  thanks
everyone  over  it's  over
so  yeah  can  I  just  ask  I  mean  what  I
think  I'm  what  what  what  it  means  if  you
have  ontic  stack  which  sits  over  like
any  of  these  classical  regions  of  but
but  what  does  it  mean  if  it  sits  over
infinitely  really  extra  point  I  mean  do
you  have
some  it's
weird  I  don't  know  it's  it's  a  bit
weird  the  norm  is  quite  different  the
norm  there  was  this  Norm  on  a  ring  in
the  sense  but  here  you  have  an  analytic
one  so  a  norm  is  is  something  quite
different  I  mean  it's  as  we  said  that
for  for  like  the  T  ring  it's  somehow  you
choose  one  nor  and  then  you  can  rescate
by  continuous  exponent  exactly  on  the  to
around  continuous  functions  from  the  b
space  to  R  yes  I  mean  this  is  actually
true  uh  in  the  the  set  of  norms  is  it
also  like  this  was  so  it's
not  actually  statement  is  that  related
to  this  SP  that  just  mentioned  that  um  I
mean  on  this  stack  of  norms  mention
of  by  scaning  the  norm  and  you  can  with
both
and  then
any  space  has  a  unique  map  to  the
sky  any  analytic  space  yeah  we  renamed
it  Tate  to  not  conflict  with  uh  yeah
so  an  stack  ofs  Z  by  yeah  exactly  which
is  rescaling  the
norms  and  govern
mentions  to
any  I'm
sorry  no  no  this  is  a  quotient  of  the  I
mean  well  except  except  for  this  AR  it's
a  qu  quo  of  the  burkovich  line  by
rescaling  but  but  there's  more  structure
on  this  because  it's  an  analytic  stack
and  not  just  a  topological  space  so  umst
action  is  free  which  is
not  uh  the  action  of  our  positive  is
because  it's  as  a  set  say  is  a  set  is
free  so  it  yeah  so  the  yeah  so  I  I  so
relatedly  like  when  you  in  some  sense  uh
so  I  I  discussed  it  what  happens  in  the
open  parts  of  these  things  but  actually
if  you  try  to  move  towards  the  the
boundary  points  what  about  the  infinity
point  in  this
question  yeah  it  just  uh  it's  Trigg
well
no  because  I  mean  you  so  from  if  you
only  look  at  the  perspective  of  norms  of
integers  then  it  might  look  like  it's  a
stabilizer  but  um  well
I  yeah
um  so  uh  right  where  oh  yeah  so  yeah  so
I  said  that  so  on  the  interior  it's  kind
of  this  this  this  uh  analytic  stack  and
it  feels  kind  of  like  a  space  like  on
the  interior  you're  just  seeing  that
unit  interval  but  base  change  to  some
gaseous  version  of  say  QP  here  or  the
real  numbers  here  but  as  you  move
towards  the  outer  points  or  towards  the
inner  point  it's  kind  of  it's  it's  more
it  really  is  more  like  a  stack  so  it's
not  that  you  know  the  the  fiber  over
this  point  is  not  uh  not  apine  for
example  neither  is  the  fiber  over  this
point
and  um  that  the  boundary  is
fixed  on  the  component
um
well  from  a  geometric  perspective  no
because  over  this  point  you  can  have
something  like
FP  luron  series  q  and  then  you  have  the
Notions  of  overon  convergent  functions
on  a  dis  there  and  then  when  you  rescale
it's  really  changing  which  dis  you're
labeling  by  a  given  positive  real  number
so  it's  not  universally  true  that
like  yes  yes  yes  exactly
yeah
oh  according  to  the  website  of  I  we  have
two  more  weeks  yes  uh  we  may  have  to
change
that  because  there  was  some  uh
yeah  apparently  on  the  IHS  website  it
says  we  have  two  more  weeks  so  but  I'll
I'll  get  I'll  get  it  fixed  because  the
number  of  talks  was  originally  more  than
the  actual  yeah  yeah  so  we
yeah  I  mean  we  did  decide  that  next  week
is  the  last  week  right  Peter
yes
okay  what's
that  iass
here  yeah  I  see  yeah  yeah  in  in  bond
classes  officially  end  that's  why  we're
stopping
\end{unfinished}