% !TeX root = AnalyticStacks.tex

\section{\ufs Generalities on analytic rings (Clausen)}

\url{https://www.youtube.com/watch?v=38PzTzCiMow&list=PLx5f8IelFRgGmu6gmL-Kf_Rl_6Mm7juZO}
\renewcommand{\yt}[2]{\href{https://www.youtube.com/watch?v=38PzTzCiMow&list=PLx5f8IelFRgGmu6gmL-Kf_Rl_6Mm7juZO&t=#1}{#2}}
\vspace{1em}

\begin{unfinished}{0:00}
e
in  the  last  lecture  Peter  was  um  well  we
we  spent  a  lot  of  time  talking  about
solid  analytic  Rings  um  in  the  last
lecture  Peter  started  to  introduce
different  kinds  of  classes  of  analytic
Rings  which  also  work  in  the  archimedian
context  so  that's  these  a  liquid  and
these  gaseous  analytic  Rings
um  what  I  want  to  do  today  is  I  want  to
uh  talk  about  some  generalities  on
analytic
rings
partially  in  in  service  of  the  the  story
Peter  is  telling  um  and  partially
because  uh  it  sort  of  should  be  done  at
some  point  it's  going  to  turn  out  when
discussing  these  new  examples  of
analytic  Rings  it's  kind  of  nice  to  have
a  good  handle  on  the  the  category  of
analytic  rings  and  how  to
manipulate  um  how  to  manipulate  in  it  so
I'll  just  be  presenting  some  various
facts  about  uh  analytic  rings  in  their
category  um  so  let  me  just  start  with
uh  let  me  start  by  recalling  the
definition
so  so  an  analytic
ring  is  a
pair
uh  uh  where  this  triangle  ring  is  a
condensed  ring  commutative  say
and  mod
R  uh  inside  mod  R  triangle  meaning
condensed  R  modules  uh  is  a  full
subcategory  closed
under  uh  limits  colimits  and
extensions  um  and  such  that  additionally
um
uh  if  you  have  anything  any  x  i  group  in
our  our  triangle  modules  from  anything
in  our
triangle  uh  to  anything  in  mod  R  uh  it
still  lies  in  mod
R  uh  for  all
I  and  finally  you  demand  that  um  that
this  the  unit  object  the  the  uh
underlying  ring  itself  should  lie  in
this  full
subcategory  let's  call  it's  called  that
last  condition
star
um
so
um  so  uh  now  I'll  make  another
definition
so  so  uh  uh  let's  say  I  don't  know  a
non-complete  analytic
ring  it's  not  a  special  kind  of  analytic
ring  it's  something  more  General  than  an
analytic  ring  so  maybe  a  pre-analytic
ring  uh  is  a
pair  as
above  uh  except  we  don't
require  star  so  it's  the  concept  you  get
by  taking  the  definition  of  analytic
ring  and  removing  this  last  condition
that  the  uh  unit  should  be  the  unit
object  should  be
complete
um  so  what's  the  reason  for  considering
this  relaxation  it  kind  of  plays  a  uh  a
technical  role  because  for  example  if
you  want  to  produce  an  analytic  ring
structure  uh  on
a  uh  an  interesting  condensed  ring  uh
then  prior  you  have  to  control  some  X
groups  in  the  category  of  our  triangle
modules  and  that  can  be  quite
complicated  to  do  because  you  would  have
to  understand  the  homological  a  prior
you'd  have  to  understand  the  homological
algebra  of  condensed  R  triangle  modules
before  you  can  get  going  on  such
calculations  and  if  this  is  for  example
the  some  liquid  base  ring  some
convergent  lant  series  it  can  be  kind  of
difficult  to  do
that  but  what  often  turn  turns  out  in
practice  is  that  there's  a  a  simpler
ring  that  you  can  map  to  our  triangle  um
in  the  case  of  the  arithmetical  rant
series  it  would  just  be  say  polinomial
and  T  and  T  inverse  um  such  that  it  is
easier  to  produce  a  pre-analytic  a
pre-analytic  ring  structure  on  that
simpler  R  triangle  um  and  then  you  get
the  you  have  that  this  condition  is  not
necessarily  satisfied  but  what  I'm  about
to  discuss  is  that  there's  in  a
completion  procedure  that  lets  to  um  go
from  any  non-complete  analytic  ring  to  a
complete  analytic  ring  and  that's  kind
of  the  most  convenient  way  to  do
things  um  and  we've  also  seen  examples
uh  when  discussing  attic
spaces  so  recall  with  this  uh  solid  RR
plus
Theory  um  when  we  were
specifying  the  category  attached  to  a
rational  open  we  took  the  category  of  R
modules  and  then  we  just  specified  a
bunch  of  conditions  um  which  actually
made  it  into  a  pre-analytic  ring  in  this
sense  um  but
then  uh  this  condition  was  not
necessarily  satisfied  so  you  would  need
to  change  the  r  triangle  to  basically
almost  the  value  of  Huber's  structure
sheath  on  that  rational  open  in  order  to
get  an  honest  analytic  ring  so  so  these
kinds  of  things  arise  in  practice  and  in
general  what  you  want  to  do  with  them  is
you  want  to  complete  them  to  get
something  that  is  an  analytic  ring  in
the  honest  sense  and  that's  that's  the
procedure  I  want  to  discuss  right
now  um  so  and  simil  so  a  map
so  just  the  last  Parts
yeah  yeah  you  still  require  that  that's
right
yeah
um
uh
right  um
so  a  map
um  a  map  of  pre-analytic
rings  is  is  just  like  um  just  like  for  a
map  of  analytic  brings  it's  a  map  of
underlying  condensed  strings
uh  such  that  uh
n  in  mod  s  implies  that  well  the
Restriction  of  scalars  of  n  uh  which  is
a  prior  just  an  R  triangle  module
actually  lands  in  mod  R  and  this  is
maybe  best  better  thought  of  as  saying
that  the  base  change
functor  um
so
um  and  then  there's  the  completion
function
the  r
completion  going  to  mod
R  and  then  you  have  the  S
completion  uh  going  to  mod  s  so  this  is
this  completion  functor  exhibiting  this
as  a  localization  of  this  and  similarly
here  and  this  condition  here  is  the  same
thing  as  saying  if  you  go  here  and  then
you  complete  that  actually  factors
through  here  um  necessarily  uniquely
because  this  is  a  localization  so  it's
saying  that  if  you  have  a  map  here  which
is  is  uh  which  is  kill  which  goes  to  an
isomorphism  here  then
um  then  it  goes  to  an  isomorphism  here
so  then  you  get  a  a  symmetric  monoidal
base  change  functor
um  uh  from  Mod  R  to
RS  mod  R  to  mod  s
sorry
okay
um
uh  so  that  defines  the  category  of
pre-analytic  rings
um  uh  and  there's  a  so  so  those  so  in
other  words  there's  a  fully  faithful
inclusion
um  some  category  of  pre-analytic
rings  into  the  category  of  analytic
Rings  oh  analytic  Rings  into  the
category  of  pre-analytic
rings  and  this  inclusion  has  uh  and  the
the  the  claim  or
proposition  uh  this  inclusion  has  left  a
joint
um  uh
moreover  uh  so  the  left  the  joint  sends
so  given  a  pre-analytic
ring
uh  our  triangle  mod
R  uh  this  left  ad
joint  uh
sends  sends  uh  our  triangle  mod
R  to  um  and  some  explanation  will  be
required  but  so  we  take  our  triangle  and
we  complete  it  with  respect  to  this  uh
this  uh  pre-analytic  ring  structure  here
so  we  apply  the  left  ad  joint  to  the
inclusion  from  Mod  R  into  mod  R  triangle
um  and  then  we  take  so  to  speak  the  same
mod
R
um  so  let  me  give  the  proof
um
um  so  first  we  need  to  make  sure  that
this  is  of  the  appropriate  type  I  mean  I
need  to  explain  how  mod  R  is  actually  a
full  subcategory  of  mod  R  triangle  hat
uh  in  order  to  for  this  to  parse  as  a
claim  and  then  I  need  to  check  that  this
setup  satisfies  the  same  axioms  as  as
this  setup  so  it  actually  defines  a
pre-analytic
ring  um
so  uh  so  first  of  all  well  I  also  I
should  also  explain  why  this  is  a
commutative  uh  a  condensed  commutative
ring  I'm
sorry  mod  R  is  the  same  yeah  yeah  yeah
so  so  first  of  all  so  this  our  triangle
hat  so  is  a  commutative
ring  condensed
ring
so  this  is  a  sort  of  a  completely
General  phenomenon  so  we  have  this  mod  R
triangle  um  and  this  uh  localization
functor  uh  to  mod  R  this  completion
functor  this  R  completion  and  this  is
symmetric
monoidal  and  cimit  preserving  and  it  has
a  right  ad
joint  which  is  the  which  is  the  uh  the
inclusion  in  this  case  but  when  you  have
a  symmetric  monoidal  functor  with  a
right  ad  joint  then  the  right  ad  joint
is  automatically  lack  symmetric
monoidal  um  and  in  particular  if  you
take  the  unit  object  here  um  and  then
you  apply  the  right  ad  joint  it  gets  a  a
commutative  algebra  structure
so
so
is  a  commutative
algebra  in  this  category  uh  in  this
tenser  category  and  therefore  a  fortiori
in  in  um  in  condensed  ailan  groups  so
it's  a  condensed  commutative
ring
um
um  so  two  we  should  check  that  uh  and
moreover  yeah  so  not  only  also  in  this
General  situation  with  the  right  ad
joint  to  a  symmetric  monoidal  funter  if
you  take  any  object  any  object  here  and
apply  that  right  ad  joint  it  gets  the
canonical  structure  of  a  module  over
this  commutative
algebra  um  again  just  purely  formally  so
uh  so  there's  this  natural
factoring  so  we  have  mod  R  including
into  a  mod  our
triangle  uh  this  actually  factors
through  uh  our  triangle  hat
modules  uh  and  then  the  forgetful
functor  um  and  the  next  thing  I  should
check  is  that  this  is  actually  fully
faithful
so
so  that  we  can  indeed  view  mod  R  not
just  as  a  full  subcategory  of  mod  R
triangle  but  of  mod  R  uh
complete  uh  our  triangle  complete  and  so
for  that  uh  so  the  proof  is  you  can  just
ask  for  HS  so  if  you  take  M  and  N  in  mod
R  uh  you  want  to  compare  h  r  m  uh  sorry
h  r  triangle  MN  to  H
our  triangle  hat
MN
um  but  uh  so  n  is  an  R  hat  triangle
module  so  therefore  this  one  is  the  same
thing  as  oh  I'm  sorry  I  think  I  should
give  myself  a  little  bit  more  room  so
this  is  the  same  thing  as  H  our  triangle
hat  from  our  triangle  hat  tensor  over
our  hat  M  to  n
um  but  now  since  n  is  complete  since  n
lies  in  mod  R
um  we  can  uh  complete  this  we  can  apply
the  completion  functor  to  this  too  uh
and  we'll  get  the  same  thing  so  this  is
same  as
H  our  triangle  hat  our  triangle  hat
tensor  our  triangle  M  hat  to  n  and  now
we  can  use  that  the  completion  is
symmetric  monoidal  so  that  we  can  move
it  inside  this  whole  relative  tensor
product  and  when  you  move  it  in  the
relative  tensor  product  and  use  the
completion  is  item  potent  you  find  that
this  is  indeed  just  a  h  r  hat
triangle  M
tan
okay
so  um  it's  now  we  have  to  verify  all  of
the  axioms  and  this  is  actually  a  little
bit  more  subtle  because  we  need  to  make
sure  that  the  X
calculations  uh  are  going  to  work  out
correctly  um  and  this  argument  so  so  let
me  so  remark  uh  this
argument  for  fully
faithfulness  uh  will  also  work  for
x
provided  that  uh  that  the  derived
completion  so  now  I  don't  know  what
derived  completion  uh  on  our  triangle  is
the  same  as  the  completion  on  our
triangle
so  so  recall  that  we  had  this  uh  notion
of  the  derived  category  of  an  analytic
ring  which  was  a  full  subcategory  of  the
derived  cat  of  our  triangle  and  we  had
the  derived  completion  functor  which  was
not  necessarily  on  on  an  even  some  on
something  sitting  in  degree  Z  does  not
necessarily  agree  with  the  the  usual
completion  it's  something  which  whose  Pi
Z  or  h0  is  the  same  as  this  but  it  could
have  higher
homology  um  for  example  in  this  uh  yeah
in  this  situation  with  these  weird  non
sheify  uh  attic
spaces  um
but  uh  in  general  we  don't  have  that  so
what  we  what  we  can
see  what  we  can  formally
see  see  is
that  uh  everything
works  if  we
replace  uh  our  triangle  Hat  by  our
triangle  hat  uh  the  derived  thing  and
use  the  notion
of  a  notion  of  derived  analytic
ring  meaning  that  if
you
uh  if  you  ask  not  to  give  an  analytic
ring  structure  on  this  ordinary  ring  but
an  analytic  ring  structure  on  this
derived  enhancement  then  using  the  same
kind  of  formal  argument  you'll  be  able
to  check  all  of  these  properties  so
limits  and  co-  limits  are  actually
automatic  um  given  that  mod  R  has  those
closure  properties  but  to  be  closed
under  extensions  in  here  is  is  more  a
priority  more  than  being  closed  under
extensions  here  but  if  you  can  make  the
X  calculations  you'll  see  that  it's  the
same  um  and  then  this  is  also  a  question
of  an  X
calculation
um  so  uh  if  you  were  to  use  a  notion  of
derived  analytic  ring  then  you'd  be
done  uh  with  that  it  wouldn't  prove  the
original  proposition  because  I  stated  it
in  the  underived  way  but  it  would  prove
something  um  so  now  I  think  uh  I  will
introduce  a  notion  of  derived  analytic
ring  kind  of  in  the  middle  of  this
proof  uh
so  so
definition  so  so  uh  an
analytic  e  infinity  ring
uh  is  a
pair  um  R  triangle  mod  r  or  maybe  maybe
I'll  say  d  of
R  uh
where  actually  it's  it's  sort  of  more
convenient  to  only  specify  the  the  part
that  that's  homologically  in  homological
degrees  greater  than  or  equal  to  zero  so
where  our
triangle  is  a  a
condensed  uh  e  infinity
ring  um
and  connective  connective  thanks
yeah  so  meaning  it  also  only  lives  in
homological  degrees  um  and  D  greater
than  or  equal  to  zero  our  subset  uh  oops
D  greater  than  or  equal  to  zero  our
triangle  is  a  full  subcategory
such  that  uh  so  it's  closed
under  limits
colimits  and  such
that  uh  internal
H  uh  in  our  triangle  from  anything  in  D
greater  than  or  equal  to  zero  I  should
say  in  D  greater  than  or  equal  to  zero
our
triangle  uh
uh  to  anything  in  D  greater  than  equal
to  z  r  is  still  in  D  greater  than  equal
to  z
r  this  full
subcategory
um  complete  or  not  complete  uh  yeah
that's  a  good  thank  you  uh  yeah  so  let's
say  a  pre-analytic
ring
um  and  then
uh  I  want  to  explain  the  relation
between  this  notion  of  analytic  ring  and
this  sort  of  derived  notion  of  analytic
ring  and  then  the  Bare  Bones  notion  on
the  aelan  level  that  we  defined  earlier
so  the  proposition  is
that
um
so  for  an  any  in
connective  condensed  infinity
ring  uh  our
triangle  uh  there  exists  a
bje
between  the  set  of  uh  pre-analytic  ring
structures
uh  on  our
triangle  uh  and  the  pre-analytic  ring
structures  so  this  is  in  the  old
sense  this  is  in  the  new  sense  um  on
just  the  condensed  commutative  ring
which  is  gotten  by  taking  Pi
0  um  and  also  uh  an  antic  matches  up
with  analytic  so  or
complete  so  I  can  also  could  also  remove
the  pre  from  both  sides  so  this
projection  on  the  level  of  pre-analytic
ring  structures  restricts  to  a  bje  on
the  level  of  analytic  ring
structures
um  and  what  is  this  gotten  by  so  here
you  have  uh  this  D  greater  than  or  equal
to  z  r  is  the  data  and  you  you  send  that
to  uh  just  those  things  in  D  greater
than  or  equal  to  z  r  which  happen  to  lie
in  the  heart  so
uh
so  let's  say  d  equals
z  uh  R  triangle  which  is  the  same  thing
as  D  equal  0  pi  r  triangle  so  you  have  a
module  concentrated  in  degree  zero  over
some  derived  ring  that's  just  the  same
thing  as  giving  a  module  usual  module
over  the  the  pi
0
so
um  and  in  this  direction  it's  given  by
sending  uh  mod  R  to  the  set  of  those  m
in  D  greater  than  or  equal  to  zero  R
such  that  h  m  is  in  mod
R  uh  for  all  I  greater  than  or  equal  to
zero
so  is  this  definition  that  be  a  problem
that  D  greater  than  zero  might  not  be
close  on  the  limit  limits  didn't  I  ask
didn't  I  ask  it  to  be  no  I  mean  a  whole
category  not  I  don't  uh  can  you  repeat
your  question  pops  Li  it  goes  to  to  like
the  greatest  than  minus
one  means  ouring  here  yeah  I  mean  yeah  I
still  have  but  we  gave  the  argument
fixing  that  uh  so  when  you  have  this
assumption  you  can  you  can  see  that  that
holds  um  when  you  if  you  take  a  an
arbitrary  direct  sum  of  copies  of  the
unit  then  that  then  you'll  see  that
products  of  a  fixed  even  higher  higher
derived  products  of  a  fixed  a  given
fixed  element  will  be  still  in  the
category  and  then  if  you  have  an
arbitrary  product  then  you  can  write  it
as  you  can  write  it  as  a  retract  of  a
product  with  a  fixed  element  by  taking
the  direct  sum  of  all  the  elements  and
uh  yeah  so  we  we  did  this  argument  in
one  of  the
lectures  I  guess  he  asked
if  infity  oh  yes  was  that  your  question
or  or  well  in  any  case  it  means  in  the
infinity  categorical  sense  it  means  R
Lim  implicitly  and  r  colim
um  right  so  let  me  make  a  remark  um  so
in  a  special
case  uh  where  where  well  R  triangle
equals  Pi  0  R  triangle  we  see  that  the
two  definitions  are
consistent  meaning  in  the  case  where  the
notion  of  rings  over  lapse  namely  where
your  derived  ring  is  just  concentrated
in  degree  Z  so  it  is  a  classical  ring
the  derived  definition  of  what  an
analytic  ring  structure  or  a
pre-analytic  ring  structure  is  matches
up  with  the  the  the  naive  definition  we
gave  on  the  level  of  a  billion
categories
um  so  that's  the  first  remark  and  the
second  remark  is  this  finishes
proof  uh  the
proof  of  the
proposition  on
completion  um  because  what  we  can  do  is
we  can  just  um  so  there's  no  obstruction
to  proving  the  proposition  on  completion
in  the  derived  SE  uh
setting  uh  and  then  we  get  a
um  but  then  we  can  move  it  down  to  the
Aion  setting  by  going  from  the  analytic
ring  structure  on  this  derived  thing
that  we  produced  to  the  analytic  ring
structure  on  its  Pi  Z  just  by  applying
this  procedure
here's  a
question  are  reallying  that  greater
equal  to  Z  is  limit  I  think  claiming
that  you  take  the  limit  in  D  greater
equal  to
Z  yes  that's  right
yes  yeah  so  somehow  there's  a  there's
something  going  on  here  so  for  example
being  claimed  being  closed  under  the
loops  funter  so  that  that  that's
actually  stronger  than  you  might  think
because  well  you  what  what  what  is  the
loops  in  in  this  greater  than  equal  to
zero  derived  category  it's  you  shift  it
down  and  then  you  kill  the  bottom  thing
so  it's  really  just  like  killing  the
bottom  thing  up  to  a  shift  um  and  so
that  kind  of  goes  into  the  argument  here
um  that's  how  you  can  see  that  an  object
in  there  is  an  object  lies  in  here  if
and  only  if  each  of  its  homology  groups
lies  in  there  and  that's  one  of  the  one
of  the  key  things  that  goes  into  this
proposition  here
um
okay  uh  I  think  this  condition  doesn't
always  match  up  so  if  you  have  something
that's  complete  the  Rock  complet
complete  but  in  the  compass  Direction  I
think  might
have  some
J
uh  and  that's  not  possible  right  just
because  of  this  fact  that  if  you  have
something  derived  complete  the  needs  of
each  of  its  homology  groups  is  also
complete  or  am  I
misunderstanding  now  you're  saying  that
you  have
pric  which  on  the  LEL  Z  is
already  it's  complete  when  you  say  it's
already  complete  but  I  don't  think
there  but
sorry  one  more  complet  match  up  it  seems
to  say  that  if  you  have
if  I  is  the  property  that  only
zero  is  complete  that  it's  already
complete  but  I  oh  you're  right  you're
right  I'm  sorry  yeah  no  you're  you're
right
uh
um
um
um  so  then  let  me  rather  say  thank  you
Peter  so  let  me  rather  say  um  an
analytic  ring  here  goes  to  an  analytic
ring  there
uh
uh  and  the  converse  is  true  if  your  ring
is  concentrated  in  degree  Z  so  in  the
setting  of  The  remark
um
um
yeah  topologically  as  space  of
sorry  the  sound  is  not  coming  through
very  clearly  so  uh  if  you
could
mean
understand  topological
ring  I  don't  no  I
understand
topolog
topology
space  I  so  here's  the  way  I  would  think
about  e  Infinity  is  it's  just  it's  a
it's  just  there's  a  general  categorical
concept  right  if  you  have  a  symmetric
monoidal  category  then  you  can  look  at
commutative  algebra  objects  in  it  and
there's  an  Infinity  categorical  version
of
that  which  implicitly  involves  things
like  infinity  operads  or  something  but
really  I  mean  um  and  then  an  e  infinity
algebra  over  Z  will  just  be  a
commutative  algebra  object  in  the
derived  category  of  Z  in  that  Infinity
category  with  the  usual  Drive  tensor
product
so  if  you  believe  in  this  Infinity
category  mumbo  jumbo  you  don't  need  to
think  about  it  in  being  built  in  terms
of  topological  spaces  and  operads
explicitly  you  can  kind  of  just  plug
into  some  simple  categorical
formalism
um
okay
so  that's  the  notion  of
completion
um  yeah  maybe  okay  I  was  I'm  reassessing
that  so  maybe  the  good  thing  to  say  is
that  you  have  an  an  analytic  ring  over
here  um  gives  you  an  analytic  ring  over
here  such  that  also  the  yeah  so  analytic
ring  structures  here  are  the  same  thing
as  analytic  ring  structures  here  uh  such
that  uh  such  that  each  pii  not  just  Pi  Z
is  complete  in  this  category  here
so
um
yeah  uh
right  um  so
now  uh  I'm  not  going  to  prove  the
proposition  this  was  proved  in  uh  in  one
of  the  older  lectures
so  you  can  look  there  for  the
argument
um  and  now  I  want  to  move  on  to  the  next
topic  which  is  co-  limits  in  the
category  of  analytic  Rings  although
maybe  I  should  make  another  well  sorry
maybe  I  should  make  another  remark  yeah
so  it's  important  to  note  that  the  uh
the  derived  categories  can  change  when
you  so  if  you  take  you  take  an  analytic
ring  structure  on  Pi  kn
it  gives  you  an  ailan  category  but  we
also  argue  it  gives  you  a  derived
category  um  that  derived  category  is  not
necessarily  going  to  be  the  same  as  the
derived  category  you  get  on  R  triangle
when  you  move  along  this  equivalence
because  the  the  higher  homology  in  R
triangle  can  can  uh  can  make  things  a
little  different  so  um  so  when  we're
doing  this  completion  on  the  level  of
naive  analytic  rings  sitting  in  degree
zero  it's  important  to  note  that  so  to
speak  the  derived  category  is  not  the
correct  one
um  the  derived  category  of  this  naive
completion  so  yeah  maybe  I'll  make  a
warning  um  so  if  so  if  r  at  triangle  mod
R  is  the
completion  of  uh  our  triangle  mod
R  then  it's  not  necessarily
true  that  uh  the  derived  category  of
this  thing  is  the  same  thing  as  the
derived  category  of  this
thing  um  but  only
if  if  you  use  derived  completion  on  the
left  hand
side  so  if  the  if  this  ring  that  you
calculate  if  you  calculate  this  derived
completion  and  it  happens  to  live  in
degree  zero  then  this  will  be  true  and
it'll  and  the  derived  categories  will
pass  along  correctly  but  if  this  happens
to  have  higher  homology  then
um  then  you  really  should  be  considering
a  derived  analytic  ring  uh  instead  of  an
ordinary  analytic  ring  and  this  was  the
phenomena  we  phenomenon  we  saw  in  in
adct  spaces  that  for  some  rational
localizations  in  very  exotic  contexts
you  get  a
derived  structure  sheath  and  then  that's
then  you  really  should  even  though  you
have  an  underlying  ordinary  analytic
ring  by  taking  Pi  Z  you  really  should  be
considering  the  derived  analytic  ring
there
sorry  how  do  the  X  computations  differ
well  they're  XS  over  different  Rings
right  right  but  like  they're  all  like  I
mean  in  the  sense  that  like  when  you
complete  it  then  like  you  have  that
condition  that  X  the  mod  R  triangle  mod
R  lives  in  mod  R  yeah  but  then  so  now
it's  still  the  same  mod  R  yeah  it'll
live  in  the  same  category  but  the
objects  won't  be  the  same  yeah  you  can
have  like  wildly  different  things  well
wildly  I  I  I  mean  you  can  have  different
things  these  are  kind  of  exotic
phenomena  I'm  not  saying  yeah  so
it's  I  haven't  actually  investigated
really  I  mean  because  the  the  phenomena
like  when  this  happens  it  tends  to  be
fairly  exotic  so  um  haven't  really
seriously  tried  to  do  those
calculations  um  okay
so  all  right  so  now  we  have  this  notion
of  completion  of  analytic  rings  and  this
lets  us  discuss
um  uh  Co  limits  of  analytic  rings
so  uh  there's  something  else  that  I
maybe  should  have  mentioned  in
connection  with  this  what  I  just  erased
so  this  uh  analytic  ring  structure  is  on
R  corresponding  to  analytic  ring
structures  on  Pi  Z  of  R  um
which  is  that  maps  of  maps  in  the
category  of  analytic  rings  in  the
derived  sense  are  the  same  thing  as  maps
on  the  level  of  derived  Rings  which  are
maps  in  the  usual  sense  in  the  old  sense
when  you  when  you  look  on  the  aelion
categories
there
um  so  okay  um  right  so  Co  limits  in  the
category  of  analytic
Rings  uh  so  so  the  uh  the  procedure  is
is  you  take  the  co
limit  uh  in  pre-analytic
rings  and  then  you
complete  and  formally  this  must  be  how
you  compute  Co  limits  just  because  the
completion  functor  was  the  left  adjoint
to  the
inclusion  um  so  what  are  what  so  then
what  is  this  Co  limit  in  pre-anal  IC
Rings  it's  actually  quite  naive  so  if
you  have  a  diagram
so  uh  r  i  um  R  triangle  I
mod  r  i  oh  I  and  I  then  the  co-  limit
is  is  given  by  taking  the  co  liit  of
condensed
rings
and  then  so  this  is  a
yeah  co-limiting  condensed  commutative
Rings  uh  oh  sorry  uh  what  no  sorry  sorry
uh  the  co  limit  of  R  triangle  I  mod  r
i  is  uh  the  pair  Co  limit  over  I  uh  our
triangle  I  and  that's  the  co  limit  in
condens  commutative  Rings  um  and  then
you  take  a  certain  full  subcategory  so
called  mod  cimit  I  ini  I
uh  RI  um  where  this  is  the  those  m  in
mod  uh  colimit
RI  uh  triangle  uh  such  that  uh  for  all  I
M  lies  in  mod  r  i  or  when  you  restrict
scalar  is  RI  triangle  rather  um  then
then  it  lies  in  that  category
there  so  it's  just  an  intersection  of  a
bunch  of  categories  uh  where  those
categories  are  just  required  when  you
restrict  scalers  you  lie  in  the
corresponding  complete  thing
there
um  so  this  is  quite
um
it's  quite  straightforward  to  from  the
definition  of  the  category  of  analytic
rings  to  see  that  this  is  the  way  to
calculate  co-  limits  um  by  definition  a
map  was  a  map  of  rings  uh  satisfying  a
certain  property  so  you  certainly  want
um  certainly  want  to  put  the  co-  limit
of  the  Rings  here  and  then  this
definition  here  is  just  tailored  so  that
you  have  the  um  correct  property
there  um
um  and  checking  that  this  satisfies  the
axioms  of  a  pre-analytic  ring  is  also
not  difficult
um  so  well  certainly  closure  under
limits  and  colimits  is  Elementary
because  the  forgetful  functor  that
you're  using  here  commutes  with  limits
and  colimits  and  the  closure  under  the
XS
is  uh  also  not  so  difficult
um  so
um  yeah
uh  you  yeah  you  just  you  you  you  just
check  it
directly  um  so  let  me  now  take  some
examples  so  filtered  Co  limits  in
analytic
rings  so  the  claim  is  that
here  uh  the  completion  is
unnecessary  so  the  this  filtered  Co
limit  of  RI  triangle  um  mod  filtered
co-limit  of  uh  r  i  is  already
complete
um  so  what  does  that  mean  that  means
that  uh  you  take  this  you  have  to  ask
whether  this  ring  lies  in  that  in  this
category  um  but  um  lying  in  that
category  means  for  every  I  you  lie  in
mod  r  i  but  uh  from  I  on  in  this
filtered  Co  liit  you  lie  in  mod  RI  by
definition  and  so  from  I  on  that's  a  co-
final  collection  here  so  you  can
calculate  this  as  a  filtered  Coit  of
things  which  satisfy  the  condition  so  it
also  satisfies  the  condition
um  so  an  example  of  this  we've  already
seen  is  that  if  R  is  a  discrete
commutative
ring  uh  then  solid
R  is  the  filtered  Co
limit  of
solid  r  i  i  ini  I  if  R  is  the  filtered
cimit  of  the
ri  maybe  you  can  say  what  you  mean  by
this  Coit  there  this  Co  liit
here  yeah  yeah  this  is  in  the  category
of  analytic
Rings  yeah  so  or  pre-analytic  ring
some  um
okay  so  if  you  understand  filtered  Co
limits  then  the  next  thing  you  should
try  to  understand  is  pushouts  so
uh  so  push
outs  so  if  you  have  a  r  goes  to  a  goes
to  b  r  goes  to  A  and  R  goes  to  B  uh
diagram  of  analytic  rings  then  you  need
to
complete
uh  the  pre-analytic  ring  which  is  a
tensor  over
RB  um  and  then  the  full  subcategory  of
those  m  in  mod  a  oh  sorry
triangle  uh
mod  a  triangle  tensor  our  triangle  B
triangle  such  that  uh  as  an  a  module
that  lies  in  mod  a  and  as  a  b  module  it
lies  in  mod
b
and  here  it  is  uh  important  to
complete  because
um  uh  this  thing  has  no  reason
to  uh  lie  in  mod  a  or  in  mod  b  and  in
fact  so  the  the  here's  a  remark  the
completion
functor  here  a
priori  involves  uh  uh
iterating  uh  a  completion  and  B
completion  this  is  a
alternating  it  one  category  or  yeah  so
I'm  doing  it  in  the  one  category  here
but  the  exact  same  thing  all  the  the
exact  same  remarks  apply  in  the  infinity
category  uh  if  you  know  the  ver  the
notion  of  derived  analytic  ring  you  just
interpret  everything  is  a  derve  tensor
product  and  use  D  greater  than  or  equal
to  zero  instead  of  mod  it's  all  there's
no  difference  in  this
discussion
yeah
um  okay  so  yeah  in  general  you'd  have  to
take  this  thing  a  completed  as  an  a
triangle  module  then  B  complete  that  as
a  b  triangle  module  then  a  complete  that
and  B  complete  that  pass  to  a  sequential
Co  limit  and  that  would  be  the
description  of  the  completion  functor
for  this  in  practice  it's  usually  not  so
bad  but  that's  um  a  priori  what  what  you
need  to
do  um  and  we  also  had  examples  of  this
in  the  solid
setting
so
um  so
example  if  we  had  one  of  our  RR  plus
settings  and  then  to  a  rational  open
and  say  of  R  uh
r+  then  we  get  a  new  analytic
ring
um  I  don't  know
um  which  was  the  completion  of  the
pre-analytic  Ring  structure  on  R  where
you  enforce  all  of  the  relations  descri
that  your  rational  open  describes  so  if
it  was  fub1  FN  over  G  then  you  require
that  gxs  inverti  on  your  modules  and
that  fi  over  G
um  uh  is  a  solid  is  a  solid  variable
with  respect  to  all  of  your
modules  um  and  then  the  the  pushouts  so
new  analytic  ring  I  I  I  kind  of  hesitate
to  call  it  O  of  U  plus  but  okay  I'll  do
it  anyway
um  maybe  um  and  then  if  you  take  O  of  U
O  plus  of  U  um  tensor  over  r  r+  with
some  other  OV  o  plus
v  uh  then  what  you're  going  to  get  is
the  same  thing  for  the  intersection  of
these  rational
opens  the  analytic  ring  corresponding  to
the  intersection  of  these  rational
Ops  I  guess  in  the  discret  case  our
discreete  it  literally  is  just  this
um  um  so  that  the  notion  of  push  out  in
analytic  Rings  is  corresponding  to
intersection  of  rational  opens  here  and
that  just  follows  from  from  the
definition  of  this
here
um  so  yeah  this  this  uh  these  push  outs
are  in  general  kind  of  the  most
important  construction  because  they
they're  geometrically  what's  supposed  to
correspond  to  pullbacks
um  and  this  business  of  having  to
complete  them  makes  for  an  additional
subtlety  compared  to  the  usual  algebraic
geometry
okay
so  uh  questions  about
that
um  is  the  procedure  becomes  easier  when
you  just  want
finite  finite  product
no  it's  the  case  of  relative  tensor
products  is  no  easier  uh  the  case  of
absolute  tensor  products  is  no  easier
than  the  case  of  relative  tensor
products  so  in  fact  I  mean  I  could
have  I  didn't  really  need  in  this  first
part  I  didn't  really  need  to  say
filtered  sifted  is  enough  if  you  know
what  sifted  means  um  and  then  General
co-  limits  just  as  general  Co  limits  can
be  decomposed  into  filtered  co-  limits
and  push  outs  General  cols  can  also  be
composed  with  into  sifted  Co  limits  and
and
co-products  so  and  concretely  the  the
completion  functor  for  this  relative
tensor  product  is  just  the  same  as  the
completion  functor  for  the  absolute
tensor  product
it's  shift  shifted  e  easy  sifted  Co  liit
is  just  as  easy  as  filtered  Co  limit
yeah  but  to  so  the  the  only  difficulty
is  about  the  finite  exactly  exactly  so
yeah  a  tensor  RB  is  no  more  difficult
than  a  tensor
B
um  so  okay  now  I  want  to  now  I  want  to
have  a  little  bit  of  fun  well  depending
on  your  definition  of  fun  uh
um
um
right  so
forus  so  I  want  to  prove  the  following
theorem  so  suppose  R  uh  is  an  analytic
ring  uh  then  the  fenus
map  which  goes  from  our  triangle  to  our
triangle  modulus  P  it's  induced  by  X
goes  to  x  to  the
p  uh  that's  certainly  a  map  of  condens
strings  um  but  this  is  actually  a  map  of
analytic
Rings  um  from  R  uh  to  well  Rod  P  by
which  I
mean  r  hat  mod  P
um  and
then  mod
R  yeah  so  P  toin  elements  yeah
um  so  you  just  induce  the  analytic  ring
structure  so  the  a  complete  module  over
our  mod  P  is  just  a  a  module  over  our
triangle  mod  P  which  is  complete  when
viewed  as  an  R  triangle
module
um  so  you  should  probably  I  mean  you  can
you're  free  to  assume  R  is
characteristic  p  in  which  case  uh  it's
really  just  a  map  from  R  to
R
uh
okay
um  so  what  do  we  need  to  do  to  prove
such  a
claim  well  according  to  the  definition
what  we  need  to  do  is  uh  so
proof  need  to
see  that  if  let  me  let  me  assume  uh  R
has  characteristic  P  so  R  is  an  FP
algebra  our  triangle  is  an  FP
algebra  uh  just  to  avoid  the  notation  of
them  modding  out  by  P  and  all  that  um  we
need  to  see  that  if  uh  m  is  in  mod
are  uh
then  uh  yeah  the  the
fenus  the  fenus  push  forward  of  M  uh
which  lies  in  M  mod  R  triangle  actually
lies
in  actually  lies  in  mod
R  and  this  is
uh  yeah  this  is  not  so  obvious  how  to
check  um  we  have  these  Maxum  of  analytic
ring  they're  all  about  closure
categorical  closure  properties  uh  in
linear  algebra  like  limits  and  colimits
and  Xs  and  so  on  they  say  absolutely
nothing  about  the
frobenius  um  they  give  no  kind  of  hint
as  to  why  this  should  be  true  um  however
let  me  make  a
remark  uh  there  is  another
perspective  on  maps  of  analytic  rings
um  so  first  of  all  so  first  of  all  if  uh
if  R  triangle  and  mod
R  uh  s
triangle  mod  s  is  a  map  of  analytic
rings
uh  then  for
all  light  profinite
sets  um
t  uh  we  get  a  map  an  R  triangle  linear
map  um  from  R  bracket  T  the  free  R
module  on  t  uh  to  the  free  s  module  on
T  Sorry  by  f  mean  frob  I  mean  frob  yeah
fenus  as  a  yeah  map  of  rings  so
yeah  thanks  uh  and  the  reason  is  well  by
by  hypothesis  of  this  being  a  map  of
analytic  Rings  uh  what  it  means  is  that
these  these  modules  here  are  actually
actually  live  in  mod  R  that's  equivalent
to  being  a  map  a  map  of  analytic  Rings
you  need  to  check  a  prior  for  all  s
modules  but  but  by  co-  limits  it's
enough  to  check  it  for  a  generating
class  in  fact  you  know  so  for  these  guys
in  fact  you  can  even  take  T  to  be  the
counter  set  if  you  like  so  that  this  is
the  thing  that  you  act  you  you  get  is
that  this  is
um  this  is  an  R  module  um  so  that  the
map  from  the  free  R  triangle  module  on  T
to  it  uh  factors  through
RT  um  and  this  is  this  is  functorial  in
s  uh  functorial  in  t
sorry  um
um  um  moreover  well
there's  so  these  aren't  quite  arbitrary
there's  a  small  extra  there's  an  extra
condition  that  in  particular  is
Satisfied  by  them  so  if  uh  if  you  do
have  a  map  from  T  to  our  triangle  uh  so
a  map  of  condensed  sets  uh  from  T  to  R
triangle  then  there's  two  things  you  can
do
um  one  you  can  make  a  map  from  R  bracket
T  to  R
triangle  um  an  our  triangle  linear  map
because  by  definition  of  analytic  ring
this  thing  is  complete  uh  so  you  go  to  R
triangle  bracket  T  and  then  you  go  to  R
bracket
T  um  uh  but  then  you  also  have  this  map
here  that  we've  assumed  exists  from  R
bracket  T  to  S  bracket
T  but  on  the  other  hand  you  can  compose
this  to  S  triangle  um  and  you
get  uh  this  thing  here  so  this  is  just  a
just  the  map  of  rings  this  is  the  thing
we  posited  to  exist  and  this  exists  for
the  same  reason  this  exists  and  that
square  uh  will  commute  when  you  have  a
map  of  analytic
rings
so  I  want  to  claim  so
LMA  uh
conversely  if  R  to  R  triangle  to  S
triangle  is  a  map  of  condensed
strings  uh  such  that  there
exist  these  Maps
um  uh  from
RT  to
St  satisfying  the  conditions
above  uh  then  R  to  S  is  a
map  R  to
s
so  this  is  nice  because  you  don't  have
to  explicitly  think  about  how  you'd
build  s  bracket  T  from  R  bracket  T
modules  if  you  just  have  the  maps  that
would  kind  of
indicate  uh  indicate  it  then  then  you
can  actually  do
it  um  let  me  maybe  give  a  hint  of  the
argument  it's  in  the  again  it's  in  the
notes  from  a  previous  iteration  of
this  so  a  hint  of  the  argument  would  be
that
so  note  that  well  would  be  to  note  that
so  so  a  hint  of  the  argument  is  that  uh
mod  R  is
monatic  over  condensed
sets  so  you  have  a  forgetful  functor
from  Mod  R  to  mod  R  triangle  which  in
turn  forgets  to  condense  sets  and  that
satisfies  the  hypothesis  of  barbec  it's
basically  a  localization  or  the  right
adjoint  of  a  localization  followed  by  a
forgetful  functor  and  so  so  this
category  can  be  understood  as  the
category  of  what  whatever  they  call  it
algebra  is  over  some  monad  here  which  is
kind  of  the  free  R  module
monad  and  then  if  you  want  to  show  that
every  s  module  is  an  R  module  it  would
be  enough  to  produce  a  map  of  monads
from  the  free  R  module  monad  to  the  free
s  module  monad  well  the  very  first  step
in  producing  such  a  map  of  monads  would
be  giving  the  map  from  the  free  object
here  to  the  map  of  the  free  object  here
and  then  there's  a  condition  you  need  to
check  compatibility  with  the  monad
structure  and  if  you  play  around  enough
you  can  reduce  the  what  you  need  to
check  to  just  uh  this  commutative
diagram
here
okay  uh  all
right  so  uh  so  let's  continue  discussing
fenus  so  now  what  are  we  reduced  to
so  we  need
to  uh  for  every  light  profite
set
uh  t  a  map  well  a  fenus  linear
map
um  uh  from  R  bracket  t  uh  to  R  bracket  t
uh  satisfying  well  it  should  be
functorial  in  T  and  it  should  satisfy
this  little  extra  condition  relating  to
when  you  take  a  map  from  T  to
R  okay
so  now  let  me  uh  recall  uh  just  in  pure
algebra  a  little  bit  about  the  fenus  so
recollection  on
fenus
um
so  fenus  is  a  map  of  rings  so  I  said
it's  a  map  of  rings  from  R  to  Rod  P  but
actually  in  some  form  it  exists  uh  for
an  arbitrary  ailan  group  so  if  m  is  an
ailan
group  uh  then  you  get  a  natural  map  from
M  uh  to  well  the  zero  Tate  chology  of  CP
acting  on  M  tensor  p  uh  so  p  is  a
prime  so  recall  that  this  is  the  uh
cernal  of  the  norm  map  or  the  transfer
map  maybe  is  a  better  name  for  it
from  M  tensor  P  CP  orbits  to  M  tensor  P
CP  fix
points
um  you  said  CP  specific  group  of  yes  uh
thank  you
yes  um  and  I'm  going  to  this  is  quite
long  notation  I'm  going  to  adopt  some
homotopy  Theory  abbrev  a  for  this  I'll
just  write  uh  pi0  of  the  CP  of  the  Tate
kology  written  in  the  style  of  hopy
theorist  so  it's  shorter  so
um  uh  and  what  is  this  map  induced  by
uh  uh  you  send  an  element  m  to  the
diagonal  tensor  M  tensor  m  p  times  which
lives  in  uh  M  tensor  p  and  it's  clearly
fixed  by  the  cylic  group  of  order  p  and
then  you  take  its  class
uh  in  the  take
chology  okay  and  the  point  is  that  of
course  this  expression  by  itself  is  not
additive  but  if  you  look  at  M  plus  where
M  plus  n  goes  you'll  have  a  bunch  of
cross  terms  and  you'll  be  able  to
recognize  that  each  of  those  cross  terms
uh  is  a  norm  uh
from  uh  from  M  tensor  P  so  it's  a  nice
exercise  if  you've  never  done  it
uh  oh
sorry  continue  over
here  so  I  should  say  you  have  a  it's  a
group  homomorphism  that's  the  special
thing  that  happens
here  um
and  what  is  the  relation  of  this
construction  with  the  fenus  on  a  ring  so
if  m  equal  R  is  a  commutative
ring  uh
then  then  you  have  this  thing  which  goes
from  R  to  R  tensor  P  Tate  CP  and  Pi  Z  of
that
um  but  then  you  can  uh  use  the
multiplication  map  from  R  tensor  P  to
R
uh  to  go  to  this  where  now  the  CP  action
is
Trivial  here  you're  using  that  the  ring
is  commutative  so  that  um  it's  a  CP
equivariant  map  multiplication  is  a  CP
equivariant  map  from  R  tensor  P  to
R  um  but  now  since  the  CP  action  is
Trivial  this  is  just  the  same  thing  as
uh  this  is  doing  nothing  and  then  the
norm  map  is  just  usually  it's  summing
over  the  action  of  the  group  but  the
group  is  Tri  else  you're  just  summing
over  P  copies  of  one  so  this  really  is
just  Rod
P  um  and  if  you  trace  through  this  is
the  frobenius  so  just  by  the  same
formula
so
okay
and  now  if  m  is  an  R
module  then  this  map  uh  this  so-called
fenus  map
from  uh  Tate
CP
um  um  this  is  a  map  of  a  bilon  groups  so
these  are  both  our
modules
uh  this  is  a  map  of  a  bilon
groups  not  a  map  of  our  modules  but  in
fact  it  is  a  map  of  our
modules  if  you  forbus  twist  uh  uh  five
twist  uh  on  the  right  hand
side  so  in  other  words  thisan  group
level  frobenius  is  kind  of  linear  over
the  the  ring  level  for  obus  um  in  the
appropriate
sense  so  what  this  suggests  is  so  now
we're  trying  to  look  for  some  uh  L  some
linear  algebra  uh  incarnation  of
frobenius  on  the  level  of  these  light
profite  sets  this  suggest  that  we  should
just  try  to  apply  this  uh  ailan  group
version  of  frus  and  and  see  what  happens
so  now  we  take  t  a  light  profinite
Set  uh  then  we  get
um  R  bracket  T  and  we  can  always  do  this
in
uh  so  now  we're  viewing  R  bracket  T  is
just  a  a  sheath  of  a  billion  groups  on
light  profinite  sets  and  we  can  apply
this  uh  construction  here  so  we  tensor
it  P
times
um  and  we  get  Tate
CP  and  then  we  take  Pi  0  so  that's  all
just  happening  at  the  level  of  a  sheaves
of  a  bilon  groups  so  so  that's  a  very
uncompleted  tensor  product  but  we  can
always  map  this  to  any  uh  further
completion  we  like  and  in  particular  we
can  map  it  to  uh  Pi  0  R  bracket  T  I
apologize  this  is  not  the  polinomial
ring  on  one  variable  this  is  the  free  R
module  on  a  profinite  set
T  um  to  tensor  over  R  uh  p  t
CP  but  you  have  this  uh  sort  of  so  to
speak  kunth  formula  so  the  tensor
products  of  the  free  modules  are  just
the  free  modules  on  the  product
um  and  then  CP  is  acting  here  as  well
Tate
CP
and  uh  we  wanted  to  make  a  map  from  RT
to  itself  which  is  forus  linear  uh  and
we've  gotten  to  RT  to  the  P  so  how  do  we
compare  them  well  we  can  put  RT  Maps
here  via  the  diagonal
embedding  and  that's  also  equivariant
for  the  CP  action  if  you  make  CP  act
trivially  here  so  you  get  a  map  like
this  and  the
claim  is  that  this  map  is  an
isomorphism  um  if  you  buy  that  then
you're  basically  done  so  CU  again  CP  is
acting  trivially  here  so  the  take
construction  uh  is  just  modding  out  by  P
so  this  is  indeed  Rod  P  bracket  T  and
then  for  similar  reasons  the  composite
is  actually  going  to  be  fenus  linear  um
and  gives  you  the  desired  construction
you  also  have  to  check  that  condition
but  the  full  argument  is  in  the  notes
analytic
PDF  what  I  want  to  highlight  is  is  um
what's  different  here  compared  to  last
time  this  was  discussed  by  Peter  in  a
previous
iteration
um  last  time  this  had  to  be  added  as
kind  of  an  axiom  on  our  analytic  rings
that  well  basically  that  this  argument
Works  which  reduces  to  this  claim  being
true  um  but  now  that  we  switched  to  the
light  profinite  set  setting  it's
actually  you  can  just  prove  it's  true  uh
for  arbitrary  R  so  that's  what  I  want  to
explain  now  um  so
to
finish  we  need  to
show  so  the
Lemma  so  let's  say
um  if  uh  so  if  no  s  if  s  is  a  light
profinite
Set  uh  with  CP
action  uh  then  the
map  um  R  bracket  uh  s  the  fix  points  the
CP  fixed  points  which  is  a  closed  subset
of  s  hence  also  a  light  profinite  set  um
to  our  bracket  s  um  induces  an
isomorphism  on  tap  chology
and  in  particular  on  take  chology  when
you  pass  to  Pi  0  so  just  take  chology
and  degree
Z
okay  so  this  is  going  to  use  some  facts
about  light  profinite  sets  which  Peter
mentions  mentioned  in  his  first  lecture
but  haven't  sort  of  well  they  have  well
they  have  been  used  for  example  for  this
projectivity  of  this  P  space  but
well  um  we're  going  to  more  directly  use
them  now  so  the  so  proof  is  so  the  first
claim  is  that
um  R  bracket
SCP  is  actually  mapping  injectivity  into
R  bracket
s  and  the  reason  is
um  any
inclusion  of  light  profinite
sets  has  a
retraction  this  was  a  fact  that  uh  Peter
proved  in  the  second  lecture  of  this
course  so  if  you  apply  it  to  the
inclusion  of  CP  and  S  you  find  there's  a
retraction  so  then  if  you  hit  it  with
any  funter  it'll  still  be  it  will  still
be
injective  um  so  what  does  this  imply
this  implies  by  the  long  exact  sequence
and  take  chology  this  means  it  suffices
to
show  that  if  you  take  r  braet  s  mod  R
bracket
SCP  uh  that  this  thing  has  Vanishing
take
Construction
um  and  the  second  claim  is  that  um
generally  if  you  have  t  inside  s  then  R
bracket  s  mod  R  bracket  t  uh  only
depends  on  the  locally  compact
space  locally  light
profinite  uh  space  which  is  the
complement  s  t  thank
you  it's  not  like  group  where  you  can
choose  yeah
okay  um  so  that's  not  a  precise  claim  so
what's  the  more  precise  version  of  the
claim  if  you  have  S  Prime  mapping  to  S
and  then  you  have  t  including  here  and
you  form  the  pullback  to  get  T  Prime  and
if  this  is  an  isomorphism  over  s  minus  t
so  you're  sort  of  blowing  up  T  so  to
speak  or  you're  choosing  a  different
compactification  of  s  minus  t
um  uh
then  r  braet  s  Prime  mod  r  braet  t  Prime
maps  isomorphically  to  R  bracket  s  r
bracket
T
and  and  this  holds
because  uh  this  Square
here  is  a  push
out  in  condensed
sets  which  is  a  little  a  little  uh  well
light  light  condensed  set  let's  say  a
little  exercise  you  can  see  using  the
definition  of  the  gro  de
topology
um  yeah  I  wanted  to  assume  it  was  a
pullback  yeah
so  um
okay  um  and  then  the  third  thing  is  that
um  if  let's  say  x  is  a  let's  say  Sigma
compact
uh  a  totally
disconnected  locally  compact  house
space  and  CP  acts  with  no  fixed  point
then  uh  X  is  actually  isomorphic  to  some
co-product  of  CP  many  copies  of  Y  or  CP
is  acting  uh  by  permuting  those
copies  uh  Sigma  compact  means  countable
Union  of
compacts  sorry  what  wase  S
Prime  you  can  choose  okay  yeah  you  it's
any  S  Prime  you  like  which  maps  to  S  by
a  map  which  is  an  isomorphism  over  this
open  subset  here  so  it's  only  it's  just
modifying  T  making  T  bigger
yeah  um
okay
um
uh  so  the  yeah  by  the  way  the  thing  is
if  you  have  a  a  light  profinite  set  and
you  remove  a  light  a  closed  subset  which
is  also  light  then  the  complement  is  one
of  these
guys  um  it's  going  to  be  Sigma  compact
because  of  the  lightness  basically
there's  a  countable  system  of
neighborhoods  of  any  closed  subset  and
then  it's  totally  disconnected  and
locally  compact  house  DWF
so
um
what  is  why  it  exists  there  exists  a  why
such  that  also  also  with  these
properties  such
that  so  I'm  saying  the  action  is  free  in
a  very  strong  sense  if  provided  the
action  has  no  fix  points  then  then  it's
free  in  a  very  strong  sense  it's  just
literally
induced
um  so  the  reason  is  that  uh  well  so  X  to
xod  CP  is  a  a  covering
map  um  that  just  uses  locally  compact
house  DWF  and  freeness  of  the  action  um
but  then  this  will  also  be  Sigma
compact  uh  locally  compact  house  DF  and
totally  disconnected  and  as  Peter
explained  uh  this  there's  an  easy
argument  that  this  implies  that  it's  a
disjoint  Union  of  countably  many  light
profinite
sets
um
and  what's
that  not  necessarily  the
GU  oh  I'm  sorry  yeah  oh  yeah  right  in
the  in  the  in  the  application  they  will
be  light  but  yeah  thank  you  I  forgot  I
should  have  maybe  said
locally  light  or  something  like
that  um  so  there  a  disjoint  countable
disjoint  Union  of  profite  sets  um  and
over  and  this  covering  map  will  have  to
be  trivial  over  each  of  these  profinite
sets  because  the  topology  is  totally
disconnected  so  locally  trivial  implies
globally  trivial  um  and  then  that
implies  this  because  the  the  Y's  or
Y  yeah  y  will  just  be  xod  CP  um  and  then
the  trivialization  of  the  cover  over
each  profinite  set  will  will  give  you
the  the  required  decomposition
there
okay  so  if  you  combine  two  and
three  what  do  you  deduce  you  deduce  that
this
um  r  s  mod  ah  you  deduced  that  R  in  that
setting  over  there  s  mod  R  CP  p  uh  is
actually  a  direct  sum  of  CP  many  copies
of  some  guy  X  or  CP  is  acting  on  this
set  here  oops
s  because  by  two  um  this  only  depends  on
the  light  profinite  set  which  is  the
complement  that's  one  of  these  guys  that
here  so  then  that  light  profinite  or
that  locally  light  profinite  set  is  um
yeah  is  a  disjoint  Union  of  CP  many
copies  of  some  guy  and  then  you  compact
a  in  a  different  way  by  just
compactifying  Y  and  then  taking  the  CP
many  disjoint  Union  copies  of  uh  of  that
to  get  a  compactification  of  X  and  then
use  that  to  calculate  the  quotient  here
and  then  you'll  find  that  the  the  module
is  is
induced  um  and  then  an  induced  module
has  Vanishing  Tate  chology
so
yeah
okay
so  this
um  this  uh  fact  that  fenus  is  a  map  of
analytic  Rings  it's  not  just  a  cute  fact
so  I'll  make  a  remark  I  don't  think  I'll
go  into  the  details
um
but  uh  this  this
theorem  on
fenus  implies  that
if
um  if  our  triangle  mod  r  or  let's  let's
do  it  in
the  in  this  setting
uh
well  a  derived  analytic
ring  or  a  pre-analytic
ring  uh
and  we
give  our  triangle  the  structure
of  an  animated  commutative
ring  uh  condensed
ring  which  gives  new  functors  s  derived
symmetric  powers  on  um  D  greater  than  or
equal  to  zero  our
triangle  which  are  the  things  that  are
used  to  build  free  animated
Rings  or
or  you  say  used  to  build  the  monad
describing  analytic  ring  or  sorry
animated
Rings  over  our
triangle  then  these  Simi  descend
to  uh  so  if  M  mapping  to  n  goes  to  an
isomorphism  in  D  greater  than  or  equal
to  r  d  greater  than  equal  0  R  so  this  is
a  map  in  D  greater  than  or  equal  0  R
triangle  then  so  it  is
Simi  uh  which  lets  you  define  the  Sim
eyes  on  the  derived  category  of  this
analytic
ring
so  this  uh  the  the  argument  for  this  can
also  be  found  in  analytic
PDF  for  this  deduction  from  the
statement  about  fbus  to  the  statement
about  symmetric  Powers  it  comes  from  a
calculation  of  what  doen  poopy  did  about
the  so-called  stable  um  stable  derived
functors  of  the  symmetric  power
functors
um  and  what  does  this  let  you  do  uh
this  lets  you  uh  do
normalization  or
completion  of  pre-analytic
rings  uh
also  in  the  animated
context  um  so  you  get  a  good
category
of  animated  analytic
rings  so  this  is  a  pair  consisting  of  uh
an
animated  commutative  uh  condensed  ring
light  condensed  ring  and  then  a  d
greater  than  or  equal  to  Zer
R  um  this  is  an  analytic  ring
structure  on  the  underlying  infinity
ring
say  or  you  could  just  specify  the  pi  not
level  thing  or  whatever  you
want  um  so  the  so  the  the  extra  uh
structure  you're  putting  on  here  is  just
that  this  gets  an  animated
structure  and  then  the  whole  discussion
Works  uh  just  fine
there
um
okay
so  uh  next  so  one  last
topic
um
um  so  maybe  I'll  I'll  call  it  a  killing
algebras  killing  algebra
objects  so  so  recall  for
motivation  that  we  presented  the  solid  Z
Theory
uh  as  uh  the  analytic  ring  structure  on
uh  on  the  integers  where  M  was  in  solid
z  uh  if  and  only  if  U  when  you  take
internal  H  from  P  to  M  and  then  you
apply  uh  this  uh  shift  minus  one  uh  to
internal  home  from  P  to  M  that  this
should  be  an
isomorphism
uh
but  this  is  equivalent  to  saying  that
say  ROM  um  from  a  to  m  equals  z
where  a  is  equal  to  P  modulo  T  minus  one
a  priori  I  should  say  derived  modulo  T
minus  one  but  actually  it's  easy  to  see
that  multiplication  by  T  minus1  is
injective
um  and  in  fact  it's  not  going  to  matter
either  way  um  as  in  the  class  of  module
satisfying  this  doesn't  really  change
um  so  it's  the  things  that  and  and  this
has  an  algebra
structure
uh  in  the  ambient  category  um  say  d  d
DZ
condensed  because  uh  P  itself  was  a
commutative  ring  object  via  addition  on
natural  numbers  and  then  we're  just
modding  out  by  a  by  an  element  of  that
ring  T  minus
one  okay
um  I'm  sorry  goes  to
isomorphism  I'm  sorry
yeah  so  if  this  is  a  map  which  goes  to
an  isomorphism  in  there  then  it
symmetric  power  also  goes  to  an
isomorphism  there  that's  the  condition
you  need  to  extend  this  symmetric  power
functor  from  this  thing  to  that
localization  of  it  D  greater  than  or
equal  to  z
r  what  would  do  back  goes  to  oh  you  take
completion  it  yeah  exactly  yeah  under
the  goes  to  under  under  the  localization
functor
yeah
um
so
um  so  General
context  so  oh  maybe  I  should  say  yeah  so
General
context  so  let's  say  C  tensor  is  a
symmetric
monoidal  uh  let's  say
presentable  uh  stable  Infinity
category  and  yeah  presentable  let  me
include  that  tensor  product  commutes
with  co-  limits  in  each
variable  um  and  let's  say  A  and  C  is  an
algebra
object  and  it  really  I  only  need  it  in
some  kind  of  weak  sense  so  let's  say
that  we  have  a  multiplication  map
literally  just  a  t  or  a  goes  to  a  um  we
have  a  unit  so  there's  the  unit  object
in  the  symmetric  monoidal  Cate  category
and  then  we  require  that  like  the
multiplication  is  either  left  or  right
unal  so
um  let's  say
one  so
Unit  t  for  a  and  then
multiplication  this  is
isomorphism
um
so  you  said  you  don't  require  any
associativity  or  no  nothing  yeah  yeah
it's  really  really
weak  okay  um  then  the  so  then  we  can
Define
uh  so  D  subset  c  uh  to  be  the  full
subcategory  of  those  M  and  C  such  that
internal  H  which  implicitly  is  an  ROM
here  uh  from  a  to  m  equals
zero  so  we're  we're  killing  a  we're
declaring  that  a  should  be  equal  to  zero
um  but  also  in  an  internal  home
sense  um  and  then  the
goal  will  be  in  in  favorable
situations  uh  give  a
formula  for  the  left  ad  joint  to  the
inclusion  to  the
inclusion  sorry  is  K  the  name  of  the
mathematician  or  just  killing
kill  killing  killing  killing  means  uh
not  not  the  mathematician  responsible
for  the  killing  I'm  that  this  kind  of
seem  probably  he  didn't  do  it  yeah  yeah
um
okay
um  so  maybe  I  should  give  some  examples
um
well  there  was  the  solid  example  that  I
just  described
so  there  was  also  solid
ZT  uh  which  was  also  obtained  by  killing
some  endomorphism  of  or  requiring  some
endomorphism  of  P  to  go  to  an
isomorphism  but  that  endomorphism  of  P
was  also  just  given  by  multiplication  by
some  element  with  respect  to  the  ring
structure  on  P  so  it's  the  same  thing  as
killing  the  the  cofiber  just  like  this
um  there's  also  kind  of  pure  algebra
examples  so  for  example  if  you  take  if
you're  in  usual  derived  category  D  of  R
and  then  you  look  at  R  mod
f  um  so  that's  Cals  D  of  R  um  then  what
is  D  well  D  is  uh  D  of  R1  over  F  so  it's
kind  of  inverting  f  in  some  algebraic
sense  is  an  example  of
this  um  another  thing  you  could  do  is
Cal  D  of  R  um  then  you  could  take  this
algebra  instead  R  bracket  f  inverse  in  D
of
R  um  then  what  is  d  d  is
the  uh  sort  of  like  f  complete  derived
subcategory  and  we're  looking  for  a
formula  for  the  derived  F
completion
okay
uh
um
so
right  so  let's  uh  so  Define
a  Define  a  functor  f  um  from  C  to  C  uh
by  the  following  so  uh  F
ofx  uh  is  internal
H  uh  from
so  uh  so  let  me  write  uh  so  sorry  so  let
me  write  C  for  the  fiber  of  the  unit
mapping  to  a
um  and  then  I'll  Define  this  functor  F
by  f  ofx  is  internal  hum  from  C  to
X  um  and  there's  a  natural
transformation  from  X  to  F  ofx  because
this  is  the  same  thing  as  internal  home
from  one  to
X  and  uh  by  construction  c  maps  to  one
so  you  get  a  map  in  the  other  way  on  the
internal
home
so  I  want  to  claim  that  this  is  a  first
step  towards  constructing  the  required
localization  so  claim
um
if  uh  m  is  in  D  then
uh  p  uh  to
M  applied  to  this  map  is  an
isomorphism
so  Hing  to  M  living  in  D  doesn't  see  the
difference  between  X  and  F  ofx
um  okay
um  so  for  the
proof  so  it  suffices  to  see
uh  that  um  the
fiber  of  X  maps  to  F
ofx  uh  is  an  a
module  the  reason  for  that  is  to  show
that  Hing  out  of  this  from  this  map  to  M
being  an  isomorphism  that's  the  same
thing  as  is  saying  that  HS  from  the
fiber  to  M  should  be
zero
um  but  um  by  definition  internal  HS  from
uh  anything  in  a  to  m  is
zero  uh  therefore  so  if  it's  an  a  module
then  it's  actually  a  unidol  yeah  maybe  I
should  say  it's  a  retract  of  a  tensor
itself  so  any  a  module  is  a  retract  of
itself  tensor  a  just  by  the  unity  axum
um  and  then  because  internal  HS  from  a
to  M  are  zero  that  implies  that  HS  any
kind  of  HS  from  A10  or  anything  to  m  is
zero  and  then  you  get  the  same  for  any
retract  and
therefore  uh  you  get  the  required
claim
um  uh  so  now  why  is  this  fiber  an  a
module  but  this  fiber  is  just  up  to
shift  maybe  it  is  just  internal  home
from  a  uh
to  X  because  a  is  the  difference  between
one  and  C  so  up  to  a  shift  the  fiber
will  be  internal  home  from  aex  and  this
has  an  a  module  structure  because  you
can  so  to  speak  act  by  a
here
okay
okay  so  now  now  now  iterate  f
in  the  following  way  so  you  get  X  going
to  f
ofx  and  then  you'll  go  to  F  of  f
ofx  um  and  one  should  actually
potentially  be  a  bit  careful  about  which
map  one  writes  down
here  um  so  what  I  want  to  do  is  I  want
to  take  uh  F  applied  to  the  previous
map  as  opposed  to  um
taking  the  instance  of  the  previous  map
with  X  replaced  by  F
ofx
okay  um  and  then  you  can  continue  on
like  this  F  of  f  of  f
ofx  um  and  then  I  claim  that
so  uh  so  toine  F
Infinity  to  be  the  co  limit  of  this
here  then  I  claim  that  I'll  do  it  over
here  so
claim  so  if
either  uh  one  this  uh  Co  limit
stabilizes
uh  for  all  X  so  if  for  example
the  every  map  from  here  onward  is  an
isomorphism
um
or
to  uh  internal  H  out  of  a  uh  from  as  a
functor  from  C  to  C  commutes  with  Co
limits  or  really  maybe  only  needs
sequential  Co
limits  uh  then
uh  X  goes  to  F  infinity  x  is  the  left  ad
joint  uh  to  the  inclusion  D  inside
C  so  for  the  proof
um  you  have  to  check  two  things  one  you
have  to  check  that  um  if  you  map  from  X
to  anything  in  D  that's  the  same  thing
as  mapping  from  F  infinity  x  to  anything
in  D  but
um  we  just  proved  that  claim  for  X  going
to  f
ofx  um  and  this  thing  is  given  by
hitting  uh  an  instance  of  that  map  with
the  funter  F  but  um  the  functor  f  is
going  to  preserve  the  property  used  in
the  proof  here  that  the  fiber  um  is  an  a
module  because  internal  H
uh  to  an  a  module  will  still  be  an  a
module  so  the  exact  same  argument  shows
that  each  of  these  Maps  also  satisfies
the  same  property  that  internal  humming
out  of  them  uh  doesn't  see  the
difference  between  one  guy  and  the  next
and  then  you  have  a  co-limit  and
internal  hming  out  of  that  is  just  an
inverse  limit  of  internal  humming  out  of
all  the  other  ones  or  humming  I  should
just  say  uh  so  that  uh  formally  just
passes  through  to  the  co-  limit  so  uh
just
as  as
for  uh  X  goes  to  F  ofx  each  F  to  the  N
going  to  f  n  +1  x  is  ISO  on  cming  out  to
D  and  then  so  is  X  goes  to  F
infinityx  uh  and  then  the  next  thing  you
need  to  show
is
um  uh  the  next  thing  you  need  to  show  is
that  um  this  guy  actually  it  lies  in  D
so  that's  that's  and  the  second  thing
you  need  to  show  is  that  F  infinityx
lies  in  D  but  to  check  that  you  need  to
check  that  by  definition  of  d  you  need
to  check  that  internal  hming  uh  out  of  a
into  F  ofx  gives  you  the  same  thing  but
by  construction  internal  hming  just
shifts  you  one  along  in  the  sequence  and
so  the  co  limit  is  going  to  be  the
same
so  by
construction
Okay  so
so  in  principle  for  example  uh  you  can
just  you  could  at  least  attempt  to  use
this  formula  to  compute
solidification  it's  all  actually  quite
explicit  this  this  in  that  case  this
this  huming  out  of  C  is  just
uh
yeah  last  thing  I  meaning  a  is  not  the
same
as  oh  oh  oh  oh  oh  oh  oh  wait
what  uh
uh  oh  for  yeah  there's  two  there's  two
things  I  yeah  there's  two  things  I
skipped  here  sorry  the  first  thing  I
skipped  was  pulling  the  co-limit  in  and
that's  where  I  would  use  hypothesis  one
or  two  to  to  be  able  to  pull  the  Homing
from  a  into  there  into  the  co-limit  um
and  then  the  second  thing  as  Peter  says
I  got  the  wrong  explanation  so
um  yeah  so  you  have  to
be
um
wait  now  I'm
blanking  uh  can  you  help  me
Peter  I  think  the  arent  should  be  that
uh  so  you  want  to  know  that  a  becomes
zero  and  the  arum  should  be
that  all  the  transitions  just  become
zero  ah  right  okay
uh  yeah
okay  right  okay  so
sorry  I  saw  I  was  actually
using  like  there  was  question  which  map
to  use  after  the
first  to  use  the  other  okay  well  when  I
when  I  prepared  this  I  wanted  to  use
this  map
um  I  want
you  from  a
to  just  becom  zero  and  you
can  yeah  um
um  internal  home  from
a  yes  but  um  but  the  same  but  you  can
use  this  I
mean  but  that  that  also  works  here
because  the  internal  home  from  a  to  x
commutes  with  the  operation  of  um
forming  F  like  applying  F
so  yeah  both  okay  okay  yeah  so  sorry  let
me  try  uh  so  so  for  this  you
check
that  internal  h  a  uh  to
each  uh
FN  FN  +  one  is  homotopic  to  the  zero  map
okay  yeah  I'm  sorry  I  I  kind
of
okay  um  right  so  yeah  so  in  in  the  case
of  solid  you  can  really  just  make  this
completely  explicit  and  this  is  some
this  operation  f  ofx  is  some  quotient  of
null  sequences  in  X  modulo  null
sequences  in  X  via  some  map  um  and  it's
basically  this  F  ofx  is  basically  you
freely  you  look  at  all  null  sequences
and  you  you  mod  out  by  their  relations
that  would  say  that  then  intuitively  you
want  to  map  that  to  M  by  by  summing  them
and  you  what  you're  modding  out  by  is
just  the  telescoping  sums  and  so  you  can
kind  of  try  to  analyze  what  happens  to
that  and  iterating  it  and  so  so  on  to
get  a  an  expression  for  the
solidification  of  a  general  module  um
it's  also  fun  to  try  to  unwind  it  in
these  cases  and  see  that  you  recover  the
classical  formulas  so  like  for
example  uh  so  here
The  Well  of  course  it's  M1  over  F  but
you  can  rewrite  that  as  uh  there's  this
wellknown
as  a  as  a  sequential  Co  limit  right  and
so  I  claim  that  that's  exactly  the  same
as  this  sequential  co-  limit  um  and  here
you  have  this  formula
um
so
uh  here  it  it  looks  different  because
it's  an  inverse  limit  um  but  what's
actually  happening  here  is  that  this
inverse  limit  is  what  you  just  see  as
the  first  at  the  first  term  and  then  all
the  maps  after  that  are  isomorphisms  so
this  this  is  just  the  the  first
iteration  of  f  ofx  f  of  f  of  M  in  this
setting  so  I  don't  know  so  it's  a  yeah
it's  fun  to  to  use  this  formula  as  to  to
understand  left  the
joints
um  maybe  I  should  also  say  that  this
this  situation
occurs
um  so  EG  if  a  is  item
potent  yeah  so  that's  what  what's  going
on  there  I'm  sorry  but  it  feels  that  in
two  and  three  you  are  ENT  in  two  no  not
in  two
so  this  is  not  this  is  not  item  poent
okay  this  is  why  you  okay  yeah  so  but  if
you  took  the  completion  no  it  wouldn't
no  yeah  in  solid  solid  would  then  but
then  you'd  be  doing  something  else  you
wouldn't  be  doing  the  algebraic
inverting  F  you'd  be  doing  the  other
thing  yeah  so  that's  yeah  okay  uh  I  went
a  little  over  time  I  apologize  for  that
um  thank  you  and  see  you  next
week
\end{unfinished}