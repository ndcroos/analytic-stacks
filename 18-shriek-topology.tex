% !TeX root = AnalyticStacks.tex

\section{\ufs !-topology (Clausen)}

\url{https://www.youtube.com/watch?v=vRUmXU8ijIk&list=PLx5f8IelFRgGmu6gmL-Kf_Rl_6Mm7juZO}
\renewcommand{\yt}[2]{\href{https://www.youtube.com/watch?v=vRUmXU8ijIk&list=PLx5f8IelFRgGmu6gmL-Kf_Rl_6Mm7juZO&t=#1}{#2}}
\vspace{1em}

\begin{unfinished}{0:00}
  e  um  let's  begin
so  be  talking  more  about  this  shriek
topology  um  but  it's  been  a  little  while
since  we've  last  met  and  so  maybe  I'll
start  with  a
recap  um
so  so
recall  well  let  me  start  at  the  very
beginning  so  so  we  have  this  category
analytic
Rings  um
and  objects  in  here  are
pairs
um  there's  different  choices  about
exactly  what  sort  of  data  you  want  to
put  in  the  second  piece
um  um  I'll  take  the  sort  of  full  derived
category  instead  of  some  connective
derived  category  or  some  ailan  category
um  where  this  is  a
condensed  um  and  in  general  we  want  to
say  animated  ring  so  we're  allowed  to
have  some  derived
phenomena  and  this  is  a  a  certain  full
subcategory
of  uh  what  you  could  call  D  of  R
triangle  which  is
um
uh  are  triangle  modules  in  derived
condensed  a  bilan  groups
um  satisfying
certain  nice  closure  axioms  for  for  this
uh  inside  there  which  I  won't  uh  recall
right
now
um  and  then  we  Define  sort  of  the
category  of  apine  spaces  to  just  be  the
opposite
category
um  and
and  we  singled  out  two
um  well  three  I  guess  classes  of
morphisms  of  of  apine  spaces  so
um  f  from  X  to  Y  in
F
is  so
proper
if  um  so  the  pullback  map
uh  from  Dy  to  DX  so  let  me  Define  d  on
the  level  of  f  by  composing  with  this
equivalence  and  taking  the  D  on  the  ring
there  um  if  this  has  a  good  R
joint  um  fow  star  where  good  means  that
right  a  joint
um  commutes  with  pullbacks  so  if  you  do
the  right  ad  joint  here  and  you  pull
back  to  some  D  of  X  Prime  that's  the
same  as  pulling  back  to  D  of  Y  Prime  and
then  doing  the  right  ad  joint  there  um
but  it  also  means  that  it  commutes  with
co-  limits  and  it  also  means  that  it
commutes  with  sort  of  the  DX  action  by
tensor  product  on  on  each  of  these  guys
here  I  a  Dy  sorry  ah  I  think  I  I  I
messed  it  up  because  this  is  sorry  you
have  sorry  we  have  y  Prime  well  yeah
right  good  means  commutes
withes  with
pullback  yeah  commutes  with  co-
limits  and  satisfies  projection  formula
yeah  well  it's  not  exactly  the  case  the
for  f  in  the  other  direction  it  is  well
okay  so  there's  a  map  in  One  Direction
and  you  ask  it  to  be  an  isomorphism  I
mean  I  don't  know  so  you  you  T  with
something  in  D  of  the  D  of  Y  yeah  the
base  yeah  uh  which  is
then  uh  pull  backed  into  D  of  X  by  the
by  the  fun  and  then  the
F
uh  ah  okay  the  right  okay  so  this  is
projection  ex  it  is  like  the  projection
for  Whata  okay
yeah
um  and  the  pullback  the  kind  of  pullback
I  mean  is  like  XY  and  then  y  Prime  and
then  X  Prime  just
to
um  okay  and  that  was
the  basically  the  same  thing  as  saying
the  the  analytic  ring  structure  is  not
really  changing  when  you  go  from  y  to  X
it's  just  the  the  underlying  ring  that's
changing  so  you  inher  inherit  the
analytic  ring  structure  from  Y  when  you
go  from  y  to  X  so  that's  the  same  thing
is  uh  the  induced  analytic  ring
structure  um  it  it's  called  an  open
immersion  induced  well  on  on  on  X  uh
right  so  you  if  so  this  is  implicitly
given  by  a  map  of  analytic  rings  from
like  uh  I  don't  know  uh  let's  say  s  to
R  so  you  have  you're  given  an  analytic
ring  structure  structure  here  and  you
want  the  analytic  ring  structure  on  here
to  be  just  inherited  from  this  one  in
the  sense  that  the  D  of  R  is  just  the
same  thing  as  R  triangle  modules  in  D  of
s  so  D  of  R  is  also  the  subset  of  D  of  R
triangle  well  it's  it's  it's  objects
after  the  morphism  are  those  which  when
you  restrict  to  S  are  yes  okay  exactly
yeah  so  it  really  so  it  looks  basically
like  an  apine  map  in  usual  algebraic
geometry  kind  of  but  we  call  it  proper
because  of
this  this  uh  this  situation  here  um  so
it's  an  open  immersion
if  if  F  upper
star  uh  D  of  Y  to  D
ofx  is  a
localization  uh  and  has  a  good  left  ad
joint
um  where  good  is  taken  in  the  exact  same
sense  as
before  um  the  maps  maybe  Go  in  different
directions  because  it's  a  left  adjoint
instead  of  a  right  adjoint  but  you  can
still  always  ask  them  to  be
isomorphisms
um  and  this  was  equivalent  to
uh  so  uh  Lo  localize  away  from
uh  uh  some  item  potent
algebra  uh  in  D  of
Y  so  such  a  local  such  a  open  immersion
for  such  an  open  immersion  there's
always  going  to  be  some  item  potent
algebra  in  D  of  Y  which  is  somehow
functions  on  the  closed  complement  uh
which  completely  determines  this  this
situation  here  it's  the  the  the  kernel
of  this  localization  will  just  be  the
modules  over  that  item  potent  algebra  so
let's  say  if  this  is  a  so  then  so  the
sense  of  this  is  that  kernel  of  star  is
just  mod
a  d  of
Y
um  okay  and  and  then  it  was
called  uh
shable  if  it
factors  as  uh  X  xar  Y  um  where  this  is
proper  and  this  is  an  open
immersion
um  and  then  um  there  are  good  closure
properties  for  all  three
classes  of
maps  so  closed  under  composition  closed
under  base  change  um  contains  all
isomorphisms  um  and  uh  kind  of  half  of
the  two  out  of  three  or  2third  of  the
two  out  of  three  properties  so  I  mean  so
if  you  have  a  map  like  this  and  a  map
like  this  and  they're  both  shakable  then
any  map  between  them  making  the  triangle
commute  is  also
shable
um
uh  and  then
um  so  we  and  then  we  get  a  six  functor
formalism  on
F  where  the  class  of
maps  for
which  the  the  next  the  the  important
functor  the  lower  shriek
functor  for  which  this  is
defined  is
exactly  uh  the  class  of  shable
Max
um  and
then  uh  F  lower  shriek  is  f  lower  star
for  f
proper  and  F  lower  shriek  equals  left
adjoint  to  F  lower  star  if  f  is  an  open
immersion  or  another  way  of  saying  this
is  that  F  uper  shriek  equals  F  uper  star
so  that's  just  um  few  past  to  wred
joints  I  think  yeah  should
be  um  a  map  from  to  star  in  general  or
um  yes  there  is  so
this  can  be  gotten  uh  because
the
um
oh  wait  just  a  sec  No  Maybe  not  maybe
not  exactly  in  this  full
generality  uh  oh  wait
uh  no  no  there  is  there  is  yeah  yeah  I
think  it  even  just  follows
from  uh  so  you  have  such  a  map  for  an
open
immersion  uh  because  the  it's  for  for
any  localization  you  have  that  kind  of
thing  where  like  the  left  ad  joint  maps
by  a  natural  transformation  to  the  right
ad  joint  and  then  you  can  just
compose  um  and  you  have  the  Cho
independent  of  the  factorization  yeah
there's  a  better  way  to  the  higher  no  no
no  yeah  there's  a  better  way  to  produce
it  which  involves  using  the  diagonal  so
uh
the  when  you  if  you  if  you  pass  to  the
diagonal  it's  enough  to  show  that  you
have  such  a  thing  for  the  diagonal  um
well  I  think  well  we'll  get  into  this  a
little  bit  more  later  I  I  was  I  missed
that  with  when  I  was  aw  because  to  the
the  closure  properties  are  are
composition  and  what  else  and  oh  a
composition  and  base  change  they  contain
all  each  of  them  contains  all
isomorphisms  and  then  there's  this
property  that  if  you  have  two  maps  with
the  same  Target  which  are  in  one  of
those  classes  then  any  map  between  them
is  also  uh  any  map  between  two  such  guys
is  is  also  in  the  yeah  is  also  in  the
same  class  or  I  could  say  that  the
classes  are  closed  under  passage  to
diagonals
um  uh  ah  this  is  equivalent  yeah  yeah
yeah  once  you  have  closure  under  base
change  I  think  yeah  I  think  this  is  some
basic  thing  okay  passes  to
dional  ah  but  this  is
intuitively  ah  intuitively  for  an  open
imion  usual  geometry  the  Diagon  is  isor
also  here  yes  also  yes  also  here  yeah
yeah  so  that's  part  of  why  uh  yeah  so
that  that  goes  into  the  other  way  of
producing  the  natural  transformation  but
never  mind
um  yeah  okay
so  uh  right  and  then  we  had  this
definition  we  had  this  result  um  so  we
had  this
theorem  or  maybe  I  should  say  definition
um  I'll
say
uh  a
map  FX  to  Y  let's  say  a  shable
map  in
F  is  a  shriek
cover
um  if
uh
um
so
uh
so  if
uh  the  or  so  if  if  the  map  from  D  of  Y
to  the  sorry  to  the  limit  of  uh
oops
um  to  the  limit  of  the  check  nerve  is  an
equivalence  where  for  the  transition
functors  we  use  the  upper  shriek  functor
uh
um  which  is  a  bit  of  a  a  strange  thing
to  do  at  first
glance  uh  but  then  we  had  this  result
that  so
for  uh  shable
F  uh  the  following  are  equivalent  so  uh
one  is  that  f  is  a  shriek
cover  so  in  that  sense  two  is  that  um  we
have  kind  of  shet  codent  so  if  you  take
the  co  limit  over  n  and  Delta  op  of  D  of
x  s  YN
um  so  then  so  then  here  you  use  the
lower  streek  functors  and  take  the  co
limit  in  this  category  PRL
um  or  or  in  uh  modules  over  D  of  Y
PRL  um  then  you  get  a  an  equivalence  so
you  had  or
in  so  PRL  is  is  what  you  said  that  L
introdu  is
is
uh  is  a  uh
uh  presentable  Infinity  category  with  Co
liit  preserving  functors  yes  okay  but
not  a  stable  any  not  no  but  I  could  also
say  oh  darn  but  the  co
limits  yeah  the  co  limits  are  going  to
be  the  same  so  it  doesn't  matter  okay
and  and  then  mode  d  ofy  means  PRL  means
it  means  yeah  so  D  of  Y  is  a  is  a  symet
monoidal  category  and  it's  in  fact  a
presentably  symmetric  monoidal  category
so  the  tensor  product  commutes  with
colimits  in  each  variable  which  makes  it
a  commutative  algebra  object  in  PRL  with
respect  to  L's  tensor  product  and  then
you  consider  modules  so  it's  a  it's  a  a
presentable  Infinity  category  which  is
tensored  over  D  of  Y  in  a  kind  of  cimit
preserving  way
um
so
um  and  then  the
third  condition  was  so  shriek  cover
meant  you  have  this  shriek  descent
descent  where  you  use  this  funny  Twisted
pullback  um  but  uh  it  turns  out  that  it
implies  that  you  get  descent  with  the
star  pullback  it  well  and  in  fact  it
implies  you  get  the  same  result  with
coefficients  so  for  all  m  in  mod  D  of  Y
P  RL
um  so
here
um  so  if  you  take  M  to  be  D  of  Y  itself
then  you  see  that  this  condition  is  just
kind  of  usual  descent  that  is  descent
with  respect  to  pullbacks  Star  pullbacks
but  in  fact  you  can  even  tensor  that
with  any
uh  any  module  and  you  still  have  descent
um  and  then  the  fourth  condition  was
this  kind  of  two  descent  this
categorified  version  uh  which  says
that  the  whole  category  of  possible  M's
uh  satisfies  descent  so  and  here  there's
only  the  only  thing  that  makes  sense  is
star  descent  but  I'll  emphasize  it  it's
just  some  naive
pullback
um
so  where  the  base  change  functors  are
just  kind  of  relative  tensor  products  in
PRL  so  one  can  ask  I  don't  know  if  it  is
a  good  question  but  instead  of  just  like
in  algebraic  geometry  usually  they
Define  like  fpqc  topology  using  quasi
compact  but  you  can  also  ask  suppose  you
have  an  arbitary  collection  of  maps  not
necessarily  finite  you  can  ask  whether
you  have  descent  for  this  can  it  happen
with  in  the  infinite  which  is  not
reduced  to  a  I  we  think  uh  maybe  I  don't
I  don't  know  if  Peter  checked  the
details  we  think  that  is  true  that  so
you  could  try  to  make  an  analogous  we
haven't  talked  about  the  gro  de  topology
yet  so  but  we're  making  it  finitary  by
definition  but  you  could  ask  whether  if
you  do  the  infinitary  analog  of  it
whether  it  reduces  it  gives  you  the  same
Gro  deque  topology  as  if  as  whether  that
Gro  de  topology  a  prior  infinitary  is
actually
finitary  um  and  we  think  yeah  so  Peter
you  checked  it  you  checked  it
carefully  yeah  yeah  okay  so  then  the
answer  is  yes  the  if  you  you  could  try
to  make  an  infinitary  version  of  the
groi  topology  I'm  about  to  describe  and
it  will  end  up  being  finitary  anyway  so
um  equivalent  this  yes  yes  so  every
cover  will  have  a
finite
subcover
um  the
basic  so  I  remember  that  in  in  the  in
the  case  of  just  what  we  started  finite
sets  if  you  have  the  counter  set  and  you
take  all  mths  from  P  to  the  counter  set
which  is  a
big  then  this  is  universally  submers
still  one  can  show  it's  universally
submersive  what  does  that  mean  to  there
is  this  like  in  the  H  topology  in  the
usual  voty  thing  with  the  H  anyway
anyway  it  means  when  you  cross  with  any
topological  space  to  check  if  something
is  open  you  can  check  the  pull  back  oh
right  right  right  right  right  yeah  so  I
don't  know  then  of  course  you  can  ask
whether  you  have
the  right  this  is  not  so  clear  I  mean
but  at  least  for  opens  it  one  can  check
it  I  mean  even  this  Universal  s  so  this
this  is  a  kind  of  candidate  for
infinitary  thing  but  I  don't  know  if  it
it  satisfies  the  chological  thing  no  I
think  uh  Peter  you  were  uh  you  you  asked
these  logicians  about  this
right  uh  yes  yeah  and  they  they  had  some
conclusion  so  so  it  can  happen  that  you
have  some  profin  set  and  some  infitary
cover  by  other  profin  sets  which
satisfies  the
S  at  least  some  example
desent  in  of  the
derived  the  street  descent  because  kind
of  full  condition  equivalently  along  the
lower
streets  using  this  perspective  you  can
show  that  it  is  fin  but  it's  really
something  that  only  works  when  you  ask
for  Street  not  for  St  so  for  stardent  it
is  it  is  possible  like  the  example  that
I  gave  of  all  all  Ms  from  P  to  the  C
said  Is  it  a
ER  well  maybe  it's  a  side  question  I
don't  know  I  think  the  specific  one  they
looked  at  was  like  you  take  a  really  big
profinite  set  and  then  you  take  a  union
of  like  small  ones  a  big  write  it  as  a
big  Union  of  of  small  ones  like  I  think
that's  the  specific  one  they  looked  at  I
don't  know  if  they  looked  at  this  one
with  P  in  the  caner
set  but  uh
yeah
um  so  uh  okay  I  think  uh  we'll  move  on
so
um  ah  right  so  yes  so  those  are  the
equivalent  conditions  for  being  a  shriek
cover  um  in  particular  you  have  some
very  nice  descent  properties  for  shriek
covers  um  but  now  I  want  the  next  thing
I  want  to  talk  about  is  how  can  you
check  these
conditions
um
um  so  here's  a  little
proposition  so  well  the  first  part  I'll
say  so  let's  say  let's  start  with  f  from
X  to  Y  shable
map  the  first  claim  is  that  if  f  is  a
shriek
cover  then  um  a  certain  condition  has  to
hold  so  um
so  if  you  look  at  the  image  of  the  lower
shriek
functor  from  D  of  Y  uh  D  of  x  to  D  of
Y  so  that's  a  certain  um  full
subcategory  of  here  um  since  this
functor  is  not  in  general  fully  faithful
that  that  image  doesn't  have  to  have  any
closure  properties  whatsoever  it  doesn't
have  to  be  a  stable  subcategory  or
anything  like  that  but  then  you  can  take
the  um  so  thi  thick  subcategory
generated  by  it  this  means
a  um  close
under  under  just  finitary  operations
like  cones  and  retracts  shifts  and
retracts  ah  so  and  uh  I  I  didn't  okay
yeah  and  retracts  I  count  that  as
finitary
um  um  ah  then  then  you  get  an  A  prior
bigger  category  you  can  ask  whether  it's
the  whole  thing  or  well  you  can  ask  it's
actually  equivalent  you  can  ask  whether
it  just  contains  the  unit  so  let  me  I'm
writing  one  y  for  maybe  you  would
normally  think  of  this  as  like  structure
sheif  of  Y  I  mean  the  the  unit  object  in
this  symmetric  monoidal
category  uh  with  respect  to  its  tensor
product
um  so  the  condition  again  is  that  this
unit  object  can  be  written  in  a  finitary
manner  in  terms  of  things  which  are
lower  shriek  from  d  ofx  and  can  it  be
written  in  an  infinitary  manner  an
infinitary  man  yes  because  of  the
because  it's  a  shover  it  follows  or  not
I
um  oh  you're  saying  well  so  so  you're
saying  if  I  say  uh  this  condition  but
with  infinitary  does  it  automatically
imply  the  same  condition  but  with
finitary  don't  know  what  is  the  relation
between  those  yeah  so  I  think  um
probably  I  mean  this  is  a  one  maybe  the
key  point  in  all  this  finitary  is  that
this  is  a  compact  object  in  D  ofy  so  I
think  uh  I  think  indeed
uh  uh  should  be
equivalent
um  yeah  so  maybe  maybe  it's  even  better
to  s  or  is  maybe  it's  I  don't  know  uh
let
me  yeah  I  guess  I  guess  it  should  be
equivalent  um
yeah
uh  sorry
okay  oh  oh  yes  yes  so  if  if  it's  a
street  cover  then  you  satisfy  this
condition  and  then  this  the  second  claim
is  that  the  converse
holds
provided  uh  f  is
either  um  and  then  first  will  be  the  uh
proper  provided  well  let  me  say  provided
either  uh  one  is  that  f  is
proper
and  uh
B  uh  f  is  I  guess  what  you  could  call
universally  locally  a
cyclic  um  let  me  explain  what  that  means
um  so  that  means  that
uh  F  uper  shriek  is
good
um  so  again  where  again  good  means
uh  commutes  with
pullback  uh  commutes  with  Co
limits  and  then  well  projection
formula  but  really  the  way  to  think
about  the  projection  formula  is  it's
giving  some  D  of  Y  linearity  and  it's
that  condition  that  you  want  to  ask  so
which  translates  into  saying  that  F
upper  shriek  should  always  be  the  same
as  F  upper  streek  of  the  unit  uh  tensor
up  our
star
so  T  are  in  D  of  X  so  there's  always
going  to  be  a  natural  map  and  you  want
it  to  be  an
isomorphism
um
yeah
okay
so  all  those  properties  of
f  are  not  true  in
general  and  they're  in  equivalent  or
they  when  you  say  good  is  the
conjunction  of  all  of  these  and  all  of
them
are  they're  cases  where  some  of  them  are
hold  and  others  don't  that's  a  good
question  there  may  be  some  non-trivial
implications  uh  maybe  if
you  yeah  there  may  be  some  non-trivial
implications  here  uh  I  think  yeah
I  yeah  that's  true  that's  a  that's  a
trivial  implication  yeah  thanks  yeah
projection  formula
and  Peter  is  saying  the  projection
formula  clearly  implies  that  it  commutes
with  colimits  so  I  didn't  I  didn't  need
to  write  the  second  one  but  okay  but
yeah
okay  so  I  mean  what  how  to  think  about
this  Ula  condition  I  mean
um  this  is  kind  of  like  the  dualizing
object  here  and  the  main  claim  is  that
the  dualizing  object  is  compatible  with
pullback  so  if  you're  thinking  of  f  from
X  to  Y  some  family  then  it's  saying  that
the  dualizing  objects  kind
of
um  very  nicely  in  the  family
so  it's  kind  of  some  kind  of
equisingularity  property  for  f  from  X  to
Y  um  okay  is  there  an  example  when  the
conver
oh  yeah  so  so
warning
uh  Converse  fails  in
general  so  let  me  give  the  example  which
will  explain  kind  of  how  you  should
think  about  this
condition  so  we  could  take  for  example
uh  so  y  to  be
spec  uh  so  let's  say
solid  Z  bracket  t
z
um  and  then  take  X  to  be  uh  X1  disjoint
Union
X2  um  where  X1  is
spec
so  spec  of  solid
ZT  or  maybe  I'll  write  it  as  U  Union  Z  I
want  to  so  I'm  going  to  write  it's  going
to  be  I  take  a  space  and  I  take  an  open
subset  and  it's  closed  complement  that's
going  to  be  the  counter  example  so  um  so
solid  ZT  sitting  inside  solid
ztz  um  will  be  U  and  then  this  is  the
complimentary  guy  so  this  will  be
solid
uh  uh  Z  Lon  series  T
inverse
spec  so  spec  is  just  this  formal
notation  for  passing  to  the  opposite
category  um
uh  then
um  so  F  so  so  let's  called  uh  so  so  this
is  mapping  there  by  J  this  is  mapping
there  by  I  and  that  induces  a  map  from
the  disjoint  Union  um  then  J  lower
shriek  uh  of  of  the
unit  is  this  sort  of  compactly  supported
thing  so  it's  the  fiber  of  the  homotopy
fiber  of  ZT  T  going  to  zon  series  T
inverse  but  uh  I  lower  star  well  I  lower
Shak  of  the  unit
there  uh  I  is  a  i  is  proper  it's  a  kind
of  a  closed  immersion  um  so  this  thing
just  gives  a  Zant  series  T
inverse  uh  and  it's  clear  that  you  can
build  uh  Z  of  T  from  this  fiber  in  this
guy  by  just  one  co-  fiber  sequence
um  and  then  the  um  the  lower  shriek  from
X  the  disjoint  Union  will  just  be  given
by  restrict  to  you  take  the  lower  shriek
there  and  restrict  to  Z  take  the  um
lower  star  yeah  take  the  lower  star
there  um  and  then  take  the  direct  Su  of
those  two  objects  so  you'll
get  uh  you'll  get  this  guy  direct  sum
this  guy  therefore  by  by  closure  under
retracts  you  if  you  allow  closure  on
retracts  you  get  each  of  them  and  then
you  get  ZT  individually  so
uh  so  so  it  is  not
shable  sorry  you  said  it  is  not  sh  no
it's  shable  sorry  no  it's  not  it's  not  a
shriet
cover  ah  it  is  not  a  stre  cover  because
the  the  the  for  the  cover  you  just  get
youu  and  Z  separately  somehow  you  get
the  derived  of  youu  cross  the  deriv  of  Z
not  a  derived  of  Y  exactly  yeah  because
uh  okay  here  you  can  you  have  some  X  or
something  that  is  not  the  same  in  yes  in
in  X  and  then  y  you  D  some  Z  yes  some
particular  yeah  or  I  mean  uh  I  mean  the
the  the  Ring  of  global  sections  on  the
disjoint  Union  is  kind  of  it's  the
product  of  the  two  rings  whereas  yeah  so
yeah  uh  okay  so  so  this  condition  is
something  like  just  set  set  you  know
surc  ity  on  underlying  sets  or  maybe
it's  like  a  cover  in  some  constructible
topology  or  something  that  could  be  how
you  think  about  this  and  then  if  you
want  to  go  from  that  to  some  honest
descent  you  need  to  assume  um  some
properties  so  this  is  analogous  to  like
in  maybe  in  topological  spaces  you  know
if  you  want  there's  descent  for  open
covers  obviously  in  topological  spaces
but  there's  also  descent  for  say  finite
closed  covers  or  there's  like  proper
descent  but  you  can't  mix  open  and
closed  and  still  expect  to  get  descent
so  that's  so  that's  kind  of  yeah  so  this
is  kind  of  some  cover  condition  some
sort  of  set  theoretic  cover  condition
and  then  if  you  assume  either  that
you're  some  generalization  of  open  which
is  this  Ula  or  some  generalization  of
closed  which  is  this  proper  then  then
you  get  honest  descent  but  you  can't  mix
them  so  what  is  the  relation  of  this
like  when  we  now  go
to  like  hub's  analytic  spaces  where
there  are  certain  topologies  like  the  V
topology  the  AR
topology  that  had  some  descent
properties  there  and  what  is  the
relation  with  the  your  Notions  here  I
I'm  not  sure  that  I  understand  the
precise  sense  of  your  question  but  we
will  discuss  uh  I  will  get  to
uh  to  discussing  Huber  stuff  in  relation
to  these  definitions  later  in  the
lecture  so  let's
postpone
um  okay  right  okay
so  the
proof
um  so  uh  so  let's  uh  let's  make  explicit
what  this  um  shriek  descent  according  to
the  definition  means
so  uh  so  we're  using  these  upper  shriek
functors  uh  those  upper  shriek  functors
are  right  adjoints  they  they're  they're
left  adint  join  is  the  lower  shriek
functor  and  there's  some
um  some  standard
some  uh  some  category  Theory  which  tells
you  that  this  comparison  functor
uh  uh  this  itself  will  then  also  have  a
left  a
joint  uh  which  is  given  by  sort  of
informally  so  you  have  some  collection
of
objects
um
um  and  what  you  do  is  you  take  their  uh
sort
of
uh  you  take  a  a  co-  limit  in  this
category  D  of
Y  uh  of  the  G  lower  shrieks  of
them  um  okay  one  has  to  make  this
precise  I  believe  lri  has  done  it  um  but
that's
uh  that's  the  gist  of  it  um  so  in
particular  for  fully
faithfulness  this  amounts  to  uh  the
claim  that  if  you  take  the  co  limit  and
and  by  G  I  mean  for  for  every  n  in  Delta
you  have  a  you  have  the  map  from  this
fiber  product  down  to  all  the  way  down
to  Y  and  I'm  calling  that  map  G  I  mean
it  kind  of  depends  on  N  but  it's  just
yeah  um  so  the  co-  limit  of  these  uh  G
lower  sh  G  upper  shriek  M's  mapping  to  M
so  this  should  be  an  isomorphism  in  D  of
Y  uh  for  all  m  and  d  of
Y
um
so  uh  if  you  take
M  to  be  the  unit  in  y  um  so  this  is  a
geometric  realization  it's  a  CO  limit
over  this  Delta  op  but  you  can  filter
that  you  can  always  write  this  as  a  CO
limit  over  the  natural  numbers  and  then
a  kind  of  a  partial  totalization  so  like
maybe  I'll  say  d  and  then  so  n  and  Delta
op  or  you  restrict  the  cardinalities  to
be  less  than  or  equal  to  D
um  and  this  is  a  finite  Co
limit  that's  a  nice  fact  about  the
Simplex  category  um  but  then  uh  since
the  unit  object  is
compact  uh  if  you  write  it  as  a  filtered
Co  liit  of  something
huh  sorry  where  are  we  in
Delta  you  WR  Delta  what  is  the  sub  the
the
oh  here  Delta  yeah  Delta  op  and  I  write
less  than  or  equal  to  D  meaning  I'm
looking  at  non-empty  finite  sets
of  nonempty  totally  ordered  sets  of
cardinality  less  than  or  equal  to  D
yeah
um  since  so
since  uh  unit  Y  is
compact  so  that's  the  key  property  here
uh  we  deduced  that  unit  Y  is  a
retract  of
some  uh  some  partial
totalization  uh  where  are
we
uh  um  but  each  of
these  uh  lies  in  the  image  of  of  eor
shriek
from  D  of  x  to  D  of  Y  because  all  of
these  structure  Maps  G  from  these  fiber
products  iterated  fiber  products  they
all  Factor  through  D  of  Y  or  D  of  X
sorry  mapping  to  D  of  Y  um  and  and  as
mentioned  before  this  is  equivalent  to  a
finite  Co  limit  this  um  this  partial  to
geometric  realization  what's  so  mean  by
a  finite  Co  liit  a  CO  liit  over  a  finite
usual  category  no  I  mean  over  a  finite
say  simplicial  set  so  over  a  finite
cimit  over  a  finite  simplicial  set  kind
of  in  L's
um
it  it  means  it  can  be  expressed  in  terms
of  iterated  pushouts  and  yeah  uh
iterated  pushouts  and  um  and  the
empty  empty  the  and  the  initial  object
so
push  outs  SC  here  it  is  a  CO  liit  over
the  usual  category  over  one  category
that  is  the  that's  a  totally  just
the
finite  sub  no  no  no
the  I  forgot  which  one  the  Delta  is  a
category  of  opposite  of  finite  ordered
Set
uh  there  simplicial  objects  or  functors
from  Delta  op  so  whatever  whatever  that
so  I  mean  yeah
um  uh  okay  so  this  is
a  I  mean  in  this  stable  setting  it's  so
the  difference  between  this  for  D  and
this  for  D  minus  one  is  just  given  by
one  of  these  objects  up  to  a  shift  or
maybe  some  uh
like
there's  I  mean  the  co-  fiber  of  the  map
so  we  have  a  a  dindex  system  here  and  if
you  look  at  the  sucess  exessive  co-
fibers  then  they're  just  described  in
terms  of  these  individual  objects  here
so  that's  I  mean  in  the  stable  setting
some  kind  of  do  con  thing  simplifies  all
of  this  I
I  yeah  and  so  a  finite  Co  limit  is
defined  in  higher  topus  yeah  yeah  so
this  is  not  more  General  than  Co  liit
over  a
usual  uh  well  well  not  no  you  have  to  be
a  bit  careful  because  a  finite  if  you
have  ai  a  category  a  finite  category  in
the  usual  sense  it  might  not  be  finite
considered  as  a  Infinity  category  or  or
say  considered  as  a  simplicial  set  so
for  example  like  the  category  with  one
object  and  like  a  finite
group  uh  worth  of  automorphisms  is  a
finite  category  in  the  usual  sense  but
it's  not  finite  as  an  Infinity
category  well  for  example  because
there's  group  homology  ology  in
arbitrarily  High  degrees  so  it's  not
finite  it's  not  finite  as  a  simplicial
set  BG  it's  not  BG  is  not  finite  as  a
simplicial
set  um  it's  you  need  infinitely  many
simplies  to  to  describe  it
so  that's  but
but  but  this  one's  okay  okay  so  you
don't  say  exactly  what  okay
yeah
um
okay  uh  right
so  um  where  are  we  now  oh  so  that  was
the  the
implication  um  and  for  the  for  that  was
part  one  so  that
proves
that  proves  part
one  um  and  then  for  part  two
um
uh  so
um  well  so  for  part  two  you  can  note
that  the  the  hypothesis
uh  implies  that  every
M  uh  are  generated  by  things  in  the
image  of  f
streak  um  because  uh  we  have  the
projection  formula  for  lower  shriek  so
you  can  just  tensor  this  statement  with
M  and  you'll  get  the  um  desired
statement  for  arbitrary
M  um  but  uh  so  then  we  want  to  deduce
deduce  at  least  the  fully  faithfulness
this
uh  um  we  can
assume  by  this  uh  claim  here  we  can
assume  uh  that  actually  m  is  of  the  form
F  shriek
n  and  then  and  then  if  we  have
a  base  change
result  for  uh
commuting  uh  F  shriek  and  say
like  or  let  me  say  some  some  like  G
Prime  lower  streek  and  and  G  upper
shriek  then  we  uh  can  uh  reduce
to
uh  a  split  situation
what  g  g  so  yeah  I'm  sorry  it's  it's  not
good  notation  but  so  G  depends  on
N  so  G  so  G  is  just  my  arbitrary
notation  for  some  one  of  these  uh  maps
from  projection  maps  from  x  t  over  y  n
to  to  Y  so  what  G  is  always  going  to  not
denote  like  some  structure  map  back  to  Y
that's  uh
I'm  now
completely  do  do  you  know  uh  projection
formula  for  is  part  of  no  is  part  of  the
no  I  now  go  don't  remember  ah  I'm  sorry
yes  so  it's  part  of  the  definition  of
what  it  means  to  have  a  six  functor
formalism  so  it  it  comes  from  the  fact
that  we  assumed  have  to  be
shable  okay  and  then  you  want  want  to
know
that  the
limit  of  but  I  think  that  it  was  not
this  it
was  you  want  to  represent  oh  I'm  sorry
thank  you  yes  yes  I  got  him  mixed  up
thank  you  very  much  yeah  I'm  sorry  yeah
yeah
um  and  all  of  these  Maps  G  are  like
compositions  of  pullbacks  of  f  so  if  F
has  one  of  these  two  properties  then  all
of  the  maps  G  will  as
well
um  so  in  the  proper
case  so  G  lower  shriek  equals  g  lower
star  and  the  base
change
follows  from  sort  of  the  upper  star
lower  shriek  base
change  by  passing  to
adjoints  um  and  in  the  Ula
case  we  have  a  sort  of  G  upper  shriek  is
G  upper  shriek  of  one  tends  to  G  upper
star  and  the  base
change  uh  follows  kind  of  more  directly
from  again  the  upper  Street
lower  star  base
change  so  there  is  some  you  have  to  make
sure  the  two  base  change  comparison  maps
that
you're  one  you  a  PRI  two  different  base
change  comparison  maps  and  you  have  to
check  they're  equal  but  okay
um  uh  yeah  in  order  to  conclude  from  one
being  an  isomorphism  that  the  other  one
also  is  but
okay  all
right  um  so  that  was  that  proves  fully
faithfulness  for  two
um  so  that  gives  fully
faithfulness  uh  in  the  in  the  shriek
descent  but  the  um  essential
surjectivity  or  so  the  the  other  adjoint
or  the  the  the  the  unit  or  the  co-unit
or  whichever  it  is  the  one  going  from
here  to  there  and  then  back  up  here
again  um  that's  proved  in  the  same  way
so
uh  handled
similarly  uh  using  base
change  to  reduce  to  the  situation  where
the  cover  is  split  So  you  you're  pulling
back  to  X  where  the  cover  is
split  so  we  have  a  functor  where  we  want
to  claim  is  an  isomorphism  we've
identified  if  an  adjoint  to  it  and  what
I've  just  explained  is  that  if  you  do
the  functor  and  then  the  adjoint  that's
the  identity  and  you'd  also  need  to
check  that  if  you  do  the  adjoint  and
then  the  functor  you  get  the  identity
and  I'm  claiming  that's  handled
similarly  using  base  change  along  X
going  to  Y  where  it  happens  for  formal
reasons  because  the  the  cover  is
split  ah  once  you  so  when  you  are
upstairs  and  go  downstairs  then  you  are
you  want  to  reduce  the  K  with  split
cover  so  it's  even  easier  even  but  you
still  need  some  base  change  yes  exactly
you  need  the  same  base  change  property
that  the  upper  shriek  commutes  with  the
lower  shriek  in  in  the  for  the  relevant
class  of
maps  okay
so  we're  part  of  the  way  towards  so  now
we've  uh  this  is  kind  of  a  more  concrete
thing  that  you  might  hope  to  be  able  to
check  so  you  have  to  be  in  one  of  these
two  situations  let  me  explain  some
special  cases
um
so  special
cases
um  so  one  will  be  closed
covers  F  I  should  say  maybe  finite
closed
covers
um  so  we  defined  a  notion  of  a  proper
map  and  we  defined  the  notion  of  open
immersion  but  we  didn't  Define  the
notion  of  closed
immersion  um  so  Al
so  if
um  if  f  is
proper  and  um  well  one  and  one  way  of
saying  it  is  the  D  the  pullback  from  Dy
to  DX  is  a
localization
uh  me  oh  yes  it  means  the  it  has  a  the
right  ad  joint  which  exists  the  right  ad
joint  is  fully
faithful  nothing  nothing
more
no  but  a  localization  in  category  theory
in  in  which  context  is  defined  now  for
categories
of
which  kind  of  I  forgot  I  I  I  know  that
people  like  in  Gabriel  and  some  ofation
don't  remember  now  what  the  so  let  me
let  me  say  that  so  this  this  is  a  funter
which  admits  a  right  a  joint  um  so  when
the  wrer  joint  is  fully  faithful  which
is  what  I'm  calling  localization
um  it  follows  that  uh  you  have  a
universal  property
for  uh  cimit  preserving  functors  out  of
here  namely  they're  the  same  thing  as
cimit  preserving  functors  out  of  here
which  kill  every  object  or  sorry  which
invert  every  map  which  is  sent  to  an
isomorphism  by  this  functor
so  so  this  is  an  analog  of  localization
for  triangulated  categories  for  example
yeah
yeah  okay
and  the  ISS  are  like  localization  for
usual  a  billion  categories  by  oh  but
okay  but  then  it's  again  the  similar
thing  at  least  for  Goen  the  categories
and  things  which  are  closed  and  I'm  not
sure  in  which  depends  if  you  assume  that
all  Co  limits  are  all  or  not
or  no  I  remember  that  in  some  old
literature  category  the  there  the  it
looked  very  hard  to  read  uh  uh-huh  no  I
don't  know  so  when  you  have  these  big
categories  the  simplest  thing  to  say  is
definitely  just  that  the  righted  joint
is  fully  faithful  so  and  that's  what
I'll  that's  what  I  mean  by
localization  um  or  the  yeah  you  know
some  unit  map  is  an  isomorphism  or
whatever  yeah  okay  but  so  since  we  if  we
we  already  know  that  f  is  proper  which
means  that  all  of  D  of  X  so  this  means
that  D  of
X  is  just  modules  over  so  the  push
forward  of  1x  and  D  of  Y  that's  the  same
thing  as  proper  if  you  if  you  remember
the
discussion  um  so  a  proper  map  is  just
controlled  by  this  single  algebra  object
in  D  of  Y  that  so  then  being  a  closed
immersion  is  the  same  thing  as  saying
that  this  this  guy  is  item
potent
algebra  so  tens  are  product  of  two
copies  of  itself  D  of  Y  is  itself  again
via  say  the  multiplication
map
um  so  this  is  since  f  is
proper
um  ah  so  now  let  me  make  a
warning
um  uh
so  in  af  so  on  the  level  of  these
apoid  uh  it's  not  true  in  general  not
generally
true  that  uh
closed  and  open
immersions
well  are  in
viction  uh  so  say  with  the  same
Target
so  that  so  an  open  immersion  is  not
necessarily  going  to  have  a  closed
complement  and  a  closed  immersion  is  not
necessarily  going  to  have  an  open
complement
um  it's  close  to  being  true  that  uh  an
open  immersion  has  a  closed  complement
the  only  uh
so  um  so  let's  so  so  let  me  expand  on
this  so  uh  given  an  open
immersion
we  get  uh  an  item  potent
algebra  so  given  an  open  immersion  so  U
to  Y  uh  we  get  an  item  potent  algebra  a
in  D  of  Y  such  that  um  you  know  D  of  U
equals  D  of  Y  mod  mod  a  d  of
Y
but  uh  it's  not  true  in
general
that  uh  a  lives  in  D  greater  than  or
equal  to  z  y  so  it's  not  necessarily
connective  um  and  this  is  the  condition
so  this  is  the  condition  uh  this  is  the
condition  needed  uh  to  get  a  closed
immersion  in
F  so  if  you  start  with  just  an  item
potent  algebra  it  doesn't  necessarily
correspond  to  a  closed  immersion  in
af  because  you  can't  it  doesn't
necessarily  have  a  correct  underlying
uh  animated  ring  it's  only  the
connective  ones  that  correspond  to
animated  Rings  not  the  non-c  connective
ones  so
um
uh  yeah  so  in  the  case  of  usual  schemes
can  you  recall  what  are  the
open  I  forgot  well  it  depends  which
funter  from  schemes  to  analytic  spaces
you're  using  so  we'll  go  into  it  but  I
want  to  say  that  this  is  analogous  to
kind  of  some  complimentary  phenomenon  in
scheme  Theory  where  um  you  know  for  an
apine  scheme
uh  you  have  a  you  have  a  closed
immersion  um  then  the  open  complement
might  not  be  apine  right  so  it  might  not
correspond  to  open  immersion  in  apine
schemes  but  it's  still  a  scheme  and  it's
kind  of  similar  here  even  in
situations  where  this  is  not  connective
usually  you  will  you  will  get  a  a  closed
complement  which  is  an  analytic  space  it
just  won't  be  an  apine  one  um  but  in  the
cas  of  scheme  is  like  you  taking  the
complement  like  a  is  qu  comp  spec  a  then
you  get  Al  GMA  U  which  is  important  but
and  and  does  this  correspond  to  to  a  CL
in  this  setup  to  like  when  you  take  D  of
a  model  this  what  do  you  get  do  you  get
D  of
some  some  which  you  call  an  openion  a
closed  here  but  I  I'd  rather  let  let  me
again  get  to  the  comparisons  with  the
classical  theories  a  bit  later  in  the
lecture  although  this  is  going  much
slower  than  I  anticipated
um  uh  so  um  yeah  so  I  mean  I  could  give
an  example  uh  so  if  you  look  at  um  so  y
equals
solid  uh  Z  bracket  I  don't  know  two
variables  X  Y  comma  Z  and  then  uh  U  to
be
solid
uh  um  then  I  invite  you  to  do  the  very
good  exercise  of  figuring  out  what  this
itm  potent  algebra  is  and  then  the  the
corresponding
a  uh  has  a
nonzero  uh  H  minus
one
um  the
situation  on  the  other  side  um  is
somehow  even  worse  so  given  in  the
closed
immersion
um  well  actually  Peter  described  the
condition  required  for  there  to  be  a
complimentary  open  so  a  closed  immersion
corresponds  to  um  an  item  potent
algebra  uh  in  the  greater  than  or  equal
to  zero  derived  category  of  Y  close
Merion  say  Z  to
Y
um  uh  then  uh  so  then  you  need
for
uh  to  get  a  complementary
open  in
af  you  need  that  uh  that
ROM  from  the  fiber  of  a  unit  of  Y  going
to  a  to  this  uh  which  is  a  functor  from
D  of  Y  to  D  of  Y
uh
uh  you  need  this  commutes
with  uh  filtered  Co
limits  and
preserves  uh  D  greater  than  or  equal  to
z  y  so  that's  the  formula  for  what  would
be  the  localization  functor  to  the
complimentary  open  and  you  need  that
that  actually  defines  an  analytic  ring
structure  in  our  sense  which  amounts  to
these  conditions
here  internal  rhom  I  should  say
yeah
um
okay
right  so  as  I  said  some  of  this  will  be
fixed  by  allowing  General  analytic  space
is  not  necessarily  apoid  but  it's
probably  probably  good  to  keep  all  this
in
mind
um
maybe  maybe  I  should  explicitly  write
that  uh  these
subtleties  will  be  to  a  large
extent  uh
fixed  by
allowing  uh  arbitrary  analytic
spaces  or  Stacks  you
know
and  not  necessarily
aine
um
okay  uh  oh  sorry  I  was  in  I  actually
didn't  plan  to  talk  about  this  I  go  but
then  it  occurred  to  me  I  should  um  so
yeah  I  was  talking  about  finite  closed
covers  right  as  an  example  of  descent  um
so  uh  suppose
given
uh  finitely
many  uh  so  Z1  ZN  uh  closed  subsets
closed
immersions  well  with  closed
immersions  uh  into  X  and  all  of  these
are  in
f
um
then  so  when  do  we  get  a  cover  so  oh
when  then  disjoint  Union  of  Z  mapping  to
X  so  it's  a  um  is  a
cover  if  and  only  if
um  you  just  plain  have  the  most  naive
form  of  descent  for  the  structure  sheath
so  um
so
uh
so
um  and  this  is  actually  going  to
terminate  at  a  finite  stage  because  um
because  it's  a  finite  closed
cover
um  this  is  the  condition  that  the  IE  the
structure  sheath  is  a  sheath  so  to  speak
well  well  yeah  just  um  so  this  is  the
this  is  the  item  potent  algebra  picking
out  Z  and  then  this  will  be  the  tensor
product  of  the  two  of  them
so  uh  that's  what  I  mean  by  the  that
defines  a  closed  immersion  which  is  the
fiber  product  in  the  category  of
analytic  spaces  or  sorry  in  fiber
product  in  af  um  and  um  that's  what  this
is  so  may  yeah  maybe  by  definition  this
is  the  fiber  product  in  af  but  then  it's
it's  also  a  closed  immersion  and  it's
exactly  uh  corresponding  to  the  item
potent  algebra  given  by  tensoring  these
two  things
together
um  so  why  is  this  true  that  this  is  the
Criterion  for  being  a  shet  cover  well  um
if  this  is  satisfied
then  certainly  uh  one  is  in  the  image  of
the  lower  star  functors  from  the
disjoint  unions  of  the  Z  because  each  of
these  things  is  so  in  this  case  lower
shriek  is  lower  star  because  the  maps
are  proper  um  so  we  get  this  condition
but  then  that  map  is  proper  so  that
implies  a  shriek
descent
um  uh  which  is  the  the  condition  that
it'  be  a  shriek  cover  on  the  other  hand
shriek  descent  implies  this  fanc  all
this  fancy  star  descent  right  um  in
particular  it  implies  star  descent  for
like  d  of  blank  so  D  of  X  is  the  limit
and  then  if  you  just  plug  in  uh  the  unit
object  in  D  of
x  uh  and  see  what  that  um  base  change
and  then  the  righted  joint  is  telling
you  it's  exactly  giving  this  uh  sort  of
structure  sheath  being  a  sheath
condition  so  yeah  um
so  so  this  follows  from
proposition
um  and  this  uh  follows  from  Star
descent  which  is  implied  by  shriek
descent
um
okay
um  okay  what  about  open
covers
um
well  yeah  I
guess  um  yeah  so  let  let  let  me  just  say
so  it  it  turns  out  that  every  open  cover
has  a  finite  refinement  if  you  say  what
it  means  to  have  an  arbitrary  open  cover
but  let  me  just  stick  to  the  case  of
finite  open  covers  there's  in  any  case
no  loss  of  generality  um  so  let's  say  we
have  a  yeah  U1  up  to  unu  n  again  mapping
to  X  by  open
immersions  and  let's  say  that  U  U1
corresponds  to  so  the  complimentary
closed  is  described  by  this  item  potent
algebra  A1  and  a  n  sorry  so  these  are
item  potent  in  D  ofx
um  then  uh  intuitively  speaking  um  these
open  subsets  should  form  a  cover  of  x  if
uh  you  know  when  you  take  the
intersection  of  all  these  guys  you  get
the  empty  set  or  intersection  of  all  the
Clos  corresponding  closed  subsets
complements  you  get  the  empty  set  um  so
the  claim  is
that  uh  is  that  um  disjoint  Union  of  UI
mapping  to  X  is  a  street
cover  uh  if  and  only  if  um  A1  tensor  dot
dot  tensor  a  n  is  equal  to  zero  D  of
X  is  it  the  case  that  the  filtered  Coit
of  itent  algebra  is  zero  implies  that
one  of
the  yeah  that's  the  that's  that's
correct  and  the  reason  is  because  the
unit  object  is  compact  so  because  an
algebra  is  going  to  be  zero  if  and  only
if  like  1  equals  z  in  in  it  and  then
that's  a  question  of  a  map  maps  from  the
unit  object  to  your  algebra  but  if  the
unit  is  compact  then  1  equals  z  will  be
witnessed  at  a  finite  stage  that's  the
reason  behind  why  an  arbitrary  open
cover  has  a  finite  refinement
yeah  yeah
um
okay  um  um
so
uh  right
um
so  um  yeah  and  well  let  me  just  for
Simplicity  uh  so  proof  for  Simplicity
Let's  Take  N  equals  2  you  can  kind  of
reduce  to  that  case  if  you  well  if  you
make  if  you  formulate  the  appropriate
categorical  analog  of  this  I  mean  the
union  of  two  of  the
U  U1  Union  U2  will  not  necessarily  be  an
A  so  to  speak  but  if  you  appropriate  the
for  if  you  formulate  the  appropriate
just  purely  categorical  version  of  this
statement  you  can  run  an  induction  to
reduce  to  the  case  n  equals  2  um  so  when
n  equals  2  um  then  uh  then  shriek
descent  well  um  so  we  need  uh  right  so
so  star
descent  again  uh  gives  that
um  what  do  I  how  do  I  want  to  say  this
um
so
um  so  if  you  look  at  what  star  descent
means  and  use  the  formula  for  um  the
upper  star  functor  which  is  this  kind  of
localization  formula  then  you  find  that
the  the  claim  of  star  descent  is  exactly
the  claim  that  you  have  a  pullback  of
this
form
um
uh  wait  I'm  sorry  I'm  getting  myself
awfully  confused  right  now
uh  just  uh
no  I'm  getting  myself  very  confused  this
is  this  is  the  star  no  this  is  the  star
descent  for  the  closed  compliment  I'm
sorry  I'm  sorry
um  uh  just
oh
so  of  course  if  something  is  is  is  a
module  over  A1  10  and  it  goes  to  zero  on
the
UI  so  so  it  goes  to  zero  in  each  stage
of  the
simpl  so  there  it  goes  to
zero  apparently  in  this  limit  in  the
infinity  categorical  well  whatever  so  so
you  have  some  SC  in  so  the  condition  not
satisfied  then  there  is
a
uh  uh
yeah  um  so
um  um  I  apologize  I  kind  of  I  assumed  I
would  be  able  to  do  this  off  the  top  of
my  head  and  I  didn't  think  about  it
carefully  this  is
um
uh
so  uh  I  mean  maybe  I  should  just  say  uh
so  well  okay  maybe  I  should  just  say
this  so
the  so  the  shriek  let  me  say  so  shriek
descent  um  so  the  streak  descent  for
this  cover  say  where  you  have  two
elements  and  both  of  them  are  mapping  by
monomorphisms  into  X  um  then  The
Descent  which  a  priori  involves  some
check  nerve  which  is  some  infinite  thing
it  actually  reduces  to  some  Myer  vurus
um  as  is  kind  of  standard  so  it's  the
same  thing  as  uh  so  D  of  U1  intersect  U2
uh  D  of  U1  uh  D  of  U2  or  maybe  I'll  make
the  maps  go  the  other  way
um
and  these  are  the  upper  shriek  maps  all
of
them  okay  uh  that  should  be  a  pullback
um  and  then  so  and  then  you  can  check
then  you  have  the  map  funter  from  this
to  the  pullback  there  and  again  it  has
this  left  ad  joint  so  the  uh  the  claim
for  the  unit  gives  that  the  unit  of  X  um
receives  an  isomorphic  map  from  uh  you
take
J1  lower  shriek  of  the  unit  on  U2  or
U1
um  oh  sorry  sorry  j  J1  upper
streek
unit
um  okay  um  uh  sorry  yeah  so  I  mean  um  I
mean  you  have  some  kind  of  my  vator
sequence  like  this  so  this  is  a  cofiber
sequence
um  and  then  you  have  formulas  for
everything  so  I  apologize  for  not
explaining  this  very  well  but  you  have
formulas  for  all  of  the  functors
involved  so  um  in  terms  of  the
corresponding  item  potent  algebras  and  I
promise  that  if  you  work  it  out  then  uh
it's  just  going  to  mount  to  the
condition  that  A1  t  or  A2  is  equal  to
zero  uh  meaning  that  this  condition  will
be  directly  equivalent  when  you  write
down  what  everything  means  to  A1  tensor
A2  equals
z  um  so  that  would  be  a  consequence  of
shriek  descent  but  on  the  other  hand  if
you  have  this  then  you  have  this  and
this  proves  that  the  unit  is  written  in
terms  of  lower  shrieks  from  your  guy  and
therefore  by  the  proposition  you  get  a
shriek  descent  as
well
I  should  have  mentioned  uh  uh  open
immersions  are  so  no  note  H  sorry  uh
open
immersions
Rula  so  J  upper  sh  one  is  one  yes  is
one
well  for  open  imion  yeah  that's  right
okay  so  you  have  and  the  J  of  this  how
is  it  given  in  terms  of  the
algebra  so  the  J  lower  shriek  um  of  one
so  This  this  term  for  example  will  be
the  fiber  of  the  unit  mapping  to
A1
yeah  I  mean  yeah  it  is  Elementary  which
is  um  then  I  made  a  mistake  by  therefore
not  preparing  it  because  I  thought  it
would  just  my  chalk  would  do  it  but  yeah
I  I'm  sorry  for  messing  this
up
um
okay
uh  so  those  were  kind  of  um  special
cases  but  we  haven't  gotten  completely
concrete  yet  because  was  still  in  some
kind  of  generality  of  closed  covers  and
open  covers  and  so  on  um
but  now  let's  give  some
examples
um  so  the  first  example  is  zariski
zariski
covers
um  so  note  that  there  is  a  functor  um
from  the  usual  category  of  commutative
Rings  uh  to  the  category  of  analytic
Rings  um  which  sends  a  commutative  ring
to
um  to  the  pair  so  we  take  R  now  we're
viewing  r  as
a  a  condensed  ring  but  it's  just
discrete  so  it  has  the  discret
topology  um  and  then  we're  actually
going  to  take  the  full  derived  C  of  R
but  now  for  emphasis  since  this  is  just
a  discret  ring  I  want  to  make  sure  we
remember  that  we're  taking  the  category
of  condensed  R  modules  so  quite  quite
large
um  no  solid  no  we  don't  need  yeah  we
don't  need  to  localize  to  solid  z  um  so
yeah  we'll  just  take  all  condensed  R
modules  um  okay  so  this
induces  uh  fun  from  apine
schemes  uh  to  what  I  was  calling  AF
which  is  maybe  apine  analytic  spaces  um
which  is  just  this  the  same  functor  but
viewed  on  the  level  of  opposite
categories
um  and  well  maybe  an  a  remark  is  that
this  this  functor  commutes  with  fiber
commutes  with  fiber
products
uh  commutes  with  finite  limits  in  fact
so  it's  also  sends  the  terminal  object
to  the  terminal
object  or  in  other  words  yeah  relative
tensor  products  in  commutative  rings  are
also  relative  tensor  products  in
analytic
rings  that  follows  from  our  discussion
of  a  relative  tensor  product  and
analytic  Rings  um  but  more  but  then  I
want  to  say
moreover  the  relative  tens  of
product  I  mean  underived  or
oh  shoot  yeah  yeah  I  say  maybe  I  should
say
derived  so  they  say  animated  yeah  ah  and
then  we  are  much  much  yeah  thank  you
thank  you  o  Foria  so  now  you  are  and
then  you  are  then  of  course  all  this
notion  of  D  are  is  much  more  yeah  much
more
uh
um  so  zisy
covers  uh  map  to  uh  shriek
covers  uh  but  now  it's  occurring  to  me  I
I  forgot  to  remind  what  this  means  so
sorry  sorry  sorry  sorry  just  a  sec
so  so  uh  sorry  so  so  note  um  yeah  we  get
a  Gro
topology  on
F  by
saying  uh  that  a
seive
ace  over  X  in
F  is  is  a  street
cover  uh  if  it
contains  uh  finitely
many  many
Yi  mapping  to  X  now  I've  switched  the
rules  of  X  and  Y  from  previously  sorry
about  that  um  well  maybe  I'll  just  call
this  XI  such  that  uh  find  it  may  shable
uh  such  that  disjoint  Union  XI  uh
mapping  to  XF  is  a  street
cover  uh  in  the  previous  sense
so  and  the  key  behind  this  besides  kind
of  obvious  properties  of  finite  drint
unions  is  the  that  a  a  shriek  the  if  you
have  a  shriek  cover  then  any  base  change
is  also  a  sh
cover  which  um
is  a  consequence  of  this  discussion  of
co-  limits  in
PRL  what  did  you  say  now
that  uh  if  you  have  a  map  which  is  a
shet  cover  then  any  pullback  of  it  is
also  a  shet
cover  or  base  change  any  base  change  of
it  it's  also  a  street
cover  uh  well  yeah  there's  uh  it's
basically  because  of  the  fact  well  well
if  you  can  think  in  terms  of  this  uh  Co
limit  in  PR  business  and  then  it's  just
because  the  the  base  change  functor  on
the  level  of  mod  dprl  the  base  change
functor  just  commutes  with  co-  limits  so
if  you  have  the  condition  there  then  you
base  change  you  get  the  condition
there  it's  a  consequence  of  two  facts
the  first  was  this  kind  of  kunth  formula
for  like  if  you  have  okay  if  you  have  if
you  have  a  base  change  in  F  then  you
have  the  D's  of  all  of  these  and  the  D
of  the  base  of  the  pullback  is  just  the
relative  tensor  product  of  the  D's  of
the  the  other  three  guys  ah  D  of  this  is
the  tens  of
produ  and  this
uh  is  and  then  you  have  that
the  and  the  desent  that  you  had  is  all
you  formulated  it  in  a  more  General  way
where  you  have  a  category  which  is  a
module  over  then  you  apply  this  somehow
and  check  that  the  maps  are  the  right
ones  yeah  yeah
somehow  yes  exactly  exactly  yeah  um  okay
so  uh  right  so  that's  that  that's  what  I
mean  by  there  Z  risky  cover  going  to  a
shriek  cover  I  guess  I  just  mean  that  if
you  take  the  you  know  has  a  finite
refinement  you  take  the  disjoint  Union
then  that  thing  will  be  a  shet
cover
um  so
um
uh  right  so  um  the  proof  is  simple  uh
indeed  uh  zisy
covers  uh  go  to  closed  covers  in  the
sense  just
discussed  so  if  you  have
a  a  principle  open  a  principle  open  in
uh  Spec  R  and  it's  given  by  inverting
some  element  and  that  inverting  an
element  is  is  gives  you  an  item  poent
algebra  um
so  uh  and  that  that  item  poent  algebra
defines  a  closed  cover  on  the  level  of
these
guys  um  and  then  the  condition  we  had  to
check  um  is  just  usual  zisy  descent  no
why  is  the  risky  cover  goes  gives  a  Clos
cover  well  the  well  so  there's  let's
separate  that  out  the  first  claim  is
that  if  you  have  a  a  principal  open  then
that  goes  to  a  closed
immersion  ah  now  it  is  called  the  Clos
imersion  rather  than  ah  okay  so  this
uh  ah  yeah  remember  this  in  the  previous
talk  yeah  I  mean  I  there's  a  picture
that  I  would  like  to
draw  like  could  you  see  when  you're
doing  analytic  geometry  you  have  more
more  more  room  you  have  more  more  room
than  in  algebraic  geometry  because  you
have  like  nearby  infinite  kind  of
infinite  decimals  almost  I  mean  I  so
usually  you  think  of  like  the  locus
where  f  equals  0  in  algebraic  geometry
like  you  think  you  have  the  locus  where
f  equals  z  and  then  the  principle  open
is  the  complement  of  this  and  then  you
think  okay  the  locus  for  f  equals  z
that's  obviously  a  closed  subset  so  the
complement  should  be  an  open  subset
right  but  um  it's  a  little  bit  of  a  lie
because
um  I  mean  uh  this  isn't  really  where  the
locus  where  f  equals  zero  it's  kind  of  a
you  know  could  also  be  just  a  like  Locus
where  F  to  the  N  is  equal  to  zero  or
something  like
this  um  so  it's  really  what  you're
really  deleting  is  not  the  locus  where  f
equals  zero  but  you're  deleting  kind  of
the  whole  formal  neighborhood  of  that
and  this  then  it  it  acquires  some
fuzz
um  and  then  I  claim  that  what  you  should
really  think  is  that  that  fuzz  is  making
this  thing  really  behave  more  like  an
open  subset  um  and  that  the  there's  a
risky  open  it  should  really  be  thought
of  as  closed  and  it  should  have  a  it
should  have  a  boundary  some  kind  of
tubular  neighborhood  there  should  be
like  um  so  it  should  really  be  a  closed
subset  and  then  the  formal  neighborhood
usual  formal  neighborhood  is  the  open
complement  that's  the  picture  I  would
like  to
suggest
um
yeah  and  when  you  go  to  the  the  solid
world  then  then  you  can  Rec  you  can
again  have  an  open  version  of  puncturing
so  if  you  then  you  then  you  can  name
this
boundary  maybe  maybe  it's  something  like
Z  lant  series  T  base  changed  along  ZT  to
R  where  T  goes  to
F  something  like  that  um  you  can  name
the  boundary  and  then  you  can  remove  it
so  it's  it's  going  to  be  a  closed  subset
you  can  remove  it  and  you  get  an  open
subset  and  then  you're
back  back  into  usual  way  of  thinking  of
having  having  a  z  risky  open  but  then
it's
not  um  yeah  you're  so  you  take  r  t
z  and  then  Z
Lan  Lan  uh  yeah  so  I'm  trying  to  I'm
saying  once  you  move  to  the  solid
framework  then  you  can  name  this
boundary  here  which  what  before  was  just
heris  and  its  name  is
this
um  so  that  will  give  it  that  gives  an
item  put  in  algebra
uh  in  D  of  r  z
solid
um  and
the  well  in  fact  it's  in  D  of  R1  over  f
z  solid  and  the
complement  the  complimentary  open  is
like  d  of  r  1  over  F  kind  of  Z1  over  f  z
so  Z  adjoined  one  over  F  whatever  see
Union  one  over
F
yeah  um  and  but  so  so  if  you  decided  to
if  you  yeah  I
mean  there's  two  different  ways  you  can
embed  schemes  into  adex  spaces  right  one
is  based  on  sending  on  the  level  of
rings  to  R  to  RZ  and  the  other  is  based
on  sending  R  to  R  comma
R  and  this  is  the  one  that  I'm
discussing  right  now  well  based  change
to  solid  z  um  and  there  Z  risky  opens
look  closed  but  if  you  use  this  one  that
corresponds  to  always  removing  the
boundaries  and  then  zarisky  opens
actually  go  to  open  immersions  um  in  the
discussion
yeah
um
so  what  do  you  mean  by  boundary  like  the
foral  neighborhood  minus  the
middle  I  don't  know  what  I  really  mean
by  boundary  if  I  don't  just  mean  this
formula  but  intuitively  I  I'm  claiming
you're  removing  kind  of  an  open  piece
from  this  chunk  and  then  the  the  the
closed  complement  should  intuitively
speaking  have  some  boundary
um  I  mean  there  going  to  be  boundaries
at  Infinity  too  if  if  your  thing  isn't
proper  and  so  I  mean  there's  there
what's  happening  at  Infinity  is  also
important  but  let's  let's  local  Let's
Pretend  We're  in  a  proper  thing  or
something
um  okay  other
questions
um
okay
so  in  uh  so  in  particular
yeah  we  get  a
functor  uh
from  like  zisy  sheaves  on  apine  derived
schemes  um  to  shriek  sheaves  um  on
F
um  um  which  uh  is  in
fact  uh  pullback  of
Topo  so  I  mean  it  commutes  with  co-
limits  and  finite  limits  that's  kind  of
consequence  of  this  so  this  is  so  this
is  one  one  way  of  embedding  usual
algebraic  geometry  into  well  I  haven't
quite  defined  what  an  analytic  space  is
it's  not  exactly  going  to  be  this  but
the
if
you're  it's  it's  close  enough  for
practical  purposes  I  mean  the  we're
going  to  do  have  some  kind  of  hyper
descent  condition  we  want  to  impose  as
well  I  thought  I  was  going  to  get  to  it
today  but  I'm  clearly  not  um  but  this  is
basically  basically  analytic  spaces  or
analytic
Stacks  modulo  some  modulo  a  couple  of
technicalities  so  this  is  shapes  with
values  and  simpli  right  so  there's  the
question  of  yeah  and  that's  related  to
whether  I  want  to  use
hypers  sheaves  or  not  so  I  mean  we
could  we  could  do  either  we  could  do  we
could  do  sheaves  with  values  and  sets
and
just  yeah  why  don't  yeah  what  I  what  I
really  will  mean  is  she  was  values
in  what  Peter  and  I  have  been  calling
anama  which  is  what  lri  calls  spaces
which  is  what  other  people  call  Con
complexes  but  the  Quasi  category  of  con
complexes  some  people  call  homotopy
types  yes
um
and  something  has  many  names  it's
because  probably  because  it's  a  subtle
concept  yeah  well
anyway
um  yeah  makes  sense  to  look  just  shs  of
sets  when  you  have  the  derived  things  so
yeah  I  think  it  still  makes  sense  yeah
still  makes  sense  I  think  so  yeah  and  it
makes  sense  to  look  at  the  OS  like  shs
of
trated  this  makes  sense  yeah  it  makes
sense  and  there  the  the  The  Descent  are
equivalent  yes  exactly  then  when  you  go
to  the  I  mean  to  all  car  complexes  then
you  have
different  potentially  different  Ascent
uh  conditions  and  when  you  R  this  you
you  use  the  weak  one  with  which  the  SEC
yes  covers  although  if  you  did  a
stronger  one  it  still  get  the  same  claim
because  it  would  be  just  a  further  a
further  localization  of  this  so  and  then
the  on  the  other  side  also  yeah  well
right  so  I  don't  I'm  not  going  to  touch
the  other  side  one  could  but  well  the
risk  shes  does  it  also  depend  on  whether
you  have  oh  yes  yes  it  does  yeah  I'm  but
I'm  not  going  to  on  this  side  I'm  not
going  to  care  too  much  but  well  and  it
was  said  in  some  talk  I  don't  remember
who  that  maybe  Peter  said  that  there
would  be  something  intermediate  between
check  descent  and  full  hyper  correct
yeah  I  I  was  going  to  discuss  it  today
but  uh  it's  not  going  to  happen
um  yeah  okay
so  um  and  then  the  other  subtlety  is
like  set  theoretic  subtlety  because  like
the  category  of  commutative  Rings  is  not
small  so  you  have  to  be  careful
considering  pre-  shees  and  sheaves  on  it
and  so  on  and  same  with  you  take  only
those  which  are  accessible  yes  yes  yes
exactly  and  then  you  have  to  prove  sheif
ification  preserves  this  accessibility
and  so  on  and  so  forth  but  okay
um  I  don't  think  uh  in  the  remaining
like  7  minutes  I  could  do  justice  to  the
the  next  topic  was  going  to  be  attic
spaces  um  I  could  rush  through  it  right
now  but  I  don't  think  that's  a  I  don't
think  that's  a  good  idea  so  I  think  I'll
stop
here
J  find  that  cover  here  by
Somey  and  well  I  don't  have  to  I  wanted
to  know  what's  for
example  should
cover  oh  oh  oh  oh  sorry  I  didn't
understand  your  question  I'm  sorry  yeah
whe  whether  in  a  tall  cover  also  gives
you  a  shet  cover  yes  it  does  it
does  right  so  yeah  so  the  yeah  so  yeah
this  is  thank  you  thank  you  I  I  think
maybe  this  is  something  I  should  do  in  a
few  minutes  um  uh  yeah
so  so  this  was  all  I  mean  a  lot  of  this
was
motivated  this  was  motivated
uh  by  Matthew  Akil  Matthews
uh  a  theory  of  descend
ability  um  so  um
so  uh  so  we  could  so  we  were  talking  in
af  right  and  then  we  were  saying  we  have
this  derived  category  of  anything  in  af
and  it's  built  on  this  condensed
framework  but  it's  clear  that  the
discussion  is  very  categorical  in
general  and  we  could  just  try  to  make
the  exact  same  definitions  in  the  world
of  ordinary  algebraic  geometry  instead
using  like  for  a  ring  you  just  have  this
usual  derived  category  and  and  see  what
the  kind  of  definitions
give  um  so  if  you  take  the  same
definitions  uh  but
with  the  pair  R
Dr  so  so  this  is  a
usual  commutative  ring  or  may  be  derived
um  and  then  this  is  the  usual  derived
category  so  instead  of  instead  of
analytic  Rings  we're  just  looking  at
commutative  Rings  um
then  some  simplifications  happen  first
is  that  every  map  is
proper  um  that's  quite  clear  uh  well
maybe  that's  the  main  simplification
that  happens  is  every  map  is  proper
because
by  definition  you  know  if  you  have  R
mapping  to  S  then  D  of  s  well  almost  by
definition  then  D  of  s  is  just  the  same
as  s  modules  and  D  of
R
um  uh  so  th  this  implies  uh  that  every
map  is
shable  and  then  um  The  Proposition  that
I  discussed  over  here
um
thing  here  there's  no  condensed  thing
here  yeah  what  is  the
sh  what's
the  what's  the  upper  sh  and  lower  sh  oh
yeah  so  it's  not  it's  not  what  you  might
think  so  the  upper  shriek  is  just  the
right  adjoint  to  lower  star  which  exists
for  General  categorical
reasons  um  it's  not  the  same  as  the
upper  shriek  in  groi  Duality  Theory  so
so
caution  so  for  say
for  say  spec  K  bracket  T  going  to  spec
K  Pi  Pi  upper  Shake  is
not  the  usual  one  and  P  low  sh  is  also
is  what  pylor  streek  is  just  pylor  star
so  that's  what  I  mean  by  every  map  is
proper  okay  so  so  the  lower  upper  sh  is
like  for  finite  map  is  the
same  finite
presentation  I
mean  Maps
it's  for  proper  yeah  for  proper  Maps
it's  yeah  yeah
yeah  but  for  arbitrary  Maps  it's  just
some  funny  thing  nonetheless  it  turns
out  to  be  useful  um  exactly  in  this
descend  ability  discussion  because  so
every  map  is  shable  every  map  is  proper
um  and  then  we  deduced  it  so  a  shriek
descent  shriek  descend  or  a  shriek  cover
is  the  same  thing  as  saying  that  the  the
so  a  street  cover  X  to  Y  is  the  same
thing  as  saying  the  unit  in  y  lies  in
the  image  of  just  so  F  lower
star  D  ofx  to  D  of
Y  and  this  is
exactly
descend  this  is  the
definition  descend  ability  uh  in
Matthew's  work  or  that's  one  equivalent
he  gives  many  characterizations
um  what  about  the  classical  grend  FP  you
see  the  sand  is  it  related  is  it  what
does  it  let  me  give  some  examples  so  so
so  actually  so  every  so  actually
every  actually  kind  of  purely
formally  uh  any  um  any  descendible
map  uh  R  to  s  in  a  ke
sense  uh  gives
a  a  sh  cover  in
a  via  this  this  functor
here  cover  in  now  in  yes  exactly
yes  um
and  Matthew  gives  many  examples  of
descendible  maps  so
um  so  examples  so  a  tall
covers  so  that's  more  General  than  zisy
um  but  also  more  generally  than  that  uh
countably
presented  uh  Faithfully  flat  Maps  so
fpqc  covers
um  so  this  is  kind  of  funny  you  need
this  uh  well  we  apparently  probably  for
real  need  this  countably  presented
hypothesis  countably
so  so  uh  that  means  that  so  like  so  you
have  a  map  from  A  to  B  which  is
Faithfully
flat  but  also  B  is  uh  presented  as  an  a
algebra  or  maybe  even  just  countably
generated  is  enough  so  B  has  a
presentation  as  an  a  algebra  I  countably
many  generators  so  the  so  here  when  you
work  with  simpli  rings  so  uned  so  the
condition  is  that  that  on  Pi  Z  it  is  f
fully  flat  and  that  the  pi  I  are  this  is
kind  I  did  not  carefully  think  about  uh
let  me  say  for  this  I  mean  ordinary
Rings  because  I  didn't  carefully  think
about  uh  the  derived  analog  sometimes
they  they  some  references  they  have  this
derived  yeah  s  with  fa  flat  where  they
Define  it  like  what  I  said  oh  yeah  so  so
it  is  true  that  case  you  need  to  you
need  to  do  it  yeah  and  and
then
face  ordinary
Rings
um
countably  generated  up  because  you  need
the
to  uh  maybe  yeah
there  is  some  uh  so  it's  equivalent  to
the  kernel  the  fiber  t  m  from  fiber  to  a
here  is  some  T  of  power  which  is  zero
yes  yes  yes  and  this  is  related  to  some
limit  higher  limit  I  mean  so  if  it  is
aimit  of  Al  n  s  which  that's  fine  too
but  in  practice  it's  the  same  thing  I
mean  okay  but  if  it  is  only  countably
generated  then  you  don't  get  the  maybe
it  maybe  it  has  to  be  countably
presented  yeah  yeah  because  you  need  to
yeah  you  need  to  be  able  to  reduce  to
accountable  ring  accountable  Bas  ring
I'm  sorry  yeah  let's  say  countably
sended  anyway  the  point
is  uh  lot  of  situations  where  you  have
classical  descent  um  you  get  this  even
stronger  shriek  descent  which  recall
gives  also  this  two  descent  and  much
else  besides  um  but  also  there's  some  um
like  if  uh  so  R  going  to  R  mod  I  or  I  is
a  nil  potent
ideal  um  and  now  you  have  to  be  very
careful
um  yeah  now  these  are  ordinary  rings  but
the  shriek  descent  is  defined  on  the
level  of  the  derived  categories  so  the
The  Descent  you  get  for  this  this  does
not
imply  does  not  imply  that  D  of  R  is  the
same  as  D  of  R  mod  I  obviously
and  the  reason  is  that  you  have  when  you
do  The  Descent  you're  doing  everything
on  the  derived  level  and  so  it's  the  uh
you  end  up  with  terms  like  this  um  in
the  cimit  diagram  and  those  are  those
are  that's  not  the  same  as  our  mod  about
go  to  Pi  z  r  does  it  have  the  same
property  R  going  to  Pi  z  r  r  going  to  Pi
z
r
um  no  no  Peter  says
no  you  can  you  can  have  put  nor  algebra
deg2
generator  and  this  has  a  module  which  by
Inver  deg2  generator  yeah  that's  right
yeah
yeah  yeah  so  so  if  you  look  at  like  I
know  Q  bracket  X  degree  xal  2  um  then
uh
then  there  is  a  ring  module  in
particular  where  you  invert  that
generator  which  goes  to  zero  when  you
mod  out  by  X  but  is  but  is  non  zero  so
so  what's  the  analog  for  simpli  Ring  of
anent  Ideal  you  should  ask  that  it  be
truncated  the  ring  be  truncated  so  it
has  only  F  so  you  should  ask  that  R  has
only  finitely  many  homotopy  groups  and
then  yeah  then  going  to  Pi  z  r  mod
I
um  yes  so  that's  how  the  some  kind  of
artinian  situation  you  could  think  I
don't  know  yeah  um  yeah  and  so  more
generally  I  think  uh  yeah  maybe  yeah  I
don't  know  who  proved  it  uh  maybe  maybe
it  was  bot  schultza  but  if  also  if  you
have  like  a  proper
map  I  I'm  I'm  talking  about  the  apine
situation  here  but  okay  you  can  extend
it  to  schemes  but  if  you  have  a  proper
map  of  nean
schemes  uh  then  you  also  get  Des
sendable  so  that's  a  generalization  of
this  this  here  A  Proper  s  map  yeah
sir  um  and  more  generally  age
cover  I  mean  that's  kind  of  just  a
combination  of  this  and  and
that  uh  let's  say  finitely
presented  um  so  uh  and  maybe  I  okay  so
anyway  there's  a  huge  class
of  very  very  much  you  but  you  do  have  to
remember  in  like  cases  like  this  and  in
non-fat  cases  you  do  have  to  remember
The  Descent  involves  a  higher  higher
higher  tourus  and  so  on  so  so  yeah  all
of  these  kinds  of  things  we'll  go  to  uh
shet  covers  in  in  our  setting  there  okay
so  now  I  I'll  really  stop  sorry  for
going
over
yeah  so  in  these  two  versions  there's
one  is  is  taking  the  classical  one  and
the  other  is  two  with  condensed
structures  that  that  would  not  make  a
difference  for  see  whether  it's  a
shepper  or
not  well  it's  a  priori  possible  that  for
some  weird  reason  well  is
it
uh  a  priori  it's  possible  that  for  some
weird  reason  you  could  have  a  a  map  of
commuity  of  rings  where  you  go  into  that
world  it  becomes  a  street  cover  because
of  some  weird  condensed  modules
I  kind  of  we'd  be  surprised  if  it
actually  happened  in  practice  but  a  I
I'm  only  claiming  an  implication  that  if
you  have  a  street  cover  here  you  get  a
street  cover  there  right  uh  wait  no  no
it's  just  tensoring  yeah  oh  no  yeah  no
aior  it's  possible  it's
possible  that  no  that  sounds  it  sounds
very  strange  but  yeah  I  don't  know  the
question  was  whether  uh  whether  it's
equivalent  to  have  if  you  have  a  map  of
commutative  Rings  whether  it's
equivalent  for  it  to  be  descendible  in
the  ordinary  commutative  algebra  a
Matthew  sense  or  and  whether  the  image
under  this  funter  is  a  shable  map  a
street  cover  in  our  sense  so  I  think
it's  equivalent  is  it  equivalent  so
what's  the
argument  well  I  me  this  important  to
theil  this  kind  of  relative  sense  but
this  is  just  meaning  that
The  Descent  and  this  S  pro  object
aism  but  this  Pro  object  is  just  an  old
discreet  Pro  well  wait
but
uh
yeah  okay  yeah
thanks
yeah  so  yeah  what  Peter  was  pointing  out
is  there's  another  characterization  of
descend  ability  which  is  in  terms  of  on
the  level  of  just  the  Rings  like  a  b  b
tensor  a  b  and  so  on  you  get  some  first
of  all  you  have  to  have  descent  you  have
to  get  a  limit  diagram  but  you  but  then
you  have  to  get  even  more  you  have  to
get  a
a  very  stable  limit  diagram  it  has  to  be
a  pro  isomorphism  between  the  the  tower
you  get  from  this  co-al  thing  the  nindex
tower  and  uh  that  proobject  should  be
Pro  isomorphic  to  the  constant  Pro
object  a  and  in  that  condition  it  is
clear  that  when  you  uh  that  it's
independent  of  which  framework  you  put
it  in  because  the  pro  category  um
because  the  the  usual  D  of  R  sits  fully
Faithfully  inside  D  condensed  R  and
therefore  Pro  of  usual  D  of  R  sits  fully
Faithfully  inside  Pro  of  D  condensed  R
so
the  yeah  and  so  the  for  the  pro  thing
it's  enough  to  work  in  homotopic  categor
it's  enough  to  work  in
weaker  for  the  pro  thing  I
mean  I  don't  know  so  but  what  you  can  do
is  you  can  pass  to  the  fiber  and  then
you're  ask  you  want  to  know  about  a
tower  being  pro  zero  and  then  it's
enough
to  to  look  in  the  homotopy
category
okay  thank  you  again  for  your
attention  so  when  it  is  pro  zero  yeah
then  is  it  the  case  that  the  that  it  is
uniformly  pro  zero  that  is  because  there
is  some  finite  do  mean  that  it's  enough
for  any  stage  it's  enough  to  to  add  a
fixed  number  exactly  exactly  exactly  yes
in  Matthew  yes  and  it  works  and  it  works
with  simping  so  he  works  in  actually  a
very  general
setup  just  like  commutative  algebra
object  in  PRL  again  um  and  then
sometimes  he  specializes  but  for  the
general  discussion
yeah  um  but  he  only  he  only  looks  at  the
proper  case  so  for  him  every  every  map
every  so  he  works  with  his  base  category
is  always  some  very  general  thing  like  C
and  C  alge  PRL  a  presentably  symmetric
monoidal  thing  but  he  only  ever  then
considers  algebra  objects  in  C  and  he's
comparing  like  C  to  mod  a  and  he's
talking  about  descend  descend  ability
kind  of  in  this  sense  so  it's  the  he's
for  him  every  map  is  proper  in  our  sense
so
object  is  like  a  symmetric  monoidal
category  yeah  and  then  you
consider
a  is
a  is  an  algebra  is  a  ah  yes  yes  yes  then
you  take  a  and  algebra  in
C  ah  so  you  walk  in  this  setup  and  then
he  defines  the
cability  in
this  context  yeah  okay  so  it  specializes
also  to  the  condensed  world  but  it  it's
um  it's  only  discussing  the  the  proper
maps  not  the  arbitrary  shable
Maps
all  right  see  you  yep  see  you
Peter  can  I  ask  you  something  um  about
what  you  said  last  time  okay  oh  I  think
it  was  last  time  uh  they  mentioned  that
we  have  this  um  decomposition  oh  yes
formula  yeah  formula  and  and  so  it's
probably  not  true  that  we  have  this  for
all  analytics  STS  right  right  but  um  for
like  we  have  these  TS  of
perfect  do  you  have  I  mean  question  have
similar  intuition  for  like  class  of
objects  for  which  yeah  I  think  it  should
be  pretty  similar
yeah  okay  yeah  this  is  something  that
will  be  relevant  because  we're  going  to
talk  about
extending  this  right  now  we  have  a  six
funter  formalism  on  AF  and  we're  going
to  talk  about  extending  it  to  more
General  analytic  stacks  and  this  kind  of
condition  will  show  up
again
yeah  and  I  also  didn't  discuss  like
fully  faithfulness
of  was  easy  to  settle  here  but  for  more
comp  in  more  complicated  situations
fully  faithfulness  of  some  funter  from
classic  analytics  basis  to  our  analytics
tax  is  not  not  at  all  um  easy  to  settle
but  uh  yeah  similar  considerations  have
to  come  into  that  as
well  the  essential  problem  is  that  this
grow  de  topology  is  quite  inexplicit
like  usual  as  a  risku  tall  you  kind  of
have  a  good  handle  on  the  covers  because
they're  you  know  finally  generated  for
example  and  you  can  kind
of  give  standard  models  for  what  they
look  like  but  in  this  setting  it's  much
more  difficult  to  get  control  over  grot
and  de  topology  so  it's  harder  to  a
prior  prove  that  locally  something  weird
can't  happen  um  but  okay
yeah
\end{unfinished}