% !TeX root = AnalyticStacks.tex

\section{Stacks (Clausen)}

\url{https://www.youtube.com/watch?v=EEH_0QhrgEg&list=PLx5f8IelFRgGmu6gmL-Kf_Rl_6Mm7juZO}
\renewcommand{\yt}[2]{\href{https://www.youtube.com/watch?v=EEH_0QhrgEg&list=PLx5f8IelFRgGmu6gmL-Kf_Rl_6Mm7juZO&t=#1}{#2}}
\vspace{1em}

The topic of today's lecture will be stacks.

So far, we've discussed the theory and examples of ``analytic rings''. Next, we will explain how to use these to build geometric objects, the ``analytic stacks''. We already saw a clue of what phenomena we want to include in this world of analytic stacks in the previous lecture, namely the Tate curve over the gaseous base ring.

Today we won't give the precise definition of analytic stack, but will provide motivation from algebraic geometry. The paradigm here is you have commutative rings, and you want to think of these as describing some sort of basic geometric objects which are called affine schemes.
\begin{align*}
  \text{commutative rings} &\leadsto \text{affine schemes}\\
  R &\mapsto \Spec R
\end{align*}
These two categories are anti-equivalent. Then you want to allow yourself to glue these affine schemes together to make more general geometric objects. 

There are two aspects to this gluing:
\begin{enumerate}
  \item What gluings are allowed?
  \item How to identify the results of different gluings? More generally, what are the maps/what is the category we get from these formal gluings of affine schemes?
\end{enumerate}

A classic example is the projective line
\[ \P^1 = \A^1_+ \cup_{\G_m} \A^1_- \]
where we glue together a ``plus'' version of $\A^1$ and a ``minus'' version of $\A^1$ along the multiplicative group $\G_m$. But we want this object to have symmetries like $\PGL_2$, which don't respect this decomposition. So there has to be something said about what are the maps between different gluings.

It's better to think of (2) first: specify an ambient category containing the category of affine schemes, and then single out a full subcategory by specifying allowable gluings of affine schemes in the larger category. This is one way to describe a class of geometric objects in algebraic geometry.
\[\xymatrix@R=2em@C=1em{
\text{category} \ar@{}[rr]|-{\supseteq} \ar@{}[dr]|-{\rotatebox{-45}{$\supseteq$}} && \text{affine schemes = $\text{CRing}^\op$}\\
& \text{single out $\{*\}$} \ar@{}[ur]|-{\rotatebox{45}{$\supseteq$}}
}\]

In the classical approach to schemes, you take the ambient category to be locally ringed spaces, so a commutative ring gives you $\Spec R$, and then a scheme is a locally ringed space which is locally isomorphic to some $\Spec R$. So the allowable gluings between affine opens are gluings along open subsets, in terms of viewing $\Spec R$ as a locally ringed space.

There's a more modern approach which says we shouldn't try to be clever about choosing an ambient category---we don't have to find the concept of a locally ringed space. We can just formally build an ambient category based on the category of commutative rings, and then work there. The category where you're allowed to glue arbitrary affine schemes is called the \textbf{category of presheaves} on affine schemes, $\Psh(\Aff)$. This is the universal category in which you can glue; more formally, it's the initial category-with-all-colimits with a functor from $\Aff$, and giving a colimit-preserving functor out of $\Psh(\Aff)$ is the same as giving a functor out of $\Aff$. There's a set-theoretic technicality here, so caution is needed; we'll discuss that later. The category $\Aff$ is not a small category, so one has to be careful when taking functor categories out of it.

Now you've formally allowed yourself to glue, but you haven't explained how you should identify the results of two different gluings. If you pass to sheaves for some Grothendieck topology, that explains how to identify gluings; more generally, how to map between two of these formal gluings. The more covers you put in your Grothendieck topology, the more maps you're going to be making, which might not be so evident from the perspective of the cover. For example, the automorphisms $\PGL_2$ of $\P^1$ may not be apparent. But you don't want to add too many elements to your cover, or else you might destroy information by identifying too many things; so there's a delicate choice to be made in the Grothendieck topology.

For example, you could take the Zariski topology, and then schemes is a full subcategory of Zariski sheaves on affine schemes such that they're locally representable, in the sense of open covers.
\[
  \text{schemes} \subseteq \Sh^\Zar(\Aff)
\]
You can define the notion of an open inclusion in $\Sh^\Zar(\Aff)$ just by reduction to the case of $\Aff$: a map $X\to Y$ in $\Sh^\Zar(\Aff)$ is an open inclusion if for every $\Spec A\to Y$, the pullback
\[\xymatrix{
  \Spec B \ar[r] \ar[d] \pullback & X\ar[d]\\
  \Spec A \ar[r] & Y
}\]
is an open inclusion $\Spec B\to \Spec A$ in $\Aff$ (this includes the condition that the pullback is in $\Aff$).

This is the perspective we're going to take on defining analytic stacks from analytic rings. However, we don't want to take schemes as a model, because more general geometric objects come up and are relevant, namely \emph{stacks}.

\subsection{Why stacks?}
% \subsection{\yt{12m13s}{Why stacks?}}
Often moduli spaces in algebraic geometry have the form $X/G$, where $X$ is a variety and $G$ is a group variety or scheme acting on $X$. So we have a quotient of some variety by some group of automorphisms.

\begin{example}[\yt{12m48s}{Moduli of elliptic curves}]
  Consider the moduli stack of elliptic curves $\M_{\text{ell},\Z[\frac16]}$, where we invert $6$ to simplify the discussion. Then this is $X/G$ with
  \begin{align*}
    X &= \Spec\Z\left[\tfrac16\right][A,B][\Delta^{-1}]\\
    G &= \G_m
  \end{align*}
  where $\Delta$ is the discriminant.

  This parameterizes elliptic curves with affine equation in Weierstrass form
  \[ y^2 = x^3 + Ax + B. \]
  It then turns out that the only isomorphisms between two elliptic curves given by the above equation are given by scalar multiplication with certain weights on the $x$ and $y$ variables, and that gives the $\G_m$ action.

  There's also many other ways of presenting the same stack. For example, you can add level structure and then mod out by a finite group,
  \[
    \M_{\text{ell},N} / \GL_2(\Z/N)
  \]
  If $N$ is sufficiently large, $\M_{\text{ell},N}$ will be represented by a variety, and then you quotient out by the finite group $\GL_2(\Z/N)$. The Grothendieck topology that we choose should be such that this is identified with $\M_{\text{ell}}$ our category, so we should at least allow \'etale covers into the story for this, and that is indeed the classic choice.

  These quotients exist in the category of schemes, and they give $\A^1_{\Z[\frac16]}$, the so-called ``$j$-line'', implemented by the $j$ function. However, this is not a good quotient. One way of measuring that is: on the moduli stack of elliptic curves, there's a natural line bundle $\omega$,
  \[ \omega = \Lie(E)^*, \]
  which is the dual of the one-dimensional vector space of tangent vectors at the origin. In other words, it's the cotangent space of the universal elliptic curve.

  This means you can write down a line bundle on $\Spec\Z\left[\tfrac16\right][A,B][\Delta^{-1}]$ which is equivariant for the $\G_m$-action, or a line bundle on $\M_{\text{ell},N}$ which is equivariant for the $\GL_2(\Z/N)$-action. However, this line bundle doesn't descend to the quotient $A^1$, so it's a bad quotient in the sense that you can have equivariant objects on the top, but they don't come from something on the bottom. Even more basic, the universal elliptic curve over $\M_{\text{ell}}$ can't be defined over $\A^1$.
\end{example}

The problem in the above example is that the action is not free. There are some elliptic curves with extra automorphisms, and because the action isn't free, the naive quotient in schemes is collapsing too much. The solution is to take the quotient in a more refined (2-)category: (sheaves of) groupoids.

Working in groupoids, there's a notion of groupoid quotient
\[ X \gq G \]
where:
\begin{itemize}
  \item the objects of $X\gq G$ are the elements of $X$
  \item $\Hom_{X\gq G}(x,y) = \{g\in G \mid gx = y \}$
\end{itemize}

There's always a map $X\to X\gq G$, and the fibers are all isomorphic to $G$, where ``fiber'' means a pullback
\[\xymatrix{
  X_x \ar[r] \ar[d] \pullback & X \ar[d]\\
  \{x\} \ar[r] & X \gq G
}\]
So this sort of allows us to pretend that every action is free; the map $X\to X\gq G$ is always like the total space of a $G$-bundle.

The only trick is you have to interpret fiber product in the sense of 2-categories: in a pullback diagram of groupoids,
\[\xymatrix{
  A \times_C B \ar[r] \ar[d] \pullback & B \ar[d]^-g\\
  A \ar[r]_-f & C
}\]
an object of $A\times_C B$ consists of a triple $(a,b,\gamma)$ where $a\in A$, an object $b\in B$, and $\gamma$ is \emph{an isomorphism} between $f(a)$ and $g(b)$ in $C$.

So it's a way of taking a quotient such that you don't really care about the difference between a free action and a non-free action. And it's such that you formally have that line bundles on $X\gq G$ are the same as equivariant line bundles on $X$ and so on and so forth.

This leads to notions such as Deligne-Mumford stack, or more generally, Artin stack. These are all full subcategories of \'etale sheaves on affine schemes.
\[
  \text{Deligne-Mumford stacks, Artin stacks}
  \subseteq
  \Sh^\et(\Aff)
\]
In the discussion of schemes, we had Zariski sheaves that we asked to be locally representable, where ``locally'' is in the sense of open covers. We also could have used \'etale sheaves, it doesn't change the resulting category of schemes. For Artin stacks, you more or less require that you have a smooth cover by affine schemes, and maybe some technical things you want to put in there as well.

The basic example is, let $X$ be an affine scheme, for simplicity let's say over some base $S$. Let $G$ be a smooth group scheme over $S$ which acts on $X$. Then you pass to the quotient in the stacky sense, $X \to X\gq G$.

So these are more general geometric objects, and this is really good for moduli theory. The theoretical justification for that is Artin representability theorem which gives concrete criteria for when a functor is represented by an Artin stack. But it's still constrained by the need for a smooth cover; in particular, this means finitely presented. So if you had some non-finite type group scheme acting on something, you wouldn't necessarily be able to take the quotient in Artin stacks.

Now you might say, why do you care? It turns out there are many $X\in\Sh^\et(\Aff)$ which are geometrically relevant, but are \textbf{not} Artin stacks.

\begin{example}[\yt{23m40s}{Formal schemes}]
  \label{ex:15-formal-schemes}
  Let $R$ be a noetherian ring, and $I\subset R$ an ideal of $R$. We want to consider the formal spectrum $\Spf(R_I)$. There are different options as to how to encode this thing, and what it should mean. One we've already discussed is Huber's theory, which includes formal schemes as an example, and is based on viewing $R$ as a topological ring. Grothendieck's theory of formal schemes is also based on viewing $R$ as a topological ring, but localizing along a smaller subset than in Huber's theory.
  
  Another way of looking at it is it's just the union of all the $n$th order infinitesimal neighborhoods of $\Spec(R/I)$ inside $\Spec(R)$,
  \[ \Spf(R_I) = \bigcup_n \Spec(R/I^n). \]
  So giving some (finite-type) data on a formal scheme should be the same as giving compatible collections of data at all of these finite levels. For example, vector bundles on $\Spf(R_I)$ should just give a compatible collection of vector bundles in the usual sense (finitely-generated projective modules),
  \[ \Vect(\Spf(R_I)) = \lim \Vect(R/I^n). \]
\end{example}

This is a very different gluing from what you think about with schemes, and even when you think about Artin stacks. There are no smooth covers in sight here; instead, you're taking some union of infinitesimal thickenings and getting a new object.

There are other examples: for some moduli problems, you really are quotienting by an infinite-dimensional group. One example that's dear to the hearts of both homotopy theorists and number theorists is the moduli of one-dimensional formal groups, the \textbf{Lubin-Tate space}. In this case the group you have to mod out by is coordinate changes on a one-dimensional formal scheme, and then there's infinitely many coefficients you have to specify. So you have an infinite-dimensional group you have to mod out by, and that doesn't fit into the framework of Artin stacks. 

This suggests using a different Grothendieck topology, say fpqc instead of \'etale, if you want to accommodate infinite type phenomena in your covers. But in addition to these, there's also a very remarkable class of examples started by Carlos Simpson.

\begin{center}
  Simpson: ``Every linear algebra category is $\QCoh(\text{some stack})$''
\end{center}

Simpson didn't literally say this, and it's much too strong to be true; this is just an slogan interpretation of his work. His work gives lots of examples of this phenomenon, where you have a natural linear-algebraic category, and then it turns out you can write down some stack whose quasicoherent sheaves are that category. There are fun examples of this already in the world of Artin stacks.

\begin{example}[\yt{28m22s}{Representations}]
  Let $G$ be an algebraic group over a field $k$. Then a special case of a non-free quotient is a point $*=\Spec k$ with an action of $G$, so
  \[ BG = * \gq G. \]
  A quasicoherent sheaf on $BG$ should be a $G$-equivariant quasicoherent sheaf on the point; but a quasicoherent sheaf on a point is just a $k$-vector space, and the $G$-equivariance exactly means you have a representation of $G$. So we get
  \[ \QCoh(BG) = \operatorname{Rep}_G(\text{$k$-vector spaces}). \]
\end{example}

\begin{example}[\yt{29m20s}{Filtered objects}]
  We can also consider $\A^1/\G_m$ for the natural action of $\G_m$ on $\A^1$ by scalar multiplication. This is a funny stack, because there's an open locus in this stack, corresponding to a $\G_m$-invariant open locus in $\A^1$, namely $\G_m$ itself. On that open locus, you're taking $\G_m/\G_m$, which is a point. So this has a point as an open subset, and the closed complement is $0/\G_m=B\G_m$. In this case, (flat) quasicoherent sheaves are given by filtered $k$-vector spaces.
  \[
    \QCoh^{\text{flat}}(\A^1/\G_m) = \{\text{$k$-vector spaces equipped with a $\Z$-indexed filtration}\}
  \]
  The restriction to flat modules is just to stay in the abelian world rather than going derived.

  Note that at the origin, we have
  \[
    \QCoh^{\text{flat}}(B\G_m) = \{\text{$k$-vector spaces equipped with a $\Z$-indexed grading}\}
  \]
\end{example}

These are Artin stacks, so there's nothing exotic there. Here's a more interesting example.

\begin{example}[\yt{31m11s}{de Rham stack}]
  \label{ex:15-dR}
  Let $k$ be a field of characteristic $0$, and let $X/k$ be a smooth variety. Then you can form, and Simpson did, what's called the \textbf{de Rham stack} of $X$. There is a presentation
  \[ X \epi X^\dR \]
  but this is not quotienting out by a group action, it's just some by equivalence relation. The equivalence relation in question is that which identifies two points if they're ``infinitesimally close'' to each other. Equivalence relations are supposed to live in the product $X\times_k X$, and what you do is you take this product and you formally complete along the diagonal,
  \[ (X\times_k X)_{\widehat X}. \]
  Then $X$ mod this equivalence relation gives $X^\dR$. The formal completion should be taken in this sense of Example \ref{ex:15-formal-schemes}, i.e.\ the union of the different scheme structures that are available on the diagonal as a closed subset.

  (The definition of $X^\dR$ makes sense in arbitrary characteristic, but some things we're about to say will not be true in positive characteristic.)

  What are quasi-coherent sheaves, let's say vector bundles, on $X^\dR$? This is the same thing as vector bundles on $X$ equipped with some kind of descent datum, but that descent datum exactly amounts to a flat connection.
  \[ \Vect(X^\dR) = \{\text{vector bundles on $X$ + flat connection}\} \]
  That's Grothendieck's interpretation of what is a flat connection. It's exactly giving descent, identifying infinitesimally close points.
  
  There's also cohomology of the structure sheaf. This gives de Rham cohomology, which is the natural notion of cohomology in the world of vector bundles with flat connection.
  \[ R\Gamma(X^\dR, \O_{X^\dR}) = R\Gamma_\dR(X/k) \]

  This is not even close to being an Artin stack either, and for different reasons from the moduli of formal groups. Here we're modding out by some formal scheme giving an equivalence relation.
\end{example}

\begin{example}[\yt{35m2s}{Prismatization}]
  More recently, Bhatt-Lurie \cite{APC,BLPrismatization} \citeme\textcolor{red}{$F$-gauges} and Drinfeld \citeme define stacks whose $\QCoh$ capture coefficient systems for various $p$-adic cohomology theories in characteristic $p$ or mixed characteristic. For example, there's a stack capturing de Rham characteristic $p$, but it is not the one of Example \ref{ex:15-dR}. You have to use the divided power envelope of the diagonal instead of the formal neighborhood of the diagonal.
  
  So there are stacks capturing prismatic cohomology, de Rham cohomology, and crystalline cohomology, as well as filtered versions of these. Moreover, the comparison theorems in prismatic cohomology between all of these various cohomology theories can be explained, so to speak, ``geometrically'' in terms of the stacks. (It's arguable how geometric these kinds of stacks are :)
  
  Maybe the better way to say it is that, a priori these comparison theorems are about comparing linear algebra categories, e.g.\ vector spaces. But it turns out there's a more fundamental explanation, which is that you have an isomorphism of \emph{stacks}. You then deduce comparison theorems of cohomology theories by passing to quasicoherent sheaves. So you promote a comparison of cohomology theories to an isomorphism of stacks.
\end{example}

I want to give another example of this phenomenon. We've seen de Rham cohomology in characteristic zero, some $p$-adic cohomology theories as well, but what about Betti cohomology? Here's a fun example which actually has quite a bit of relevance for the course, so that's why I'm going to mention it.

\begin{example}[\yt{38m6s}{Betti stacks}]
  ``Betti cohomology'' is the algebraic geometers' term for, if you have a complex variety, then you take singular cohomology or sheaf cohomology with constant coefficients on the underlying topological space with the analytic topology. For example, if you have a compact Hausdorff space $S$, then I claim that you can make a stack.
  
  How do you do it? You use the old familiar idea: you find a surjection from a profinite set $T$, and then you have some fiber product $T\times_S T$.
  \[\xymatrix{
    T\times_S T \ar@<-.5ex>[r] \ar@<.5ex>[r] & T \ar@{->>}[r] & S
  }\]
  Since $T\times_S T$ is a closed subset of a product of two profinite sets, it'll also be a profinite set. Write $T_0\defeq T$ and $T_1\defeq T\times_S T$.
  
  So your compact Hausdorff space $S$ is a quotient of an equivalence relation in the category of profinite sets. We can then apply $C(-,\Z)$ (continuous $\Z$-valued functions) followed by $\Spec$ to get a groupoid in the category of schemes, in fact in the category of affine schemes. We define the Betti stack of $S$ as the quotient of this equivalence relation in the category of sheaves for the fpqc topology on affine schemes. 
  \[\xymatrix{
    \Spec C(T_1,\Z) \ar@<-.5ex>[r] \ar@<.5ex>[r] & \Spec C(T_0,\Z) \ar@{->>}[r] & S^\Betti
  }\]
  You could also replace $\Z$ with a commutative ring $k$.

  What are quasi-coherent sheaves on Betti stacks? These are just usual sheaves of abelian groups on the topological space $S$.
  \[ \QCoh(S^\Betti) = \Sh(S,\Ab) \]
  That's a fun exercise. So, coherent cohomology on the Betti stack $S^\Betti$ is just usual topological cohomology of the topological space $S$. More generally, we can embed condensed sets into stacks via the above procedure, using the presentation of a condensed set via profinite sets.

  \note{(something I didn't hear)}

  What you have to check to see that is that, if you have a surjective map of profinite sets, then it goes to a faithfully flat map on the level of continuous functions. That's not that hard to do: $C(T_i,\Z)$ are filtered colimits of continuous functions of finite sets, which as rings are copies of products of $\Z$. In fact, for \emph{any} map of profinite sets $T_1\to T_0$, the induced map $C(T_0,\Z)\to C(T_1,\Z)$ is flat, and then if it's surjective it's faithfully flat. You also need to check that fiber products in profinite sets correspond to relative tensor products; this follows again by a reduction to finite sets.
\end{example}

So now we're faced with this somewhat baffling array of different stacks, some of which don't resemble Artin stacks in the least. But we want them because they're convenient ways of encoding different linear-algebraic and geometric phenomena. 

\textbf{Question:} how to define a reasonable subcategory of $\Sh^\fpqc(\Aff)$ containing all these examples?

\textbf{Answer:} $\Sh^\fpqc(\Aff)$ (modulo set theory)

In other words, it's not clear that there's any other answer to this question than the entire category $\Sh^\fpqc(\Aff,\An)$. So there is content in the answer, but it didn't necessitate introducing anything wasn't present in the question. 

\subsection{Desired examples of analytic stacks}
% \subsection{\yt{47m44s}{Desired examples of analytic stacks}}
This is also the approach we will take in defining analytic stacks. We will define a Grothendieck topology on $\AnRing^\op$ and then take sheaves with respect to it (again modulo set theory). Before getting into the details of exactly which Grothendieck topology, and these set theoretic technicalities as well, let's see what kind of phenomena we want to capture, so what the examples should be. In some sense, we've already seen some.

\begin{example}[\yt{48m6s}{Adic spaces}]
  We certainly want that any adic space in the sense of Huber should give rise to an analytic stack. We already explained how the basic ingredient in adic spaces, namely Huber pairs $(R,R^+)$, give rise to analytic rings, and we explained something about how the formalism of analytic rings lets you glue. But we didn't quite discuss how you can use that to then glue these analytic rings together to get some kind of analytic stack.

  We saw that at least you can localize the category of modules over that analytic ring along Huber's spectrum, but we didn't quite discuss how you can use that to then glue these analytic rings together to get some kind of analytic space. But we certainly want the kind of gluing that shows up in Huber's theory, gluing along rational open subsets in the topology defined by a basis of rational opens, we want that kind of gluing to be allowed and to give you an analytic space.
\end{example}
  
\begin{example}[\yt{49m8s}{Complex analytic spaces}]
  We also want any complex analytic space to give an analytic stack, say over $\C^\gas$ or $\C^{\liq_p}$. So the kind of gluing allowed should also incorporate gluing along open subsets in complex analytic geometry.
\end{example}

\begin{example}[\yt{49m53s}{Algebraic stacks}]
  Another even more basic thing is we want the world of analytic geometry to be a generalization of the world of schemes, and even of algebraic stacks (in some sense---maybe not precisely the fpqc topology discussed above, but a slight modification of that). These should live over the universal base $\Z$, with trivial analytic structure ($\Mod(\Z)=\Cond\Ab$). So universally, over any analytic ring, if you have some algebraic object, you can get an analytic object.
\end{example}
  
\begin{example}[\yt{51m4s}{Banach rings}]
  Any Banach ring $R$ should also give rise to an analytic stack. This in some sense matches Berkovich's theory, in the same way that the affinoid analytic stacks coming from pairs $(R,R^+)$ match Huber's picture. There's a small interesting tidbit here, which is that the stack that we'll assign to a general Banach ring will actually not be affinoid, it will really be a stack. So, if you take $\Z$ with the usual archimedean norm, it will go to an actual stack that's not affinoid; instead, it's a stack which in some sense corresponds to $\M(\Z)$, the Berkovich spectrum of $\Z$, so that's a fun little twist.
\end{example}

\begin{example}[\yt{52m22s}{Coefficient systems}]
  As above, there should be analytic stacks whose $\QCoh$ capture various coefficient systems for cohomology (sheaves of abelian groups, vector bundles with connection, prismatic $F$-gauges etc.)
\end{example}

\begin{example}[\yt{52m58s}{Tate curve}]
  We want to be able to define the Tate curve as well as its uniformization over $\gasopen$. We also want to have machinery to prove it's algebraic. So we have this curve that we define via uniformization. We take the analytic $\G_m$ and we quotient by the multiplication by $q$, and we get something which Peter argued was a smooth proper curve, and it has an identity section. So then, if you have some Riemann-Roch theorem, then you can see that you have a projective embedding. And if you have some GAGA theorem, then you'll be able to see that it has to be algebraic. So we want Riemann-Roch, and we want GAGA.
\end{example}

\begin{example}[\yt{52m20s}{Comparison theorems}]
  On the theme of GAGA, we want that various linear algebraic comparison results should promote to isomorphisms of stacks. GAGA is one such.
  
  GAGA is a general principle which applies in different contexts. But for example, in the world of complex analytic spaces, it says that if you have a proper algebraic variety $X$ over the $\C$, so it has an analytification $X^\an$ which is compact, then it says that coherent sheaves and their cohomology in the algebraic and in the analytic sense agree.
  
  So that's a question about making a comparison between two linear algebraic categories. It's saying algebraic coherent sheaves are the same as analytic coherent sheaves. And one thing we would like our formalism to do is to promote that to an isomorphism of stacks.

  Another example, again in the complex analytic context, would be the comparison between Betti cohomology and de Rham cohomology. This should also promote to an isomorphism of stacks.
\end{example}

\subsection{Analytification}
% \subsection{\yt{56m48s}{Analytification}}
Recall that adic geometry in the solid context really got started once we noticed that there's a nice subset of $\A^1_{\Z^\solid}$, namely the closed unit disc, which we were thinking of as open. Or you could think of it in terms of its complement. or the translation of that back to the origin. In the end, algebraically speaking, this came from an idempotent algebra
\[ \Z\psr T \in \Solid_{(\Z[T],\,\Z)} \]
Once we had this idempotent algebra, then we could move it to infinity via the change of variables $T\mapsto T^{-1}$, and then that gave us the closed unit disc, and then that let us tie into the $(R,R^+)$ theory, where we could ask that the elements in $R^+$ actually land in the closed unit disc as opposed to just being maps to $\A^1$.

Similarly, over $\gasopen$, you can define a ``subset'' of $\A^1_{\gasopen}$, corresponding to an idempotent algebra of ``functions which are convergent in some (unspecified) disk around the origin''. So it's germs of functions defined at the origin, so to speak. Formally, it's going to be the filtered colimit
\[ P[q^{-1}] \defeq \colim\left(P \xra q P \xra q P \xra q \dotsb \right) \]

In the solid case, $P$ was itself idempotent and turned into $\Z\psr T$. In the gaseous setting, $P$ is not idempotent in $\Mod_{\Z[T]}(\Mod_{\gasopen})$, but it becomes idempotent after we take this colimit to shrink the open unit disc down to the origin.

% \begin{remark}
% Sorry, we write sheaves and what are they valued in? One group? Points, yeah. So, let's say group points, but of course, I mean, you might as well, yeah, I mean, but all the examples are just groupoids. And in fact, many of the examples, even though they're called stacks, are actually just sheaves of sets. So, like Simpson's Duram stack, for example, is just a sheaf of sets. It's a Duram space somehow, but anyway, somehow they call these things stacks anyway.

% But yeah, so it's going to be similar to the situation with analytic rings and where we had sometimes theoretically we had derived things that were animated or whatever and so on. Theoretically we had them, but in practical examples, they didn't really show up. And similarly here, theoretically we're going to allow sheaves with values in anma, but in practical cases, they'll be at most groupoids. I mean, that's, yeah.
% \end{remark}

The idempotent algebra $P[q^{-1}]$ satisfies many of the same properties as $\Z\psr T$, and it again lets us import Huber's theory of pairs to this context. What you do is you take this idempotent algebra functions of germs at the origin in the affine line, you move it to infinity and you get germs at infinity.

Let $R$ be of finite type over $\gasopen(*)$, and assume that $\Z$ is bounded in $R$. Then you get an analytic ring structure on $R$, namely $(R,\Mod_R(\Mod_{\gasopen}))$. Every element $f\in R$ gives a map
\[ \AnSpec(R)\xra f\A^1, \]
and we pass to the ``subset'' where all such maps land in the locus ``away from $\infty$''. This gives a new analytic space $\Spec(R)^\an$, the ``analytification'' of $\Spec(R)$.

\begin{example}
  When $R=\Z[T^\pm]$, we get $\G_m^\an$. After base change to a ring where $2$ is bounded, this is the object of the previous lecture, the thing we quotiented out by to get the Tate curve.
\end{example}

So now we have two different contexts (solid and gaseous) in which you can import Huber's theory of pairs into the world of analytic spaces. It turns out that they can be glued together, and so can the idempotent algebras. First notice that the gaseous theory makes sense over $\Z[q]$: all we had to do to define the gaseous theory was we had to write down the endomorphism $1-qt$ of $P$, and then ask that it become an isomorphism in our theory. And to do that, you didn't need to require $q$ to be topologically nilpotent. So this gives an analytic ring $\Z[q]^\gas$.

\begin{itemize}
  \item if we set $q=0$, we get the uncompleted $\Z$ theory
  \item if you set $q=1$, you get the solid theory, $\Z^\solid$
  \item if you require $q$ to be topologically nilpotent and a unit, you get the gaseous theory, $\gasopen$
  \item if you work away from the locus where $q$ is topologically nilpotent, then you are working over a theory where you force both $q$ and $q^{-1}$ to be gaseous, and that implies that $1$ is gaseous, which means you're in the solid theory, so you get $\Z[q^\pm]^\solid$.
\end{itemize}

Then we can define a certain quotient of
\[ \Spec(\Z[q^\pm]^\qgas) \]
which more or less parameterizes choices of a notion of ``analytification''. So, anytime you have this variable $q$ which you've declared to be gaseous, you can form this colimit and you'll get an idempotent algebra at zero, and you can move it to infinity, and then you get a notion of analytification, as discussed.

However, different choices of $q$ can give rise to the same thing. If two choices of $q$ differ by a bounded unit, so something in $\G_m^\an$, then you'll get the same algebra. So the quotient we want is 
\[ \Spec(\Z[q^\pm]^\qgas)/\G_m^\an \]
where $\G_m^\an$ acts by multiplication on $q$.

This is a different role of stacks in the theory. Here we have a stack which is a quotient of ($\AnSpec$ of) an analytic ring by an equivalence relation, and which is in some sense parameterizing choices of analytic geometry over a given base ring. Let $R$ be an analytic ring with a map
\[ \Spec R^\tri(*) \to \Spec(\Z[q^\pm]^\qgas)/\G_m^\an, \]
which we call a ``gaseous structure'' on $\Spec R^\tri(*)$ (and again assume $2\in R$ is bounded). Then we get two functors from $R^\tri(*)$-schemes to analytic stacks over $R$, plus a natural transformation.

Say $X=\Spec A$, where $A$ is an $R^\tri(*)$-algebra. We can view this as an analytic stack over the uncompleted $\Z$, and then you can base change that to $R$ to get $X_R\defeq X\times_\Z R$. But then you also have a subset
\[ X^\an_R \xra\subseteq X_R, \]
the analytification over $R$, given by requiring that all the functions land in the part of $\A^1$ that's away from $\infty$. Now there's a general theorem.

\begin{theorem}[\yt{1h18m18s}{GAGA}]
  If $X\to\Spec(R)$ is proper (and finitely presented), and every element $f\in R(*)$ is bounded, then $X_R^\an \to X_R$ is an isomorphism.

  \note{some discussion between Peter and Dustin about whether finitely presented is really necessary. It might not matter for GAGA, but could for some other things.}
\end{theorem}

\begin{remark}
  This implies completely formally that $D(X^\an_R)=D(X_R)$, which is some form of GAGA. Non-formally, maybe with some more conditions on $R$ (but satisfied in practice), this implies that $\Vect(X^\an_R)=\Vect(X_R)$, which is classical GAGA. 

  Basically you can do algebraic geometry over the uncompleted $\Z$ theory, and then that means you can do algebraic geometry over any analytic ring just by base change.
\end{remark}

One of the things that maybe I should have been emphasizing before launching into this whole discussion is that over the completed $\Z$ theory, you can basically do algebraic geometry. And then that means you can do algebraic geometry over any analytic ring just by base change. For example, we were considering $\A^1$ over an arbitrary base ring. It's not analytified yet, but when you have extra structure on your analytic ring, then that picks out a choice of what it means to analytify an algebraic variety.

All the classical GAGA theorems are special cases:
\begin{itemize}
  \item complex-analytic GAGA after Serre (using $\C^\gas$ or $\C^{\liq_p}$)
  \item Grothendieck's formal GAGA: \note{look up statement of formal GAGA}if you have again complete noetherian ring $R$, and a proper scheme over it, then coherent sheaves on that is the same thing as coherent sheaves on the formal scheme you get by formally completing or, in other words, compatible collections of coherent sheaves on all the various $n$ potent thickenings there.

  In terms of Huber pairs, you would work over
  \[ \Spa(R^\land_I, R^\land_I) \to \Spec(\Z^\solid) \]
  so $\Spa(R^\land_I, R^\land_I)$ inherits the notion of analytification from $\Z^\solid$ based on the closed unit disc.
  
  \item non-archimedean GAGA over $\Q_p$ or any complete non-archimedean field. For this you would take your analytic ring to be $\Q_p^\solid$. But you don't put the gaseous structure on it which factors through the map to $\Spec\Z^\solid$, rather you put the gaseous structure on it which corresponds to
  \begin{align*}
    \Spec(\Q_p^\solid) &\to \Spec(\Z[\hat q^\pm]^\qgas)\\
    q &\mapsto p
  \end{align*}
  That's the one that picks out the notion of analytification that corresponds to usual analytification of algebraic varieties over your non-archimedean field.
\end{itemize} 

\begin{remark}
  You could try to use the other gaseous structure on $\Spec\Q_p^\solid$ where you factor through $\Spec\Z^\solid$. But GAGA won't apply in this setting, since then $1/p$ will not be bounded, and if you make it bounded in the solid setting you'll kill everything.

  Nonetheless, there is still a different gaseous structure on $\Q_p^\solid$ obtained by factoring it through $\Z_p^\solid$. In a sense the above gaseous ring structure corresponds to some kind of overconvergent version of rigid geometry, while the gaseous ring structure that factors through $\Z_p^\solid$ corresponds to usual rigid geometry.
\end{remark}

\subsection{\ufs Addressing set-theoretic technicalities}
\begin{unfinished}{1:29:22}
things that maybe I should have been
emphasizing before launching into this
whole discussion is that over the
completed Z Theory you you basically you
can do algebraic
geometry
um and then that then that means you can
do algebraic geometry over any analytic
ring just by just by base change so
that's um so that's kind of like
considering the apine line like I was
considering the apine line for example
over an ariary base ring it's not it's
not analytify yet but when you have
extra structure on your analytic ring
then that uh then that picks out a
choice of what it means to analytify
also in algebraic
variety um and
so
uh and I'm also conf exit are fine or
yeah when I was describing kind of
explicitly what it is I was looking at
the case where X is apine but in general
both things glue and so you then then
you can Define uh you can doesn't need
to be it doesn't need to be Aline no
yeah
um so so all the all classical Gaga
theorems
uh are special
cases so
there's um so for example groen's formal
Gaga well the complex analytic
Gaga um so Sarah's original Gaga um take
for example C gaseous or P liquid or
whatever you
like
um uh there's also formal
Gaga uh which is you know if you have a
a again complete nean ring and then you
have a a proper a proper uh scheme over
it then coherent sheaves on that is the
same thing as coherent sheaves on the
formal scheme you get by formally
completing or in other words compatible
collections of coherent sheaves on all
the various n potent thickenings there
um then you would take a spff well maybe
I should say
Spa so in terms of Huber pairs you would
work you would work over this um which
lives over a solid
Z and therefore inherits the notion of
analyics based on
the the closed unit
dis
um it also includes like non-
archimedian GMA
Gaga uh say over I don't know over QP or
any complete non- archimedian field so
there you would take your analytic ring
you would take to be a spec of say QP
solid
um but you wouldn't put the gaseous
structure on it which factors through
the map to spec Z rather you'd put the
gaseous structure on it which
corresponds to mapping topc of Z q hat
plus or minus one Q
gasas uh where Q goes
to uh Q goes to P say so St say standard
choice of topologically mil potent
unit um that's the one that picks out
the notion of analy ification that
corresponds to usual anal analy
ification of algebraic varieties over
your non- archimedian field um you could
have also chosen the other gashes
structure on this guy where you factor
through uh spec Z solid um and that
would give you a different statement of
Gaga in fact a different theorem so yeah
so we're
Mark and three could also
choose
uh the solid
z uh gaseous
structure on uh on solid
QP that gives a an a priori different
Gaga
theorem oh the the analy ification of A1
will then just be the open unit or the
closed unit
dis so um yeah so there's so let me
illustrate H that cannot
beis it has
what
sorry dation
sorry1
ofp
canale so they must
also1 so iation of A1 in that context is
oh wait ah you're right you're right
you're right I'm sorry you're right I'm
sorry yeah that's that that open your
dis is when you just force T to be
analytic but there's no reason
why
uh that should also Force something like
uh T over P to be analytic yeah no ah
same Gaga thank
you yeah I miss uh I I was not the same
guys it's
just no it is the same isn't it because
you you would
make you should always assume that your
whole
ah
you're right I forgot that Axiom shoot
yeah so I'm sorry you need
ah oh boy than uh thanks again so you
need
uh to
assume uh that every element in here is
is bounded
uh
uh yeah
so I was kind of making the same mistake
again over and over um yeah I'm sorry uh
so you don't get a Gaga statement at all
in this context because
um one over p uh is never going to be if
you try to make one over P bounded in
that sense then you're just going to
kill everything so in in the solid QP
setting but nonetheless um there's a
point I wanted to make which is that
there is still a different gaseous
structure on solid QP obtained by taking
um uh obtained by factoring it through
solid zp
and in a sense the difference between
this gaseous ring structure and this
gaseous ring structure is sort of the
difference between usual rigid geometry
and um and some kind of overon
convergent version of rigid
geometry
um
yeah uh
okay
so is there something weaker than
n something weaker than n could you
explain the
question a result that would put uh
identify x r with some
less uh
regular could you explain uh okay maybe
okay yeah this assumption that this uh
bound it yeah yeah
so what is the issue if it's if you
don't assume that it's empty like the if
you don't assume that everything in here
is bounded then when you try to do the
analyics that I described then in
particular you're always forcing all of
the scalers IE elements of of here uh
you're forcing them to be bounded anyway
so if you don't if you don't have this
assumption then your analyics changing
your your base so we have so a special
case is when you take x equals Spec R
and you try to analytify that what this
assumption is saying is basically that
then it doesn't change it like specr is
its own
analytify ring where it's where this
thing has been forced to be bounded but
in the case of solid QP if you try to
force with that with that with that
gaseous structure factoring through
solid Z if you try to force all of the
scalars to be bounded um you're just
going to get the zero ring because
yeah yeah
um
right
yeah um so yeah so that was the
uh yeah x equals specr was giving a
problem there my
apologies
um
okay
uh right
so that was um kind of explaining one
example of where a linear algebraic
comparison result uh can be promoted to
an isomorphism of
stacks um more generally we want
relations um so
relations between the various kinds of
stacks uh various
examples so for already in Peter's talk
he had this map from the T
curve uh with parameter Q to the
topological Circle
so and we want this to make sense as a
map of analytic stacks
so where I already explained how such a
thing is an algebraic stack and
algebraic Stacks will embed into
analytic Stacks even over the the
initial analytic ring um and then we
have the Tate curve and so this kind of
analytic space should map to that kind
of analytic space but also if x is a
complex analytic
space uh then you want
um X to map
to uh its underlying topological Space X
of c
um and if say R is a banak
ring then you want its well sort of its
liquid spec uh to M map to the burkovich
Spectrum
um
so
uh yeah so there's all these we're going
to have all these various different
classes of analytic spaces and we want
the category to be such that we can make
maps between them when we expect
comparisons um so for example it's well
known that the coherent chology of a
comp complex analytic space localizes on
the underlying topological space um
that's kind of more or less by
definition in a lot of ways um and that
kind of linear algebraic relation is
supposed to be explained by the
existence of such a map in this in this
large category of analytic
spaces um so all right
um I
think
uh maybe I'll say one word about the set
theoretic technicalities just to get it
out of the way
um
so so addressing
uh
um
so um well let's go back to our
uh algebraic analog so so this was a
Comm the opposite category of
commutative
Rings um so we have fpqc sheaves on
there but again this is not a legitimate
definition because this is not a small
category and this by definition is a
full subcategory of functors from uh
commutative Rings uh to say sets or
groupoids or whatever you like um and
there's well-known problems with such
things that the
morphisms in this funct category involve
a some more than a sets worth of data so
you have morphisms that aren't sets but
are some sort of bigger entity so that's
a real pain um so what is the
fix
um uh consider so instead
of instead of pre- shees uh on on F
consider uh what are called accessible
prees
um which are the same thing as
accessible
functors uh from commutative rings to
sets and what are accessible
functors
uh if and only if there exists a
cardinal Kappa regular
Cardinal uh such that uh fun F such that
f commutes
with with Kappa filtered
coletes
um so what does this mean uh the the
first choice of Kappa is Alf not
um um and in general when you think of a
regular Cardinal you shouldn't try to
think about what it is as a set what you
should really think of is the collection
of sets that are of smaller cardinality
that's what Kappa is really indexing
it's don't think of it as the
cardinality of some set think of it as
indexing the so this corresponds to
finite
sets um those are the sets of
cardinality smaller than Kappa and then
the notion of Kappa filtered Co limit in
that case is just the usual notion of
filtered Co limit so if your funter
commutes with filtered colimits you're
okay but if Kappa gets
bigger uh you have
fewer uh Kappa filtered Co
limits so to be a so for example if you
take Kaa equals lf1 then you're indexing
the countable
sets and you're only then a Kappa
filtered Co liit would be one where
every countable set has a cone as
opposed to just every finite set so
there are fewer kapper filtered colimits
that means there's more examples of
functors which commute with Kappa
filtered
colimits um and
uh uh
Kappa um so so the nice thing uh is that
every uh for
every X in
F uh the un
uh the on its image under the on
embedding is always
accessible and in
fact
accessible is equivalent to being a
small co-
liit of representable functors
hxs so if we think of
these apine schemes is the basic
building blocks and Co limits as our
gluing procedure then we're saying kind
of an obvious thing we're only allowed
to glue a sets many worth of things
together at a time and that's exactly
what's captured by this notion of
accessible
functor and
moreover uh fpqc shif
foration uh preserves accessible sheaves
or pre- shees
sorry that was actually that's a theorem
of
Waterhouse
um so that um
you can um yeah so you can also impose
the sheath condition at will without
running into set theoretic
technicalities um but this accessible PR
is not presentable right it's not a
presentable C never there
AIC yes okay yeah we have we have to
prove it yeah yeah yeah it has to be
proved so what what makes this work so
these are not kind of in some sense non
trivial claims the claims made here so
what makes this
work so the first thing is that this
category that we're working with uh is
presentable in fact compactly
generated
uh that's what makes the theory of
accessible prees work so it makes that
the condition they're commuting with
Capa filtered col is the same as being a
small cimit of
representes um and then there's
something about the gro de topology
which is that
every uh fpqc
cover
um uh in
F is a a filtered is a there exists a
Kappa such that um and in this case it's
lf1 is a is a Kappa filtered
limit of Kappa comp
compact uh cap fpqc
covers so every uh every flat cover of a
ring is a alf1 filtered Co limit of
countably presented flat covers of that
ring um and this kind of condition
having some a priori bound
on the basic fpqc covers I cardinality
bound is what you can use to um to check
this kind of fpqc shic result and um
so well so maybe I'll say then so
then at
least I we haven't explained what our gr
de topology is but we' least explained
what the category of analytic Rings is
so so there's a theorem um
so analytic
Rings is uh
presentable in fact it's Kappa compactly
generated for
Kappa uh true to the ALF not plus
so
so so that you Capa small means
cardinality less than Continuum um
so the uh that's what lets you kind of
formally avoid all of these uh set
theoretic difficulties involved with
taking pre- sheaves and sheaves on a um
a big
category uh maybe
yeah oh no I won't add anything that's
fine okay thank you for your attention
this category of analytic Rings is
naturally ened of a condensed animal
right I mean
mapping mapping home mapping space
is a condens condens animal yes so we
could also consider ened seeds
instead of usual SE it it seems also
natural to me but you take non-enriched
we take non-enriched yeah and
um it's confusing
the it's
confusing
um what can I
say
um
so in particular this analy the category
of analytic uh Stacks is
not um no it's not that's right
yeah um right different
way because also okay but how do you use
that to get an
enrichment uh
I
mean yeah well maybe there's some
adjoint funter is
there I mean you're saying I mean okay
it's yeah it's not even clear it's
enriched over itself
right
well I mean I don't know yeah but no I
guess I guess you're right you're you're
saying you just take your profinite set
or whatever and then you cross it with X
and map it to y yeah no yeah that's
right yeah so there is an sorry thanks
yeah so you you could take take so in in
analytics tax is enriched over light
condensed sets because you can define an
s-v valued point to be just a a map like
this um but uh it's different from the
enrichment there's essentially no
relation with the enrichment you had on
analytic Rings because we imported these
things via kind of very trivial analytic
Rings just continuous functions and it
um yeah it's all all in all it's a
little bit confusing and um yeah so for
example I think in some recent Works
in in with ptic coefficient systems and
like like V vishan was telling me that
she wants to work with like sheaves of
solid aelan groups on the proall side
site and then you know on the protol
site you already have profinite sets
there so you're somehow you have you're
doing condensed in two orthogonal
directions at the same time but
apparently that's the correct thing for
her and it's just it's it's all a little
bit confusing
but
yeah
yeah I want to explain how this space be
seen as an analytical space um well this
by this I just meant the topological
space um and maybe I make some
assumptions to make sure it's compactly
generated otherwi I mean sorry
metrizable otherwise I have to modify a
little bit what I mean maybe but um yeah
just so the same way that uh that this
thing was a a topological space and
therefore an analytic stack so but this
you could think of as providing an
analytic stack structure on this a
non-trivial analytic stack structure on
this this this topological space
here
yeah um what do you mean by two years
Bond what do I mean by by
one CU is
bounded two is bounded two is bounded ah
yeah so what I meant was that the yeah
so we have some Spec
R and I was looking at the apine line
and then so two is a a function on this
well if you have any element in the
ground ring it gives you a section of
the projection and you can ask that that
section be away from the the local ring
at Infinity so so we had this ring of of
uh ring living at the origin this
filtered cimit of P along scaling the
coordinate by Q then we can put it at
infinity and we can ask that this F not
meet that Locus at Infinity in the sense
that well in the sense that if
you that if you take some relative
tensor product like R mod f and then
that that ring that lives there then you
get zero um
so
um uh right and that if that happens we
say that f is bounded or that f is of
analytic so um it doesn't yeah it
doesn't get too close to
Infinity um so you have to ask that
condition for the scaler two um in order
for some of the claims I made to hold
um yeah and for the Gaga you have to ask
that condition for all elements in the
Basse ring
um
yeah I feel particularly silly about
forgetting that because I mean basically
we we already basically wrote up the
argument in the previous lectures on
complex analytic spaces and that aim was
explicitly included there so I just kind
of forgot about it
um okay other
questions
maybe at Point you said so you wrote
what what was the G structure on Spec R
on Spec R yeah uhhuh said this amount
to Spec
R G GM did it say did I say spec RQ
plusus one I meant spec zq plusus one ah
okay spec oh I'm sorry I'm sorry I meant
Z I meant Z I'm sorry but is it possible
to
elaborate or maybe or
maybe
well I mean the point is that if you
have a map to that stack then you get to
Define such a Ring The Ring of germs at
zero
um uh which lets you run this machine of
of building
analytic so yeah but so why you divide
by G because you don't really want to
specify exactly you don't want to
specify Q so for example you know if you
have like a a tate ring or whatever an
analy T Tate Huber ring or whatever then
there exists a topologically no potent
unit but you know it doesn't play any
role I mean so or at least up to some
bounded difference it doesn't play any
role so that's exactly what this
quotient is doing it's saying you have
some say topologically nil potent unit
um but it shouldn't it should only
matter up to some bounded difference and
so you define this GM analytic and then
you say that if you have q and you
multiply it by something bounded both
both away from zero and away from
Infinity then that should play the same
role in particular it should give rise
to the same item poent algebras and the
same theory of
analytic and so
on by the way I don't see you multiply
by something about it I mean rather that
you can mully by something one or you
raise to raise to some
power but this gets
confus s model is not really antic gen
at
all uh it
is I mean you can have something that's
like Q nor itself but if you multip by
something bound that's too large then
you would actually not have something
anymore yeah but
we that's
true
um that's true wait
uh there but it's not
really I thought I wrote the thing
down
um but maybe we can just
sort I thought I wrote this thing
down um just a
second
ah okay sorry yeah maybe it's not the GM
maybe it's the uh so you can also Define
the locus
where uh the coordinate is bigger than
or equal to one or sorry some in some
over convergent sense and the locus for
T is less than or equal to zero maybe
it's the thing that's uh that lies in
between
there um that's not a yeah that's a
group yeah
um we will discuss later in the
course
yeah but now I'm concerned okay anyway
okay we'll discuss it later in the
course thanks everyone
\end{unfinished}