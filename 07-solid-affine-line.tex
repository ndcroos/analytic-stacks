% !TeX root = AnalyticStacks.tex

\section{\ufs The solid affine line (Clausen)}

\url{https://www.youtube.com/watch?v=fUjn2rGw9SA&list=PLx5f8IelFRgGmu6gmL-Kf_Rl_6Mm7juZO}
\renewcommand{\yt}[2]{\href{https://www.youtube.com/watch?v=fUjn2rGw9SA&list=PLx5f8IelFRgGmu6gmL-Kf_Rl_6Mm7juZO&t=#1}{#2}}
\vspace{1em}

\begin{unfinished}{0:00}
e
today  we're  going  to  be  talking  about  a
little  bit  of  geometry  maybe  the  solid
Aline  line  um  so  let  me  start  with  a
recap
um  of  what  we've  seen  so  far  um  so  we
had  this  category  of  solid  abilon
groups  um  which  was  a  full  subcategory
of  uh  light  condensed  a  bilion
group  um  and  this  was  some  kind  of
analog  of  uh  complete
non-  archimedian
topological  aan
groups  but  it's  kind  of  um  from  a  formal
perspective  easier  to  work  with  it's  an
ailan
category  um  in  fact  it's  yeah  so  it's
ailan  category  closed
under  uh  limits
colimits
extensions
uh  and  many  other  things  besides  and  um
also  we  had  this  left  ad  joint  called
solidification  um  and  there's  a
symmetric  monoidal  structure  here  which
you  think  of  as  a  completed  tensor
product  and  and  such  that  this
completion  functor  is  symmetric
monoidal
um  and  uh  the  definition  was  in  terms  of
this  object  p  uh  which  was  kind  of  the
free  module  on  a  null
sequence  so  the  definition  was  that  m  is
solid
uh
uh  so  the  definition  was  that  it's  solid
if
um  uh  this  map  from  null  sequences  to
null  sequences  given  by  identity  minus
shift  is  an
isomorphism  um
okay  so  that's  what  we  had  last  time  um
now  I  want  to  start  doing  a  little  bit
of  geometry  so  we're  going  to  be  modest
and  look  at  the  Aline  line  which  is
actually  the  most  important  case  um  so
about  this  P  so  it  turns  out  that  this  p
uh  even  before  solidification  so  p  is  a
ring  so  there  are  two  shets  one  can  look
at  because  you  can  shift  one  to  the  left
and  one  to  the  right  yeah  or  bothos  is
equivalent
uh  I  want  the  one  no  I  think  only  one  is
good  I  mean  I  want  the  one  that's
injective  uh  so  p  is  a  ring  and  maybe
I'll  it'll  be  clarified  in  a  sec  maybe
well  p  is  a  ring  and  in  fact  there's  a  a
ring  map
uh  so  from  the  polinomial  ring  in  one
variable  to  p  and  the  shift  is
multiplication  by
T
um  uh  so  that's  that's  what  you  have
before  solidification  but  after
solidification  something  uh  extremely
nice  happens  so  when  we
solidify  well  we  actually  already
explained  what  the  free  module  on  P  is
so  the  the  the  uh  there's  a  compact
projective  generator
here  uh  which  was  a  countable  product  of
copies  of  Z  that's  kind  of  the  basic
object  from  which  everything  is  built
and  you  can  get  that  object  object  just
by  solidifying  this  basic  um  sequence
space  p  and  if  you  take  into  account  the
ring  structure  what  happens  when  you
solidify  is  you  just  get  the  power
Series  ring  in  one
variable  so  in  the  solid  context  the
universal  carrier  of  a  null  sequence  is
just  this  power  Series  ring
um  but  moreover  there's  a  a  very  nice
property  of  this  situation  so  so  LMA  how
do  you  multiply  things  in  P  that  looks
like  a
cent  by  something  which  is  not  uh  the
null  sequences  is  like  ring  out  unit
usually  yeah  so  let  me  explain  how  you
get  the  ring  structure  just  very  briefly
so  it's  uh  it's  better  to  think  of  this
construction  as  being  attached  to  n
rather  than  n  Union  Infinity  so  more
generally  if  you  have  a  set  say  or  a
countable  any  countable  set  you  can
Define  measures  on  it  as  the  free  module
on  any  compactification  module  the  free
module  on  on  the  on  the  boundary  and
it's  independent  of  the
compactification  and  it's  functorial  for
proper
Maps  so  now  and  it's  symmetric  monoidal
you  can  see  that  using  Independence  of
the  compactification  if  you  take  the
product  of  two  countable  sets  then
measures  on  one  tens  or  measures  on  the
other  is  measures  on  the  product  so
using  that  and  the  fact  that  addition  on
the  natural  numbers  is  a  proper  map  has
finite  fibers  you  can  see  that  addition
on  natural  numbers  induces  this  ring
structure  on
P
okay  all  right  um  it's  the  same  and  you
know  addition  on  natural  numbers  also  in
some  sense  induces  this  so  that's  why  uh
yeah  all  right  uh  so  the  Lemma  is  that
uh
so
uh
so  if  you  do  the  tensoring  in  the  solid
world  uh  of  this  power  Series  ring  with
itself  over  the  polinomial  ring  uh  you
just  get  the  power  Series  ring  again
maybe  multiplication  map  is  an
isomorphism  you  could  see  and  even  the
even  the  derived  tensor
product  and
well  this  is
um  this  is  quite  Elementary  uh  because
we  know  how  to  well  we  know  how  to  do
tensor  products  in  solid  Z  Peter  gave
lots  of  examples  of  calculations  in  this
category  and  the  tensor  product  of  two
of  these  basic  guys  uh  is  just  another
one  of  them  um  and  so  if  you  do  uh  so
proof  If  you  do  Z  power  series  T  let's
say  T1  uh  tensor  over  solid  z  uh  Z  power
series
T2  that's  just  uh  Z  power  series  T1
T2  by  the  basic  calculation  of  the  solid
tensor  product  and  then  if  you  want  to
get  to  this  uh  you  just  do  modding  out
by  you  just  identify  the  two  different
variables  T1  minus  T2  and  uh  Power
Series  ring  modulo  T1  T2  is  is  the  power
Series  ring  in  one
variable  um  there's  maybe  another
perspective  Peter  also  mentioned  this
fact  that  if  you  take  the  solid  tensor
product  of  two  derived  complete  things
then  the  result  is  still  derived
complete  so  well  maybe  you  have  to  worry
a  little  bit  about  this  but  morally  if
you  want  to  check  this  is  an
isomorphism  both  sides  are  Drive  T
complete  you  can  check  at  modulo  T  and
modul  T  it's  really  a  triviality  um  so
kind
of  okay  um  the  the  derived
solid  uh  Tor  product  is  it  defined  as
the  derived  T  product  then  you  solidify
or  is  it  how  how  is  Def  exactly  yeah
it's  the  you  can  either  say  it's  the
derived  the  left  derived  functor  of  the
solid  tensor  product  or  you  can  say  that
you  take  the  derived  tensor  product  and
then  derived  solidify  it  it's  it's  all
the
same  and  do  need  to  know  there  are
somehow  enough  flat  things  to  construct
the  the  you  don't  need  to  know  that  but
you  but  we  but  it  is  actually  true  that
all  of  these  guys  are  flat  I  don't  know
if  Peter  did  Peter  mention  that  I  forget
uh  yeah
um
okay  uh  yeah  so  a  a  prior  you  need  to
resolve  both  variables  but  since  you
have  flat  objects  you  don't  you  would
only  need  to  resolve  one  um  okay  so  what
is  the  interpretation  of  this  so  this  uh
this  corresponds  to  sort  of  the  apine
line  you  could  say  and  then  since  this
uh  satisfies  this  item  potent  property
uh  sort  of  like  a  a  when  you  invert  an
element  in  a  ring  you  kind  of  get  the
same  thing  you  can  think  of  this  as
corresponding  to  some
Subspace  of  the  apine  line  so  a  Subspace
as  opposed  to  just  a  random  space  with
them  with  a  map  to  the  Aline  line  and
that's  the  interpretation  of  this  item
potency  so  then  you  could  ask  okay  how
how  should  one  think  of  this  Subspace
what  Subspace  is  this  um  so  the  naive
thing  to  say  would  be  that  it's  the
formal  completion  of  the  Aline  line  at
the
origin  um  well  because  it's  a  power
Series  ring  but  that  naive
interpretation  is  not  the  correct  one
and  so  so
so  does  not  correspond  to  a
formal  uh
neighborhood  of  zero  in  A1  and  the
reason  is  that  that  that  interpretation
is  not  stable  under  base  change  so  so
look  at  base
change  by  which  I  mean  this  is  all
implicitly  over  Z  but  we  could  tensor  it
to  any  ring  or  any  solid  ring  and  see
what  pops  up  um  and  let  me  do  a  little
calculation
so  so  let's  uh  let's  base  change  to  a
non-  archimedian  field  say  I'll  just
take  the  simplest  example  QP  and  we'll
see  what  pops
up  uh  so  we
take  the  Aline  line  over
QP
um  uh  oh  sorry  well  maybe  I'll  just  say
yeah  we  we  take  we  take  QP  and  then  we
do  a  we  we  do  a  solid  tensor  product
with  uh  this  thing  here  and  we  want  to
compare  this  to
qpt  um  so  this  so  QP  is  zp  with  P
inverted  and  the  solid  tensor  product
commutes  with  co-  limits  in  both
variables  so  this  is  the  same  thing  as
zp  solid  tensor  product
uh  Z  power  series  T  and  then  you  invert
P  but  this
zp  uh  has  a  resolution  by  two  copies  of
so  so  Zu  U  maybe  mod  U  minus  P  it  has  a
resolution  by  two  copies  of  this  power
Series  ring  uh  so  we  know  how  to  do
these  tensor  products  and  it  just  does
the  naive  thing  of  you  can  pull  in  the
limits  uh  so  the  result  of  that  is  that
you  get  zp  power  series
t  uh  and  then  invert  p  on  the
outside  which  is  not  the  same  thing  as
the  formal  completion  at  the  origin
namely  QP  power  series  T  so  it's
contained  in  there  and  it's  the  subset
where  the
coefficients  uh  are  bounded  in  piic
Norm
so  so  what  is  the  interpretation  then  of
of  this  ring  uh  this  ring  is  the  ring  of
functions  on  the  open  unit
dis  uh  in  in  A1  over
QP
so  if  you  you  could  ask  uh  if  you  take
say  maybe  after  an  arbitrary  non-
archimedian  field  extension  of  QP  you
could  ask  when  you  plug  in  a  value  in
that  field  uh  when  will  this  when  will
such  a  series  converge  and  the  answer  is
it  will  converge  if  and  only  if  the
number  you  plug  in  has  absolute  value
strictly  less  than  one  the  Ring  of
bounded  functions  actually  what's  that
it's  the  Ring  of  bounded  holomorphic
function  of  open  Unity  ah  yes  uh  what
not  the  full
because  sometimes  yeah  there's  another
one  Ah  that's  true  yeah  that's  the  bound
thank  you  yeah  yeah  that's  right
sometimes  people  use  the  the  not
necessarily  like  the  robing  and  stuff
which  is  not  about  I  don't  know  that  all
kinds  of  things  yeah  I  don't  know
what  well  okay  certainly  all  these
functions  Converge  on  the  open  unit  dis
um  but  they  don't  converge  past  that
um  so
uh  well  so  it  yeah  so  then  the
interpretation  to  that  this  suggests  is
that  uh  that  uh  this  uh  ZT  uh
Z  this  should  be  thought  of  as
corresponding  to  some  sort  of  open  unit
dis  and  it's  just  that  over  a  discrete
ring  like  Z  you  can't  tell  the
difference  between  the  open  unit  dis  and
the  formal  completion  at  the  origin  but
the  difference  pops  out  when  you  make  a
a  base  ch  change  to  something  that  has
more  more  freedom  to  play
with
okay
um  but  in  uh  rigid  geometry
uh  uh  it's  not  really  the  open  unit  dis
that's  fundamental  uh  usually  it's  the
closed  unit  dis  um  so  what  about  the
closed  unit  disc
well  one  way  to  understand  the  closed
unit  disc  if  you  understand  what  an  open
unit  dis  is  you  can  imagine  putting  the
open  unit  disc  at  Infinity  instead  so
then  you're  looking  at  then  you  get  the
complement  of  the  closed  unit  disc  and
then  that's  one  way  of  talking  about  the
sorry  the  complement  of  the  you  yeah  you
get  the  complement  of  the  closed  unit
disc  and  the  complement  of  a  thing  is  is
just  as  good  as  describing  the  thing  so
let's  take  this  guy  and  put  it  at
Infinity  instead  so  the
complement  well  this  is  all  just  at  the
level  of  a  uh  loose  thinking  so  we  can
take  this  power  Series  ring  in  one
variable  uh  and  we  can  tensor  it  over
well  but  let  me  call  the  variable  T
inverse  instead  because  I  want  to  be
thinking  of  putting  it  at  infinity  and
then  I  tensor  over  ZT  inverse  with  uh  ZT
inverse  so  then  I'm  puncturing  at
Infinity  so  that  I  actually  live  in  the
apine  line  so  I  have  a  homomorphism  from
ZT
um  uh  and  what  is  this  ring  well  I'm
just  taking  this  power  Series  ring  and
I'm  inverting  the  the  variable  there  so
another  way  of  describing  this  is  as  Z
power  Series  in  t
inverse  so  that's  the  open  unit  dis
centered  at
Infinity
um  uh  so  if  we  want  to  understand  the
closed  unit  dis  we  should  in  some  sense
be  uh  kind  of  localizing  away  from  this
complement  or  another  way  of  saying  that
is  that  we  should  be  killing  this  object
so  so  to
get  we  need  to  kill  uh  kill  this  thing
here
so
um  so  well  I'm  just  going  to  make  just  a
preliminary  so  this  Z  power  series  T
inverse  is  a  module  over  ZT  as  I  just
exhibited  and  it  also  has  a  very  simple
resolution  so
uh  note
um  so  uh  well  ZT  inverse  we  can  think  of
that  as  let's  call  let's  disambiguate
and  call  the  variable  U  um  we  can  think
of  it  as  Z  power  series  U  and  then
invert  U  um  but  one  way  to  invert  U  is
to  adjoin  the  thing  that  you  want  to  be
uh  the  generator  uh  I  mean  want  to  be
the  inverse  and  then  enforce  the
equation  that  that  says  that  they're
inverse  to  each
other  um  and  when  you  look  at  it  like
this  uh
so  so  I  I'll  rewrite  that  so  we  get  ZT
inverse  uh  has  a  two-term  resolution
where  you  have  Z  power  series  U  bracket
T  and  Z  power  series  U  bracket
t  uh  and  here  you  have  uh  U  minus
one  so  two  term
resolution  and  what  are  these  objects
well  these  are  just  the  base  changes  of
our  fundamental  uh  p  uh  not  to  solid  Z
but  to  solid  z  uh  and  then  adjoin  a
variable  T  so  so  these  are  resolutions
by  the  compact
projective  uh
generator  uh  of  the  category  of  of  z
braet  t  modules  in  solid  ailion
groups
so  killing  this  thing  should  be  the  same
thing  as  requesting  this  map  to  become
an
isomorphism  and  now  this  suggests  the
following  based  on  an  analogy  with  the
definition  of  solid  up  there
so  so
definition  so  let's  say  let's  let's  say
we  have  a  one  of  these
guys  let's  say  that
is  ZT
solid
um  if  and  only  if  when  you  take
uh  internal  H  from  uh  P  to  M  and  then
internal  H  from  P  to
M  uh  and  then  you  take  the  map  which  is
given  by  so  now  U  corresponds  to  the
shift  and  T  is  the  extra  thing  we  have
acting  on  M  because  m  is  a  z  bracket  T
module  so  we  have  shift  time  T  minus  one
uh  we  want  this  map  to  be  an
isomorphism  and  then  this  note  up  here
is  saying  that  that's  the  same  thing  as
requesting  that  if  now  I  should  maybe
pass  to
Arham
uh
it's  the  same  thing  as  requesting  that
there  are  no  R  homs  from  this  object
we're  trying  to  kill  into
M  okay
so  now  uh  theorem  is  basically  that
um  this  definition  the  name  is  well
chosen  so  that  this  ZT  solid  Theory  over
ZT  is  very  similar  to  the  solid  Z  Theory
over  z  uh
so  so  let's
say
and  we  can  even  I  mean  everything  is
also  inside
condensed  little
light  this  uh  is  an  aelon
category  closed
under  limits  and
colimits
extensions  um  so  if  uh  if  m  is  an
arbitrary  say  condensed  ZT
module  uh
then  internal  x  uh  and  N  is  in  solid
Z  ZT  then  internal  X  over  ZT  from  uh  M
to  n  uh  is  also  in  solid
ZT  uh  there  exists  d  uh  there  exists  a
left  ad
joint  to  the  inclusion
uh  call  it  uh  let's  say  well  I'm  going
to  at  least  temporarily  denote  it  by
upper  T  solid  we're  forcing  the  variable
T  to  be  solid  you  could  say  so  that's
uh  z  z  and  Z  is  the  same  thing  one  you
wrote  in  the  first  yes  I  wrote  I  use
different  notation  yeah  sorry  yeah  this
is  this  is  supposed  to  be  the  same  as  as
this
yeah
yeah  um  there  exists  symmetric  monoidal
structure  on  this  solid  this  drive
solidification  over
ZT  um  and  then  uh  the  drived
analog  uh  holds  as
well  so  the  derived  analogs  of  all  these
claims  hold  and  moreover  the  drive
theory  is  naive  in  the  sense  that  an
object  in  the  derived  category  is
derived  solid  if  and  only  if  each
homology  group  is  solid  in  the  sense  of
the  ailion  category  and  it  is  the
derived  category  of  the  aelan
category  um  okay  so  that's  all  the  kind
of  formal  stuff  but  but  that  in  the
solid  Z  Theory  it  was  also  important  to
to  get  a  to  understand  what  was  going  on
it  was  important  to  understand  the  the
basic  compact  projective  generator  which
is  always  gotten  by  just  well
solidifying  the  sequence  space  um  so  let
me  make  a  claim  about  that  so  the  if  you
take  the  Z  power  series  well  maybe  I'll
just  I'll  say  let's  not  think  of  it  as  a
ring  we're  thinking  of  it  as  a  module
now  so  if  you  take  a  product  of  copies
of  z  uh  you  base  change  it  to  Z  bracket
T  and  then  you  solidify  you  get  well  a
product  of  copies  of  Z  bracket
t
um
so  maybe  a  little  remark  about  an
interpretation  of  it  so  I  said  solid  Z
was  kind  of  analogous  to  complete  non
archimedian  topological  ailan  groups  non
archimedian  means  Say  by  definition  that
there's  a  basis  of  neighborhoods  of  the
identity  consisting  of  a  billion
subgroups
um  so  what  would  be  the  analogous
interpretation  of  solid  ZT  it  would  be
that  you  have  a  as  you  have  a  complete
non-  archimedian  thing  where  moreover
there's  a  basis  of  neighborhoods
consisting  of  ZT  submodules
and  you  you  can  think  of  I  mean  the  and
for  a  basic  example  of  a  a  ZT  a  ZT
module  which  is  non  archimedian  but  does
not  satisfy  that  property  you  can  think
of  this  ring
um  you  can  not  find  a  basis  of
neighborhoods  of  zero  which  are  stable
under  multiplication  by  T  because
they're  stable  under  multiplication  by  T
inverse  instead  and  so  but  in  some  sense
the  the  theorem  is  that  if  you  kill  just
that  one  guy  then  you've  explained  the
difference  between  the  two
Notions  um
okay
so  so  in  the  previous  Theory  without
light  it  was  a  little  bit  different  it
was  not  defined
using  this  P
but  well  okay  it  basically  was  done  done
this  way  uh  I  mean  maybe  we  didn't  make
this
explicit  and  we  maybe  more  talked  about
uh  just  this  but  we  definitely  talked
about  this  um  yeah  but  it  does  make  it
more  clear  to  to  to  think  of  it  that
that  way  with  the
p  okay  so  the
proof  yes  confused  about  the  ZT  itself
is  ZT  solid  ah  so  it  is  that's  part  of
the  that's  part  of  the  theorem  because
I'm  claiming  in  particular  that  this  is
ZT  solid  and  and  ZT  is  a  retract  of  this
so  it's  something  that  we  need  to  prove
and  we  will  prove  it  but  it's  not  solid
what  ZT  is  UN  solid  ZT  oh  no  it  is  it  is
so  as  an  ailan  group  you  mean  or  yes  no
no  every  discrete  ailan  group  is  solid
because  it's  generated  under  co-  limits
by  Z  no  it's  an  important  point  Thank
you  for  for  bringing  it  up
yeah  other
questions  okay  and  Z  ZT  power  series  is
also  solid  ZT  power  series  is  also  solid
yes  yes  because  well  because  it's  a
limit  of  things  I  mean  you  can  build  it
from  limits  and  Co  limits  from  ZT  so
it's
yeah  all
right  uh  so  for  the  proof  uh  well  the
all  of  these
properties  so  all  except  the
last
those  are  that's  exactly  the
same  those  the  arguments  for  all  of
those  things  were  completely  formal  just
based  on  the  fact  that  you  have  this
internally  projective  object  p  and
you're  asking  that  some  endomorphism  of
internal  HS  from  P  to  M  become  an
isomorphism  that  was  all  that  Peter  used
when  he  was  proving  the  analog  of  these
claims  for  uh  solid  Z  so  the  fact  that
we  have  a  good  formal  the  theory  is
already  contained  in  there  and  then  in
Peter's  lecture  the  hard  part  was
identifying  the  free  modules  that  when
you  solidify  this  sequence  space  p  that
you  actually  just  get  a  product  of
copies  of  Z  fills  up  the  whole
thing  thankfully  uh  so  that  that  part  is
actually  going  to  be  easier  here  because
we  already  have  solid  Z  and  um  and
basically  thanks  to  this  interpretation
of  killing  some
object  so
so  the  claim  uh  the  key  claim  is
that
um  the  that  you  can  give  an  explicit
formula  for  this  derived
solidification  uh
so  I  apologize  for  the  janky  notation
now  this  looks  really  bad  uh  D  derived
derived  solidification  with  respect  to  T
so  uh  that  this  is  just  an  internal  ROM
over  z  braet
t  uh
from  the  quotient  of  this  ring  by  this
ring  to
M  and  note  that
this  this  is  this  does  have  the  required
map  from  M  uh  because  m  is
ROM  over  oh
sorry  oh  sorry  minus  one  uh  shift  by
minus  one
uh  soor  I'm  just  trying  to  read  what
that  last  thing  says  that  that's  and  m  t
t²  yeah  l  so  derived  solidification  yeah
the  deriv  functor  of  this  thing  here
yeah  which  I  I  kind  of  prefer  actually
to  think  of  as  just  the  derived  analog
of  it
um
okay
and  the  AR  underline  is  either  in  all
condense  or  only  in
solid  uh  it's  the  same  so  basically
because  of
this  I  mean  they  yeah  the  derived  the
the  so  when  you  compute  of  course  you
have  Gren  the  categories  so  you  have  a
well  defined  even  Rec  bounded  complexes
but  how  do  you  check  exactly  I  think
Peter  must  have  briefly  said  something  I
don't  remember  but  how  do  you  check  that
the  the  AR  or  underline  are  the  same
for  some  Junction  arum  I  don't  remember
exactly  I  mean  you  basically  I  don't
know
I  like  I  said  I  I  sort  of  tend  to  think
on  the  derived  level  from  the  beginning
but  the  basic  point  is  that  everything
has  a  resolution  by
these  uh  internally  projective  guys  and
then  on  those  they're  the  same  and  then
it's  some  derived  limit  of  that  and  it's
just  I  di  internally  projective  in  h  in
uh  in  all  condensed  in  light  condensed  I
mean  so  but  it's  really  not  that's
that's  really  not  necessary  either  I
mean  this  also  works  in  the  in  all
condensed  so  I  mean  I  I  claim  it's
formal  and  I  also  claim  I  don't  want  to
get  into  the  detail  right  now  so  uh  yeah
so  let's  maybe  discuss  after  if  you
still  have
questions  um  okay  uh  right  so  the  proof
of
claim  so  this  follows
formally
from  uh  the  fact  that  this  ring  Z  power
series  T  inverse  uh  is  item  potent
over
uh  over
ZT  which  follows  from  the  very  first
description  I  gave  of  it  as  the  base
change  of  the  we  check  that  the  power
Series  ring  is  item  potent  and  we  put
that  at  Infinity  um  and  the  item  potency
is
preserved  so  for  example  using  so  that
the  item  potency  of  that  so  what  is  this
thing  here  it's  just  the  the  homotopy
fiber  of  the  inclusion  of  this  into  this
and  this  is  the  the  base  ring  and  you
can  easily  check  that  item  this  item
potency  is  equivalently  to  equivalent  to
the  derived  item  potency  of  this  object
and  that  means  that  if  you  take  this
expression  and  you  apply  it  again  you
get  the  same  thing  back  so  that's  kind
of  an  item  potent  operation  and  again
the  same  item  potent  will  prove  to  you
that  if  you  take  this  operation  and  you
Arham  from  this  guy  uh  you  get  zero  so
this  thing  is  ZT  solid  and  that's  not
quite  everything  you  need  to  check  but
it's  basically  everything  you  need  to
check  it's  an  item  potent  operation  and
it  uh  m  is  solid  if  and  only  if  this  map
is  an
isomorphism  um
so  it's  a  it's  just  completely  formal
and  I  won't  write  out  all  of  the  all  of
the
details  um  so  maybe  I'll  just  say  note
so
this
yeah
okay  so  that's  a  a  formula  for  the
derived  solidification  of  a  general  Z
bracket  T  module  but  uh  something  nice
happens  if  your  Z  bracket  T  so  we  still
haven't  proved  the  claim  um  but
something  nice  happens  if  your  Z  bracket
T  module  is  base  changed  from  a  a  solid
Z  module
so  and  next
claim  uh  is  that  the  functor
from
the  sort  of  pullback  functor  from  solid
Z  to  solid  ZT  so  sending  a  a  module  M  to
uh  you  take  M  tensor  Z  ZT  and  then  you
solidify  um  this  functor  uh  is  T
exact  uh
preserves  limits  and
colimits  and  it  sends
uh  z  uh  to
Z  to  uh
ZT  so  if  we  if  we  prove  this  then  we're
in  particular  we  in  particular  get  this
claim  because  this  is  of  that  this  is
that  functor  applied  to  product  of
copies  of  Z  I  claim  the  funter  commutes
with  products  so  then  we  could  just  try
to  it's  enough  to  understand  what
happens  with  Z  but  I  already  also
claimed  that  Z  goes  to  ZT  so  uh  we'd  be
done  so  the
proof  um  so  we  just  take
this  formula  for  the
solidification  um  and  we  plug  in  the
case  where  m  is  uh  also  induced  so  we
get  that
so  M  tensor
ZT  uh  LT  solid  this  is  the  same  thing
as  I'll  just  write  it  m  brackets  T  by
that  I  mean  m  t  for  ZZ  brackets  T  um  and
now  we're  Computing  an  rhom  over  ZT  so
the  way  to  do  that  is  to  disambiguate
the  two  occurrences  of  T  and  then
equalize  them  at  the  end  so  that's  uh
calculated
so  um  let's  say  Z  power  series  U  well
this  is  so  u  z  power  series  U
um
uh
uh  so  two  term  complex  I'll  write  it
vertically  so  to  speak  um  so  then  you
equalize  U  and  T  or  no  sorry  U  and  T
inverse  no  uh
yeah
um
uh  or  so  let  me  let  me  say  we  no  I
should  I  should  say  rather  that  we
equalize  T  and  the  and  the  and  the  shift
operator  on  here  which  is  induced  by
multiplication  uh  by  T  here  so  let's
say  let's  denote  that  by  shift  here  and
then  I'm  switching  variables  uh  from  T
to
U
um
okay
uh
but  uh  note  that  now  we're  taking  R  homs
in  solid  Z  but  this  is  one  of  our
compact  projective  guys  so  huming  our
humming  out  of  that  commutes  with  Co
limits  and  this  Mt  is  a  CO  limit  uh  it's
just  uh  so  we  can  put  that  out  here  like
this  and  then  we're  taking  uh  this  thing
bracket  T  and  we're  modding  out  by  T
minus  something  so  that  means  that  or
we're  not  modding  out  by  it  that  modding
out  would  be  the  co-  fiber  we're  doing
the  fiber  instead  but  we  also  had  a
shift  and  those  two  actually  cancel  and
in  the  end  what  you  get  is  that  this  is
ROM  over
z  uh  from  u  z  power  series  U  uh  to
M  and  that's  the  key
formula
oops  and  uh  we've  kind  of  along  the  way
we  sort  of  of  lost  the  ZT  module
structure  on  this  thing  uh  but  you  can
recover  it  it's  induced  by  uh  so  the
multiplication  by  T  is  here  given  by
sending  U  to  zero  uh  u2  to  u  u  Cub  to
u2  Etc  so  if  you  do  that  there  then
that  induces  an  endomorphism  here  and
that's  the  multiplication  by  T  on  this
resulting
thing  sorry  I  I  miss  what  you  say  so
what  what  is  the  isomorphic  to  this
green  uh  this  this  this  bit  a  somewhat
silly  notation  uh  is  homotopy  fiber  of
this  map  or  some  kind  of  shift  of  a
mapping  cone  and  that's  the  same  as  this
it's  also  the  same  as
this  okay  and  now  we're  done  basic  well
now  this  functor  again  this  is
internally  projective  in  solid  Z  so  this
fun  is  T  exact  and  it  preserves  limits
and  Co  limits  in  solid  Z  but  limits  and
Co  limits  in  solid  ZT  or  are  calculated
on  the  underlying  level  because  it's
just  part  of  a  module
category
um  and  then  the  last  claim  is  that  Z
goes  to  uh  just  the  usual  polinomial
ring  in  one  variable  but  what  happens
when  you  plug  in  Z  here  you're  taking  R
homs  from  this  product  of  copies  of  Z  to
Z  it  turns  into  a  direct  sum  of  copies
of
z  um  and  you  can  check  that  uh  what  you
get  is  just  the  usual  Z  bracket  T  even
with  the  ZT  module  structure  or  really
there's  a  natural  comparison  map  which
you  see  to  be  an
isomorphism  okay  so  there  was  this  some
someone  prove  it's  not  related  directly
to  the  theory  but  someone  proved  just
obr  and  gr  at  home  from  the  product
infinite  product  cop  of  Z  to  Z  is  a  is  a
direct  but  this  is  not  here  you  are
doing  it  in  this  is  easier  this  is
easier  yeah  yeah
yeah  okay
so  uh  we  can  also  do  other  examples  of
such  a  thing  so  let's  do  another  example
so  that  finishes  the  sorry  that  finishes
the  proof  of  the  theorem  so  we  now  have
a  grip  on  this  solid  ZT  Theory  and  I
want  to  advance  the  interpretation  that
the  solid  ZT  theory  is  is  like  uh  like
working  over  the  the  ZT  without  the
solidification  is  like  the  Aline  line
and  ZT  with  the  solidification  is  like
the  closed  unit  dis  that  was  kind  of  the
interpretation  that  I  started  with  um
but  let's  do  an  example  again  in  non-
archimedian  let's  base  change  to  a  non-
archimedian  field  and  see  what  happens
uh  so  let's  take  another  example  in
above  so  let's  look  at  QP  T  so  that's
functions  on  the  Aline  line  and  now  we
want  to  restrict  to  the  closed  unit  dis
in  the  sense  that  we've  just  described
so  we  take  the  T  solidification  of
this
um  so  we  this  is  an  instance  of  this
this  funter  here  um  and  I  just  said  this
functor  has  all  the  properties  in  the
world  so  it  commutes  with  co-  limits  so
I  can  again  take  the  T  to  the
outside  oh  sorry  the  one  over  P  to  the
outside  and  it  also  um  commutes  with
limits  so  I  can  take  the  the
ptic  so  I  can  take  limit  Over  N  of  Zod  P
to  the
NZ  bracket
t  uh  T  solidified  and  then  one  over  P  at
the
end  um  and  now  we're  applying  this
funter  there  to  Zod  P  to  the  N  but
that's  just  two  copies  of  Z  so  um  you
get  the  same  answer  as  for  Z  it's  just
discreet  so  this  is  just  inverse  limit
Over  N  of  Z  mod  P  to  the
NZ  bracket
t  uh  one  over
P  or  in  other  words  it's  ZT  P
completed  P  completed  one  over  p  and
these  are  the
functions  uh  on  the  closed  unit
dis  kind  of  as  as
desired  okay  so  maybe  it's  I'll  take  a
five  minute
break  uh  maybe  I  forgot  to  give  the
interpretation  of  the  coefficients  here
so  again  this  is  some  power  series  with
coefficients  in  QP  but  here  the
coefficients  uh  tend  to  zero
periodically
um  so  it's  a  smaller  ring  of  functions
so  it  converges  on  a  bigger  region  and
you're  allowed  to  be  now  on  the  bound
well  boundary  so  to  speak  I
mean  yeah  it's  yeah
yeah
um  okay  now  I  want  to  let's  play  a  fun
game
uh  so  I've  been  talking  about  the  open
unit  disc  and  the  closed  unit  disc  or
maybe  I'll  say  yeah  or  the  closed  unit
disc  and  the  complement  of  the  closed
unit  disc  but  in  it's  kind  of  there's
always  this  fun  question  of  what  exactly
is  open  and  what  exactly  is  closed
because  in  in  rigid  geometry  the  closed
unit  disc  is  considered  as
open  um  and  the  open  unit  dis  well  I
suppose  it's  also  open  in  a  sense  but
it's  not  quasi  Compact
and
um  so  let's  play  a  game  also  you  can  use
a  belov  Viewpoint  which  changes  somehow
is  another  way  another  then  it  is  closed
not  way  yeah  so  I  claim  that  our
formalism  is  going  to  give  a  definite
answer  to  this  question  what's  open  and
what's  closed  yeah  okay  well  all  right
um  where  was  I  oh  yeah  what's  closed  and
what's
open  so  let's  let's  take  a  look  at  the
the
the  the  categories  you  could  say  we've
attached  to  the  various  geometric
objects  so  for  the  apine  line  we  said
we're  just  thinking  of  it  as  the
polinomial  ring  in  one  variable  but  then
the  linear  algebra  category  is  just  ZT
modules  and  solid  ailan  groups  and  let
me  actually  go  to  the  derived
perspective
so
uh  um  but  then  inside  there  we  had  this
derived  solid  so  we  had  D  of  a
solid  uh
ZT  um  but  actually  it's  better  to  think
of  that  as  a  as  a  quotient  really  um
because  the  the  symmetric  monoidal
functor  is  the  derived  solidification
functor  uh  which  goes  like  this
so  uh  but  then  on  the  other  hand  we  had
the  complement  uh  and  the  complement
was  uh  the  open  unit  disc  or  rather  the
open  unit  just  move  to  Infinity  so
that's  this  thing  and  that  corresponds
to
d  uh
mod  uh  luron  Series  in  t
inverse  um  solid  Z  I  think  someone  asked
before  but  I  for  about  that  you  had  the
solidification  which  is  one  ad  there
should  be  another  on  the  other  side  wait
wait  wait  wait  uh  so  then  and  then  this
is  just  the  natural  includ  well  it's
it's  a  forgetful  functor  a  priori  you're
forgetting  the  extra  module  structure
but  it's  in  fact  a  fully  faithful
inclusion  because  of  the  item  potency  uh
the  item  potency  of  this  thing  means
that  there's  at  most  one  ZT  module
structure  on  any  ZT  Z  lauron  series  T
inverse  module  structure  on  an  e  z
module  um  so  this  is  actually  like  a
localization  sequence  or  what  have  you
so  this  is  the  symmetric  monal  quotient
of  this  by  this  thing  which  is  kind  of
an  ideal  in  there  um  and  what  kind  of
adjoins  do  we  have  and  which  of  them  are
well  behaved  and  so  on  so  well  we  know
that  this  one  has  a  right  ad  joint  uh
which  is  given  by  the
inclusion
um  but  it  also  has  a  left  ad  joint
so
so
um  and  the  left  ad  joint  is  given  by
sending  M  uh  to
uh  I  guess  the  the  fiber  so  you  take  M
and  then  you  base  change  it  to
Infinity
uh  and  you  take  that  uh  cone  diagram
here  so  it's  the  fiber  of  uh  M  mapping
to  m  t  over  ZT  with  the  Z  lant  series  T
inverse  so  uh  and  this  left  ad  joint  is
in  a  sense  better  behaved  than  the  right
ad  joint  the  the  naive  inclusion  and  by
the  the  measurement  that  it's  better
behaved  well  they  both  commute  with  co-
limits  but  but  this  one  satisfies  a
projection  formula  with  respect  to  this
fundamental  functor  which  the  symmetric
monoidal  functor  so  this  left  ad  joint
is  kind  of  linear  over  these  two
symmetric  monoidal  categories  so  this  is
better  so  uh  and  I  I'm  going  to  call
this  one  J  upper  star  and  this  one  J
lower  shriek  and  this  one  J  lower  star
this  inclusion
here
um  and  on  the  other  hand  for  here  uh  we
have  this  funter  here  which  I'll  call  I
lower  star  the  inclusion  there  uh  so  it
has  a  left  ad  joint  also  I  upper  star
which  is  just  the  base  change
functor  uh  which  is  symmetric  monoidal
so  that's  kind  of  the  the  more
fundamental  one  from  this  perspective
but  it  also  has  a  a  joint  I  upper  shriek
which  is  given  by  some
arom  um  but  it's  less  well  beh  paved
again  uh  because  here  this  one  satisfies
a  projection  formula  with  respect  to
this  one  so  this  is  a  linear  this
functor  is  linear  over  this  symmetric
monoidal  category  just  like  this  funter
is  linear  over  this  symmetric  monoidal
category  so  star  is  an  adint  the
inclusion  has  anint  on  it  does  but  that
one  we  don't  talk  about  yeah  now  this
was  the  question  I
yeah  yeah  um  because  already  this  one's
not  as  nice  as  I  mean  the  good  ad  joint
is  is  actually  up  here  so  um  so  the
interpretation  that  this  suggests  so
this  is  exactly  like
uh  uh  if  you  have  a  if  you  have  a
topological  Space
X  and  then  you  have  z  a  closed
subset  and  then  U  uh  the  complement  open
then  you  get  if  you  have  on  DED
categories  of
sheaves  uh  you  have  exactly  the  same
thing
where  uh  you  have  the  open  over  here  you
have  this  thing  satisfies  a  projection
formula  you  happen  to  have  this  other
guy  but  it's  not  as  well  behaved  um  and
then  you  have  I  lower  star  uh  from  d
shees
z  uh
z  um  I  upper  star  and  then  I  upper
shriek  so  it  satisfi  formally  speaking
it  behaves  exactly  the  same  way  and
actually  they're  both  special  cases  of
the  same  thing  which  is  you  have  a
symmetric  monoidal  category  um  and  you
have  an  item  potent  algebra  in  it  and  uh
it  it  generates  the  whole  situation  in
this  case  you  have  this  category  and
this  item  potent  algebra  here  you  have
this  category  and  then  the  push  forward
of  the  structure  sheath  from  the  closed
sub  or  push  forward  of  the  constant
sheath  from  the  closed  subset  and
it  completely  analogous
formalisms  um
so  that  tells  us  that  from  our
perspective  uh  the  closed  unit  disc
should  definitely  be  thought  of  as  open
and  the  open  unit  dis  should  definitely
be  thought  of  as  closed  so  as  far  as  I'm
concerned  the  question  is
settled  um  but  also  uh  this  this  another
funny  thing  is  that  if  you  think  about
this  perspective  it's  also  telling  you
that  zarisky  opens  and  usual  algebraic
geometry  should  be  thought  of  as  closed
and  I  think  that's  also  true
um  um  I  can  explain  why  people  want  to
know  uh  all
right  but  not  for
usual  you  don't  like  no  you  start  you
start  from  the  usual  theory  of  like
patching  of  sh  like  is  used  in  many
context
like  in  I  mean  it  was  used  but  then  the
for  usual  is  used  for  the  usual  topology
like  oh  the  risk  of  a  but  then  you
somehow  develop  it  further  and  decide
that  now  that  the  risk  opens
are  well  close
this  I  think  I  I'd  like  to  try  to
convince  you  but  but  not  right  now
uh  all
right
open  sorry  which  import
open  I  don't  understand  the
question  no  I  heard  the  question  I  don't
understand  it  what  because  you  said  the
closed  things  correspond  to  an  ENT
object  in  symmetric  monal  category  so
you  said  that  there  risky  open  is  like
closed  yeah  yeah  so  for  example  a
distinguished  open  is  just  the  ring  with
the  function  inverted  that's  item  potent
yeah
yeah  so  I  mean  it's  so  so  maybe  I  should
say  quasi  compact  opens  or  closed  it's
probably
yeah  the  deriv  direct  image  of  the
structure  ship  is  important  yeah  yeah  um
okay  all  right  so  now
uh  um  yeah  so  this  by  the  way  this
uh  yeah  so  this  this  we're  kind  of  going
to  be  guided  by  this  sort  of  thing  in  in
setting  up  the  definition  of  analytic
stack  and  so  on  like  the  idea  of  so  one
of  the  things  we  discovered  is  that  when
you  move  to  this  condensed  solid  what
whatever  context  that  you  you  actually
get  six  funter  formalisms  in  large
generality  on  derived  categories  of  so
to  speak  quasi  coherent  sheaves  and  um
and  they  have  I  mean  they  really  have
nice  interpretations  so  for  example  if
you  think  of  this  thing  as  being  what
sits  at  Infinity  then  it  makes  sense
that  this  is  extension  by  zero  because
you're  taking  your  sections  and  then
you're  killing  the  ones  that  live  near
the  boundary  and  it  really  all  just
plays  quite  nicely  and  we  showed  how  to
give  proofs  of  things  like  sidity  and  so
on  using  these  formalisms  and  um  so  we
take  this  perspective  seriously  that  the
derived  categories  and  the  funs  between
them  are  going  to  dictate  to  us  what
what  the  geometry  looks
like  um  there  was  oh  there's  another
thing  I  remembered  in  the  break  that  I
forgot  to  mention  is  that  so  I  said  that
this  is  T  exact  and  preserves  limits  in
Co  limits  but  I  want  to  caution  you  that
this  this  uh  T  solidification  from  from
solid  ZT  is  not  t
exact  uh  but  it's  only  uh  so  it's  only
off  by  one  so  so  um  I  said  it  was  given
by  R  H  humming  from  this  object  which
has  a  two-term  compact  resolution  so
it's  only  off  by  one  from  being  T  exact
so  it  sends  everything  connective  goes
to  something  connective  everything
anti-c  connective  goes  to  something
that's  at  most  a  one  shift  of  something
anti-c  connective  um  so  it's  not  deriv
t-  solidification  is  not  exactly  exact
but  it's  it's  very  controlled  but  the
composed  map  is  yes  exactly  so  each  the
first  map  is  the  exact  and  the  composed
map  and  the
second  map  is  not  except  that's  right
yes  it's  a  bit
funny  all
right  um  okay  so  now  I'm  going  to  say
something  like  I  want  to  motivate  what
I'm  going  to  do  in  the  rest
of  so  I'll  say
Vista  so  where  what  is  one  place  we're
probably  going  to  go  is  uh  so  we  want  to
look  at  solid
rings  so
IE  commutative  algebra  objects  in  this
tensor  category  solid  z
um
uh  and  then  we  want  to  see  uh  so  sort  of
generalizing  our  discussion  when  R  is
the  a  r  is  z  bracket  T  we  want  to  see
some  subsets  like  closed  unit  discs  open
unit  discs  things  like  this  but  of
course  when  you  have  a  big  algebra
you'll  get  many  more  such  subsets  and  we
want  to  organize  the  organize  uh  what
you  see  in  a  in  a  nice  way  and  it's  not
necessarily  the  most  General  thing  you
can  do  but  what  we're  going  to  do
basically  is  just  take  the  things  you
things  you  see  over  Z  bracket  T  and  kind
of  Base  change  them  along  all  possible
maps  from  ZT  to  r  that  is  uh  all
possible  functions  in  R  um  and  what
we're  going  to  end  up  with  is  some  is
the  statement  this  this  derived  category
of  uh  mod  R  solid
Z  even  this  tensor  category
uh
localizes  uh
along  the  valuative  Spectrum  the
spectrum  of  all
valuations  uh  of  well  the  underlying
usual  commutative  ring  of  this  solid
commutative
ring  localizes  so  you  take  the  valuative
spectrum  of  r  ah  just  of  R  viewed  as  a
as  a  discrete  ring  yeah  as  a  discrete
ring  and  the  valuative  spectrum
is  okay  it's  like  a  residue  field  and
valuation
without  the  valuation  doesn't  have  to  be
integral  on  R  just  any  evaluation  yes
that's  right  it's  a  big  which
is  and  it  has  the  topology  which  is  okay
I  know  the  topology  but  it's  kind  of  the
it
was  yeah  is  topology  which  induces  the
topology  and  things  like  spa  and  so  that
yeah  but  it's  kind  of  because  sometimes
one  gets  Ms  which  are  not  SP  in  this
okay  but  okay  it's  possible  to  consider
so  it's  yeah  yeah  it's  possible  to
consider  it  indeed  and  it's  you  know  a
priori  more  General  okay  and  so  so  I  I
won't  remind  you  what  evaluation  is
right  now  but  I  will  but  I  will  remind
you  that  this  space  is  similar  to  usual
spectrum  of  a  commutative  ring  in  that
it  has  a  basin  basis  of  uh  quasi  compact
opens  I  mean  it's  a  spectral  space  and
it  even  has  a  particularly  nice  basis
which  is  in  some  sense  analogous  to  the
distinguished  apine  opens  in  algebraic
geometry  and  while  a  distinguished  Aline
open  in  algebraic  geometry  is
parameterized  by  a  single  element  of  the
ring  uh  here  you  kind  of  have  to  take  a
little  bit  more  data  so  so  this  uh  so
the  basic
opens  are  the  so-called  rational
opens  uh  so  X  you  have  to  take  finitely
many  functions  uh  and  then  you  form  this
thing  here  where  the  interpretation  is
that  uh  you  invert  G  so  we're  inside  the
zisy  distinguished  open  for  G  but  then
we  shrink  further  by  requiring  that  the
the  sort  of  F1  be  the  the  absolute  value
of  F1  be  less  than  or  equal  to  the
absolute  value  of  G  the  absolute  value
of  f  to  all  of  the  absolute  values  of
these  guys  should  be  less  than  or  equal
to  the  absolute  value  of
G  um
G  yes  that's  what  I  said  yeah  g  yeah  so
so
uh
right  and  so  I'm  saying  that  this
localizes  on  this  meaning  you  have
actually  a  sheath  of  categories  a  sheath
of  symmetric  monal  categories  um  and
what  you're  going  to  attach  to  this
thing  uh  is  a  version  of  the  solid
theory  of  the  solid  Theory  so  you  look
at  this  those  m  in  D  mod  are  solid
Z  such  that  well  first  you  want  to  say
that  multiplication  by  G  on  M  is  an
isomorphism  and  second  you  want  to  say
that  if  you  take  this  uh  internal  homs
from  P  to  M  uh  and  internal  HS  from  P  to
M  uh  and  you  take  fi  *  shift  minus  one
this  should  be  an  isomorphism  for  all  I
so  you  can  or  in  other  words  uh  oh  fi
over  G
sorry  so  in  other  words  you  want  so  you
think  of  these  all  these  guys  as  Maps
from  say  spec  spec  R  so  to  speak  to  the
apine  line  then  you  want  that  g  lands  in
the  uh  standard  you  know  zisy  nonzero
Locus  and  you  want  that  the  FI  over  G  uh
land  inside  the  closed  unit
disc
so
so
so
okay  so  I'm  not  going  to  go  into  details
about  that  um
but  uh  where  where  should  I  go  now  I
don't  know  maybe
here  this  as  PB  just  refers  to  the  local
corespond
to
uh  I  mean  yeah  the  only  I  mean  I  don't
uh  you  can  think  of  it  as  a  local  but
it's  also  a  topological  space
um
yeah  and  the  points  have  a  nice
description  and  so
on  uh  okay  so  what  do  I  want  to  do  today
well  or  part  part  maybe  partly  do  today
so  in  particular  so  I'm  claiming  this
category  local  izes  along  the  space  and
in  particular  you  get  a  structure
sheath
uh  uh  on  the
space  and  um  what  I  want  to  do  in  the
next  bit  so  goal  for  rest  of
lecture
is  make
this  this  structure  of  Chief
explicit  and  uh  compare  well  maybe
probably  start  to  compare  to  hub's
Theory  so  we're  we're  eventually  going
to  produce  this  data  by
very  easy  formal  means  but  it  requires
some  language  and  setup  so  we  can't  do
it  yet  but  I  want  to  make  at  least  this
part  of  it  explicit  uh  already  at  the
beginning  okay
so  let's
see  so  um  let  me  make  a  so  ah  let's
let's  let's  start  with  this  generality
so  we  have  a  solid
ring
um  um  and  let's  take  an  element  F  in  r
or  really  I  should  say  in  the  underlying
discret  ring  of  R  in  case  there's
ambiguity  um  and  that  in  particular  well
that  that  gives  you  a  map  from  ZT  to  R
which  sends  T  to  f
um  and  then  we  can  sort  of  see  how  these
Loi  that  we've  identified  uh  can  be  or
correspond  to  properties  of  R  so  let  me
make  a
definition  um
so  uh  f  is  topologically  nil
potent  uh
if  uh  this  map  Factor
through  this  homomorphism  I  should  say
factors  through  the  power  Series
ring  and  uh  f  is  power
bounded  uh  if  uh  well  I  want  to  say  that
if  if  this  map  well  the  map
geometrically  factors  through  the  closed
unit  dis  um  but  the  way  to  say  that  is
uh  if  so  let's  say  if  R  is  actually  so  R
is  a  algebra  over  ZT  in  particular  it's
a  module  over  ZT  and  we  can  ask  that  it
be
solid  in  the  first  definition  the
factorization  is  unique  yes  it's  Unique
if  it  exists  because  of  this  item
potency  so  that's  an  interesting  fact
actually  this  is  also  the  free  module  on
a  null  sequence  and  so
um  so  even  though  there  are  sort  of
house  solid  aing  groups  that  have  like
non-h  house  dworf  Behavior  Uh  still  uh
kind  of  this  limit  is  unique  if  it
exists
um
okay  so  basically  in  the  definition  of
solid  you're  imposing  that  certain
limits  exist  uniquely  even  even  though
you  have  non-house  dorf
Behavior  Uh  okay  so  that's  the  same
thing  as  saying  that  uh  if  you  take  H
from  P  to  R
h  p  to  r  f  *  shift  minus  identity  that
this  is  isomorphism  that's
just  oh  someone's  talking  hello
yes  yes  the  first  condition  lit  saying
that  St  nor  sequence  which  is  the  PO  of
and  zero  so  it's  literally  yeah  yeah
it's  I  mean  yeah  I  was  going  to  I  mean  I
was  going  to  explain  the  relation  with
classical  definitions  but  it's  it's  yeah
it's  quite  immediate  for  this  one  that
it's  the  same  as  the  classical
definition  so  maybe  I'll  just  repeat
what  Peter  said  uh  so  this  is  the  same
thing  as  uh  this  P  thing
solidified  so  yeah  this  map  sends  T  to  F
but  it's  a  ring  homomorphism  so  it  sends
t  to  the  N  to  F  to  the  N  um  and  then
saying  that  that  factors  through  p  is
the  same  thing  as  saying  that  this
sequence  uh  so  1  f  f  s  so  on  extends  to
a  null  sequence  which  is  B  basally
exactly  the  same  thing  as  being
topologically  nil
potent
um  yeah  the  new  phenomenon  in  the  solid
case  is  that  it's  Unique  if  it  exists
such  a  limit  that's  kind  of
fun  uh
okay  it  factors  through  in  the  sense  of
Rings  or  in  the  sense  of  the
uniqueness  factorization  okay  you  can
ask  factorization  as  a  condens  group  or
in  the  sense  of  rings  in  the  sense  of
rings  yeah  so  you  could  have  another
factorization  just  in  the  sense  of
condens  yes  that's  true  that's  true  yeah
yeah  yeah  yeah  I  was  maybe  being  too
imprecise  earlier  thank  you  yeah  or  in
the  sense  of  ZT  modules  is  enough  or  I'm
not  sure  I  don't  think  so  so  yeah  thank
you  for  the  thank  you  for  the  comment
yeah  just  as  Rings  just  as
Rings  okay  um  so
all  right  so  here's  a  a  Lemma  giving
basic  properties  ah  so  no  sorry  let  me
say
so  so  we  write  uh
r0  so  this  is  again  kind  of  standard
notation  in  Huber's  Theory  so  set  of
power  bounded
elements  and  our
z0  uh  set  of  topologically  nilp  putting
elements
which  is  a  Subs  well  yeah  I'm  going  to
prove  that  uh  yeah  so  Lemma  is  that  R
zero  uh  inside  this  ring  here  uh  is  an
integrally
closed  sub
ring  and  uh  R  z0  first  of  all  is
contained  in  r0  and  it  is  a  radical
ideal
so  I'm  still  confused  about  this  n
sequence  you  something  is  n  sequence  in
the  just  the  sense  that  the  map  extends
to  a  another  sequence  then  you  don't
know  that  it  factors  as  a  ring  that  this
is  a  ring  map  that's  that  seems  correct
to  me  yes  okay  so  it  is  doesn't  mean
topologically  important  in  your
definition  in  your  sense  but  on  the
other  hand  maybe  uh  yeah  no  I  mean  yeah
you're  right  uh  yeah
well  it's  the
same  it's  okay  that's  a  good  point  but
if  you  have  something  that  comes  from  a
uh  if  you  have  something  that  comes  from
a  house  doorf  topological  ring  uh  for
example  what  if  if  the  target  is  quasi
separated  then  independently  of  asking
about  the  algebra  structures  then  the
limit  is  unique  if  it  exists
so  yeah  oh  quasi  separated  was  this  uh
analog  of  how  DF  in  the  in  the  condensed
setting  so  so  still  if  you  start  with
yeah
I  then  it  is  automatically  a  ring  map  if
you  yes  because  of  density
basically  okay  so  we  are  not  sure
whether  topologically  OS  in  the  null
sequence  sense  is  always  the  same  as
this  well  okay  I  mean
house  door  I  mean  yeah  all  right
um  so  hello  um  okay  so
proof  so  why  is  this  uh  a  subring  um
well  I  guess  maybe  the  first  thing  to
check  is  that  it  has  a  unit  that's  part
of  what  I  mean  by
subing  um  but  if  you  look  at  it  that's
the  definition  of  solid  that's  one  way
of  saying
it
so  put  f  equals  1  here
okay  um  okay  now  I  want  to  show  that  if
f  and  g  are  in  there  uh  I'll  prove  that
it's  a  subing  by  showing  that  any
polinomial  so  if  f  is
a
uh  and  if  you  apply  any  polinomial  to  f
and  g  then  you're  still
uh  in  there  uh  so  how  can  we  do  this  so
kind  of  maybe  there's  different  ways  of
doing  it  but  I  think  the  cleverest  one
is  maybe  so  we  so  we  can  look  at  the  map
from  the  polinomial  ring  in  two
generators  to  R  which  sends  X  to  F  and  Y
to
G  and  our  hypothesis  is  that  R  is  a
solid  ZX  module  and  it's  a  solid  zy
module  and  what  we  want  to  conclude  is
that  for  any  map  here  uh  from
the  polinomial  ring  in  one  generator  T
that  R  is  a  solid  ZT  module
okay
so  so  we  can  so  first
resolve  uh  R  by  it's  it's  a  solid  it's
definitely  solid  as  an  ailan  group  so  we
can  resolve  resolve  resolve  R  by  these
guys  we  can  make  a  res  resolution  of  r
by  direct  sums  of  our  compact  projective
generators  um  but  we  know  that  R  is  a  z
bracket  X
solid  uh
so  but  uh  then  so  we  so  if  we
solidify  uh  if  we  if  we  solidify  with
respect  to  X  then  what  does  this  turn
into  it  turns  into  direct  sum  um  ah
sorry  R  is  also  a  ah  I'm  sorry  I'm  sorry
we  should  I  should  put  I  should  give
myself  my  variables  uh  X  and  Y  cuz  R  was
also  a  module  over  Z  bracket  XY  um  and
now  I  solidify  with  respect  to  X  um  and
I  get  direct  sum  of  product  of  copies  of
z  braet
x  bracket  y's  by  the  properties  of
derived  solidification  that  I  proved
earlier  but  then  and  then  it  doesn't
change  R  so  that's  still  a  resolution  of
R  and  then  we  solidify  uh  with  respect
to  Y  and  we  get  a  direct  sum  of  produ  of
copies  of  Z  bracket  X  Y  Again  by  the
same  uh
reasoning  so  uh  in  total  we  see  that  R
can  be  resolved  by  these
guys  but  each  of  these  is  clearly  uh  but
each  of  these  is  solid  over  ZT  because
it's  a  limit  it's  a  co-limit  of  limits
of  things  which  are  solid  over
ZT
this  is  solid  over  ZT  because  it's
discreet  um  and  then  this  is  a  product
and  that's  a  direct  sum  so  in  total  it's
it's  solid  over
ZT
okay
uh  so  I  didn't  directly  use  the
definition  of  solid  instead  I  used  the
description  of  these
generators
so
um  right  now  how  about  showing  that
our  uh  the  topologically  nil  potent
elements  are  all  power  bounded  um  you
can  use  a  very  similar  argument  so  what
we  have  is  a  map  from  uh  power  series  X
to  R  which  sends  X  to
f  um  which  means  that  R  becomes  a  Z
power  series  X
module  uh  so  now  if  we  resolve  R  by
direct  sums  of  products  of  copies  of  z
uh  uh  we're  allowed  to  tensor  this  with
uh  Z  power  series  T  but  we  know  how  to
compute  this  tensor  product  they're  both
just  products  of  copies  of  Z  so  this  is
direct  sum  of
uh  product  copies  of  Z  power  series  T
and  this  is  also
solid  over  ZT
all  um  so  we  conclude  that  R  is  solid
over  ZX  oh  t  i  I  switched  to  T  somehow
switch  to  X  somehow  so  so  we  conclude
that  this  composition  uh  R  is  actually
solid  as  as  a  ZT
module
oh  I  didn't  prove  oh  I  forgot  to  prove
it's  integrally  closed  oh  I'm  sorry
uh  uh  oh  let's  let's  do  that  uh
so  so  it's  again  a  very  similar  argument
so  let's  say  that  we  have  a  an  equation
of  the  form
uh
um  so  where  all  CI  are  power
bounded  um  so  again  we  just  make  the
universal  things  we  have  Z  bracket  x0  up
to  xn  minus  one  and  over  that  we  have
the  ring  where  you  adjoin  another
variable  uh  let's  call  it  t  and  then  you
set  the  equation
uh
uh
yeah  and  we  have  our  solid  ring  R  and  by
hypothesis  we  have  a  map  here  such  that
when  we  compose  to  here  uh  R  becomes  a
solid  module  over  each  of  the
variables  and  what  we  want  to  show  is
that  when  you  compose  here  uh  it  becomes
a  solid  module  over  this
variable
so  uh  I'll  just  say  it  quickly  in  words
you  use  the  same  trick  so  you  resolve  R
first  as  just  a  zx0  up  to  xn  minus  one
module  in  solid  Aon  groups  and  you
solidify  with  respect  to  each  of  the
variables  you  find  yourself  built  out  of
product  of  copies  of  this  ring  but
you're  also  a  module  over  this  so  you
can  tensor  up  to  this  but  that's  a  a
finite  free  module  over  that  ring  so
then  that  will  just  go  inside  the
products  and  you  find  that  you're
resolved  by  a  direct  sum  of  products  of
copies  of  these  guys  and  then  because
this  individually  is  solid  and  solid  is
closed  under  limits  and  colimits  you
deduce  that  that's  solid  as  well  so  the
key  here  is  just  that  this  is  finite  as
a  module  over  this  so  that  tensoring
with  it  you  can  bring  inside  the  product
finite  free  yeah
although  well  that's  not  really
necessary  the  Rings  no  Theory  and  you
can
resolve  yeah  finite  would  be  enough  in
fact
yeah  you  mean  finite  with  all  with
representation  by  finite  well  it's  an
ethereum  ring  I  mean  yeah
yeah  okay
um
so  uh  oh  and  the  rest  so
yeah  I  mean  I'll  leave  the  rest  to  you
it's  completely  analogous  arguments  like
why  uh  why  it's  an  ideal  and  why  it's
a  why  it's  a  a  radical  ideal  even  um  so
fun  exercises  in  that  style  of
argument
so  um  so  now  we  can  describe  this
structure
sheath  uh  so  now  suppose  we  have  a  again
a  solid  ring
and  then  we  have  G  and  F1  G  comma  fub1
comma  FN  in  R
solid
um  then  the  claim  is
that  well  so  now  yeah  so  there  exists  a
universal  or  an
initial
solid
ring
uh
so  and  I'll  write  I  want  to  just  I  don't
want  to  use  exactly  the  same  notation  as
who  so  I'll  decorate  it  with  a  solid  in
the  super  stri  script  um  with  a  map  from
R  uh  such  that  uh  so  first  of  all  G
becomes
invertible  uh  in  there
and  the  second  thing  is  that  uh  fi  over
G  is  power
bounded  uh  in
R  uh  for  all
R  so  that  kind  of  uh  encodes  the  idea  I
was  talking  about  with  you  want  fi  over
G  to  go  to  the  closed  unit  dis  you  want
GI  to  be
invertible  um  so  the  proof
is  uh  you  can  just  construct  the  guy  so
um  you  can  start  with
so  well  you  could  first  invert  G  but  I'd
rather  maybe  I'll  do  invert  g  at  the  end
so  what  you  can  do  is  you  can  take  R  and
adjoin  just  polinomial  variables  X1  to
xn  and  then  solidify  with  respect  to  all
of
them  which  recall  um  does  something  when
R  is  not  discret  remember  we  had  the
example  of  one  variable  and  R  was  QP
then  this  gave  us  the  Tate  algebra
um  uh  and  then  we  can  say  that  these
variables  are  supposed  to  be  fi  over  G
so  maybe  uh  GX  IUS  F  so  G  gx1  minus  fub1
uh  gx2  -
F2  gxn  minus  FN  and  then  it  doesn't
matter  at  which  point  you  invert  G  but
let's  do  it  at  the
end  and  um  this  kind  of  obviously
satisfies  the  correct  Universal  Property
so  we  first  we  freely  adjoin  solid
variables  uh  then  we  impose  the
relations  which  guarantee  that  kind  of
thing  there's  one  one  small  thing  to
check  which  is  that  after  you  take  this
in  which  these  are  definitely  solid  and
then  you  do  these  operations  you  need  to
see  it's  still  solid  but  that's  because
it's  all  Co
limits
um
okay  uh
so  um  so  this  is  the  kind  of  thing
you're  looking
at  um  and  now  I  want  to  make  a  a  caution
here
uh  yeah  know  exactly  so  just  a  sec  so
this  is  this  is  not
exactly  the  value  of  the  structure
chief  on  uh  X  F1  FN  over
G  it
is  pi  0  of  the  value  of  the  structure
so  you  have
AED  yes  so  I  said  that  this  I  when  I
remember  when  I  was  discussing  this  six
funter  formalism  and  this  localizing  the
category  I  was  always  careful  to  say  it
was  the  derived  category  and  in  fact
these  things  really  only  work  uh  these
six  funter  formalisms  and  so  on  really
only  work  at  the  derived  level  and  what
that  means  is  in  what  you're
producing  a  prior  is  a  derived  chath  not
an  ordinary
chath  um  so  so  that's  uh  okay  but  but
now  I'll  say  but  in  in  most  practical
cases  almost  all  practical
cases  uh  all  Pi  I  equals  z  for  I  bigger
than
zero  so  in  practice  it  doesn't  seem  to
cause  trouble  but  it's  an  important
thing  to  keep  in  mind  and  the  second
warning
the  second  warning  is
a  so  even
if  R  is  an  is  is  very  nice  some  kind  of
Huber
ring  uh  then  uh  this  quotient  uh  may  not
be
uh  so  this  uh
this  may  not  be  uh  quasi
separated  so  um  okay  so  it's  a  kind  of  a
nonous  of  quo  of  of  of  restricted  form
of  Series  in  gender  yes  exactly  so  this
this  will  always  be  some  this  will
always  be  the  usual  Tate  algebra  and
many  generators  that  follows  basically
from  the  arguments  we  gave  for  any  sort
of  Huber  ring  R  but  then  when  you  take
this  quotient  the  ideal  generated  by
these  elements  might  not  be  closed  so  in
principle  you  could  be  having  a
non-house  dwarf  quotient  here  um  so  you
start  when  you  say  Huber  ring  you  mean
complete  hu  yes  I  mean  complete  Huber
ring  thank  you  yes  yes  um  but  again  but
this  theory  is  defined  for  complete
ubering  as  well  or  no  this  theory  is
only  defined  for  complete  Huber  Rings
the  theory  I'm  discussing  because  you
cannot  Define  the  condensed  of  any
topological  thing  you  can  Define  it  but
it  won't  be  solid  unless  unless  the
thing  is  complete  in  general  ah  okay  you
wanton  Sol  it  so  you  won't  complete  and
then  the  other  guys  when  you  do  this  it
is  is  still  solid  but  doesn't  come  from
it's  kind  of  nonous  of  yes
yes  yeah  but  again  uh  in  almost  all
practical  cases  or  maybe  all  practical
cases  uh  it  is  quasi
separated  I.E  house
dwarf
um  so  and  then  so  so  what  I  was  trying
to  aim  for  is  the  relation  to  hu's
Theory  so  so  I  but  I  I  so  I  Pro  I  guess
I  probably  won't  get  there
uh  so  if  so  I  think  we'll  probably  we'll
definitely  discuss  this  more  detail
later  but  just  as  a  preview  so  if  if  R
is  a  a  Huber  ring  and  don't  worry  if  you
don't  know  what  that  is  it's  just
certain  nice  kind  of  topological  ring
that  people  use  in  non  archimedian
analysis  ah  and  I  let  say  complete  h  no
just  hubber  I'm  just  doing  the  Rings  now
I'm  not  not  doing  the
pairs
um  and  if  uh  so  and  if  the  ideal
generated  by  F1  up  to  FN  uh  inside  R  is
open  which  is  the  condition  that
describes  rational  opens  in  the  space  of
continuous  valuations  as  opposed  to  the
space  of  AR  arbitrary  valuations  then  in
this  case  hubber
defines
uh  uh
the  Ring  of  functions
here  and  you  could  ask  what  the
relationship  is  with  this  thing  that
comes  from  the  solid  Theory  um  and  this
is  the  uh  quasi  separ
ification  uh  of  this  more  generally
defined  thing  that  we  have
here  so  the  solid  thing  or  maybe  for
emphasis  I  should  maybe  put  the  pi  Z  as
well  so  you  can  get  the  Huber  thing
functorially  from  this  more  uh  General
thing  in  particular  from  our  structure
Chief  but  they're  not  necessarily  equal
in  general  but  in  all  practical  cases
sort  of  they  are  equal  that's  the  the
general  outline  of  the
story  so
General  okay  is  not  and  so  so  in  general
if  you  always  use  the  structure  shft  do
always  satisfies  the  yes  so  it's  always
shy  yes  yeah  so  and  yeah  so  maybe  that's
an  important  point  to  mention  there  was
this  little  flying  the  ointment  in
Huber's  theory  that  for  in  the  general
setting  uh  he  had  he  defined  a  structure
sheath  except  it  wasn't  a  sheath  it  was
only  a  prief  and  in  all  all  practical
cases  it  was  a  sheath  but  still  it  was  a
little  bit  the  general  theory  was  kind
of  missing  something  something  for  that
reason  um  that's  fixed  by  this  so  if  you
don't  if  you  no  longer  care  about  things
being  quasy  separated  and  non-derived
then  you  get  a  a  good  a  good  plain  old
structure  sheath  and  you  get  even  more
you  get  derived  category  of  quasy  gent
she  which  localizes  and  also  you  you  get
the  possibility  of  um  uh  of  of  defining
all  of  these  things  even  without  this
condition  being
present
um  okay  I  think  probably  I'll  stop  there
thank  you  for  your
attention
yes  do  we  have  any  use  of  don't
continuous
varations  any  use  yes
uh  uh  I  know  Coles  likes  them  so  he  he
came  he  has  this  has  this  notion  of  AD  o
spaces  and  and  those  correspond  to
certain  yeah  but  you  don't  really  use
them  um  I  think  he  used  them  but  you  oh
me  well  look  I  I  just  build  the  theory  I
don't  he  calls  him  ad  hoc  spaces  I  don't
think  he  developed  the  theory  he  just
kind  of  did  it  in  an  example  um  so  he
wanted  to  he  wanted  to  exactly  include
things  like  this  open  unit
disc  kind  of  uh  so  this  yeah  things  like
the  zp  I  think  in  his  paper  where  he
does  this  GL2  mod  P
langland  uh  is  that  right  yeah  I  don't  I
don't  know  so
so  he  wants  to  this  this  thing  is  a
prototypical  example  of  a  topological
ring  that's  not  a  Huber  ring
so  um  so  in  the  usual  theories  of  viid
analytic  geometry  the  open  unit  disc  you
can't  think  of  it  as  an  apine  space
because  the  the  ring  that  it  corresponds
to  is  not  one  of  your  allowed  rings  so
they  always  so  you  always  have  to  think
of  the  open  unit  disc  as  a  union  of  the
closed  unit  discs  contained  in  it  it's
this  non-  quasy  compact  thing  to  fit  it
inside  Huber's  Theory  but  Coles
Advocates  that  sometimes  it's  better  to
to  just  work  with  this  thing  as  if  it
were  apine  and  that's  something  that  our
Theory  easily
accommodates  yeah  there  are  people  who
who  did  I  forgot  the  name  of  this
and  construct  some  notion  of
rigid  I  forgot  the  name  some  draw  some
paper  on  this  on
like  variant  of
three  I  about  the  name  of  the  there  was
some  paper  some  time  ago  but  not
sure
okay  any  other  questions  yes
yeses  every  Sol  z
albra  uh  yes  so  you're  yeah  so  this  this
notion  of  analytic  ring  we  haven't
defined  it  yet  it  kind  of  organizes  a
lot  of  discussion  but  indeed  yeah  if  you
have  a  if  you  have  a  solid  if  if  you
have  algebra  in  solid  Z  modules  you  get
what  we  call  an  induced  analytic  ring
structure  so  it's  just  all  you  take  all
I  mean  I  said  the  module  category  you
have  and  that  that  sits  inside  condensed
our  modules  and  that  is  an  analytic  ring
yeah  but  in  case  of  Z  we  had  something
more
natural  more  natural  it's  arguable  it's
more  complete  uh  there're  two  things
they  exist  they  do  I  mean  they  you  want
them  both  I
think
and  can  be  extended  can  be  extended  to
any  discreet  ring  right  yes  that's
right  not  in  general  not  not  General  Z
solid  Z  algebra  well  for  a  general  solid
Z  Algebra  I  don't  know  of  any
necessarily  know  of  any  like  completely
canonical  well  like  maybe  I  mean  one
thing  you  could  do  is  you  could  take  all
of  the  Power  bounded  elements  and  force
all  of  them  to  be  solid  that  would  be
kind  of  the  maximally  complete  thing
that  you  can  get  via  the  stuff  that
we've  developed  Ved  but  there  could  be
there  could  be  further  completions  of
that  as  far  as  I  know  I  mean  I
uh  you're  welcome  yes  with  the
definition  of  solid  anal  Rings  we  used
in  the  very  first  lecture  if  we  use  this
instead  of  a  regular  ring  here  do  we  get
something  similar  to  like  Huber's  theory
for  Huber  pairs  instead  of  just
regular  could  you  could  you  repeat  the
question  I'm  not  not  sure  I  understood
in  the  very  first  lecture  we  find  like
solid  analytic  ring  solid  analytic  Rings
yes  yes  yes  with  the  ring  and  the  direct
category  uh  if  we  try  applying  something
similar  here  to  analytic  Rings  what  do
you  mean  what  do  you
mean  ah  you  mean  like  this  discussion  of
this  this  thing  here  oh  yeah  yeah  yeah
you  can  do  that  yeah  but  do  we  recover
like  oh  yeah  you  you  recover  the  theory
of  with  with  the  r+  in  there  as  well
yeah  so  elements  R  plus  you  require
solid  solidity  exactly  exactly  not  for
all  of  the  guys  exactly  exactly  so  you
can  you  can  you  can  yeah  and  we  will
discuss  this  you  get  you  can  add  that
extra  flexibility  into  the  picture  yeah
and  the  fact  that  R  z0  is  this  is
automatic  that  those  are  the
to  they're  automatically  solid  exactly
yes  okay
yeah  so  it  actually  it  actually  fits
remarkably  well  with  Uber's  Theory
a  Prett  the  like  you  take  tens  product
of  analytic  R  you  have  to  first  like
solid  one  another  have  to  take  yeah  yeah
usually  yeah  so  here  it  still  the  same
here  you  don't  have
well  these  all  these  solidifications  all
all  commute  with  each  other  so  there  you
don't  need
to  yeah  they're  all  given  by  R  homing
out  of  out  of  something  and  any  two  R
homing  out  of  commute  with  each  each
other
yeah  other
questions  okay  well  see  you  on  Friday  or
next  Wednesday  or
whenever
\end{unfinished}