% !TeX root = ../AnalyticStacks.tex

\section{\ufs !-topology (Clausen)}

\url{https://www.youtube.com/watch?v=vRUmXU8ijIk&list=PLx5f8IelFRgGmu6gmL-Kf_Rl_6Mm7juZO}
\renewcommand{\yt}[2]{\href{https://www.youtube.com/watch?v=vRUmXU8ijIk&list=PLx5f8IelFRgGmu6gmL-Kf_Rl_6Mm7juZO&t=#1}{#2}}
\vspace{1em}

\begin{unfinished}{0:00}

Let's begin. So I'll be talking more about this Zariski topology, but it's been a little while since we've last met, and so maybe I'll start with a recap.

Recall---well, let me start at the very beginning. We have this category of analytic rings. The objects in here are pairs---there are different choices about exactly what sort of data you want to put in the second piece. I'll take the sort of full derived category instead of some connective derived category or some abelian category, where this is a condensed---and in general, we want to say animated ring, so we're allowed to have some derived phenomena. This is a certain full subcategory of what you could call $D$ of $R$-triangle, which are triangle modules in derived condensed abelian groups, satisfying certain nice closure axioms for this, which I won't recall right now.

And then we define the category of affinoid spaces to just be the opposite category. And we singled out two---well, three, I guess---classes of morphisms of affinoid spaces. 

So $f$ from $X$ to $Y$ in $\mathcal{F}$ is proper if the pullback map from $\mathcal{O}_Y$ to $\mathcal{O}_X$ has a good right adjoint. Good means that the right adjoint commutes with pullbacks and also with colimits, and satisfies the projection formula. 

The maps called open immersions are those where the pullback map $\mathcal{O}_Y \to \mathcal{O}_X$ is a localization and has a good left adjoint, again in the same sense as before. For such an open immersion, there will always be some idempotent algebra in $D$ of $Y$ which is somehow the "functions on the closed complement" and determines this situation.

Called shable if it factors as $X \xrightarrow{f} Y$, where $f$ is proper and $Y \xrightarrow{g}$ is an open immersion. There are good closure properties for all three classes of maps: they are closed under composition, closed under base change, and contain all isomorphisms. Furthermore, if you have a map like this and a map like this, and they're both shable, then any map between them making the triangle commute is also shable.

We then get a six-functor formalism on $\mathcal{F}$, where the class of maps for which the important functor, the lower shriek functor, is defined is exactly the class of shable maps. The lower shriek functor $f_!$ is $f_*$ for $f$ proper, and $f_!$ is the left adjoint to $f^*$ if $f$ is an open immersion. This can be obtained by passing to the diagonal, as the closure properties imply that the classes are closed under passage to diagonals.

We then had a definition: a shable map $f: X \to Y$ in $\mathcal{F}$ is a shriek cover if the map from $\mathcal{D}(Y)$ to the limit of the Čech nerve is an equivalence, where we use the upper shriek functor for the transition functors. We then had a result that for shable $f$, the following are equivalent: (1) $f$ is a shriek cover, and (2) we have a Čech-type equivalence, where we use the lower shriek functors and take the colimit in the category $\mathrm{PRL}$ (presentable $\infty$-categories with colimit-preserving functors) or in modules over $\mathcal{D}(Y)_\mathrm{PRL}$. Here, $\mathcal{D}(Y)$ is a presentably symmetric monoidal category, so the tensor product commutes with colimits.

Colimits in each variable make it a commutative algebra object in PRL with respect to L's tensor product. Then you consider modules, so it's a presentable $\infty$-category which is tensored over $\mathcal{D}(Y)$ in a kind of colimit-preserving way.

The third condition was the "shriek cover" condition. You have this "shriek descent," where you use this funny twisted pullback. But it turns out that it implies that you get descent with the star pullback, and in fact it implies you get the same result with coefficients for all $M$ in $\mathrm{Mod}_{\mathcal{D}(Y)}(PRL)$. 

If you take $M$ to be $\mathcal{D}(Y)$ itself, then you see that this condition is just the usual descent, that is, descent with respect to pullbacks. But in fact, you can even tensor that with any module, and you still have descent.

The fourth condition was this kind of "two-descent," this categorified version, which says that the whole category of possible $M$'s satisfies descent. Here, the only thing that makes sense is star descent, but I'll emphasize it's just some naive pullback, where the base change functors are just relative tensor products in PRL.

One can ask, instead of just the fpqc topology in algebraic geometry, using quasi-compact, whether you have descent for an arbitrary collection of maps, not necessarily finite. We think that is true, that you could try to make an analogous infinitary version of the Grothendieck topology, and it will end up being finitary anyway. Peter checked this carefully, and the answer is yes - every cover will have a finite subcover.

In the case of just finite sets, if you have the conerter set and take all morphisms from $\P$ to the conerter set, which is big, this is universally submersive. This means that when you cross with any topological space, to check if something is open, you can check the pullback.

Of course, you can ask whether you have the right descent properties, which is not so clear. But at least for opens, it seems to work. Peter discussed this with some logicians, and they had some conclusions. It can happen that you have some profinite set and some infinitary cover by other profinite sets which satisfies the descent, at least in some examples, using the perspective of the derived "shriek" descent condition. Equivalently, along the "lower shriek," you can show that it is finitary, but it's really something that only works when you ask for "star descent," not for "shriek descent."

So those are the equivalent conditions for being a "shriek cover." Now, I want to talk about how you can check these conditions.
That image doesn't have to have any closure properties whatsoever. It doesn't have to be a stable subcategory or anything like that. But then you can take the thick subcategory generated by it. This means a closure under just finitary operations like cones, retracts, and shifts.

So, you get a priori a bigger category. You can ask whether it's the whole thing, or whether it just contains the unit. Let me write $1_Y$ for what you would normally think of as the structure sheaf of $Y$, the unit object in this symmetric monoidal category with respect to its tensor product.

The condition again is that this unit object can be written in a finitary manner in terms of things which are lower shriek from $\mathcal{D}(\mathcal{X})$. Can it be written in an infinitary manner? Yes, because it's a sheaf, it follows.

Oh, you're saying that if I say this condition but with infinitary, does it automatically imply the same condition but with finitary? I don't know what the relation between those is. I think the key point is that in the finitary case, this is a compact object in $\mathcal{D}(Y)$. So I think indeed they should be equivalent.

Maybe it's even better to say that if it's a proper, universally locally acyclic map, then you satisfy this condition. And the converse holds, provided either $f$ is proper, or $f$ is universally locally acyclic. Let me explain what that means: it means that $f_!$ is "good", which means it commutes with pullbacks, colimits, and satisfies the projection formula.

Those properties of $f$ are not true in general, and they're not equivalent - there are cases where some hold and others don't. There may be some non-trivial implications, but the projection formula clearly implies that $f_!$ commutes with colimits, so I didn't need to write that separately.

The main claim is that the "universally locally acyclic" condition is kind of like an equisingularity property for $f: X \to Y$, where the dualizing object is compatible with pullback. But the converse fails in general. Let me give a counterexample...


Category. So let's call this mapping by J, and this mapping by I. This induces a map from the disjoint union. Then J lower shriek of the unit is this compactly supported thing, which is the fiber of the homotopy fiber of ZT going to the series T-inverse. But I lower star, well I lower shriek of the unit there, I is proper, it's a closed immersion, so this thing just gives a Zariski series T-inverse. It's clear that you can build Z of T from this fiber and this guy by just one cofiber sequence.

Then the lower shriek from X the disjoint union will just be given by restricting to Z. Take the lower shriek there and restrict to Z, take the lower star there, and then take the direct sum of those two objects. So you'll get this guy direct sum this guy. Therefore, by closure under retracts, if you allow closure on retracts, you get each of them, and then you get ZT individually.

So it is not a sheet cover, because for the cover you just get Y and Z separately, somehow you get the derived of Y cross the derived of Z, not the derived of Y. Okay, here you can have some X or something that is not the same in Y, and then you lose some Z. 

So this condition is something like just set-theoretic surjectivity on underlying sets, or maybe it's like a cover in some constructible topology or something. And then if you want to go from that to some honest descent, you need to assume some properties. This is analogous to, in topological spaces, there's descent for open covers, but there's also descent for finite closed covers or proper descent, but you can't mix open and closed and still expect to get descent.

So this is some sort of set-theoretic cover condition, and then if you assume either that you're some generalization of open, which is this étale, or some generalization of closed, which is this proper, then you get honest descent. But you can't mix them.

So the proof - let's make explicit what this shriek descent according to the definition means. We're using these upper shriek functors, and there's some standard category theory which tells you that this comparison functor itself will then also have a left adjoint, which is given by taking a colimit in this category D of the G lower shrieks of them. So in particular for fully faithfulness, this amounts to the claim that if you take this colimit, it should be an isomorphism in D of Y for all M in D of Y.

If you take M to be the unit in Y, this is a geometric realization, it's a colimit over this Delta op. You can filter that, you can always write this as a colimit over the natural numbers and then a partial totalization. And this is a finite colimit, that's a nice fact about the simplex category. But then, since the unit object is compact, if you write it as a filtered colimit of something...

Let's look at sets of cardinality less than or equal to D. Since the unit interval Y is compact, we can deduce that it is a retract of some partial totalization, where each of these lies in the image of the shriek functor from D of X to D of Y. All of these structure maps from these iterated fiber products factor through D of Y. This is equivalent to a finite colimit, a colimit over a finite simplicial set. 

In the stable setting, the difference between this for D and this for D-1 is just given by one of these objects up to a shift. The successive cofibres are described in terms of these individual objects. So in the stable setting, some kind of rewriting simplifies all of this.

A finite colimit is defined in a higher topos. This is not more general than a colimit over a usual category, but you have to be careful, as a finite category in the usual sense might not be finite when considered as an infinity category or a simplicial set.

For part two, the hypothesis implies that every M are generated by things in the image of the shriek functor, using the projection formula. We want to deduce at least the fully faithfulness. We can assume that M is of the form F shriek N, and then with a base change result, we can reduce to a split situation.

And all of these maps $G$ are like compositions of pullbacks of $f$, so if $F$ has one of these two properties, then all of the maps $G$ will as well.

In the proper case, $G_!$ equals $g_*$, and the base change follows from the $f^* g_! \cong g_* f^*$ base change by passing to adjoints. And in the proper case, we have a sort of $G_*$ is $G_*$ of $1$ tending to $G^*$, and the base change follows more directly from again the $f^* g_* \cong g^* f^*$ base change. So there is some---you have to make sure the two base change comparison maps that you have are equal, but okay.

In order to conclude from one being an isomorphism that the other one also is, but okay. 

So that was that---proves fully faithfulness for two. So that gives full faithfulness in the $!$-descent, but the essential surjectivity, or so the other adjoint, or the unit or the counit, or whichever it is, the one going from here to there and then back up here again---that's proved in the same way, handled similarly using base change to reduce to the situation where the cover is split. So you're pulling back to $X$ where the cover is split, so we have a functor where we want to claim is an isomorphism. We've identified an adjoint to it, and what I've just explained is that if you do the functor and then the adjoint, that's the identity. And you'd also need to check that if you do the adjoint and then the functor, you get the identity, and I'm claiming that's handled similarly using base change along $X$ going to $Y$, where it happens for formal reasons because the cover is split.

Okay, so we're part of the way towards---so now we've, this is kind of a more concrete thing that you might hope to be able to check. So you have to be in one of these two situations. Let me explain some special cases.

Special cases. So one will be closed covers, finite closed covers. We defined a notion of a proper map and we defined the notion of open immersion, but we didn't define the notion of closed immersion. So if $f$ is proper and, well, one way of saying it is that the pullback from $\mathcal{D}_Y$ to $\mathcal{D}_X$ is a localization---the right adjoint exists, and the right adjoint is fully faithful. Nothing more, no, but a localization in category theory---in which context is defined now for categories---of which kind I forgot. I know that people like in Gabriel and some---I don't remember now what the. So let me say that---so this is a funtor which admits a right adjoint, so when the right adjoint is fully faithful, which is what I'm calling localization, it follows that you have a universal property for limit-preserving functors out of here, namely they're the same thing as limit-preserving functors out of here which kill every object, or sorry, which invert every map which is sent to an isomorphism by this functor. So this is an analog of localization for triangulated categories, for example.

Algebra: So tensors are the product of two copies of itself. $\mathcal{D}_{\mathscr{Y}}$ is itself again via the multiplication map.

Since $f$ is proper, let me make a warning. On the level of these $\infty$-topoi, it's not generally true that closed and open immersions are in bijection, with the same target. So an open immersion is not necessarily going to have a closed complement, and a closed immersion is not necessarily going to have an open complement.

It's close to being true that an open immersion has a closed complement. The only...Let me expand on this. Given an open immersion $U \to \mathscr{Y}$, we get an idempotent algebra $a$ in $\mathcal{D}(\mathscr{Y})$ such that $\mathcal{D}(U) = \mathcal{D}(\mathscr{Y}) / a$. But it's not true in general that $a$ lives in $\mathcal{D}^{\geq 0}(\mathscr{Y})$. This is the condition needed to get a closed immersion in $\mathsf{Aff}$. If you start with just an idempotent algebra, it doesn't necessarily correspond to a closed immersion in $\mathsf{Aff}$ because it doesn't necessarily have a correct underlying animated ring. It's only the connective ones that correspond to animated rings, not the non-connective ones.

In the case of usual schemes, can you recall what are the open... Well, it depends on which functor from schemes to analytic spaces you're using, so we'll go into it. But I want to say that this is analogous to some complementary phenomenon in scheme theory, where for an affine scheme, you have a closed immersion, but the open complement might not be affine. It might not correspond to an open immersion in affine schemes, but it's still a scheme. And it's kind of similar here, even in situations where this is not connective, usually you will get a closed complement which is an analytic space, it just won't be an affine one.

In the case of schemes, like taking the complement of $\mathrm{Spec}(A)$, you get $\mathrm{Spec}(A)$, which is important. But does this correspond to a closed immersion in this setup? When you take $\mathcal{D}(a)$, what do you get? Do you get $\mathcal{D}$ of some...which you call an open...a closed here? But I'd rather let me again get to the comparisons with the classical theories a bit later in the lecture, although this is going much slower than I anticipated.

I could give an example. If you look at $\mathscr{Y} = \mathbb{A}^2_{\mathrm{sol}}$ and $U$ to be $\mathbb{A}^1_{\mathrm{sol}}$, then I invite you to do the very good exercise of figuring out what this idempotent algebra is, and then the corresponding $a$ has a nonzero $\mathcal{H}^{-1}$.

The situation on the other side is somehow even worse. Given a closed immersion, well, actually Peter described the condition required for there to be a complementary open. A closed immersion corresponds to an idempotent algebra in the $\geq 0$ derived category of $\mathscr{Y}$. Then you need for there to be a complementary open in $\mathsf{Aff}$, you need that the internal $\mathcal{R}\mathrm{hom}$ from the fiber of the unit of $\mathscr{Y}$ going to $a$, which is a functor from $\mathcal{D}(\mathscr{Y})$ to $\mathcal{D}(\mathscr{Y})$, you need this commutes with filtered colimits and preserves $\mathcal{D}^{\geq 0}(\mathscr{Y})$. That's the formula for what would be the localization functor to the complementary open, and you need that that actually defines an analytic ring structure in our sense, which amounts to these conditions.

Right, so as I said, some of this will be fixed by allowing general analytic spaces, not necessarily

Okay, I was talking about finite closed covers right as an example of descent. So, suppose we have finitely many closed subsets $Z_1, \dots, Z_n$ with closed immersions into $X$, and all of these are in the image of some map $f$. When do we get a cover? Well, the disjoint union of the $Z_i$ mapping to $X$ is a cover if and only if the structure sheaf satisfies the most naive form of descent. 

This is actually going to terminate at a finite stage because it's a finite closed cover. The condition is that the structure sheaf is a sheaf, which is given by the tensor product of the item-potent algebras associated to the closed immersions. 

Why is this the criterion for being a sheaf cover? If this is satisfied, then we are in the image of the lower-star functor from the disjoint union of the $Z_i$, since each of these is closed. This implies $\mathfrak{f}$-descent, which in turn implies the fancy $\star$-descent, giving the structure sheaf condition.

Now, what about open covers? It turns out that every open cover has a finite refinement. So let's consider the case of a finite open cover $U_1, \dots, U_n$ mapping to $X$ by open immersions. The claim is that the disjoint union of the $U_i$ is a sheaf cover if and only if the tensor product of the corresponding item-potent algebras in $\mathcal{D}(X)$ is zero. The reason is that the unit object is compact, so this condition is equivalent to one of the algebras being zero, which happens if and only if the cover is a cover.

For simplicity, let's focus on the case $n=2$. Then the $\star$-descent condition gives that...


If you look at what $\star$ descent means and use the formula for the upper $\star$ functor, which is this kind of localization formula, then you find that the claim of $\star$ descent is exactly the claim that you have a pullback of this form.

Wait, I'm sorry, I'm getting myself awfully confused right now. No, I'm getting myself very confused. This is the $\star$ descent for the closed complement. I'm sorry, I'm sorry.

Of course, if something is a module over $A_1 \otimes A_2$ and it goes to zero on the $U_i$, so it goes to zero in each stage of the simplicial diagram, so it goes to zero apparently in this limit in the $\infty$-categorical sense. So if this condition is not satisfied, then there is a $\star$ descent.

I apologize, I kind of assumed I would be able to do this off the top of my head and I didn't think about it carefully. 

Let me say, the $\star$ descent for this cover, where you have two elements and both of them are mapping by monomorphisms into $X$, then the descent, which a priori involves some Čech nerve which is some infinite thing, it actually reduces to some Mayer-Vietoris. As is kind of standard, it's the same thing as $D(U_1 \cap U_2)$ being the pullback of $D(U_1)$ and $D(U_2)$, with the upper $\star$ maps. Then you can check that you have the map functor from this to the pullback, and again it has this left adjoint, so the claim for the unit gives that the unit of $X$ receives an isomorphic map from $J_1^{\star}$ of the unit on $U_2$ or $U_1$.

You have some kind of Mayer-Vietoris sequence like this, so this is a cofiber sequence. And then you have formulas for everything. I apologize for not explaining this very well, but you have formulas for all of the functors involved in terms of the corresponding idempotent algebras. If you work it out, it's just going to amount to the condition that $A_1 \otimes A_2$ is equal to zero, meaning that this condition will be directly equivalent when you write down what everything means to $A_1 \otimes A_2 = 0$. 

Open immersions are special, so $J^{\star}$ is one for open immersions. The $J^{\star}$ of this, how is it given in terms of the algebra? This term, for example, will be the fiber of the unit mapping to $A_1$. I made a mistake by not preparing this properly, because I thought it would just come to me, but yeah, I'm sorry for messing this up.

Let's give some examples now. The first example is Zariski covers. Note that there is a functor from the usual category of commutative rings to the category of analytic rings, which sends a commutative ring $R$ to the pair $(R, \mathrm{D}(R))$, where $\mathrm{D}(R)$ is the full derived category of $R$ modules in the category of condensed $R$-modules.

Viewed on the level of opposite categories, and maybe a remark is that this functor commutes with fiber products. In fact, it also sends the terminal object to the terminal object. In other words, relative tensor products in commutative rings are also relative tensor products in analytic rings, which follows from our discussion of relative tensor products in analytic rings. 

Moreover, the relative tensor product---I mean, the derived one---covers $f$ a map to $f$ a shriek covers. But now it's occurring to me that I forgot to remind what this means. So, note that we get a Grothendieck topology on $F$ by saying that a sieve over $X$ in $F$ is a shriek cover if it contains finitely many $Y_i$ mapping to $X$, such that the disjoint union $\coprod Y_i$ maps to $X$ as a shriek cover in the previous sense.

The key behind this, besides the obvious properties of finite disjoint unions, is that if you have a shriek cover, then any base change is also a shriek cover. This is a consequence of the discussion of colimits in PRL. Basically, the base change functor on the level of Mod-PRL just commutes with colimits, so if you have the condition there, then base changing, you get the condition. 

The proof is simple. Indeed, Zariski covers go to closed covers in the sense just discussed. If you have a principal open in Spec $R$, given by inverting some element, that inverting an element gives you an idempotent algebra, which defines a closed cover on the level of these guys. And the condition we had to check is just usual Zariski descent.


The whole formal neighborhood of that, and this then it acquires some fuzz. I claim that what you should really think is that the fuzz is making this thing really behave more like an open subset, and the Zariski open should really be thought of as closed, and it should have some kind of tubular neighborhood. So it should really be a closed subset, and then the formal neighborhood is the open complement. That's the picture I would like to suggest.

And when you go to the solid world, then you can again have an open version of puncturing. So you can name this boundary, maybe something like $\Z\langle\text{series } T\rangle$ base changed along $\Z[T]$ to $\R$ where $T$ goes to $F$, something like that. You can name the boundary and then you can remove it, so it's going to be a closed subset that you can remove, and you get an open subset. Then you're back to the usual way of thinking of having a Zariski open.

But then it's not---you take $\R[T]$, $\Z$, and then $\Z\langle\,\rangle$. So I'm saying once you move to the solid framework, then you can name this boundary here, which before was just heuristic, and its name is this. That will give it an input in algebra in $\mathcal{D}$ of $\R[z]_{\text{solid}}$, and the complementary open is like $\mathcal{D}$ of $\R[1/f(z)]_{\text{solid}}$, so $\Z[1/f(z)]$.

I mean, there are two different ways you can embed schemes into analytic spaces---one is based on sending $R$ to $R\langle z\rangle$, and the other is based on sending $R$ to $(R, R)$, and this is the one I'm discussing right now, where you base change to the solid $\Z$. And then the Zariski opens look closed, but if you use this one that corresponds to always removing the boundaries, then the Zariski opens actually go to open immersions.

So what do you mean by boundary? I don't really know what I mean by boundary, if I don't just mean the formal neighborhood minus the middle. Intuitively, I'm claiming you're removing kind of an open piece from this chunk, and then the closed complement should intuitively have some boundary. I mean, there will be boundaries at infinity too if your thing isn't proper, and what's happening at infinity is also important, but let's pretend we're in a proper thing or something.

In particular, we get a functor from Zariski sheaves on derived schemes to sheaves on $\F$, which is in fact the pullback of some topology. So it commutes with colimits and finite limits, which is a consequence of this. This is one way of embedding usual algebraic geometry into what I haven't quite defined what an analytic space is, but it's close enough for practical purposes. We're going to have some kind of hyperdescent condition we want to impose as well, which I thought I was going to get to today but I clearly didn't. But this is basically analytic spaces or analytic stacks, modulo a couple of technicalities.

Conditions and when you analyze this, you use the weaker one with which the Serre-Swan theorem covers, although if you did a stronger one, it would still give the same claim, because it would be just a further localization of this. On the other hand, yes, well, I'm not going to touch the other side - one could, but the risk is that it also depends on whether you have... Yes, yes, it does. But I'm not going to care too much about that side. It was said in some talk, I don't remember who, that maybe Peter said that there would be something intermediate between Čech descent and full hypercover.

I was going to discuss it today, but it's not going to happen. Okay, so the other subtlety is a set-theoretic subtlety, because the category of commutative rings is not small, so you have to be careful considering presheaves and sheaves on it. And the same if you take only those which are accessible - yes, yes, exactly. And then you have to prove that sheafification preserves this accessibility and so on and so forth. But okay, I don't think in the remaining 7 minutes I could do justice to the next topic, which was going to be adic spaces. I could rush through it right now, but I don't think that's a good idea, so I'll stop here.

I wanted to know what, for example, should cover... Oh, oh, oh, oh, sorry, I didn't understand your question. I'm sorry. Whether in a tall cover also gives you a sheaf cover - yes, it does, it does. Right, so this is something I should do in a few minutes.

So this was all motivated by Matthew Emerton's theory of descendability. So we were talking in AF, and then we were saying we have this derived category of anything in AF, and it's built on this condensed framework. But it's clear that the discussion is very categorical in general, and we could just try to make the exact same definitions in the world of ordinary algebraic geometry instead, using the usual derived category of a ring, and see what kind of definitions that gives.

If you take the same definitions, but with the pair R and DR, where R is a usual commutative ring or maybe derived, and DR is the usual derived category, then some simplifications happen. First, every map is proper, which is quite clear - well, maybe that's the main simplification that happens, that every map is proper. This implies that every map is also sheafifiable, and then the proposition I discussed earlier also holds in this setting, where there's no condensed thing involved.

The upper shriek and lower shriek here are not the same as the ones in Grothendieck duality theory - they're just some categorical adjoints. Nonetheless, it turns out to be useful in this descendability discussion, because every map is sheafifiable and every map is proper. And then we deduced that a shriek cover, X to Y, is the same thing as saying that the unit in Y lies in the image of F lower star D of X to D of Y, and this is exactly the definition of descendability in Matthew's work, or one of the equivalent characterizations he gives.

What about the classical Grothendieck flat and faithfully flat descent? Is it related? Let me give some examples. Actually, every kind of purely formally descendible map, R to S in an $\infty$-category sense, gives a sheaf cover in the usual sense via this functor. And Matthew gives many examples of descendable maps, such as a tall cover.

So, this is kind of funny. You need this, well, we apparently probably really need this countably presented hypothesis. That means that you have a map from $A$ to $B$ which is faithfully flat, but also $B$ is presented as an $A$-algebra, or maybe even just countably generated is enough, so $B$ has a presentation as an $A$-algebra with countably many generators.

The condition is that on $\pi_i \Z$, it is faithfully flat and the $\pi_i$ are countably generated. This is related to some limit, higher limit. I mean, if it is a limit of $A_n$'s, that's fine too, but in practice it's the same thing. If it is only countably generated, then you don't get it. You need it to be countably presented, because you need to be able to reduce to a countable base ring.

The point is, in a lot of situations where you have classical descent, you get this even stronger Čech descent, which also gives you descent and much else besides. But also, there's some issues if you have $R \to R/I$ where $I$ is a nilpotent ideal. The Čech descent is defined on the level of derived categories, so the descent you get for this does not imply that $D(R)$ is the same as $D(R/I)$ obviously.

The reason is that when you do the descent, you're doing everything on the derived level, and you end up with terms like this in the limit diagram, and those are not the same as $R/I$. Going to $\pi_i \ZR$, does it have the same property as $R \to \pi_i \ZR$? No, you can have a polynomial algebra with a degree 2 generator, and this has a module which by inverting that degree 2 generator goes to zero when you mod out by $X$, but is non-zero.

So, the analog for simplicial rings of a nilpotent ideal is that you should ask that the ring be truncated, so it has only finitely many homotopy groups. Then going to $\pi_i \ZR/I$ would work.

More generally, if you have a proper map of derived schemes, then you also get Čech-descendable. That's a generalization of this. And more generally, any faithfully presented cover, that's kind of a combination of this and that. There's a huge class of very, very much, but you do have to remember that in non-flat cases, the descent involves a higher topos and so on. All of these kinds of things will go to Čech covers in our setting.

Question: Was it equivalent to have a map of commutative rings be descendible in the ordinary commutative algebra Matthew sense, or for the image under this functor to be a shable map, a strict cover in our sense?

So, what's the argument? This is important to understand in this kind of relative sense. The descent and this pro-object formalism mean that this pro-object is just an old discrete pro-object. Okay, yeah.

Peter was pointing out that there's another characterization of descendibility, which is in terms of just the rings - like A, B, B tensor A B, and so on. You have to have descent, you have to get a limit diagram. But then you have to get even more: you have to get a very stable limit diagram. It has to be a pro-isomorphism between the tower you get from this co-augmented thing, the n-index tower, and that pro-object should be pro-isomorphic to the constant pro-object A.

In that condition, it's clear that it's independent of which framework you put it in, because the pro-category - the usual D(R) sits fully faithfully inside D(condensed R), and therefore Pro of usual D(R) sits fully faithfully inside Pro of D(condensed R). So for the pro-thing, it's enough to work in the homotopy category - it's enough to work in a weaker framework.

What you can do is pass to the fiber, and then you want to know about a tower being pro-zero. It's enough to look at it in the homotopy category. Okay, thank you again for your attention. 

When a tower is pro-zero, is it then the case that it is uniformly pro-zero? That is, because there is some finite domain, it's enough for any stage to add a fixed number? Exactly, yes, in Matthew's work. And it works with Simpson, who works in a very general setup, just like commutative algebra objects in presentably symmetric monoidal categories. He only looks at the proper case, so every map is proper in our sense.

So the setup is a symmetric monoidal category C, and then you consider an algebra A in C. He defines descendibility in this context. It specializes to the condensed world but only discusses the proper maps, not the arbitrary shable maps.

\end{unfinished}