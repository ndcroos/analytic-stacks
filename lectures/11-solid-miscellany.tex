% !TeX root = ../AnalyticStacks.tex

\section{\ufs Solid miscellany (Clausen)}

\url{https://www.youtube.com/watch?v=87-wuqGA8GE&list=PLx5f8IelFRgGmu6gmL-Kf_Rl_6Mm7juZO}
\renewcommand{\yt}[2]{\href{https://www.youtube.com/watch?v=87-wuqGA8GE&list=PLx5f8IelFRgGmu6gmL-Kf_Rl_6Mm7juZO&t=#1}{#2}}
\vspace{1em}

\begin{unfinished}{0:00}

Okay, so I'd like to start. Welcome back, everyone. We're about ready to move on to a new topic, but I just want to finish off our discussion of the solid theory with some miscellaneous facts.

Recall that we had this procedure: if we had a pair, say $A$ and $A^+$, where $A$ is some solid ring and $A^+$ is a subring of power-bounded elements, then we associated to this a certain solid analytic ring specified by $A/A^+$. This was obtained by forcing all the elements in $A^+$ to become solidified variables.

Recall also that we had this on the level of the derived category. If $A^+$ is not a subring of $\Z$, then when you enforce this condition, you might lose $A$, and it might no longer live in that category, so you'd have to apply localization and get a new $A$, a sort of completion in some sense.

We also had this defined as a full subcategory of the derived category of $A$ consisting of those objects whose homology lies in $A^+$ for all $i$. We showed that you have a nice left adjoint with the same properties as the left adjoint here: symmetric monoidal, and so on. But there was this subtlety that this is not necessarily, in the full generality of an analytic ring, equal to the derived category of this. The basic reason is that if you take some basic object here and apply derived solidification, it might not live in degree zero.

In fact, the generator here is the compact projective generator you take from the solid $\Z$-theory and then tensor it, in the sense of the solid $\Z$-theory, with this solid ring $A$. We know that it doesn't matter whether you take the derived tensor product or the ordinary tensor product, because as Peter showed in one of his lectures, this is flat in solid $\Z$.

So if you did the derived analog of this definition, it would literally just be the derived category of this $\mathcal{A}$-linear category. But then what's the generator of this derived category? You take this object here and you localize it with respect to this derived localization functor $\mathrm{der}A^+\text{-solid}$.

I want to explain that this lives in degree zero, which means it's a compact projective generator for the $\mathcal{A}$-linear category $\mathrm{Mod}_{A^+\text{-solid}}$ and the derived category we defined there. As long as your underlying solid ring is not derived in any way, the whole theory is, so to speak, flat and determined on the $\mathcal{A}$-linear level.

Does this change from what was said before, or was it not clear in the previous talk? I mean, it was not clear that this was the case.

Okay, so how do you access this derived $A^+$-solidification? Being $A^+$-solid means that you're solid for any variable mapping in. What I want to claim is that this derived $A^+$-solidification can actually be written as a filtered colimit. It's all happening elementwise in $A^+$, and you can write $A^+$ as a union of rings which are finitely generated over the integers, let's say $R \subset A^+$ with $R$ of finite type over $\Z$. Then you can treat $M$, which is a priori an $A$-module, as an $R$-module and do the derived $R$-solidification.

The point is, well, there are two things to check. First, you need to check that this is indeed derived $A^+$-solid. And you also need to check that in this functor, if you have something which is...

That if you map out to anything derived $\mathbb{A}^+_{\mathrm{solid}}$, then it doesn't see the difference between this expression and this expression. Why is it derived $\mathbb{A}^+_{\mathrm{solid}}$? Because derived $\mathbb{A}^+_{\mathrm{solid}}$ is the same thing as derived $\R_{\mathrm{solid}}$ for all $\R \subset \mathbb{A}^+$ of finite type, just because it's simply an elementwise condition on the algebra.

So certainly the $\R$ term in this thing is derived $\R_{\mathrm{solid}}$, but also any further term in the filtered colimit after the $\R$ term is also going to be derived $\R_{\mathrm{solid}}$ because it's even has an even stronger property of being derived $\R_{p',\mathrm{solid}}$ where $\R_p'$ is bigger than $\R$. So there's a cofinal system of things in this filtered colimit which are derived $\R_{\mathrm{solid}}$ for any fixed $\R$, and we know that a filtered colimit of $\R_{\mathrm{solid}}$ things is $\R_{\mathrm{solid}}$, so it's going to be $\R_{\mathrm{solid}}$ for all $\R$ and therefore it's going to be $\mathbb{A}^+_{\mathrm{solid}}$.

And then mapping, if you map out to something $\mathbb{A}^+_{\mathrm{solid}}$, then in particular it's $\R_{\mathrm{solid}}$ for every $\R$, and by a similar argument you see that there's no difference on maps from $M$ to this or this filtered colimit to this. 

Okay, so it suffices to analyze, to show that if you take this product $\Z \otimes_\Z \mathbb{A}$ and then you derived $\R$-solidify, this lives in degree zero. But this, we can do it in two steps. So we can take, we can see this as you first base change to the solid $\R$ theory, so you take this thing and then you derived $\R$-solidify, and then you tensor in the derived tensor in the solid $\R$ theory with $\mathbb{A}$.

So it suffices to see that, well, first of all, that product $\Z \otimes_\Z \R_{\mathrm{solid}}$ derived $\R$-solidify, well, this lives in degree zero and is flat with respect to our solid tensor product.

Okay, by the way, let me note before I continue with the, so basically we have to analyze $\R_{\mathrm{solid}}$ when $\R$ is finite type over the integers and prove analogues of what Peter already explained for $\Z_{\mathrm{solid}}$. So we have to calculate this

Solidification and so on, it always just throws the variables inside the product instead. Here, we recall that if you do the $\Z[T]$-solidification of a module tensor over $\Z$ with $\Z[T]$, this commutes with limits in $M$, and it sends $\Z$ to $\Z[T]$. We're also using that the different solidifications commute, so that you can actually do the solidification with respect to all of them by just doing them one by one. When you do the $X_1$-solidification and then $X_2$-solidify, it'll still remain $X_1$-solid. You also use that if it's $X$-solid and $Y$-solid, then it is solid for everything generated in the subring.

So, to solidify for $R$, it's enough to solidify for all the variables in $R$. In the general case, if we have a quotient $X_1 \dots X_n \twoheadrightarrow R$, then the $\Z$-solid tensor $R$ can be derived. You can get it in two steps: first, solidify with respect to a generating set for $R$, and then do an algebraic base change from this ring to the ring $R$. The derived tensor product is then the product of $\Z[T_1] \dots \Z[T_n]$, and we're left with analyzing this and proving that it's equal to a product of copies of $R$.

The product of $\Z[T_1] \dots \Z[T_n]$ is $R$-flat, and this can be shown using the fact that $R$ can be resolved by potentially infinite resolutions of finite free modules. For each of those finite free modules, it's clear that you just bring it into the product. This argument works in the derived category, and we don't need to worry about resolving the original module.

So, we've proved that this derived solidification lives in degree zero, and it's flat with respect to the tensor product. Now, we want to analyze the structure of solid $R$-modules in more detail.

From this, being a compact projective generator is again that solid $R$ is just the end category of its compact objects, which are the finitely presented ones. These are the things that are cokernels of maps between the compact projective generators.

But this second claim is---you could say---coherence. This subcategory is an abelian category, is closed under kernels, cokernels, and extensions. This is the same as---the first part is obvious, or sorry, well the first part follows from the fact that we have a compact projective generator of this category. Then the compact objects will all be built from finite colimits from that, and they will be generators, and it follows formally that it's the end category of that.

So, what we really need to check is that we have this---this is an abelian subcategory. Now, let me remind you a bit about some classical commutative algebra. So, a commutative classical commutative ring is called coherent if you have the analogous property in the setting of discrete modules, so that the finitely presented $R$ modules form an abelian subcategory. But you can check by playing around with short exact sequences that you only need to---there's only really one thing to check, which is that every finitely generated ideal is actually finitely presented. And those same arguments in this setting show that it suffices to check that every finitely generated subobject of a product of copies of $R$ is finitely presented.

So, for coherent rings, now I'm going to use a bit this notion quasi-separated. Note that if you have any subobject of a product of copies of $R$, well, this is quasi-separated, i.e., Hausdorff, and any subobject of anything quasi-separated is also quasi-separated. So, this is also quasi-separated. So, it suffices to show that every finitely generated quasi-separated solid $R$ module is actually finitely presented. So, let me make a lemma.

If $M$ in solid $R$ is quasi-separated, then the following are equivalent: (1) $M$ is finitely presented, (2) $M$ is finitely generated, and (3) $M$ is an inverse limit of $M_n$'s, where each $M_n$ is a finitely generated discrete $R$ module, and the transition maps are projective. The category of these is just the countable pro-category of the category of finitely generated $R$ modules, so the maps between such inverse limits are just the maps of pro-objects.

The non-trivial implications are (2) implies (3) and (3) implies (1). For (2) implies (3), if $M$ is finitely generated by definition, we have a surjection from a product of copies of $R$, and then there's some kernel. But since $M$ is quasi-separated, it follows that this inclusion is a quasi-compact map of condensed sets, which means that this object is quasi-separated, so it's just some filtered union of compact Hausdorff spaces. And this means that when you restrict this inclusion to any of those compact Hausdorff spaces, you get a closed subset, but this is one of those sequential spaces, metrizable spaces in fact, and so being a closed subset on any compact subset is the same thing as just being a closed subset in the topological sense with the product topology. This closed subset is also a module, the same associated condensed module that we are talking about.

So, then we can look at the projection onto the first $n$ coordinates, and let $K_n$ be the image. It follows that $K$ is the inverse limit of

Because if we know that this guy is finitely generated, then it fits into something like this, and we just prove that the kernel is of the same form, and then it will follow that it's finitely presented.

So, but then this is actually elementary. You have like $M_1$ surjecting onto $M_2$ surjecting onto $M_3$, you can make some finite free module rejecting onto $M_1$, and then you have some kernel of this map here, and then you can make some $R$-direct some $D_2$ rejecting onto that, and then you can make a system $R$-direct some $D_1$, $R$-direct some $D_1$ plus $D_2$, and in the inverse limit you get a product of copies of $R$ rejecting onto $M$.

So, in fact, if you're wondering about surjection on the level of condensed objects, in fact you can even get a topological splitting for the map of topological spaces, so by kind of section here, make a compatible section there and so on, if you're not worried about linearity, which you're not, then you can get this quite easily.

Okay, so that's the proof of the Lemma. So, the quasy separated objects are just these kind of classical $R$-modules, and that also proves the coherence: extensions are easy, extensions are formal, yeah, it doesn't require anything, because it's projective, projective generator, so you can lift exactly.

Actually, because you have this compact projective generator, you're equivalent to the category of modules over a ring, and so it's really, you can actually just quote the usual theorem from module theory, not commutative algebra, sorry, thank you.

Wait, we're still proving the theorem. Ah, well, it was used in order to say that this was the compact projective generator, so when we were---so a priori, you take product $\Z$ tensor up to $R$ and then solidify with respect to $R$, and only in the finite type over $\Z$ case when you had this surjection from the polynomial ring, yeah, but for any noetherian ring, you can take the product of copies of, yeah, you if you just say that you build a category that has this as compact projective generator with the endomorphisms, what they have to be, then the rest of the arguments here work for an arbitrary noetherian ring.

Okay, and we're not quite done with the first theorem because we still need to show that the product of copies of $R$ is flat. So now we make a claim that if $M$ is finitely presented in solid $R$, then if you take $M$ derived tensor product $R$ solid with product of copies of $R$, you just get a product of copies of $M$, and in particular, this lives in degree zero.

No, and that implies that this theorem or this claim implies that this guy is flat, because the derived tensor product with any finitely presented object lives in degree zero, but every object is a filtered colimit of finitely presented objects, so the derived tensor product with any object will live in degree zero, which is the same thing as saying you have flatness.

So, incidentally, to define the right-derived tensor product, maybe we didn't spend time on this, but in the usual approach, you need to know there are enough objects to relate to $\mathcal{T}$ to define classically $\mathcal{T}$. So here, what is the approach, what's the meaning assigned $\mathcal{D}_\mathcal{T}$
Definition: It's not actually necessary to know anything about flat objects. For condensed abelian groups, we do know there are enough flat objects, so that's not needed to discuss them. But I was just speaking off the top of my head about how I think about it. Yes, free objects on a condensed set will themselves be flat.

Okay, so where was I? Ah, we need to prove this claim. The proof is: we can make a surjection where the kernel is finitely generated but also quasi-separated, so it will be finitely presented. We can then continue to make an infinite resolution, which reduces us to the case where M itself is a product of copies of R. Then we can remember that this was a product of copies of $\Z$ tensored with R, and that solidifies to a product of copies of $\Z$. 

This nice object we were calling $\mathbf{P}$, the $\mathbb{N}$-indexed union, has the property that its product with itself is isomorphic to itself in the expected way.

Okay, any other questions? This works because the product of copies of a projective object is projective. But you're right that this requires a classical treatment - I couldn't have used that fact without defining the derived category of the solid category first.

You're absolutely right that I should have added the claim that the derived solidification is the left derived functor of solidification, and that the derived tensor product is the left derived functor of the tensor product. This all follows from the flatness and living-in-degree-zero properties we've established.

To define the derived tensor product without solidification, you can just work in the derived category of modules over a commutative ring, and use the standard construction there, without needing flat objects. The key is to have a way to do derived tensor products in the underlying category, which you can then solidify.

Okay, let me spend a bit more time on the homological algebra of solid R. Peter also explained something else in the case of solid $\Z$, that every finitely presented module has an infinite resolution by products of copies of $\Z$.

But it actually has a two-term resolution by products of copies of $\Z$, so that at least from the perspective of finitely presented objects, you have sort of homological dimension one for solid $\Z$, just like you have for usual $\Z$. I want to present a generalization of that.

Theorem: If $R$ is of finite type over $\Z$ and is a regular ring of dimension $d$, then if you have $M$ in solid $R$ finitely presented and $N$ in solid $R$ arbitrary, the $\text{Ext}^i(M, N)$ vanish in degrees bigger than the dimension.

A corollary of this will be that if you have a finitely presented module, then you actually get a projective resolution of length at most $d$ or $d+1$, depending on how you define length. And then we already saw that the projective objects are flat, so this also implies a bound on the Tor groups.

However, this statement does not extend to non-finitely presented $M$. Unlike in the case of discrete modules over a regular noetherian ring of finite dimension, where you actually have the $\text{Ext}$ vanishing for all modules, the argument for going from the finitely presented case to the general case does not work in this context. In fact, I think you can have arbitrarily high non-vanishing $\text{Ext}^i$ even for solid $\Z$.

Let me give a fun exercise you can try to do. Take two distinct prime numbers $p$ and $q$, and consider the module $\Z/p\Z \oplus \Z/q\Z$. Then there is a non-zero $\text{Ext}^2$, obtained by switching the roles of $p$ and $q$. This suggests that there can be arbitrarily high non-vanishing $\text{Ext}^i$ groups, even in ZFC.

The issue is that when going from the finitely presented case to the general case, you need to analyze derived inverse limits along arbitrary filtered systems, and the cardinality issues come into play. If you are in a model with the Continuum Hypothesis, then perhaps you can get a bound. But in general, it seems that arbitrarily high non-vanishing $\text{Ext}^i$ groups are possible.

Let's now try to prove the theorem. The first case is when $N$ is a classically finitely generated module over the noetherian ring $R$, and $M$ is a quasi-separated, finitely presented solid $R$-module. In this case, the claim is that the $\text{Ext}^i(M, N)$ groups vanish for $i$ greater than the dimension of $R$.

Let's just say discrete is the same thing as the filtered colimit of $\mathit{xt}_i$ over $R$, so this is $R$-solid $\mathit{xt}_i$ over $R$ from $m$ to $n$. Didn't you prove that presents imply quasi-separation?

No, certainly not. In solid $\Z$, something like $\Z_p/\Z$ will be finitely presented but not quasi-separated. So what was proved is that if you're quasi-separated and finitely generated, then you're finitely presented. But if you're finitely presented, you are not necessarily quasi-separated. For example, $\Z_p/\Z$ is finitely presented.

So if a map from a product of copies of $R$ is not coaccessible, that does not necessarily imply it's not quasi-separated. That's right, I misunderstood this. That's important.

So we need to show that we can pull the inverse limit out of the $\mathbf{R}\mathbf{Hom}$. By the way, the claim for internal $\mathbf{R}\mathbf{Hom}$ follows from the claim with underlying sets, just replacing $n$ by continuous functions with values in $n$ for some profinite set $S$.

We have the Mittag-Leffler resolution, some kind of usual identity minus shift sort of thing, or maybe it's a shift times $F$ or some kind of shift map. This distinguished triangle tells you that the question of pulling $\mathbf{R}\mathbf{Hom}$ the inverse limit out is the same as the question of pulling out a product and turning it into a direct sum. We can resolve $\prod_n M$ by $\prod_n R$ to reduce to $\mathbf{R}\mathbf{Hom}(\prod_n R, n) \cong \bigoplus_n \mathbf{R}\mathbf{Hom}(R, n)$, which follows from $R$ being projective.

So from this, it follows that we get the desired vanishing of $\mathit{xt}_i$ if $n$ is a classical discrete $R$-module and $M$ is finitely presented and quasi-separated. Now if $M$ is arbitrary finitely presented but not necessarily quasi-separated, you can always resolve $M$ by a product of copies of $R$, and then the kernel will be both finitely presented and quasi-separated. Analyzing the long exact sequence, you would get the desired result but with a loss of one degree of homological dimension: $\mathit{xt}_i$ would vanish for $i > D+1$, where $D$ is the Krull dimension of $R$.

So this naive argument only gives $\mathit{xt}_i(M, n) = 0$ for $i > D+1$. We have to do a little bit of extra work to improve this to the conclusion that $\mathit{xt}_i(M, n) = 0$ for $i > D$.

We didn't use regularity here, but I should maybe recall the classical fact that if $R$ is regular of Krull dimension $D$, then the classical $\mathit{xt}$ groups between discrete modules vanish in degrees greater than $D$. So the question for quasi-separated things in the solid context reduces to the classical question in discrete commutative algebra, and that's where we use regularity.

We're going to do a less naive argument. Instead of just a product of copies of $R$, we can surject again from a product of copies of $R$, and now at this point we take the kernel. This is still quasi-separated, and it suffices to show $\mathit{xt}_i(M, n) = 0$ for $i > D-2$. We bought ourselves one extra drop in the vanishing bound because we continued the resolution one step more.

This will work assuming $D \geq 2$; the case $D = 1$ can be handled by similar arguments in low dimension. Now we want to find a nice expression of $n$ as an inverse limit of finitely generated $R$-

It's a closed submodule here. We can always just look at the first method. We analyze this map here. This is a map from a product of copies of $R$ to a product of copies of $R$ again. That's the same thing as a map on the pro-system. If you restrict to any initial chunk of this, then there's some corresponding initial chunk here where the map projecting onto that factors through projecting onto this chunk, and then just a map of discrete modules there. 

If you reindex and allow finite free modules here instead, then you can assume that this map here comes from an inverse limit of compatible maps from the nth initial chunk here to the nth initial chunk here, just by reindexing. And then if you take the kernels of those maps, we can let $n$ be the inverse limit over $n$ of $n_n$. But the other thing we can do is we can let $n'_n$, which will be contained in $n_n$, be the image of $n$ mapping to the inverse limit of $f_n$ mapping to $f_n$.

Number one has some good and some bad properties. The good news is that you have this $x_i - 2$ vanishing, $x_i n_n x = 0$ for all $i$ greater than $d - 2$. But the bad news is no guarantee that this system $n_n$ is Mittag-Leffler, which is needed for $x_i n x = \text{filtered colimit}$ over $n$ of $x_i n_n x$. 

In the argument I presented, the case where the transition maps were surjective, the argument really only used that it was a Mittag-Leffler system to get this resolution. And in the second case, the situation is opposite: Mittag-Leffler, but no guarantee that $x_{d-1} n'_n x = 0$. 

We know that $n$ is the limit of both in the condensed category, limiting the condensates of both of these systems. But you can produce an example of a map from $\lim n$ to $\lim n'$ for which the kernels would not be Mittag-Leffler.

What we have to do is find something in between that has the good properties of both. The reason we have this nice property here is that it's a kernel of a map between projective objects. But for this $n'_n$, you don't know that it's a kernel of a map between projective things, you only know that it's sitting inside a projective thing. So you get vanishing of $x_d$ one better than for a general module, but you wouldn't get vanishing a priori in degree $d-1$. 

We're going to find $n'_n$ sitting in between them, such that we get $x$ vanishing and $n'_n$ is Mittag-Leffler. If we get this, then we're done, because the inverse limit, we would get the desired $x$ bounds for the inverse limit of the $n'_n$, but that's sandwiched in between these two things which have the same inverse limit, which is the thing we're interested in.

The obstruction is a question of depth. The Auslander-Buchsbaum formula tells you that for a regular Noetherian local ring, the depth is the codimension.

Projective dimension of a module plus the depth of the module is equal to the dimension of the Ring. Because we have the one better estimate on the projective dimension, the only rings that are going to give us obstruction are the local rings at prime ideals, which are actually maximal ideals. So the local ring has maximal dimension, and the only obstruction is going to be moving from a situation of depth one at that maximal point to depth two.

At a maximal ideal, the module only has depth one. Yes, yes, yes, many closed points, and you know that it lies inside the other one because the other one is reflexive. And then you have to establish, but only at the closed points, the classical argument works. That's correct, yes.

Okay, maybe I'll just repeat what was said. So what you can do, if you let's say for simplicity you only have a problem at one maximal ideal (in general you'd fix a problem at finitely many), and then if you increase the number of them, eventually the situation would stabilize by an argument. So let's say there's a problem only at one maximal ideal. Then you can look at the inclusion of Spec R minus that closed point into Spec R, and then you replace the module by the extension-restriction of the module. That receives a map from the module, but it's in fact an inclusion due to the depth one assumption on the module.

But then, on the other hand, by this procedure of this extension and restriction, depth is characterized in terms of local cohomology at the maximal ideal, and that's exactly what comes up when analyzing the difference between this and say the derived version of this. And using that, you can see that this improves the depth to depth two at the closed point. On the other hand, if you were to perform the same construction for the original module, since we already know this has depth two, you wouldn't have been changing it. So you really are producing something sitting in between which fixes the problem at a given closed point. Then there's the noetherian property which tells you well, it's only finitely many closed points that are going to be involved, so you can, by the same procedure, fix all the problems for the module in a compatible manner in the tower. Okay, so that's how you've got it to be depth two at all the closed points, which gives you the desired vanishing for these modules.

And then what about the Mittag-Leffler condition? This tower of modules sits in a short exact sequence with the module and the quotient module. To show that this is Mittag-Leffler, you need to know that the quotient is Mittag-Leffler, and that is Mittag-Leffler because the transition maps are surjective. But this one here is supported at finitely many closed points and is a finitely generated module over the ring, and that implies that it's finite as an abelian group. Because remember, our ring was finite type over $\Z$, the maximal ideals all have residue fields which are finite fields, and a compactness argument shows that any inverse system of finite abelian groups satisfies the Mittag-Leffler property.

Okay, so that gives the vanishing. To sum up, we've seen that if the module is finitely presented and solid over the ring, and if the object $X$ is a discrete module over the ring, then we get the desired vanishing.

Now, suppose $X$ is quasi-separated and finitely presented in solid $R$. We can write $X$ as an inverse limit of $X_n$ with surjective transition maps. Then the $R$-homs into that will just be the inverse limit of the derived inverse limit of the $R$-homs into each of those terms. In principle, you might get one worse again because there could be a lim$^1$ in the last inverse system, but because the transition maps are surjective and $D$ is the largest degree at which you have nonzero vanishing, you actually see that on the $X$-dual, you get surjective maps, so there's no lim$^1$ potentially giving you something in degree $D+1$. So you get the claim for $D$.

And then for $X$ arbitrary finitely presented, you resolve it by, and the $X$ with values in $X$ can only be better than the $X$ with values in these two, so you get that situation as well. And then

Arbitrary, so if you write $X$ as a filtered colimit of $X_i$, where the $X_i$ are finitely presented, then $\underline{Hom}(M, X)$ from any finitely presented $M$ to $X$ is the filtered colimit of $\underline{Hom}(M, X_i)$, by the pseudocoherence of $M$. So you can resolve $M$ by a product of copies of $R$, and then this follows from products of copies of $R$ being compact projective.

What about $\underline{X}$? It's the same---from this point on, this claim for an arbitrary discrete module implies the same claim for $\underline{X}$, because it's the same as the $S$-valued points of this thing, which is the same thing as $\underline{Hom}(M, S)$ for continuous functions from $S$ to this discrete thing, which is just another example of a discrete thing. So then you get this there, and then from that point on, all of the arguments actually work at the internal level.

Okay, including the Lim 1 argument, yes, because you reduce it to showing that a Lim 1 vanishes in condensed abelian groups, and the terms in the system are discrete and the system is Moeglin, and those properties are preserved by taking continuous functions with values in that---it's still a system of discrete things and the transition maps are still Moeglin or subjective.

In the condensed abelian group, again, you have only up to Lim 1, because you can compute it termwise. Yes, because countable products are exact. Okay, and you can compute the Lim 1 term by---no, maybe this is not, you can compute Lim 1 on each object, not necessarily projective.

You don't have projective computations like in any site where it's etale, you cannot compute, but here you can compute the Lim 1 on any test condensed set. Uh, no, no, for some things, yeah, so certainly for this $\underline{P}$ object you can, because that's projective. And then that's enough for solid, because we know that the solidification of $\underline{P}$ generates. So that's one argument you could give. I'm sure there are other arguments as well.

But you know, the Lim 1 is always just going to be the sheaf of the, you know, the condensed Lim 1 will always be the sheaf of the naive sectionwise Lim 1, and so if you prove the vanishing of the Lim 1 sectionwise, that's enough to prove. Yes, I missed where coherence comes from. Ah, right, so that comes from the claim that the finitely presented objects form an abelian category, which has as a corollary that for any finitely presented objects, you can build an infinite resolution where all of the terms are products of copies of $R$.

You're welcome. And to have non-quasi-separated finitely presented objects, the ring has to be of dimension at least two, or non-quasi-separated finitely presented has to have dimension at least one at least one. Yeah, so if you have a, if you're over a finite field, then every finitely presented object is quasi-separated, but once you move to say the integers, then $\Z_p/\Z$ is an example of something that's not quasi-separated but is finitely presented.

Also, the step of reducing from quasi-separated finitely presented to discrete is easy---it's not Lim 1 here. Like, which step? Sorry, this step here. Yeah, you see that there's no Lim 1 because the only Lim 1 that can contribute to degree $D+1$ is the Lim 1 of the $X_D$'s, and because on the level of these guys, we know that there's nothing, there's no $X$ for any discrete module in degree $D+1$. You see that if you have a surjective map, with then $X_D$'s into it will also be surjective, because the obstruction is an $X_{D+1}$ of the kernel, which vanishes, so then the

Ways to localize the same category. This is by no means the most general possible localization. It's the one we discussed because it's the one that's most closely related to Huber's Theory. And our goal in this class is to explain the basic definitions in our Theory and their relations to more classical theories of analytic geometry.

But I just want to point out very briefly that there's also a whole different avenue you can go for localizing this, which even more radically, I guess, departs from Huber's formalism. So we already saw a small departure in that we were allowed to do slightly more localizations of a Huber ring than before, because of this difference between valuations that are less than one on topologically nilpotent elements and ones which are valuations which are continuous in Huber sense. But kind of even more drastic things are possible.

So, in fact, if $C \subset \mathrm{Spec}\, R$ is any constructible subset---so that means an intersection of a quasi-compact open with the complement of a quasi-compact open, so it's some locally closed subset of $\mathrm{Spec}\, R$ which is sort of finitely presented---and of those, a finite union. Yes, yes, yes, yes. So the basic objects which sort of form a basis for the constructible topology would be something like you take $D(g)$ intersect the complement of the common zero set of finitely many functions. 

To this, you can assign an idempotent algebra in $\mathrm{D}(R\langle\langle z\rangle\rangle)_\mathrm{solid}$, namely, to this basic object, $D(R[1/g]) \cap V(F_1,\dots,F_n)$, you assign the thing you get when you take $R$ and then you invert $g$ and then you derived complete along $F_1,\dots,F_n$, which this final object only depends on the constructible subset. And you know, by the way, if $R$ is an Artin ring, then this derived completion is just the usual completion. But I want to emphasize that I'm taking this derived completion in the derived category of solid Abelian groups, so I'm, so to speak, putting the inverse limit topology on this derived inverse limit here.

No, such things then you say that you ah you view your constructive subset as a union of locally closed, you also look at the intersection, so on. So for each of them, each finite intersection, you do this, and then you take the what? Okay, you take the limit of this type thing, and it won't be concentrated in homology, they'll have some cohomological degree, yes. So is it the case that the intersection of two basic things, what you get, is a tensor product? Yes, okay, so it is. Intersection, oh yes, yes, that's right, sorry.

Right, so maybe I shouldn't actually say that you can assign idempotent objects to any constructible subset, maybe I should just say that you can assign a idempotent object to any basic constructible subset. So you want to take the kind of the derived inverse limit of that idempotent object associated to all basic things inside the given thing, but then you have a problem to prove properties of this. I don't know if it follows that it is an algebra, or you're looking at those subsets, what? Yeah, constructible, locally closed, is if it's constructible, locally closed, so take $\mathrm{R}\Gamma_\mathrm{cts}$ of this, so it will, could have some cohomological degree depending on the number of $\alpha$ and $\gamma$ to cover, yes, yes, yes, yes. So that's---but thank you, Peter, that's better. 

Yeah, so then the only derived behavior comes from the union of principal opens, thanks. Yeah, that's what I should be saying. Okay, so now why are these idempotent? So recall that the solid tensor product, they, if me, idempotent means that, let's say that this is idempotent, means that a tensor over $R$ solid derived $a$ is the same thing as derived complete $\mathrm{R}\langle\langle z\rangle\rangle_\mathrm{solid}$ $a$. Is derived complete, so to check

So, you have a ring, and you have finitely many elements, or any number of elements, I suppose. And the derived solid tensor product of connective objects, so in homological degrees, will still be derived complete, why? Because we proved it, or yeah, we---well, Peter gave an argument for the special case of like $p$, and then the general case is quite similar, so he gave the heart of the argument of the general case. 

You have to write, yeah, you have to, you have to do some work, you have to do some analysis, but it's not um, yeah, it's something we've discussed in previous lectures.

Okay, so now, and then, so now the claim that I would like to make is that if $C_1$ up to $C_N$ are locally closed, constructible, and if their union, set-theoretically, is all of $\Spec R$, then these guys cover, let's say, these the corresponding $A_i$ these idempotent algebras, cover $\mathrm{D}_\et(R^\mathrm{zt})$ solid, in what sense? So in the sense that if you have $M$ in the derived category of $R^\mathrm{zt}$ solid, such that $M \otimes A_i = 0$ for all $i$, then $M = 0$.

But, yes, thank you, thank you very much, yeah. So, and this implies that you get $\mathrm{D}_\et(R^\mathrm{zt})$ solid is some limit over---so in the first term, you have a product of product over $i$ of $A_i$ modules in $\mathrm{D}_\et(R^\mathrm{zt})$ solid, and then you have a product over $i, j$ less than $j$, modules over the tensor product $A_i \otimes A_j$ in the same thing, and then so on, you'll have some finite check, uh, thing. What, sorry, cover if they, what?

I like that, they're giving it a big hug, that cover. Uh, yeah, so if they love $\mathrm{D}_\et(R^\mathrm{zt})$ solid, okay. Anyway, and the basic idea, idea of the proof is that it suffices to show that $R$ is generated by $A_i$ modules for varying $i$.

So, for if I just take the example of Spec $R$ is $\mathrm{D}_f$ union $Z(f)$, then you use, then $r_{\frac{1}{f}}$ is okay, because it comes from here, and then the difference between $R$ and $r_{\frac{1}{f}}$ is, uh, so generated in a finite manner, finitary manner. But this is the union of like $R/f^N$ with some shifts, uh, this is actually the fiber over here, and this is actually an $R_f$-complete module, so it lives in $Z(f)$. So, if you take all modules over $R_f^{-1}$ and all modules over $R_f$-complete, and then generate things just by triangulated category nonsense, eventually you're going to hit $R$, and then tensoring with anything, you'll see that you hit anything, and then using that, you can

Inverting and it famously matters in which order you do this. But it all comes from this sort of commutative situation nonetheless in this way of setting it up here. I want to make the point that these things are like kind of like higher local fields that people have studied, and they tried to study them from their perspective of topological rings and topological fields. And there were just terrible problems. Matthew's not here right now, but he even wrote a paper explaining that everything is horrible. But if you put them in the condensed world, then they're just perfectly well-behaved objects that you can work with. It arises from this natural procedure and it's idempotent even in the $\infty$-category sense over this ring.

So, I invite you to try to draw pictures of these covers to get an understanding for what's going on with these kinds of constructions. But okay, I think I'll stop here.

Yes, for finite-type $\Z$-algebras $R$, the solidification of the countable product of $\C$ and $\R$ is just a countable product of $\C$ and $\R$. This is also true for something like the ring of powers of $\Z$, which is not finite-type, but it's also not discrete. So, yeah, it's determined by the topological ring structure.

You could also take a finitely generated ring and an arbitrary ideal, and look at the completion, and then you get the same statement. The basic compact projective generator is just the product of copies of the $I$-adic completion of $R$. And that's the same time if the ring is not finite-type. You also said it's the union of finite-type algebras. Is it easy to see for this? No, but $\Z[[T]]$ — sorry, that was a statement about discrete rings. So, in general, if you have some ring that's complete with respect to a finitely generated ideal, you'll get the good answer if your ring modulo that ideal is finitely generated. Maybe this is a good way to understand it.

If the ring modulo the ideal is finite-type over $\Z$, then you'll get the naive answer, I think. Yeah, sounds reasonable at least. Solidification is determined by the underlying topological ring structure.

Yes, so the same formula doesn't work for $\Z[[T]]$. $\Z[[T]]$ is a discrete ring, yeah. As a ring, you're looking at $\Z[[T]]$ as a discrete ring, yeah. Well, then it doesn't hold. No, but if you look at $\Z[[T]]$ non-discrete modules, the solid theory is the same as the solid theory in $\Z[T]$.

Once you've decided to be complete, then it's enough to solidify stuff modulo that ideal you're complete along. Okay, yeah, so does that address your concern?

Okay, so what again? You take a discrete ring and take the associated condensed ring, and you consider solid relative to the elements of the ring. Okay, just each element of the ring. And then you solidify, so if you consider this condensed ring with the product topology, you could either solidify every element in the underlying discrete ring or you could just solidify $T$, and it's the same. In fact, you don't even need to solidify $T$. I mean, you
Proposition for finite type. So if I just, for example, like $\Z_p$, the $p$-adic integers, yeah, was there a computation? Oh, you're wondering about these homological dimension results for $\Z_p$, yeah? So there you get, uh, yeah, now I have to remember. I think, um, yeah, there you should get also, it should just only be $X_1$, uh, yeah. That's actually even easier to show than in the case of solid $\Z$, because $\Z_p$ is compact, so you can directly see that all the finitely presented objects there are also quasi-separated, and it's the they're also all those objects are quasi-separated, and they're just like, yeah, inverse limits of finite $\Z_p$ modules, uh, countable inverse limits of finite $\Z_p$ modules, and that makes the whole analysis much easier.

To summarize, so for let us say $R$ is a field, yes, then the finitely presented are quasi-separated. In fact, they are all given quite simply by you just have the either infinite or final product of the $\mathcal{A}_\bullet$. So that's okay for our domain, except the fact that you can have a, the torsion group is maybe not trivial, but still it's a product of, uh, you have a resolution, you have product copies of $R$, and then the Koszul is something like product of maybe, ah, actually you can use the the Artin-Rees Lemma to get rid of this anyway. So you can, it's always this, this okay. So it is a resolution of length 2, like like before, because we don't have to worry. And so we get, and then in dimension at least two, you do your, the pro, the, uh, that we discussed, yes. Okay, that's the, yeah. And then there's if you wanted to move to non-discrete rings, then you'd have to do more analysis and so on.

Yeah, yep, okay, thanks everyone.

\end{unfinished}