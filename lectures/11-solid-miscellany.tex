% !TeX root = ../AnalyticStacks.tex

\section{\ufs Solid miscellany (Scholze)}

\url{https://www.youtube.com/watch?v=87-wuqGA8GE&list=PLx5f8IelFRgGmu6gmL-Kf_Rl_6Mm7juZO}
\renewcommand{\yt}[2]{\href{https://www.youtube.com/watch?v=87-wuqGA8GE&list=PLx5f8IelFRgGmu6gmL-Kf_Rl_6Mm7juZO&t=#1}{#2}}
\vspace{1em}

\begin{unfinished}{0:00}
e
okay  so  I'd  like  to  start  um  welcome
back  everyone  uh  so  I  think  um  we're
about  ready  to  move  on  to  a  new  topic
but  I  just  want  to  um  finish  off  our
discussion  of  the  solid  Theory  with  some
miscellaneous
miscellaneous  uh  facts
so
um  so
recall  that  we  had  this  uh  procedure
that  if  uh  we  had  a  pair  say  a  A+  for
this  is  some  solid
ring  and  this  is  A+  uh  a  sub
ring  of  power  bounded
elements  uh  oh  sorry  just  one
um  then  we  Associated  to  this  well  a
certain  solid  analytic  ring  uh  which
was  specified  by  this  mod  a
A+
um  and  it  was  um  obtained  by  uh  forcing
all  of  the  elements  in  A+  to  be  Sol  to
become  solidified
variables  um  but  recall  that  we  also  had
the  thing  on  the  level  of  the  derived
category  and  this  is  so  what  would  if
you  if  if
A+  if  A+  is  not  a  subing  of  a  z  ah  then
then  when  you  enforce  this  condition  you
might  lose  the  uh  a  a  might  not  might  no
longer  live  in  that
category  so  you'd  have  to  apply  the
localization  and  get  a  new  a  a  sort  of
sort  of  the  completion  in  some  sense
okay
yeah  um  but  we  also  had  this
uh
so  and  recall  that  this  was  defined  by
uh  this  was  defined  by  as  it  was  defined
as  a  full  subcategory  of  here  consisting
of  those  objects  whose  homology  lies  in
here  for  all  I  and  we  showed  that  uh  you
have  a  nice  left  joint  with  the  exact
same  properties  as  the  left  ad  joint
here  symmetric  monoidal  and  so  on  and  so
forth  but  there  was  this  subtlety  that
this  is  not  necessarily  um  well  in  the
full  generality  of  an  analytic  ring  this
is  not  necessarily  equal  to  the  derived
category  of  this  and  the  basic  reason  is
that  if  you  take  some  basic  object  here
and  apply  derived  solidification  it
might  not  live  in  degree  zero  uh  that's
kind  of  the  obstruction  um  so  in  fact  uh
the  generator  here
so  so  there's  a  compact  projective
generator
which  is  equal  to  you  take  the  compact
projective  generator  we  know  In  Love
from  the  solid  Z  Theory  and  then  you
simply  tensor  it  in  the  sense  of  the
solid  Z  Theory  with  this  solid  ring
a  um  and  that  we  know  that  it  doesn't
matter  whether  you  take  the  derived
tensor  product  or  the  ordinary  tensor
product  there  because  as  Peter  showed  in
one  of  his  lectures  this  is  flat  in
solid
z  um
so  that's  what  means  that  if  you  did
made  that  the  if  you  did  the  derived
analog  of  this  definition  it  would
literally  just  be  the  derived  category
of  this  ailion
category  um  but  then  what's  the
generator
here  compact  generator  generator  of  this
derived  category  uh  sorry  uh  of  this
thing
so  so  what  do  you  do  you  take  this
object  here  and  you  localize  it  with
respect  to  this  derived  localization
functor
uh  so  you  derived  a  plus
solidify  and  uh  maybe  I'll  say  uh  so
theorem  so  first  the  I  want  to  explain
is  that  this  this  lives  in  degree
zero
and  that  means  it's  a  a  compact
projective
generator  for  the  aelan
category  uh  mod  a  A  plus
solid  and  uh  and  the  derived  category  of
this  ailan
category  uh  is  the  is  the  drive
category  we  defined
there  so
as  long  as  your  underlying  solid  ring  is
not  derived  in  any  way  then  in  fact  the
whole  theory  is  you  know  or  that  doesn't
have  any  the  whole  theory  is  kind  of
flat  so  to  speak  uh  determined  on  the
aelan  level
um  does  it  so  is  it  a  change  from  what
was  said  before  or  or  because  I  remember
that  it  was  not
clear
in  the  previous  talk  I
mean  it  was  not  clear  that  this
was  well  that's  that's  why  I'm
explaining  it  now  yeah  yeah
um  okay  so  so  how  what  is  the  how  do  you
access  this  derived  as  solidification
so  to
so  being  being  uh  being  A+  solid  means
that  you're  solid  for  any  variable
mapping  in  so  um  what  I  want  to  claim  is
that  uh  this  uh  so
claim  is  that  the  this  derived  A+
solidification
um  uh  you  can  actually  write  this  as  a
filtered  Co  limit  so  it's  all  happening
element  wise  in  A+  and  you  can  write  A+
as  a  union  of  of  rings  which  are
finitely  generated  over  the  integers  so
let's  say  R  subset  A+  R  finite  type  over
z  um  and
then  you  just  forget  to  write  the  square
in  a  A  plus  d  oh  yes  thank  you  yes  I  did
yeah
uhhuh  yeah
because  iritating  asking  the  difference
between  when  there  is  a  square  yes  yes
uh  sorry  that's  yeah  um  so  you  so  you
can  take  M  which  is  a  priori  an  a  module
but  you  can  A+  maps  to  a  and  therefore  R
maps  to  a  you  can  treat  it  just  as  an  R
module  um  and  then  you  can  do  the
derived  uh  uh  r
solidification
um  so  this  is  a  filtered  Co  limit  so  the
point  is  well  there's  two  things  to
check  first  of  all  you  need  to  check
that  well  there's  a  map  you  need  to
check  that  this  is  derived  A+
solid  um  and  you  also  need  to  check  that
um  in  this  functor  if  you  have  something
which  is  uh
well  yeah  and  then  you  need  to  check
that  if  you  map  out  to  anything  derived
uh  A+  solid  then  it  doesn't  see  the
difference  between  this  expression  and
this
expression  um  so  why  is  it  derived  A+
solid  um  because  derived  a  plus  solid  uh
is  the  same  thing  as  uh
so  is  the  same  thing  as
derived  uh  our
solid  uh  for  all  our  subset  a  plus
finite  type  just  because  it's  simply  an
element  wise  condition  on  the
algebra  um  and  uh  so  so  certainly  the  r
term  in  this  thing  is  derived  R  solid
but  also  any  further  term  in  the  uh
filtered  Co  limit  after  the  r  term  is
also  going  to  be  derived  R  solid  because
it's  even  has  an  even  stronger  property
of  being  derived  like  RP  Prime  solid
where  RP  Prime  is  bigger  than  R  um  so
there's  a  co-  final  system  of  things  in
this  filtered  cimit  which  are  derived  R
solid  for  any  fixed  R  and  we  know  that  a
filtered  Co  limit  of  uh  of  R  solid
things  is  R  solid  so  it's  going  to  be  R
solid  for  all  R  and  therefore  it's  going
to  be  A+
solid
um  and
um  so  and  then  mapping  if  you  map  out  to
something  a  plus  solid  then  in
particular  it's  R  solid  for  every  R  and
by  a  similar  argument  you  see  that
there's  no  difference  on  maps  from  M  to
this  or  this  filtered  cimit  to  this
okay
so  so  it  suffices  to
analyze  to  show  uh  that  if  you  take  this
product
z  uh  tensor
Z  solid  a  uh  and  then  you  derived  R
solid  uh  R
solidify  uh  this  lives  in  degree
zero  um  but
uh
this  uh  we  can  do  it  in  two  steps  so  we
can  take  we  can  see  this  as  you  take  you
first  base  change  to  the  solid  R  Theory
uh  so  you  take  this  thing  and  then  you
derived  R
solidify  uh
uh  and  then  you  tensor  in  the  derive
tensor  in  the  solid  R  Theory  um  with
a  so  it  suffices  to
see  uh  that
um  uh  it  suffices  to  see  that  well  first
of  all  that  product  z  uh  tensor  zolid
are  derived
solidify  uh  well  this  lives  in  degree
zero  and  is
flat  uh  with  respect  to  our  solid  tensor
product
okay
so  by  the  way  let  me  not  to  before  I
continue  with  the  so  basically  we  have
to  analyze  R  solid  when  R  is  finite  type
over  the  integers  and  prove  analoges  of
what  Peter  already  explained  for  zolid
so  uh  we  have  to  calculate  this  basic
generating  object  make  sure  it  lives  in
degree  zero  and  we  also  have  to  check
that  it's  flat  um  but  before  I  go  on  and
do  that  let  me  note  corollary  which  is
kind  of  a  corollary  of  this  filtered
cimit  type  of  argument  so  uh  if  um  so  we
have  a  compact  projective  generator  of
this  aelon  category  and  I  basically  just
analyzed  for  you  that  it's  gotten  by
some  filtered  cimit  of  uh  the  thing  you
see  over  a  finite  type  ring  um  so  let  me
say  that  uh  so  so  if  R  is
arbitrary  commutative  ring
uh  then  well  one  thing  you  can  say  is
that  uh  solid
R  um  by  which  I  mean  solid  R  comma
R  uh  so  you  solidify  all  of  the  all  of
the  elements  in  R  uh  this  uh  is  equal  to
in  of  the  finitely  presented
uh  elements  in  in  solid  are  um  and  the
finally  presented  things  are  the  things
which  are  just  co-  kernels  of  uh  of  maps
between  finite  direct  sums  well  you
don't  need  that  but  of  these  comp  of
this  single  compact  projective  generator
that  you  have  here  um  and  so
also  uh  if  you  want  to  know  about  this
finitely  presented  category  it  formally
reduces  to  the  finite  type
case  uh
um  so  in  some  sense  all  of  the
completion  is  happening  at  the  level  of
finite  type  sub
subrings
um  and  then  the  rest  is  just  some
algebraic  filtered  cimit
um  all
right  uh  so  right  so  now  we  need  to  do
this  analysis
here
um  so  I'll  make  a  a
claim  uh  which  is  basically  that  it  it
works  just  like  in  the  case  of  solid  Z
so  if  you  take  product  Z  tensor  solid  z
r  and  then  you  derive
solidify  uh  then  this  is  just  the  same
thing  as  a  product  of  copies  of  R  and  if
I  forget  to  say  an  index  set  it's  always
going  to  be  countable
um
okay  sorry  finite  yes  thank  you  yeah  our
finite  type  over  Z  thank  you  very  much
so  by  this  discussion  if  R  is  not  a
finite  type  what  you  would  instead  be
seeing  is  so  if  R  not  finite
type  then  uh  it's  equal  to  the  union
over  all  finite  type
subrings  of  the  product  of  copies  of  RP
Prime  which  is  smaller  than  the  product
of  copies  of
r
okay  uh  so  proof
so  so  so  first
consider  uh  the  polinomial  ring  on
finitely  many  uh
variables
um  then  we've  uh  essentially  already
seen
this  um  so  we  saw  that
the  derived  uh  T  solidification  of
of  a  free  module  um  commutes  with  all
limits  and  colimits  and  uh  and  sends  Z
to  just  the  to  to  Z  bracket  T  so  um
well  so  if  you  do  X1  solidification  so
uh  uh  then  that  just  gives  you
product
um
uh
uh
uh  X2  xn  and  then  you  just  keep  going
you  do  X2  solidification  X3
solidification  and  so  on  and  it  always
just  throws  the  variables  inside  the
product  instead  um  and  in  the  end  you
you  do  end  up  with  product  of  zx1  up  to
xn  so  here  we're  so  here  we
recall  uh  that  the  if  you  do  if  you  want
to  if  you  do  ZT
solidification  of  a  a  module  tensor  over
Z  with  Z  bracket  T  um  this  commutes  with
limits  uh  in  m  and  it
sends  Z  to  Z  bracket  T  and  we're  also
using  that  uh  the  different
solidifications
commute  so  that  you  can  actually  do  the
solidification  with  respect  to  all  of
them  by  just  doing  them  one  by  one  and
when  you  do  the  X1  solidification  and
you  X2  solidified  it'll  still  remain  X1
solid  you  also  use  that  if  you  solid  if
it's  X  and  Y  solid  then  it  is  Sol  for
everything  yes  in  the  subing  generated
yeah  yeah  yeah  exactly  so  that  you  know
so  that  to  solidify  for  R  it's  enough  to
solidify  for  all  the  variables  in  R  yeah
don't  well  right  but  we  we're  using  it
oh  well  this  is  yeah  this  I  guess  it
also  follows  yeah  that's  true  the
argument  for  that  was  basically  this
anyway  but  okay  um  it's  probably  good  to
mention  it  in  any
case
um  okay  um  now  in  the  general  case
so  let's  say  we  have  a  a  quotient  so  X1
xn  rejecting  on  to
R  uh
then  uh  sorry  uh  product
z  uh  tensor  Z  solid
R  derived  R
solidify  um  you  can  get  it  in  two  steps
again  you
can  uh  zolid  uh
uh  uh  and
then  uh  you've  already  solidified  with
respect  to  a  generating  set  for  R  so
you're  actually  already  are  solid  and
the  only  thing  you  need  to  do  is  do  an
algebraic  base  change  from  this  ring  to
the  ring  R  so  derived  everything  should
be
derived  um  but  we  already  figured  out
what  this  is  that  this  was  the  product
of  zt1  up  to
TN  and  then
we  uh  so  we're  left  with  just  analyzing
this  um  and  proving  that  it's  equal  to  a
product  of  copies  of  R
but  uh  so
zt1  up  to  TN  is
nean  that  means  R  can  be
resolved  by  potentially  infinite
resolution  but  a  resolution  by  finite
free
modules  and  for  each  of  those  finite
free  modules  it's  clear  you  just  bring
it  into  the  product  um  and  then  it
follows  that  the  same  holds  for  r  on  the
level  of  this  derived  tensor
product  um  so  then
so  you  can  look  at  the  class  of  zx1  up
to  xn  modules  for  which  this  claim  is
true  with  r  being  replaced  by  that
module  and  that  class  contains  the  Ring
Of  course  and  therefore  it  contains
anything  that  can  be  resolved  by  uh
direct  sums  of  copies  of  the  ring  as  a
also  the  product  the  infinite  product  of
copies  of  zt1  TN  is  I  think  flat  over
zt1  TN  again  it's  a  property  of  Nan  or
coherent  yeah  at  least  in  the  as  a  the
usual  product  but  here  it's  condensed
Mass  prod  but  still  I  think  for  every
condensed  set  it's  the  value  of  this  on
it  probably  is  yeah  yeah  that's  another
that's  another  argument  that's  another
possible  argument  but  I  mean  for  me  this
is  a  little
more  a  little  more  more
straightforward
yeah  yeah  because  the  flat  I  I  believe  I
don't  remember  the  flat  thing  just
uses
coherence  ahh  okay  so  the  flat  the
product  of  flat  is  flat  is  coherent
dring  but  now  you  use  ideal  is  is  a
finite  Ty  to  actually  see  that  when  you
quotient  out  you  get  a  product  of  copies
of  R  otherwise  it  would  you  would  get  a
limit  of  product
of  of  Co  liit  of  product  of
finite  fin  type  so
and  and  again  it  is  condens  set  by
condens  set  it's  the  same  it's  it's
verified  by  Elementary  way  yeah  I  mean
for  me  this  is  pretty  Elementary  but
okay  I  I  don't  know  no  because  in  theory
to  compute  this
tensel  L  you  have  to  resolve  on  the  left
I  mean  t  is  different  meaning  I  mean
when  you  B  change  from  of  ring  to
another  ring  you  are  supposed  to  resolve
the
original  not  okay  you  can  also  compute
it  by
resolving  the
other  I  don't  need  to  I  mean  I  I  really
this  argument  just  works  in  the  D
category  I  don't  need  to  I  don't  need  to
worry  about  resolving  this  it  just  I  the
the  argument  is  I  I  claim  there  are  no
gaps  in  this
argument  no  because  when  you  reserve  a
ZN  you  are  living  the  categories  of  R
and  you  have  a  d  category  of  some  of  no
no  no  no  no  then  but  the  map  is  Def  find
already  so  the  yeah  the  to  cheog  you  can
pass  to  the  D  categor  if  no  I  have  a
resolution  in  this  algebraic  category
and  then  I  just  I  have  this  tensor
funter  taking  me  to  this  DED  category
and  that  tensor  funter  preserves  co-
limits  and  I  get  a  resolution  there  it's
it's  really  um  it's  really  fine  um  okay
so  yeah  this  is  this  is  quite  technical
I'm  afraid  um  but  uh  so  we've  proved  now
this  uh  this  claim  that  this  drive
solidification  lives  in  degrees  zero  and
that  was  the  first  thing  we  needed  to
show  but  the  second  thing  we  needed  to
show  is  that  it's  flat  uh  with  respect
to  the  tensor
product  um  so  for  this  we  want  uh  we're
going  to  analyze  this  category  of  solid
R  modules  in  a  little  more  detail  so
this
structure
of  solid
R
um  so  again  so  let  me  say  maybe  theorem
so  again  R  finite  type  over
z  uh  then  well  something  we  already  know
um  from  this  being  a  compact  projective
generator  is  again  that  solid  R  uh  is
just  the  in  category  of  its  compact
objects  which  are  the  finitely  presented
ones
um  so  these  are  the  things  that  are  co-
kernels  of  maps  between  the  compact
projective
generators
um  but  this  CLA  second  claim  is  uh  so
you  could  say
coherence  uh  this  category  this
subcategory  uh  is  an  ailan  category  is
is  closed
under  uh  under  kernels  co-  kernels  and
extensions
this  is  same  as  this  Corine  uh  which
Coraline  yeah  the  first  part  is  the
first  part  is  obvious  or  sorry  well  the
first  part  follows  from  the  fact  that  we
have  a  compact  projective  generator  of
this  category  then  uh  the  compact
objects  will  all  be  built  from  finite
co-  limits  from  that  and  they  will  be
generators  and  it  follows  formally  that
it's  the  end  category  of  of
that
um
so  what  we  really  need  to  check  is  that
we  have  this  this  is  an  a  bilon
subcategory  now  let  me  remind  you  a  bit
about  some  classical  commutative  algebra
so  a  commutative  classical  commutative
ring  is  called  coherent  if  you  have  the
analogous  property  in  the  setting  of
discrete  modules  so  that  the  finitely
presented  R  modules  form  in  a  billion
subcategory  but  you  can  check  by  some  by
playing  around  with  short  exact
sequences  that  you  only  need  to  there's
only  really  one  thing  to  check  uh  which
is  that  every  uh  finitely  generated
ideal  uh  is  actually  finitely  presented
and  those  same  arguments  in  this  setting
uh  show  that  it  suffices  to
check  so
every  finitely  generated
subobject  of  product  of  copies  of  R  is
finally
presented  so
so  CF  coherent
Rings  now  I'm  going  to  use  a  bit  this
notion  quasi  separated  so  note  that  uh
if  you  have  any  subobject  of  product  of
copies  of  R  well  this  is  quasy  separated
I.E  house  dwarf  and  any  subobject  of
anything  quasy  separated  is  also  quasy
separated  so  this  is  also  quasy
separated  um  so  it  suffices  to
show  for
this  it  suffices  to  show  that  every
finitely  generated  quasi  separated  uh
solid  R  module  is  actually  finitely
presented  so  let  me  make  a  Lemma
so  so  if  m  in  solid
R  is  quasi
separated  uh  then  the  following  are
equivalent  um  one  is  that  m  is  finitely
presented  two  is  that  m  is  finitely
generated  and  three  is  that  well  you  can
say  exactly  what  M  looks  like  uh  so  M  uh
is  an  inverse  limit  of  mn's  or  each  MN
is  a  finitely  generated  R  module
discrete  R  module  um  and  the  transition
Maps  forus
projective
and  the  and  the  category  of
these  uh  is  just  the  countable  Pro
category  of  the  finite  of  the  category
of  finitely  generated  R
modules  so  the  maps  between  such  inverse
limits  are  just  the  the  maps  of  pro
objects  Pro  objects  with  subject  of
transition  yes  thank  you  thank  you  yes
yes
yes
yeah
yeah  okay
so
so  the  non-trivial  implications  are  two
implies  3  and  3  implies
1  um  so  for  two  implies
3  uh
so  well  we  finally  generated  by
definition  means  we  have  Sur  some
surjection
from  product  of  copies  of
R  and  then  there's  some  kernel
um  but  since  m  is  quasi
separated  uh  it  follows  that  this
inclusion  is  a  quasi  compact
map  of  of  of  condensed
sets  which  means  so  uh  this  object  is
quasi  separated  so  it's  just  some
filtered  Union  of  compact  house  doorf
spaces  and  this  means  that  when  you
restrict  this  inclusion  to  any  of  those
compact  house  door  spaces  you  get  a
closed
subset  but  um  this  is  one  of  those
sequential  spaces  metrizable  spaces  in
fact  um  and  so  being  a  closed  subset  on
any  compact  subset  is  the  same  thing  as
just  being  a  closed  subset  so  this  is
the  same  thing  as  just  a  closed  subset
in  the  topological
sense
uh  with  the  product
topology  CL  Subs  which  is  also  in  our
module  this  is  the  same  associate
condensed  module  is  the  one  that  we  are
talking  about  right  uh  so  let  me  so  then
um  but  now  we  can  look  at  for  any  n  we
can  look  at  the  projection  so  we  have  K
uh  subset  product  R  we  can  look  at  the
projection  onto  like  the  first  n
coordinates
uh  and  then  we  can  let  knen  be  the  image
in  here
uh  then  it  follows  that  uh  K  is  just  the
uh  the  inverse  limit  of  K
NS  uh  and  these  have  subjective
transition  Maps  I.E  so  K  is  as  in
three  no  you  know  that  m  is  as  in  three
but  this  also
uh  right  I  wanted  to  know  that  m  is  as
in  three  but  this  also  follows  yes
um  uh  and  that  yeah  then  yeah  and
M  analogously  be  uh  inverse  limit  Over  N
of  MN  uh  where  MN  is  the
product
uh
okay  and  note  that  this  argument  shows
that
um  if  you  so  in  particular  if  you  start
with  an  object  as  in  three  uh  yeah  okay
so  now  let  yeah  so  now  um  we  also
see  uh  to  finish  I.E  to  show  that  3
implies  1  it  actually  suffices  to
show  uh  3  implies
2  because  if  we  know  that  this  guy  is
fin  generated  then  it  fits  into
something  like  this  and  we  just  prove
that  the  kernel  is  of  the  same  form  and
then  it  will  follow  that  it's  fin
presented
um  um  so  but  then  this  is  actually
Elementary  so  you  have  like  M1
subjecting  onto  M2  subjecting  onto
M3  you  can  make  some  uh  finite  free
module  rejecting  onto
M1  um  and
then  you  have  some  here  uh  kernel  of
this  map  here  uh  and  then  you  can  make
some  R  direct  some  D2  rejecting  onto
that  uh  and  then  uh  you  can  make  a
system  R  direct  some  D1  R  direct  some  D1
plus
D2
uh
uh  um  and  in  the  inverse  limit  you  get  a
product  of  copies  of  R  rejecting  on  to
m
so  in  fact  uh  if  you're  wondering  about
surjection  on  the  level  of  condensed
objects  in  fact  you  can  even  get  a
topological  splitting  for  the  map  of
topological  spaces  so  by  kind  of  section
here
make  a  compatible  section  there  and  so
on  if  you're  not  worried  about  our
linearity  which  you're  not  then  uh  you
can  get  this  quite
easily
um  okay  so  that's  the  proof  of  the
Lemma  so  the  quasy  separated  objects  are
just  these  kind  of
classical  these  classical  R
modules  um  and  that
also  uh  proves
the  um  it  proves  the  uh  the  coherence
here  in  extensions  is  easy  extensions  is
formal  yeah  yeah  it  doesn't  require
anything
uh  because  it's  projective  projective
generator  so
it's  yeah  you  can
lift
exactly
I  mean  actually  uh  because  you  have  this
compact  projective  generator  you're
equivalent  to  the  category  of  modules
over  a  ring  and  so  it's  really  it's  you
can  actually  just  quote  the  usual
theorem  from  commutative
algebra  um  okay  yes  all  right  like  so
here  um  so  if  M  was  finally  generated  in
particular  the  kernel  also  had  this
really  nice  property  of  being  of  having
property  three  as  well  uh  if  M  was  quasi
separated  and  finitely  generated  yeah
and  so  that  in  particular  implies  that
if  m  is  finally  you  can  find  a
resolution  of  by  really  nice  like  yeah
that's  right  that's
right  you  said  you  said  something  about
Comm  in  fact  the  endomorphisms  of  the
projective  object  is  an  associative  yes
oh  yeah  sorry  yeah  not  not  commutative
algebra  sorry  thank  you  thank  you  it's
not  communative  algebra  it's  it's  module
Theory  over  non-commutative  ring  yeah
thank  you
yes
wait  we're  still  proving  the  theorem  uh
we're  done  with  that  theorem  where  did
we  use  the  fact  that  was  finite  type
over  Z
sorry  ah  well  it  it  was  used  in  order  to
say  that  this  was  the  compact  projective
generator  so  when  we  were  so  a  prior  you
take  product  Z  tensor  up  to  R  and  then
solidify  with  respect  to  R  and  only  in
the  finite  type  over  Z  case  when  you  had
this  rejection  from  The  polom  Ring  yeah
but  for  any  narian  ring  you  can  take  the
product  of  copies  of  yeah  you  if  if  you
just  say  that  you  build  a  category  that
has  this  as  compact  projective  generator
with  the  endomorphisms  what  they  have  to
be  then  the  rest  of  the  arguments  here
work  in  the  for  an  arbitrary  etherion
ring
yeah
okay  um  right  and  we're  not  quite  done
with  the  uh  the  first  theorem  because  we
still  need  to  show  that  uh  product  of
copies  of  R  is  flat  so  now  but  so  now  we
make  a  claim  that  if  uh
m  is  finitely
presented
uh  in  solid
R  uh  then  if  you  take  M  derive  tensor
product  R  solid  with  product  of  copies
of  R  you  just  get  a  product  of  copies  of
M  and  in  particular  this  lives  in  degree
zero
no  um  and  that  implies  uh  that  this
theorem  or  this  claim  implies  that  this
guy  is
flat  uh  because  the  derived  tensor
product  with  any  finally  presented
object  lives  in  degree  Z  but  every
object  is  a  filtered  cimit  of  finitely
presented  objects  so  the  drive  tensor
product  with  any  object  will  live  in
degree  Z  which  is  the  same  thing  as
saying  you  have
flatness  so  incidentally  to  define  the
the  right  pens  of  maybe  we  didn't  spend
time  on  this  but  in  the  usual  approach
you  need  to  know  there  are  enough
objects  to  relative  to  T  to  define
classically
t  so  here  what  is  the  approach  what's
the  meaning
assigned  d  t  do  you  prove  that  there
enough  L  it's  not  really  necessary  so
first  on  the  level  of  the  derived
category  of  light  condensed  ailan
groups  um  I  mean  you  can  take  this
perspective  of  sheaves  with  values  in
the  OR  hypers  sheaves  with  values  in  the
derived  Infinity  category  and  then  you
can  just  sheify  the  sectionwise  derived
tensor  product  of  ailan  group  derived
sheify  the  sectionwise  tensor  product  of
derived  ailan  groups  and  that's  a  fine
definition  of  the  derve  tensor  product
on  that  basic  thing  and  then  it  just
gets  ported  to  all  of  the  other  contexts
uh  you  know  you  have  a  ring  object  in
there  you
have  uh  then  you  just  do  a  relative
tensor  product  and  then  uh  you  have  some
localization  but  we  already  argued  how
to  get  the  symmetrical  structure  on  the
localization  of  given  given  by  an
analytic  ring  structure  and  so  on  and  so
forth  so  it's  not  actually  to  make  the
definition  it's  not  actually  necessary
to  know  anything  about  flat
objects  so  for  for  condensed  a
billion  ah  okay  so  let  us  say  without
Sol  you  just  have  the  T  of  okay  there  we
know  we  know  initially  there  enough  FL
in  indeed  in  fact  in  in  light
condensability  groups  we  do  know  there
are  enough  flat  objects  so  that's  I
didn't  need  that  Infinity  stuff  really
to  you're  right  yeah  yeah  but  I  mean
yeah  I  have  to  say  I  was  kind  of  off  the
top  of  my  head  just  saying  how  I  think
about  it  yeah  these  free  object  free
object  on  a  light  condensed  set  will
itself  be  flat
and  um  yeah  okay
uh  okay  so  where  was  I  ah  so  we  have  we
have  to  Pro  to  finish  the  uh  discussion
of  so  far  we  just  need  to  prove  this
claim
um  so
proof  um  so  we  are  finally  presented  so
we  can
make  uh  a  surjection  here  where  the
kernel  is  finitely
generated  but  the  kernel  is  also  uh
quasy
separated  um  so  you  know  it'll  also  be
finitely  presented  well  we  can  then
it'll  we  can  basically  just  continue  to
make  an  infinite  resolution
so  uh  and  that  reduces
us  to  the  case  uh  where  M  itself  is
equal  to  this  product  of  copies  of
R
um  and  then  uh  and  then  we  can  remember
that  this  was  product  of  copies  of  Z
tensor  R  and  then  R  solidified  and  then
that  uh  will  reduce  us  to  the  analogous
fact  for  products  of  copies  of  Z  that  it
distributes  over  uh  infinite
products
so  for  prod  infinite  product  of  Z  for
Z  ah  then  you  reduce  to  to  Li  condens
abil  groups  is  infinite  produ  of  Z
and  yeah  there  are  even  the  thing  so
then  you  have  this  nice  object  that
solidifies  to  a  product  of  copies  of  Z
this  uh  this  thing  we  were  calling  P
basically  n  Union  infinity  and  that
light  condensed  set  has  the  property
that  if  you  take  its  product  with  itself
it's  it's  isomorphic  to  itself  in  the  in
the  expected  way  um  so  or  like  that  that
object  for  n  times  itself  is  the  same  as
that  object  for  n  cross  n  and  yeah  you
just  well  anyway  it  was
explained
um
okay  um  any  other
questions
because  this  is
projective  uh  this  is  the  prod  copies  of
the
projec
generate  okay  but  if  you  did  not  define
t  l  is  a  classical  Left  Right  F  you
cannot  use  the  fact  that  for  projective
object  I  mean  this  requires  a  classical
treatment  yes  but  right  but
so  the  actually  theun  of  the  T  right
that  that  I  could  have  added  to  the  um  I
could  have  added  that  to  uh  yeah  that's
right  that's  another  important  claim  in
addition  to  the  fact  that  the  derived
category  of  the  ailan  category  is  the
derived  category  that  we  defined  you
also  have  that  the  derived  solidifi  or
derived  solidification  is  the  left
derived  functor  of  solidification  you
have  that  the  deriv  tensor  product  is
the  left  dve  functor  of  the  tensor
product  and  you  can  resolve  an  either
variable  this  all  follows  once  you  know
all  of  this  this  flatness  and  living  in
degree  zero
stuff  so  again  to  so  the  foundational
way  you  define  tensor  L  for  your  deriv
category  of  solid  is  what  I  mean  without
using  flat  objects  yeah  so  the  way  you
do  it  is  that  you  of  course  it  lives  in
some  dve  category  without  without  solid
so  you  can  just  take  tensor  there  and
solidify  yes  exactly  of  course  you  have
to  know  that  that  somehow  then  part  of
this  story  is  that  solidification  commut
I  mean  if  two  things  have  the  same  after
I  mean  you  have  to  know  but  then  to
define  derived  without  solidification
then  then  it  is  just  derived  over  this
is  just  modules  over  a  ring  supp  you
already  know  over  Z  of  the  find  yeah
then  over  a  ring  you  use  flat  object  or
what  do  you  use  you  you  never  have  to
use  flat  objects  to  make  the  definitions
okay  so  how  do  you  define  the  the
derived  tenso  product  for  without  the
solid  yeah  yeah  you  have  a  you  you  have
a  commutative  ring  in  solid  Z  and  you
want  to  do  dve  tensor  product  I  mean  it
explicitly  it's  you  just  you  know  it's
some  geometric  realization  of  a  bar
construction  where  all  the  tensor
products  over  solid  Z  are  derived  I  mean
so  Lori  sets  it  up  if  you  have  a
commutative  algebra  object  in  a
symmetric  monoidal  Infinity  category
then  you  get  a  tensor  structure  on  our
modules  in  there
um  okay  so  you  can  have  in  this  General
setup  you  have  some  way
to  yeah  I  mean  it's  the  it's  basically
like  you  take  M  tensor  Ln  M  tensor  LR
tensor  Ln  You  Know  M  tensor  LR  tensor
squar
okay  and  then  you  have  some  simplicial
object  and  you  take  the  co  limit  all  in
this  Infinity
sense  okay  so  you  have  this  General  way
of  viewing  the  tens  of  products  and  then
it  always  in  all  cases  there  enough  flat
objects  and  you  can  use  flat  or  okay  in
in  the  in  in  what  we've  discussed  so  far
the  solid  theory  yes  that's  what  we've
just  proved  although  as  you  see  it's  not
a  formal  argument  by  any  means  you
really  have  to  analyze  uh  what's  going
on
other
questions
okay  um
so  oops  so  I  want  to  spend  just  a  little
more  time  on  the  homological  algebra  of
solid  R
um  prove  because  Peter  also  explained
something  else  in  the  case  of  solid  Z
basically  that  every  finitely  presented
module  it  doesn't  just  have  an  infinite
resolution  by  products  of  copies  of  Z
but  it  actually  has  a  two-term
resolution  by  products  of  copies  of  Z  so
that  kind  of  at  least  from  the
perspective  of  finitely  presented
objects  you  have  sort  of  homological
Dimension  One  for  solid  Z  just  like  you
have  for  usual  Z  so  um  I  want  to  present
a  generalization  of
that  um
so
so  uh  theorem  so  if  R  is  finite
type  over  Z  and  is
regular  a  regular  ring  of  Dimension
d  uh  then  if  you  have  a  m  in  solid  are
finitely  presented  and  you  have  n  and
solid  are  arbitrary
say  uh  then  X  I  even  internal  X  why  not
uh
uh  the  XS  vanish  uh  in  degrees  bigger
than  the
dimension  and  the  to
also  right  so  a  corollary  of  this  will
be  that  if  you  have  a  finitely
presented  uh  module  then  you  actually
get  a  projective  resolution  of  length  uh
in  most  well  depending  on  how  you  define
length  D  or  D+  one  yeah  maybe  yeah
um  and  then  we  already  saw  that  the
projective  objects  are  flat  so  this  also
implies  a  bound  on  the  tours  so  so  a
corollary
is  so  if  if  m  is  and  that  and  that
passes  from  fin  presented  to
arbitrary
uh
let  me
say
hi
yeah
okay
um  let  me  caution
uh  uh  this  this  this  this  statement  does
not
extend  to  non-finitely
presented  M  so  unlike  unlike  in  the  case
of  discrete  modules  over  a  nean  ring
regular  nean  ring  of  finite  Dimension
but  there  you  actually  have  the  X
Vanishing  uh  for  all  modules  um  there's
some  extra  argument  for  going  from  the
finitely  presented  case  to  the  general
case  that  argument  does  not  work  in  this
context  and  in  fact  um  I  think  you  can
have  arbitrarily  High  AXS  even  for  solid
Z  so  let  me  give  a  here's  a  fun  a  fun
exercise  you  can  try  to  do
so  you  take  two  distinct  prime  numbers
uh  this  is  just  just  cute  so  here's
something  that's  not  finitely
presented  uh  because  you're  quotienting
by  something  which  is  z  Z1  over
L  um  then  if  you  do  there's  an  X2  which
is  nonzero  and  it's  given  by  switching
the  roles  of  P  and
L  uh  so  that's  just  if  you  want  to  have
a  little  fun  with  some
calculations  and  the  n  x  to  and  you
claim  also  there  arbitrary  highx  I
believe  that  so  certainly  there  so
certainly  uh  there  exists  a  model  of  zfc
in  which  there  are  arbitrarily  highx  so
there's  logicians  analyzed  a  particular
uh  X  Group  which  I  think  Peter  also
mentioned  in  one  of  his  lectures  but  I
think  even  in  zfc  it  should  be  possible
to  produce  a  arbitrarily  High
non-vanishing  X  I  mean  even  under  in  any
model  of
zfc  I  mean  you  run  to  a  priori  if  you
want  to  go  from  the  finitely  presented
case  to  the  general  case  you  have  to  an
analyze  derived  inverse  limits  along
some  arbitrary  filtered  system  but  do
you  have  a  boundary
cardinality  because  if  you  are  in  a
model  with  the  Contin  hypothesis  for  FS
yeah  and  if  the  you  look  at  or  maybe
generalize  I  don't  know  if  you  look  at
all  possible  finite  types  of  modules
yeah  so  they  M  it's  related  to  maths
from  product  of  Z  to  product  of  Z
yeah  then  cality  of  this  space  is
like  one  of  those  some  finite
aln  then  we  have  chological  Dimension
results  for  this  yes  yes  yes  yes  yes  so
it  suggests  that  that  there  the  would  be
bound
for  sub  for  quotients  of  s  by  by  of  a
project  something  but  then  there  is  a
classical  argument  in  ring  which
probably  extends  to  this  case
that  that  if  you  know  the  X  Vanishing
for  this  kind  ring  mod  IDE  and  you  have
it  for  arbitrary  yeah  but  so  but  you
need  to  but  that  the  analog  of  that
argument  shows  you  need  to  analyze
something  like  an  arbitrary  subm  module
of  product  of  copies  of  Z  not
necessarily  finally  generated  and  yeah
and  yeah  and  then  yeah  but  I  guess  yeah
there's  a  bound  on  the  cardinality
there  yeah  so
yeah  you're  right  that  the  cardinality
of  what's  that
Peter  it  could  be  that
something
have  maybe  yeah  it's  possible  yeah
um
but  if  the  cardinality  of  the  Continuum
is  larger  than  alfn  for  any  n  then  then
you  should  probably  get  um  okay  uh
arbitrary  higher  X  anyway  let's  let's  uh
let's  not  go  there
um  right  uh  where  are  we  ah  so  let's
let's  try  to  prove  this
so
um  so  the  first  claim  so  first  we're
going
to  so
first
case  uh  we're  going  to  take  n  so  ah  so
we  well
so  so  by  the  Sorry  by  the  same  argument
um  sorry  just  a  sec  let  maybe  I'll  do
that  at  the  end  so  the  first
case  let's  say  that  n  is  actually
a  know  a  classically  finitely  generated
module
over  so  the
uh  nean  ring
R
um  and  then  let's  take  M  to  be  quasy
separated  a  finally  presented  solid
R  so  then  that's  an  inverse  limit  of
mn's  uh  over  all
n  uh  with  surjective  transition
maps  and  in  this  case  the  claim  is
um  that  X
I  uh  from  M  to  n  uh  or  I  guess  it
doesn't  even  need  to  be  finely  generate
let's  just  say  discreet  uh  is  the  same
thing  as  the  filtered  Co  limit  of
xti  over  R  so  this  is  R  solid  um  xti
over  R  from  MN  to
n
didn't  you  Pro  that  par  present  impli  qu
separation  uh
no  no  certainly  not  uh  I  mean  in  solid
the  solid  Z  case  something  like  ptic
integers  mod  integers  will  be  fin
presented  but  not  quasi
separated  so  what  what  did  be  Pro  proved
that  if  you're  quasi  separated  and
finally  generated  then  you're  finally
presented
uh  and  if  you're  fin  presented  are  you
ah  you  are  not  necessarily  qu  SE
exactly  exactly  yeah  so  for  example  zp
modulo
z
uh  is  finally
presented  so  if
a  map  from  product  of  copies  the  product
of  copies  of  R  is  the  coal  is  not  clly
separate  not  necessarily  that's  right
okay  I  misunderstood  this  uhuh  yeah
that's  important  um
yeah
um  so  we  need  to  show  that  so  we  need
that  the
arom  uh  you  can  pull  out  the  inverse
limit  uh  basically  uh  by  the  way  the
claim  for  internal  follows  from  the
claim  with  underlying  it's  just
replacing  n  by  like  continuous  functions
with  values  in  N  for
some  uh  for  some  profinite  set  s  um  but
so  uh  we  have  the  mitag  Leer  so  we  have
the
resolution
uh  some  kind  of  usual  identity  minus
shift  sort  of  thing  or  maybe
it's  um  or  shift  times  F  or  some  some
kind  of  shift  map  um  so  this
distinguished  triangle  here  so  that
tells  you  that  the  question  of  pulling
rhom  the  inverse  limit  out  of  the  ROM  is
the  same  as  the  question  of  uh  pulling
out  a  product  and  turning  it  into  a
direct  sum  um  and  then  we  can
resolve  product  MN  by  product
R  to
reduce  to
ROM  product  r  n  equals  direct  sum
R
RN  uh  which  follows  from  this
being  uh  projective  and  yeah  and  the
basic  calculation  on
home
okay  so  from  this  it  follows  that  we  get
the  desired  bound  uh  on  X  if  n
is  uh  a  classical  discret  R  module  and  M
is  finally  presented  and  quasy
separated
now  you  can  always  resol  so  now  M
arbitrary  finally  presented  not
necessarily
quasi  separated  then  you  can  always
resolve  oops
M  by  product  of  copies  of  R  and  then  the
kernel  will  be  both  both  uh  finitely
presented  and  quasy
separated  and  uh  then  you  analyzing  the
long  exact  sequence  in  X  mapping  out  to
n  you  would  get  the  desired  result  but
with  uh  you'd  lose  one  degree  of
chological  Dimension  you'd  get  that  x  i
vanishes  for  I  greater  than  D+
one  um  so  the  xti  vanishing  here  for  I
greater  than  D  when  you  Chase  through
the  long  exact  sequence  would  only  give
the  vanishing  in  degrees  greater  than  D+
one  there
so  uh  so  this  naive
argument  only
gives  uh  x  i  m  n  =  0  for  I  greater  than
D  +
1  so  we  have  to  do  a  little  bit  of  extra
work  to  improve  this  for  that  conclusion
there  we  didn't  use
regularity  uh  well  we  did  because  I
should  maybe  recall  the  classical  fact
so
weall  so  if  R  is  regular
mean  Dimension  D  then  the  classical  X
groups  you  know  between  discrete
modules  so  this  so  that  the  the  question
now  for  quasy  separated  things  in  the
solid  context
reduces  to  the  classical  question  in
discret  commutative  algebra  and  that's
where  we  use  regularity
yeah  um  so  we're  going  to  do  a  less
naive  argument  so  we're  going  to
continue  the  resolution  one  more  step
so  so
instead  so  we  have  a  product  of  copies
of  R  we  can  Sur  again  from  a  product  of
copies  of  R  and  now  at  this  point  we
take  a
kernel
um  and  we  get  a  resolution  like
this  um  this  is  still  quasy  separated
and  then  it  suffices  to
show  uh  to  so  to
prove  uh  x  i  m  ah  different  n  I'm  so
sorry
um  ah  so  let  me  let  me  say  x
uh  for  I  I  greater  than  D  it  suffices  to
show
uh  for  I  greater  than  D  minus  2  so  we
bought  ourselves  one  extra
uh
uh  drop  um  because  we  continued  the
resolution  One  Step
more  um  sayal  Z  and  one  have  to  be  head
that's  true  yeah  so  this  will  I'll  be
assuming  so  assuming  D  greater  than  or
equal  to  two  the  case  D  equals  1  well
you  can  use  similar  arguments  in  low
Dimension  yeah  thanks
Peter  one  the  same  as  for  integers  I'm
not  entirely  sure  but  a  modification  of
this  argument  for  D  greater  than  or
equal  to  two  will  will  also  work  for  D
equals
1  um  right  so  it  suffices  to  prove  this
x  Vanishing  so  now  we  want  to
um  uh  now  we  want  to  find  a  nice
expression  of  n  as  an  inverse  limit  of
finitely  generated  R  modules  so  there's
actually  two  different  two
ways  of
writing
uh  n  as  an  inverse
limit  of  a  finitely  generated  R
modules  so  the  first  way  is  just  the
what  we  have  the  argument  we  already  had
where  this  is  um  this  is  the  kernel  of  a
map  between  quasi  separated  things  so
it's  a  closed  subm  module  here  um  so  we
can  always  just  look  at
the  um  well  maybe  I'll  make  that  the
second  method  meod  sorry  the  first
method  would  be  uh  we  analyze  this  map
here  um  this  is  a  map  from  product  of
copies  of  R  to  product  of  copies  of  R
again  that's  the  same  thing  as  a  map  on
the  Pro  system  so  if  you  uh  if  you
restrict  to  any  initial  chunk  of  this
then  there's  some  corresponding  initial
chunk  here  where  the  the  map  projecting
onto  that  factors  through  projecting
onto  this  chunk  and  then  just  a  map  of
discrete  modules  there  so  if  you  reindex
and  allow
like  uh  finite  free  modules  here  instead
so  uh  then  you  can  you  can  assume  that
uh  this  map  here  comes  from  an  inverse
limit  of  compatible  maps  from  um  you
know  the  initial  chunk  the  nth  initial
chunk  here  to  the  nth  initial  chunk  here
just  by  just  by
reindexing  um  and  then  if  you  take  the
kernels  of  those  Maps  so  you  take  uh
so  well  let  me  say  I  don't  really  need
it  to  be  a  product  anymore
so  so  we  take  an  inverse  limit  of  this
type  and  we  say  that  this  is  given
termwise
um  uh  so  then  we  can  take  so  we  can  let
uh  so
let  uh
so  then  we  have  n  equals  inverse  limit
Over  N
NN  but  the  other  thing  we  can  do  is  we
can  let  uh  n  Prime  n  which  will  be
contained  in  NN  uh  contained  in  FN  be
the
image  of  n  mapping  to  inverse  limit  of
FN  mapping  to  FN  or  the  particular  FN
inverse  limit  FM  mapping  to  the
particular  FN
um
so  uh  number  one  has  uh  so  number  so
each  of  them  each  of  these  presentations
of  n  has  something  good  and  something
bad  about  it  so  uh  good  news  for
one  is  that  you  have  this  uh  XD  minus  2
Vanishing  x  i  x  i  NN  x  equals  zero  for
all  I  bigger  than  exactly  what  you  want
D  minus
2  um  but  the  bad
news  no
guarantee  that  this  system  NN  is  MOG
leer  uh  which  is  needed  for
uh  X  to
I  uh  NX  equals  filtered  Co  limit  Over  N
of
xti  and
NX  so  that's  uh  that's  unfortunate  so  in
the  AR  the  argument  I  presented  here  in
the  case  where  the  transition  I  said  the
transition  Maps  were  subjective  but  the
argument  really  only  used  that  it  was  a
MOG  Lefler  system  um  to  get  this
resolution  here  um  and  then  in  two  prime
the  situ  in  two  the  situation  is
opposite  uh  MOG  leer  in  fact  the
transition  maps  are  rejective  by
construction  but  no
guarantee  that
X  um  D  minus  one  uh  n  Prime  n  x  equals
z  so  n  Prime  n  is  this  condensed  module
of
no  no  it's  just  discreet  so  any  any
subobject  of  a  discret  object  is  itself
discreet  all
right  so  we  know  that  this  is
a  but  we  know  that  n  is  the  limit  of
both  in  the  condens  it's  limiting  the
condenses  of  both  of  these  systems  yes
okay  but  and  certainly  of  you  can
produce  an  example  of  map  from  limn  to
limn  Prime  for  which  the  the  the  kernels
would  not  be  Meed  left  yes  exactly  yes
you  can  yeah  URS  in  nature  yeah
yes
um  right  so  what  we  have  to  do  is  find
something  in  between  that  has  the  good
properties  of  both  so  I  apologize  this
argument  is  a  bit  technical
by  the  way
um  the  reason  we  have  this  nice  property
here  uh  is  that  it's  a  kernel  of  a  map
between  uh  projective  objects  uh  so  it's
it's  like  the  it's  the  you  know  so  if
you  took  the  co-  kernel  then  it  would  be
two  steps  away  from  that  Co  kernel  so
the  vanishing  in  degrees  greater  than  D
for  the  co-  kernel  will  imply  The
Vanishing  in  degrees  greater  than  D
minus  2  for  that  long  kernel  over  there
that's  why  you  have  this  and
um  so  and  but  for  this  NN  Prime  you  you
don't  know  that  it's  a  kernel  of  a  map
between  projective  things  you  only  know
that  it's  sitting  inside  a  projective
thing  so  You'  get  Vanishing  of  XD  one
better  than  for  a  general  module  but  you
wouldn't  get  Vanishing  a  priori  in
degree  D  minus  one  so  that's  what  we
need  to  fix
um  so  what  is  the  difference  so  find  so
we're  going  to  find  uh  n  Prime  n  sitting
in  between
them
uh  and  such  that  we  get  X
Vanishing
uh
uh  uh  and  uh  NP  Prime  n  is
mler
so  if  we  get  this  then  we're
done  uh  because  the  inverse  limit  uh  we
would  get  the  desired  X  bounds  for  the
inverse  limit  of  the  NN  primes  but
that's  sandwiched  in  between  these  two
things  which  have  the  same  inverse  limit
which  is  the  thing  we're  interested  in
so  the  vanishing  of  X  for  this  inverse
limit  would  imply  it  for  uh  our  our
desired  Vanishing  for
n
okay
so  now  we  use  a  bit  of  commutative
algebra  so  so  what  is  the  obstruction  we
have  a  module  where  we  know  the
vanishing  in  degree  D  but  we  don't  know
the  vanishing  in  degree  D  minus  one  uh
the
obstruction  uh
having
uh
uh  this
is  uh
that  so  that  the  obstruction  is  in  some
sense  concentrated  at  the  closed  point
so  it's  a  question  of  depth  so  the
Outlander  bbam  formula  tells  you  that  uh
at  a  for  a  regular  nean  local  ring  the
projective  dimension  of  a  module  plus
the  depth  of  the  mod  module  is  equal  to
the  dimension  of  the  Ring  um  and  because
we  have  the  uh  we
have  uh  the  one  better  estimate  on  the
projective  Dimension  um  the  you  know  the
only  rings  that  are  going  to  give  us
obstruction  are  the  local  rings  at  at
Prime  ideals  which  are  actually  maximal
ideals  so  that  the  local  ring  has
maximal  Dimension  and  the  only
obstruction  is  going  to  be  moving  from
situation  of  depth  one  at  that  maximal
point  to  depth  two  so  so  is  the  fact
that
at  a  maximal
ideal  uh  so  n  and  Prime  n  only  known  to
have  depth
one
yeah  yeah
yeah  yeah  yeah  many  closed  points  and
you  know  that  it  lies  inside  the  other
one  because  the  other  one  is  as  this  yes
is  reflexive
yes  and  so  uh  and  then  you  have  to
establish  but  only  close  point  the
classical  L  argument  work  that's  that's
correct
yes
okay  all
right  maybe  I'll  just  repeat  what  oer
said
so  so  what  you  can  do  if  you  let's  let's
say  for  Simplicity  you  only  have  a
problem  at  one  maximal  ideal  in  general
you'd  fix  a  problem  at  finitely  many  and
then  if  you  increase  the  number  of  them
eventually  the  situation  would  stabilize
by  an  etherum  this  so  um  so  let's  say
there's  a  problem  only  at  one  maximal
ideal  then  you  can  look  at  the  inclusion
of  so  Spec  R  minus  that  closed  point  x
uh  closed
Point  into  Spec  R  and  then  you
replace
n  Prime  n  by  uh  J  lower  star  J  upper
star  of  n  Prime
n  and
uh  that  receives  a  map  from  n  Prime  n
but  it's  in  fact  an  inclusion  due  to  the
depth  one  assumption  uh  on  our
module  um  but  then  on  the  other  hand  by
this  procedure  of  this  extension  and
restriction  so  depth  is  you  can
characterize  depth  in  terms  of  local
chology  at  the  maximal  ideal  and  uh
that's  exactly  what  comes  up  when
analyzing  the  difference  between  this
and  say  the  drive  version  of  this  and
using  that  you  can  see  that  this  uh
improves  the  depth  so  this  is  depth
two  uh  at
X  um  on  the  other  hand  if  you  were  to
perform  the  same  construction  for  NN
since  we  already  know  this  has  depth  to
you  wouldn't  have  been  changing  it  so
you  really  are  producing  something  uh
sitting  in  between  which  fixes  the
problem  at  a  given  closed  point  then
there's  the  neoness  which  tells  you  well
it's  only  finitely  many  Clos  points  that
are  going  to  be  involved  so  you  can  by
the  same  procedure  you  can  fix  the
problem  all  the  problems  for  n  Prime  n
um  and  in  a  compatible  manner  in  the
tower  okay  so  that's  then  you've  got  it
to  be  depth  to  at  all  the  closed  points
which  gives  you  the  desired  X  Vanishing
for  these  uh  N  double  Prime
ends  uh  and  then  what  about  the  mid
leer
so
so  uh  mtag  ller
so  this  Tower  N  double  Prime  n  sits  in  a
short  exact  sequence  with  n  Prime
n  uh  and  the  quotient  n  Prime  n  mod  n
Prime
n  so  to  show  that  this  is  Midler  you
need  to  know  that  this  is  mogler  or  it
suffices  to  show  that  this  is  MOG  Lefler
and  that  is  MOG  Lefler  this  one  we  know
because  the  transition  maps  are
subjective  but  this  one  this  one  here  is
supported  at  but  it's  supported  at
finitely  many  closed
points  and  it's  a  finitely  generated
module  over  the  ring  and  that  implies
that  it's  finite  as  an  ailan
group  because  remember  our  ring  was
finite  type  over  Z  the  maximal  ideal  is
all  of  course  have  you  know  the  resid
field  is  a  finite  field  and  a
compactness  argument  shows  that  any  uh
inverse  system  of  finite  aan  groups
satisfies  the  mogler  property  so
compactness  gives  M
left
okay  it's  a  bit  of  a
yeah
yeah
um  okay  so  that  gives  the  X  Vanishing  so
so  the  to  sum
up  summary  of  what  we've  done  so  far
we've  seen  that  if  m  is  fin
presented  solid  R  and  if  x  I've  switched
to  x  uh  is  a  discrete  module  you  know
over  the  ring  then  we  get  the  desired  X
Vanishing
um  now  uh  so  now  suppose
x  if  x  is  quasi  separated  and  finally
presented  in  solid
R  you  write  X  as  an  inverse  limit  of
xn  uh  with  surjective  transition
Maps
um  and  then  the  R  homs  into  that  will
just  be  the  uh  inverse  limit  of  the
derived  inverse  limit  of  the  R  homs  into
each  of  those  terms
now  in  principle  you  might  get  one  worse
again  because  there  could  be  a  limb  one
in  the  last  inverse  system  but  because
that  last  uh  uh  in  the  that's  that's
talking  about  the  XDS  into  this  but
since  these  transition  maps  are
subjective  and  D  is  the  largest  degree
at  which  you  have
nonzero  X  Vanishing  you  actually  see
that  on  XDS  the  you  get  surjective  uh
Maps  so  that  there's  no  limb  one
potentially  giving  you  something  in
degree  D+  one  um  so  the  you  get  the
claim  for
D  um  and  then  uh  X  arbitrary  finitely
presented  then  you  resolve  it  by
uh  and  the  X  with  values  in  X  can  only
be  better  uh  then  the  X  with  values  in
these  two  so  you  get  that  you  get  that
situation  as  well  and  then  finally  for
an
arbitrary  so  for  X
arbitrary  uh  so  if  you  write  X  as  a
filtered  Co  limit  of  x  highs  or  these
are  fin  presented  then  uh  X  to  I  uh  from
any  finit  presented  M  to  X  is  the
filtered  Co  limit  of  x  i  oh  no  I
shouldn't  use  I
XJ  M  to  XI  by  pseudo
coherence  of  M  so  you  can  resolve  M  by
product  of  copies  of  R  and  then  this
follows  from  product  of  copies  of  R
being  compact  projective  what  about  X
line  it's  the  so  yeah  I  should  from  this
point  on  so  uh  this  claim  for  an
arbitrary  discret  module  implies  the
same  claim  for  underline  because  it's
the  same  as  uh  s  valued  points  of  this
thing  is  the  same  thing  as  X  I  from  M  to
continuous  functions  from  s  to  this
discrete  thing  which  is  just  another
example  of  a  discrete  thing  so  then  you
get  this  there  and  then  from  that  point
on  all  of  the  arguments  actually  work  at
the  uh  at  the  internal  a  level
okay  including  the  Lim  one  argument
including  the  limb  one  argument  yes
because  you  reduce  it  to  to  an
X  you  reduce  it  to  showing  that  a  limb
one  vanishes  in  condensed  to  billion
groups  and  the  terms  in  the  system  are
discrete  uh  and  the  system  is  MOG  Lefler
and  that  those  properties  are  preserved
by  and  if  you  take  continuous  functions
with  values  in  that  it's  still  a  system
of  discreet  things  and  the  transition
maps  are  still  mler  or  subjective  even
yeah  and  and  in  the
condensed  a  bilan  group  again  you  have
up  to  only  up  to  Li  one  yes  you  don't
have  more  because  you  can  compute  it
termwise
yes  because  because  products  are  because
countable  products  are
exact  that's  the
reason  ah  okay  the  the
countable  ah  okay  okay  and  you  can
compute  the  Lim  one  term  by  no  maybe
this  is
not  you  can  compute  Lim  one  on  each
object  not  necessarily  project  okay  you
don't  have
projective  Compu
one  it's  not  like  in  any  site  where  it's
it's  delicated  you  cannot
compute  but  here  you  can  compute  the  Lim
one  on  any  test  condensed  Set
uh  no  no  for  some  things  yeah  so
certainly  for  this  P  object  you  can
because  that's  projective
um  and  then  that's  enough  for  solid
because  we  know  that  the  solidification
of  P
generates  um  so  that's  that's  one
argument  you  could  give  I'm  sure  there
are  other  arguments  as  well  but  yeah  I
mean  but  you  know  the  limb  one  is  always
just  going  to  be  the  sheif  of  the  uh  you
know  the  condensed  limb  one  will  always
be  the  sheif  of  the  naive  sectionwise
limb  one  and  so  if  prove  the  vanishing
of  the  of  the  limb  one  section  wise
that's  enough  to
prove  yes  I  missed  where  coherence  comes
from  ah  right  so  that  comes  from  the  the
the  claim  that  the  finitely  presented
objects  form  an  ailan  category  which  has
as  a  corollary  that  for  any  finitely
presented  objects  you  can  build  an
infinite  resolution  where  all  of  the
terms  are  products  of  copies  of
R  you're
welcome  and  to  have  non  qu  separated
presented  object  the  ring  has  to  be  of
Dimension  at  least  two  or  non  quasy
separated  finitely  present  has  to  have
Dimension  at  least  one  at  least  one
yeah  so  if  you  have  a  if  you're  over  a
finite  field  then  the  every  every
finitely  presented  object  is  quasi
separated  but  once  you  move  to  say  the
integers  then  then  as  I  said  zp  mod  Z  is
an  example  of  something  that's  not  quasi
separated  but  is  finitely  presented
also  summary  the  step  of  reducing  to  uh
reduce
from  qu  separated  fin  presented  to  fin
to  discrete  one  uh  is  that  EAS  to  it's
not  L  one  here  like  which  Step  sorry
this  step  here  yeah  you  see  that  there's
no  limb  one  because  the  the  only  the
only  limb  one  that  can  contribute  to
degree  D  plus  one  is  the  limb  one  of  X
D's  and  because  because  on  the  level  of
these  guys  we  know  that  there's  nothing
there's  no  X  for  any  discrete  module  in
degree  D+  one  you  see  that  if  you  have  a
surjective  map  uh  with  then  uh  XDS  into
it  will  also  be  surjective  because  the
obstruction  is  an  XD  plus  one  of  the
kernel  which
vanishes  so  then  the  XDS  is  also  a  mular
system  yeah
okay  that  was  uh  that  took  longer  than  I
anticipated
um  I  had  a  couple  of  other  topics  I
wanted  to  discuss  but  at  this  point  I
probably  have  to
choose
for  so  maybe  I  will
um  U  maybe  I  just  want  to  make  a  point
um  and  illustrate
it
um  so  so
recall  so  if  R  is  now  a  discret
commutative
ring  so  we  we
showed  that
that  solid  uh  or  D  let's  say
RZ  uh  solid
localizes
over
uh
this  valuative  spectrum  of  of  the
Ring
um  this  localization  had  the  property
and  this  is  maybe  why  it's  um  convenient
to  anal  one  reason  why  it's  convenient
to  analyze  that  if  you  on  the  on  this
discret  level  if  you  take  the  unit
object  here  which  is  r  and  then  you
restrict  to  any  any  of  these  basic  quasi
compact  opens  then  you  still  just  get
another  discrete  ring  in  fact  if  it  was
the  rational  open  like  given  by  F1  up  to
FN  over  G  then  the  ring  was  just  R1  over
G  so  you  with  this  localization  you're
kind  of  staying  in  the  world  of  of  your
ambient  ring  being  discreet  but  this  is
by  no  means  uh  there  are  other
are
also  other
ways  to
localize  the  same  category  this  is  by  no
means  the  most  General  possible
localization  it's  the  one  we  discussed
because  it's  the  one  that's  most  closely
related  to  Huber's  Theory  and  our  goal
in  this  class  is  to  explain  well  the
basic  definitions  in  our  Theory  and
their  relations  to  more  classical
theories  of  analytic  geometry  but  I  just
want  to  point  out  very  briefly  that  um
there's  also  so  a  whole  different  avenue
you  can  go  uh
for
um  localizing  this  which  even  more
radically  I  guess  departs  from  Huber's
formalism  so  we  already  saw  a  small
departure  in  that  we  were  allowed  to  do
slightly  more  localizations  of  a  Huber
ring  than  before  because  of  this
difference  between  valuations  that  are
less  than  one  on  topologically  Neil
potent  elements  and  ones  which  are
valuations  which  are  continuous  in  Huber
sense  um  but  kind  of  even  more  drastic
things  are  possible
so
um  so  in
fact  so  if  uh  let's  say  C  subset  Spec
R  is  any  constructible
subset  so  that  that  means  an
intersection  of  a  quasi  compact  open
with  the  with  a  complement  of  a  quasi
compact  open  so  it's  some  locally  closed
subset  of  specr  which  is  sort  of
finitely  presented  and  of  those  uh
finite  Union  of  those  yes  yes  yes  yes  so
the  ba  the  basic  let  me  say  the  basic
objects  which  um  sort  of  form  a  basis
for  the  constructible  topology  would  be
something  like  you  take  D  of  G
uh  uh
intersect
um  the  complement  of  yeah  the  common
zero  set  of  finitely  many
functions
uh
uh  uh  then  you  can  define  an  item  potent
algebra  in  uh  in  d  r  z
solid  uh
namely  uh  to  this  on  the  level  of  this
basic
object
um  D  of  the  complement  of  the  zero  locus
of  G  and  then  intersect  the  zero  locus
of  F1  to  FN  uh  to  this  you  can  assign
the  thing  you  get  when  you  take  R  and
then  you  invert  G  and  then  you  uh
derived  complete  along  F1  up  to
FN  which  uh  this  final  object  only
depends  on  the  the  constructible
subset
um  and  you  know  by  the  way  if  R  is  an
etheron  then  this  derived  completion  is
just  the  usual  completion  but  I  want  to
emphasize  that  I'm  taking  this  derived
completion  this  is  a  discret  object  but
I'm  taking  this  dve  completion  uh  in  uh
solid  ailan  groups  or  deriv  category  of
solid  ailan  groups  so  I'm  so  to  speak
putting  the  inverse  limit  topology  on
this  uh  derived  inverse  limit  here
um  no  such  things  then  you  you  say  that
you  ah  you  view  your  constructive  subset
as  a  union  of  locally  closed  you  also
look  at  the  intersection  so  on  so  for
each  of  them  each  fin  intersection  you
do  this  and  then  you  take  the  what  okay
you  take  the  limit  of  this  type  thing
and  it  won't  be  concentrated  in  hom
they'll  have  some  chological  yeah  so  is
it  the  case  the  intersection  of  two
basic  thing  what  you  get  is  a  TENS  of
product  yes  okay  so  it's  it  is
okay
uh
intersection  oh  yes  yes  that's  right
that's  right  sorry
uh  right  so  maybe  I  shouldn't  actually
say  that  you  can  assign  item  poent
object  to  any  constructible  subset  maybe
I  should  just  say  that  you  can  uh  or  is
it  okay  or  no  well  let  me  just  say  that
you  can  assign  a  item  potent  object  to
any  basic  constructible  subset  so  you
want  to  take  the  kind  of  the  derived
inverse  limit  of  that  important  object
Associated  to  all  basic  thing  inside  the
given  thing  but  then  you  have  a  problem
to  prove  properties  of  this  I  don't  know
if  it  follows  that  it  is  an  algebra
orru  looking  those  subset  May
what  yeah  constructible  locally  closed
is  if  it's  construc  locally  closed  Som
take  R  GMA  of  this  so  it  will  could  have
some  chological  degree  depending  on  the
on  number  of  Alpha  and  C  to  cover  yes
yes  yes  yes  so  that's  but  thank  you
Peter  that's  that's  uh  that's  better
yeah  so  then  the  only  derived  Behavior
comes  from  uh  kind  of  uh  Union  of
principal  opens  thanks  yeah  that's  what
I  should  be  saying  um  uh  okay  so  now  why
are  these  item  potent
so  so  recall  that
um  uh  recall  from  one  of  Peter  Peter's
lecture  that  um  the  solid  tensor
product  theyif
me  uh  item  potent  means  that  yeah  thanks
so  let's  say  that  this  let's  say  that
this  is  a  item  potent  means  that  a
tensor  over  R  uh  solid  derived  a  is  the
same  thing  as
a  um  of  derived
complete  RZ  solid
yeah
um  is  derived
complete  so  to  check  the  item  potency
you're  asking  that  that  that  implies
that  uh  so  you're  asking  whether  a  map
between  derived  complete  objects  by  this
fact  is  an  equivalence  and  that  you  can
check  by  reducing
modulo
um  uh  modulo  the  uh  regular  or  the
sequence  F1  through  FN  that  you're
um  that  you're  looking  at  and  then  it
just  becomes  kind  of  an  obvious  fact
about  a  a  ring  tensor  itself  being
itself  uh  over  ring  tensor  over  itself
being  itself  rather  any  localization  of
a  ring  tensor  over  that  ring  with  the
localization  is  itself
so  you  did  something  like  this  before
yes  in  one  of  Peter's  lectures
he  proved  the  well  at  least  a  special
case  and  and  claimed  it  worked  in
general  that  the  if  you  take  a  solid
tensor  product  of  two  derived  complete
things  um  exactly  in  a  setting  like  this
so  you  have  a  ring  and  you  have  finitely
many
elements  um  or  any  number  of  elements  I
suppose  and  the  drived  solid  tensor
product  of  connective  objects  so  in
homological  degrees  uh  uh  will  still  be
derived  complete  why  because  we  proved
it  or  yeah  we  Pro  well  Peter  gave  a
argument  for  the  special  case  of  like  P
um  and  then  but  the  the  general  case  is
quite  similar  so  he  he  gave  the  the
heart  of  the  argument  of  the  general
case  yeah  you  have  to  write  yeah  you
have  to  you  have  to  do  some  work  you
have  to  do  do  some  analysis  but  it's  not
um  yeah  it's  something  we've  we've
discussed  in  previous
lectures
um  okay  so  now  uh  and
then  so  now  the  claim  that  I  would  like
to  make  is  that  if
C1  up  to  CN  are  locally
closed  uh
constructible  and  if  their
Union  say  set
theoretically  is  all  of  Spec  R
than  uh  these  guys
cover  let's  say
these  the  corresponding  a  A1  these  item
potent
algebras  uh  cover
uh  Dr  RZ
solid  uh  in  what  sense  so  in  the  sense
that
uh  so  if  you  have  m  in  deriv  category  of
RZ  solid  uh  such  that  M  tensor  a  i
equals  Zer  for  all  I  then  m  equal
zero
but  C  yes  thank  you  thank  you  very  much
yeah
yeah  so  uh  and  this  implies  that  you  get
uh  Dr  RZ  solid  is  some  limit  over  so  in
the  first  term  you  have  product  of
product  over  I  of  AI  modules  in  DRZ
solid  and  then  you  have  product  over  I  J
uh  less  than  J  it  doesn't  matter  modules
over  the  tensor  product  AI  tensor  i  j  in
the  same  thing  and  then  so  on  you'll
have  some  finite
check  uh
thing  what  sorry  cover  if  they  love
yeah
I  like  that  they're  they're  they're
giving  it  a  big  hug  that
cover
um  uh  yeah  so  if  they  love  DRZ  solid
uh  okay
anyway
um  and  the  basic  idea  uh  idea  of  the
proof  is  that  it  suffices  to
show  that  R  is  generated  by
uh  AI
modules  for  varying
I
um  but
um  so  so  for  if  I  just  take  the  example
of  uh  Spec
R  is  DF  Union  Z  of  f  uh  then  you  use
then  r  1  over  f  is  okay
because  it  comes  from  here  and  then  the
difference  between  R  and
r1f  uh  is  um  so  generated  in  a  finite
manner  finitary
manner  um  but  this  is  the  union  of  like
R  mod  f  to  the  N  with  some
shifts  um  this  is  actually  uh  the  fiber
SAR
here  uh  and  this  is  actually  an  RF
complete
module  so  it  uh  lives  in  in
ZF  um  so  if  you  take
all  modules  over  RF  inverse  and  all
modules  over  RF  complete  uh  and  then
generate  things  in  just  by  triangulated
category  nonsense  eventually  you're
going  to  hit  R  and  then  tensoring  with
anything  you'll  see  that  you  hit
anything  and  then  using  that  you  can
check  that  you  get  this  kind  of  covering
situation
so
uh  now  let  me
just
um  say  what  this  looks  like  in
examples
so  uh  so  all  of  these  yeah  so  we  have
the  sort  of  the  category  attached  to
each  element  in  our  cover  the  category
attached  to  the  pair  wise  intersections
is  just  modules  over  this  tensor  product
which  will  also  be  item  potent  so  let's
take  a  so  example  uh  let's  say  r  equal
zxy
um  so  let's  take  C1  equal  Z  X  Plus  or  -1
y  uh  so  we've  got  the  whole  Locus  where
uh  X  is  non  zero  and  then  we  can  look  at
the  locus  where  X  is  zero  but  Y  is  also
non
zero  um  and  then  we  can  look  at  the
locus
where  both  X  and  Y  are
zero
that's  the  power
series  uh  oh  sorry  I  should  add  X  and
then  complete  at  long  X  so  that's  z
braet  y  plus  or  minus  one  power  series
X
um  so  then  what  happens  when  you  start
taking  intersections  of  these  guys  the
most  interesting  thing  is  what  pops  out
when  you  take  the  intersection  so  the
algebra  attached  to  C1  intersect  C2
intersect  C3  um  you  can  just  calculate
these  things  using  the  fact  using  this
basic  fact  that
um  uh  uh  solid  tensor  product  of  derived
complete  things  is  derived  complete  um
so  what  do  you  get  so  when  you  do  C1
intersect  C2  you're  tensoring  this  with
this  that's  just  inverting  X  so  then  you
get  uh  z  y  plus  or  minus  one  laurant
series  X  and  then  you  should  tensor  that
over
zxy  uh  in  Z  solid  with  this  third
guy
XY  um  now  both  X  and  Y  are
inverted
um  uh  but  if  you  imagine  not  inverting  X
then  you'd  have  a  power  series  here  and
you  could  use  this  derived  X  complete
trick  to  see  that  you  can  move  this
power  series  onto  the  inside  here  and
you  get  that  this  is  Z  laurant  series  y
power  series  X  with  X  inverted  which  is
just  laurant  series  X
so  you  get  this  kind  of  object  here
where  you're  in  you  you're  iterating  op
so  you  iterate  in  general  when  you
intersect  a  lot  of  these  things  together
you  iterate  completing  and
inverting  and  it  famously  it  matters  in
which  order  you  do
this  um  but  it  all  comes  from  this  sort
of  commutative  situation  nonetheless  in
this  way  of  setting  it  up  here  and  I
want  to  make  the  point  that  the  so  these
things  are  like  kind  of  like  higher
local  fields  that  people  have  studied
and  they  tried  to  study  them  from  their
perspective  of  topological  Rings
topological  fields  and  there  were  just
terrible  problems  Matthew's  not  here
right  now  but  he  even  wrote  a  paper
explaining  that  everything  is  horrible
um  but  uh  if  you  put  them  in  the
condensed  world  then  they're  just
perfectly  well-  behaved  objects  that  you
can  work  with  um  it  arises  from  this
natural  procedure  and  it's  item  potent
even  in  the  DED  sense  over  this  ring  and
um
it's  just  uh  it  just  functions  uh
functions  quite
well  um  so  I  invite  you  to  try  to  draw
pictures  of  these  covers  um  to  get  an
understanding  for  what's  going  on  with
these  kinds  of  constructions  but  um  okay
I  think  I'll  stop
here
yes  for  finite  Type  zra  R
expain  R  solidification  of  the  countable
product  of  C  and  R  is  just  a  count
countable  product  of  yes  and  I  thought
it  also  true  for  something
like  the  ring  of  power  Z  that's  true
which  is  not  finite  time  that's  right
but  it's  also  not  discret  so  so  yeah
it's  detered  by  yeah  so  you  could  also
take  a  finally  generated
ring  uh  and  an  arbitrary  ideal  and  look
at  the  completion  and  then  you  get  the
same  statement  so  the  okay  uh  the  basic
the  compact  projective  generator  is  just
the  product  of  copies  of  the  itic
completion  of
r  i  see  and  that's  a  same  time  if  ring
is  not  fin  at  type  you  also  said  it's
Union  of  five
is  it  easy  to  see  for
this  no  but  Z  double  bracket  so  sorry
that  was  a  statement  about  discret
rings  so  in  general  if  you  maybe  if  you
have  some  uh  ring  that's  complete  with
respect  to  a  finitely  generated  ideal
you'll  get  the  good  answer  if  your  ring
modulo  that  ideal  is  finally  generated
maybe  this  is  a  good  way  to
if  is  the  ring  module  the  ideal  is
finite  type  over  Z  then  then  you'll  get
the  the  naive
answer  I  think  yeah  sounds  reasonable  at
least
yeah  solidification  solid  is  determined
by
underr
yes  so  the  same  formula  doesn't  work  for
Z  bra  what  so  Z  double  bracket  T  is  a
discrete  ring  yeah  as  a
ring  you're  looking  at  Z  double  bracket
t  as  a  discrete  ring  yeah  well  then  it's
just  no  then  no  then  the  doesn't  no  it
doesn't  hold
no  but  if  you  if  you  look  at
Z  what  to  say  if  you  look  at  Z  double
bracket  t  uh  non-discrete
modules  in  Z  double  bracket  T  discreet
uh  solid  Theory  that's  the  same  thing  as
same  as  in  in  uh  in  ZT
solid  Theory  so
um  once  you've  decided  to
yeah  yeah  once  you're  complete  then  it's
enough  to  solidify  uh  stuff  modulo  that
ideally  you're  complete  along  okay  yeah
so  does  that  address  your  your  concern
yeah
okay  so  what  again
you  discrete  ring  and  takech  the
associal  condensed  ring  and  you  consider
solid  relative
to  to  ah  relative  to  the  elements  of  the
Ring  yeah  okay
just  each  element  of  the
ring  and  then
you  so  if  you  consider  this  condensed
ring  with  the  product  topology  you  could
either  solidify  every  element  in  the
underlying  discret  ring  or  you  could
just  solidify  T  and  it's  the
same  in  fact  you  don't  even  need  to
solidify  T  I
mean  you  don't  need  actually  it's  just
it's  just  the  same  as  in  zolid  theory  so
if  you  look  at  a  Z  power  series  T  module
in  zolid  then  it'll  automatically  be
solid  with  respect  to  everything  so  in
fact  this  follows  because  I  mean  we
showed  that  the  topologically  nil  potent
element  ments  will  always  be  solid  um
and
uh
yeah  um  and  then  modular  topical  noo
elements  all  you  have  is  z
which  is  is  there  paper
on  picture  I  mean  paper  of  P  about  this
about
what
this  here  he  gave  he  talked  about  it  in
this  class
yeah  are  you  wondering  if  it's  written
down  somewhere  yes  it  is  uh  Guido
Bosco  wrote  down  at  least  the  case  of  P
like  complete  and  I  think  also  I  think
also  isn't  it  right  Peter
uh
uh  I  don't  remember  the  title  of  guo's
paper  but  I  think  also  Lucas  man  wrote
yeah  Lucas  man  wrote  The  General  case  in
his
thesis  and  the  in
the  I  think  so  I  forget  but  I  forget  the
title  of  the  paper  in  which  it's  contain
but  he  has  some  appendix  on  ptic
functional  analysis  from  the  solid
perspective  the  title  of  the
paper  you  say  I  said  the  in  this  in  this
here  it's  has  some  appendix  to  one  of
his  papers  whose  title  I  can't  remember
and  the  appendix  is  called  something
like  solid  uh  functional  analysis  or
solid  ptic  functional  analysis  something
like  this  which  kind  of  writes  down  the
the  argument  that  Peter  went  through  it
was  before  but
okay
yeah  just  yeah
yeah  very  quick  um  maybe  this  was  a
computation  done  earlier  in  the  course
I'm  just  forgetting  but  like  regarding
the  control  of  the  X  groups  like  The
Vanishing  of  of  them  like  so  we  had  the
proposition  for  finite  type  so  if  I  just
like  for  example  like  zp  zp  the  ptic
integers  yeah  was  there
computation  oh  so  you're  wondering  about
these  homological  Dimension  results  for
zp  yeah  so  there  you  get  uh  yeah  now  I
have  to  remember  I
think  um  yeah  there  you  should  get  also
it  should  just  only  be  X  ones  uh  yeah
that  that's  actually  even  that's  easier
to  show  than  in  the  case  of  of  solid  Z
because  zp  is  compact  so  you  can  there
you  can  directly  see  that  all  of  the
finitely  presented  objects  they're  also
they're  they're  it's  quasy  separated  and
it's
the  they're  also  all  those  objects  are
quasy  separated  and  they're  just  like
yeah  inverse  limits  of  finite  zp  modules
uh  countable  inverse  limits  of  finite  zp
modules  and  that  makes  the  whole
analysis  much
easier  so  to  summarize  so  for  let  us  say
R  is  a  field  yes
then  the  fin  presented  are  qu  separate
in  fact  they  are  all  given  quite  simply
by  you  just  have  the  either  infinite  or
final  product  of  the  cas  so  that's  okay
for  our
domain  except  the  fact  that  you  can  have
a  the  parar  group  is  maybe  not  trivial
but  still  it's  a  product  of  uh  you  you
have  a  resolution  you  have  product
copies  of  R  and  then  the  Kel  is
something  like  product  of  maybe  ah
actually  you  can  use  the  the  AR  Swindle
to  get  rid  of  this  anyway  so  you  can
it's  always  this
this  okay  so  it  is  a  resolution  of
length  to  like  like  before  because  we
don't  have  to  worry  okay  and  so  we  get
and  then  in  dimension  at  least  two  you
do  your  the  pro  the  uh  that  we  you
discussed  yes  okay  that's  the  yeah  and
then  there's  if  you  wanted  to  move  to
non-discrete  rings  then  you'd  have  to  do
more  analysis  and  so  on  and
yeah
yep  okay  thanks
everyone
\end{unfinished}