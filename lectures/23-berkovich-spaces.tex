% !TeX root = ../AnalyticStacks.tex

\section{\ufs Berkovich spaces II (Clausen)}

\url{https://www.youtube.com/watch?v=vXZC3WzKZgo&list=PLx5f8IelFRgGmu6gmL-Kf_Rl_6Mm7juZO}
\renewcommand{\yt}[2]{\href{https://www.youtube.com/watch?v=vXZC3WzKZgo&list=PLx5f8IelFRgGmu6gmL-Kf_Rl_6Mm7juZO&t=#1}{#2}}
\vspace{1em}

\begin{unfinished}{0:00}
This lecture is about Berkovich spaces, which is a topic I began discussing last time.

Let me remind you of the classical setup. You have a Banach ring, which is a ring equipped with a norm that is submultiplicative and satisfies the triangle inequality. The norm of 1 is equal to 0 or 1, unless $R = \Z$. And $R$ is complete.

%todo insert earlier Banach ring latex def here.

% 1:45
Berkovich assigns to this a compact Hausdorff space, $\BerkSpec(R, \norm{\cdot}) \subseteq \prod_{f \in R} [ 0, \norm{f} ]$, which is a subset over the product over all elements in R of the closed interval from 0 to the norm of f. A point in this space is denoted as X, and it is literally an evaluation map from R to the non-negative real numbers, which is now multiplicative and satisfies the triangle inequality, with 1 being 1.

\begin{remark}
$\R$ is not complete with respect to this $\norm{\cdot}_X$. In fact, there could be many elements of norm zero. However, if you complete $R$ with respect to any norm X, you get a complete valued field $\mathcal{K}(X)^{\text{hat}}$, which is sort of the residue field in the sense of Berkovich theory.
% TODO: Latex notation of K residue field
There are basically three cases:
\begin{enumerate}
\item archimedian $\iff$ $(\R, \norm{\cdot}^{\alpha})$ or $(\C, \norm{\cdot}^{\alpha)$, $\alpha \in [0, 1]$
\item non-archimedian but discrete, then you have the trivial norm $\norm{\cdot}_0 = \text{trivial norm}$
\item non-archimedian non-discrete, $\norm{\cdot}_{normalize}^{\alpha}$
\end{enumerate}
In the archimedian case, there is no real variety in complete normed fields, as it is either $\R$ or $\C$. In the non-archimedian discrete case,. In the non-archimedian non-discrete case, the value could come from a discrete valuation, and there is an ambiguity in the normalization.
\end{remark}

% Some text about the discussion seems to be missing here

% 6:36
Now, the goal is to promote $\BerkSpec(R,{\norm{\cdot}}$ to an analytic stack, using the stack of norms, $\mathcal{N}$. A map from an analytic stack $\mathcal{X}$ to $\mathcal{N}$ is the same as a certain map from $\P^1_{\mathcal{X}}$ to $[0, \infty)$, satisfying multiplicativity.

% 7:50
Last time, we investigated the geometry of $\mathcal{N}$, and saw that it lies over $\ExtBerkSpec(\Z)$,the extended Berkovich spectrum of $\Z$. The usual triangle inequality is what made the difference, the reason you couldn't take an arbitary power $\alpha$ here. If you take an arbitrary power $\alpha$, that basically behaves like a norm, but it is really just a quasi-norm. So you have to put some constants in front of here.

\textbf{Question:} Plus, you have also the kind of the limit at, 
% 8:42
\textbf{Answer:} Yes, there's also the limit at $\infty$. 
% 8:50
In this stack of norms, there's no triangle inequality imposed. It turns out that what you get, is this non-strict triangle inequality, where all powers of the usual archimedean norm are allowed. Also, there's a some kind of very strange limit point where your norm takes infinite values on natural numbers.
% 9:12
Then you have the archimedean ones, so $2$-adic absolute values which end here in a point which has both characteristic $0$ and characteristic $p$ Behavior, but where the norm of two equals zero and then for the other primes $p$.

% todo, make drawing
9:34
We also kind of saw what on the interior of these line segments, you were getting the the gaseous $\R$ theory $\R^{\gas}}, on the interior here you're getting the gaseous 2-adic numbers, and then as you move along the normed, the norm is changing, but the analytic ring is not. In $\Q_3$ and so on, and then here you have things living, you have $\F_2$ living for example, you have things living in characteristic 2 there, living in characteristic 3 here, and here you have things living in characteristic 0.

% 10:09
In some sense, you can imagine that the points of this stack correspond to something like these complete valued fields, or the minimal choices of complete valued fields like you have $\R$, $\Q_p$, you have discrete $\F_P$, you have discrete $\Q$.

% 10:32
So, that's kind of a substitute for the notion of multiplicative valuation, but then we still have to input our Banach ring $R$ into the construction in order to get something non-trivial. Here's the definition, and I don't know what good notation is, I'll write it like this.

% Definition
% todo macro for 'Nu', stack of norms?
$\Spec^\Berk (R, \norm{\cdot}) \subseteq \Nu \cross \Spec(R, \text{triv})$ 

This will be an analytic stack, which will be a substack subset of stack of norms cross, and then some affine analytic stack which is just we take $R$ with the trivial analytic ring structure. What I mean by this, is that you take an $R$, $R$ is a Banach ring, so it has a topology, but you can also view it as a light condensed ring. 

\textbf{Question:} So using the topology, you consider those condensed set?
\textbf{Answer:} Yes.

And then trivial analytic ring structure.

\textbf{Question:} that is, all modules are allowed?
% 11:41
\textbf{Answer:} Yes.


It's the full condensed derived category of this condensed ring.
In this map, so it's a condition that you can check in practice, and so is it the case that you have. Since $R$ is cool, you can think of it as a limit of C sub Rings, which are account many elements. Yes, so you can probably reduce to, no, but it maybe, it's not Dimension. Well, I'm not sure, but no, no, but you can look at as a CO liit over finally generated sub rings, and then, and that's always embedded in a finite I mean, okay, okay.

Okay, yes, so that's one way. So, there is indeed a canonical way to write this as an inverse limit of finite dimensional metrizable spaces, but let's not do it. Let's just work with this setup here. And is it true that you have inverse limits in analytic stat? Yes, you have, you have all limits in analytic stacks. Ah, okay, so you can syn of it as a limit over the stack, is a limit over the, well, this isn't a stack now. This is just, I'm just viewing this as, when you have those nice sings, you can construct.

Okay, so it's not, no, but then you can take for the nice subing, you can take the stock Associated to m in those limit, you get something which is, yes, you get something. Then you get something, the only problem I have with that, it's not a real problem, but just is that if you started with something which happened to be already be finite dimensional and metrizable, then you'd be non-trivially writing it as an inverse limit of other finite dimensional metrizable things, and I mean, you know, so let me just not, let me just not get into it.

% 18:32
\textbf{Question:} When $R$ is Tate \note{not sure}, then we we also show that we can localize over the $\Spa$, and then there's a map from the $\Spa$ to the. I mean, that's right, that's right. 
% 18:45
There's a commutative diagram that you can write down, where here you have a map from the Huber space mapping to this, and then you have some this thing, and then you have the the solid guy here mapping to that. And this last Arrow, no, when I 

% 19:05
\textbf{Answer:} I'll discuss in more detail what this looks like in the Tate case, and then you'll see.

In particular, I just want to highlight that we get a structure sheaf on this topological space, and even a structure sheaf of $\infty$-categories, so a theory of quasi-coherent sheaves on the usual Berkovich space. 
% 19:42
I'll explain in the Tate case how it's pretty easy to calculate this structure sheaf and see what it's doing in the case of rings $R$ like $\Z$. Well, $\Z$, you can kind of do it by hand, but it would be interesting to compare. 
% 20:02
So I'll explain how you compare to the cases Berkovich discussed, but it would be interesting to compare to Pooneau \citeme{} who kind of more or less by hand described structure sheaves in certain cases over $\Z$. So this is a different different approach where you define something which is a priori structure sheaf, and then you have to calculate it, which can be done in principle, but you know, takes a while. In Poineau's case, he explicitly assigns the value, and then he has to maybe prove some, prove some descent results. So here we automatically get some sort of infinity descent, but then you have to calculate the value.

\textbf{Question:} Okay, to like I know this and should be easy to say what I on a disc the structure sheaf. \note{can't hear this well}

\textbf{Answer:} It does reduce to seeing what goes on on a disc, but to see what goes on on a disc, I'll explain what you need to do to do these calculations, and you'll see that it is like...

\textbf{Question:} But probably there are like in the case of Huber rings, there are probably some derived phenomena.
% 21:27
\textbf{Answer:} Yes.
\textbf{Question:} Because when you want to look at things like the algebra of functions on close or open disc, anyway, you you quotient something by certain, I mean, you have non closed ideals \note{not sure}, so probably you need to work in some derived sense to get the right, to get what you get from your function theory, you should probably have derived rings which are complete. 
% 21:54
\textbf{Answer:} Yes. We do. okay, I mean, that's..

I do believe that in cases like what Poineau considers, where it's you're starting with a discrete ring, say $\Z$, that the calculations are quite feasible.
% 22:15
But if you start with a more arbitrary Banach ring, it is maybe not so obvious how to do the calculations.

So, where are we?
% 22:30
So I want to explain why this is true. 

Proof
Let's give charts.
Recall that we had charts for the norms.
Functor of points

So, if we pull back along this, then we look at the Spec Berk R, Norm, Norm, and then $\Spec \Z_q$ plus or minus 1, and then we get this Universal disc or some Universal annulus over here. Let me just call it y.

For this $Y$ thing, we've already got the norm, and we've kind of artificially adjoined an element of Norm 1/2. But the only thing we need to ensure to go from this to this is we need to give the second map, we need to give the map to Spec of R with a trivial thing, and we need to see that they agree, and we need to enforce the condition that this Norm condition that NF lands inside that part there.

So, to get $y$, you just take $\Spec$ of $\Z_q$ hat plus or minus 1 gas cross $\Spec R$, and then pass to the closed subsets given by the idempotent algebra obtained by taking the norm F inverse of these closed subsets.

This Berkovich spectrum here has a fairly simple cover by an affine, and to calculate this afine, the most difficult part for a completely general $R$ is already in the first step, that is making this product and calculating what analytic ring this is, and in particular, calculating what the underlying condensed ring is, because what this tensor product involves is you have to calculate the gaseous localization of $R$.

% 27:30
Once you have that, I'm going to explain that calculating these idempotent algebras is not that hard. So, once you know the ring you have here, then you're passing to certain idempotent algebras over it, and that's not the hard part of the calculation.

% 27:40
In particular, if you're in a situation where you can do this calculation, then it's quite feasible to calculate $Y$, which is giving a presentation for this stack. We also saw that the descent for this cover is quite simple. It happens at the zero stage, so the structure sheaf here will just be a retract of the structure sheaf there, so the ring of functions on this $Y$.

% 28:13
% TODO add Q&A session here

% 29:27
The map to the Berkovich spectrum is interesting. On the $\Spec^\Berk$, you have the universal norm, let's call it $N$, and then for every element in $R$, you have, by $\phi$ \note{todo}, a map to the structure sheaf here. Then, we get a map from $\Spec^{\Berk} (R, \norm{\cdot} )$ to the product over all f in $R$ of the zero norm of f.
% 30:28
But since we enforce that every element of $R$, the norm $N$ of that element is bounded by the prescribed norm on our Banach ring $R$, that in particular implies that $\norm{2} \leq 2$, which means that, by last time, the triangle inequality holds for $N$, and that means that this map lands inside the Berkovich spectrum.

% 31:33
% TODO add Q&A session here

This construction, we could also keep some of the which does not satisfy the triangle inequality, right? 
% 32:11
It's natural from the perspective of this stack of norms that we've been discussing, as we've seen to relax the triangle inequality, and you can do that. I mean, the formalism is quite general.
% 32:26
The reason I stuck to the classical thing is just because it's the classical thing.
\textbf{Question:} you have for the triangle inequality. You assume that 
\textbf{Answer:}  you could require that there exists a constant such that, you know, this, that's that's one thing you could do. 
% 32:42
\textbf{Question:}Okay, but then if you want to, so this still defines a uniform structure, so you can say it's complete, and, and it is not equivalent by slightly changing things to an actual norm if you have this. 

\textbf{Answer:} Well, I don't know 
\textbf{Question:} because for fields, just by power H, is that true? This wouldn't be, I think this is something, may be, maybe it's true, but you could also conceivably allow the norm to take infinite values and try to build that into things as well. I just wanted to stick with the classical thing.

\textbf{Question:} So there is a theorem about the scaling. No, if you have a norm with the kind of, I don't know how it's called, with the constant. So, the theorem that for fields, I think you only get the Pu classification with an $\alpha$. 
\textbf{Answer:} I think you're right. 
\textbf{Question:} But for rings, you don't know that because it's it's delicate because if Y is small, this doesn't imply this condition doesn't imply that the norm of X and of X plus Clos.
% 34:09
\textbf{Answer:} Yes, I agree, it could be subtle for a general ring. I don't want to make any claims. I actually want to, stick to the classical setting.

% 34:24
Okay, so this was me explaining the general case, but there is, and in the general case, why, sorry, this, this, this thing, despite the notation with the spec, this, um, is not going to be affine. So, for general $(R, \norm{cdot}), $\Spec^\Berk (R, \norm{\cdot})$ is not affine, so it's not the spec of an analytic ring. Um, so for example, well, if we look at $\Spec^\Berk (\Z, \norm{\cdot})$, and the usual Archimedean absolute value, which is the maximal, the every every norm you could put on Z will have to be less than or equal to this one, so this is kind of the the choice that gives you the biggest possible Berkovich spectrum. This is just this locus where, two, absolute value of two, is less than or equal to two inside this stack of norms, and it really is a stack as you can see at the at the points that live at the boundaries.

So let me make an assumption star, so there exists, let's say $\pi \in R$, such that $\norm{\pi} < 1$ , $\pi$ is a unit in the ring, and I want it to be that it strictly multi-, like the norm strictly multiplies when you multiply by the norm is multiplicative with respect to multiplying by pi. $\norm{\pi f} = \norm{pi} \cdot \norm{f}$

% 36:31
So this condition is obviously not satisfied here, but it's satisfied quite broadly, so, example

\begin{example}[\yt{36m40s}{Nondiscrete valued field}]
% three examples
\begin{itemize}
\item Any nondiscrete valued field, admits such a norm, sometimes they say nontrivially valued field. 
\item 
\item Any Tate-Huber ring has a norm defining the topology satisfying * $\pi$ pseudo-uniformizer % Added at 40:48
\end{itemize}

\end{example}
So you there you have multiplicativity for all elements, and then if it's not discretely valued, then there's something with norm between 0 and 1, and it that'll be a unit, and of course there is a slight ambiguity in valued field because sometimes it refers to absolute value, sometimes to crude valuations, it could be higher rank, and then

$
\phi R \to R^{'}
\norm{\phi(f)}_{R^{'}} \leq \norm{f}_{R} \forall f \in R
$

Is $\leq$ say a constant, nor effect this. Define still upap on ver spaces. I suppose that your definition does not depend. That is, if let's say you have two Noes which are equivalent in this sense by constants, then all of this will be the same for the two Noes. I suppose, but well, is to check something not quite. So I mean, let me say what I can say, and then, I just, okay. But in, under this assumption, you mean or in general, I think I agree with that. So that's why I want to postpone the discussion a bit. 

% todo: add missing text
% 39:07
submultiplicative property of the norm

% 39:11
\textbf{Question:} So actually, in the Berkovich theory, \note{todo} it is natural to consider a homomorphism of .
%39:25
I suppose that your definition does not understand, if you have two norms that are equivalent in this sense by constant, then all of this will be the same

% 39:40
\textbf{Answer:} Not quite. Let me say what I can say 

I want to postpone the discussion a bit.

% 40:11
Okay, so in particular, I mean, the classical settings in Berkovich geometry, you work over some fixed field which is often non-discrete, and you're working with Banach algebras over that field, and they will certainly satisfy this condition $\star$. But if you're working over a discrete ring, you won't have this condition $\star$. 
% 40:46
Also, any Tate-Huber ring has a norm defining the topology, satisfying $\star$ with $\pi$ a pseudo-uniformizer. 
% 41:06
So also some mixed characteristic examples exist.

% 41:15
Right, so, where am I? So then, claim: if $\star$ holds, then $\operatorname{Spec}_{\mathrm{Berkovich}} (R, \|\cdot\|)$ is affine and corresponds to an analytic ring structure on the condensed ring $R$. Moreover, this analytic ring structure only depends on the condensed ring $R$, not on the norm satisfying $\star$.

I think you're right, yes. So similarly for a morphism, if it is only with a constant, you could still apply the same thing. I think you're right, $\mathcal{O}_{\mathcal{E}/\mathcal{S}}$, so this, because in some, I remember that in some text, I don't know if the book of Berkovich, in some place they consider such things, which is more, apparently more natural, because I don't know the why, but it's probably, you can't.

% 43:22
\textbf{Answer:} I think you're right that because our norms are by definition multiplicative, I mean, these geometric norms that you can argue exactly as you suggested.

% 43:32
That's a good point, thanks. Okay, right, so, what doesn't depend on this thing, you have this universal norm $N$, so the universal norm $N$ on $\Spec^{\mathrm{Berkovich}} (R, \norm{\cdot})$ does depend on the norm you choose on $R$, but any two choices are equivalent under some map $\alpha$ from $\operatorname{Spec}_{\mathrm{Berkovich}} R$ to the positive reals, so Norm passes to Norm to the $\alpha$. For some map $\alpha$ from $\operatorname{Spec}_{\mathrm{Berkovich}} R$ to the positive reals, so there's a scaling action, I'm referring to the fact that there's a scaling action which sends a norm and a continuous function $\alpha$ to the norm you take the norm and you compose with the $\alpha$ map on exponentiation map on on zero Infinity exponentiation by $\alpha$ on Zero Infinity.

So, the proof... I'm going to explain how to produce this analytic ring structure on the Condensed ring R. so take $\pi$, as in star then we get a map from $\Z_q$ to the Condens spring R which sends $q$ to $\pi$. Um, but $\pi$ is of norm less than one, which implies that it's topologically nilpotent - its sequence of powers tend to zero. That implies that it factors through this ring here, and it's also a unit, by assumption. So, we get a factoring through this ring here.

Then we need to check, or we want to check, that $R$ is gaseous. Recall that this gaseous theory was a non-trivial analytic ring structure which was produced by taking, by realizing that the category as a full subcategory of over this ring, and then there's this completion procedure which changes the underlying ring to this gaseous thing, but the category of modules was just described at this level.

And this is something very straightforward, because the definition of gaseous was that some map from $\P$ to $\P$, $\P$ being the universal null sequence, so to speak, namely $1 - t * q$, should be an isomorphism on maps to $R$. But when you map out to R from this null sequence space to your Banach space, um, you're just getting the space of null sequences in the Banach space, so that's equivalent to saying that if you look at the space of null sequences in R, and then you have some $1 - q * shift$, this should be an isomorphism of condensed,  of Banach alien groups, say, um. But this is, but it's easy to see what the inverse is supposed to be, and to write it down, you need to just, you need that if, sort of $F_n$ is a null sequence, then you can sum, $F_n \pi^n$ and still get an element in $R$, um, and, the condition on Pi and the usual triangle inequality stuff, lets you write this down. It's just the limit of the Cauchy sequence, um, so it's quite straightforward to check that R is gaseous, um, and then we can just take the induced analytic ring structure.

So, what? Oh, I said I meant to, I keep saying liquid instead of gaseous, it is gaseous. Well anyway, this, take induced analytic ring structure, from $\Z_q \pm 1$ gaseous.

Now, recall that on, on the spec of $\Z_q$ hat plus or minus one gaseous, we have a universal norm, with the norm of Pi strictly equal to this. Okay, now maybe let me say R is not the zero ring. It's the zero ring, I leave the claim as an exercise. So then, if it's not the zero ring, then this Pi will have to have norm bigger than zero and less than one, um, and then on this, we have the universal norm where the norm of Pi lands inside this singleton subspace, maybe I'll write it like that to remind you that this is kind of a subset and not a value.

Then we can by functoriality of norms, but because norms pull back, we get a, get a norm on what is Pi, what? Ah, $q$, $\pi$ is $q$. Pi is our, we're fixing one of these guys, which exists by hypothesis.

The $q$, thank you. I'm sorry, yes, thank you.

So we get a norm on $R$ with an induced analytic ring structure, and the claim, so this normed analytic ring,
The Singleton value pi and it's easy to see just by tracing through the construction that this is universal with respect to that or with respect to those that structure. So what's the difference with the thing we're trying to compare to? It's that in instead of having a condition on just the norm of Pi, which we have here, we have a condition on the norm of every element. So we need that this condition on a norm is equivalent to the condition that Norm of f is contained in zero f for all F and R. So we need this equivalence.

One direction is quite easy. If you have this, then you apply it to Pi and to Pi inverse and you deduce that. The key is to see that just telling you what the norm of Pi is, then I know I've constrained the norm of every element in our offering. For that direction and because it's good to know, we will calculate the Universal norm.

More precisely, what is a normed analytic ring structure? Recall that it amounts to specifying some item potent algebras over $P_1$. So we'll calculate the algebras. For all less than c, a norm in our Sense on an analytic ring is implicitly just telling you what the overon convergent functions are on a disc of arbitrary radius centered around the origin. My claim is it's going to be the usual thing from Berkovich theory, so this is equal to filtered co limit of let's say radius bigger than C of you take the, I'll explain what this is afterwards, but kind of you can make a universal Banach ring where the norm is less than or equal to $R$.
% 59:459
\textbf{Question:} Is this local or global now? I don't know if I have fin time, how much time you need. Well, you take the time you need and when it's precise, ask again.

\textbf{Question:} Okay, so this is like the formal series that when you replace each coefficient by the absolute value and replace T by the ab by $R$, you it converges, the sum is is finite. 
% 1:00:46
\textbf{Answer:} So this is the set of well, it's just the coefficients but let's say $f_n T^{n}$ such that some absolute value $f_n$. 
% 1:01:04
Okay, so this is the norm on \note{todo}. I was also making claims last lecture about calculations of the what these overconvergent functions were in various cases. So I'd like to explain how to make these calculations and it turns out there's a trick where you really don't have to do anything, it's just kind of purely formal.
% 1:01:50
There's a trick to calculating
 
\textbf{Question:} Of course this is in the condensed, now you have to view this as a condensed, 

\textbf{Answer:} This is Banach so it's condensed, and then this filtered colimit is taking place in the condensed category. So, what is it that we're calculating here actually? We're trying to calculate we have the universal thing over this gaseous base, which we more or less wrote down, and then we have to tensor it with the gaseous tensor product, so over this analytic ring with $R$ where here $q$ goes to $\pi$ and we have to, I mean actually you know a priori it's a derived tensor product but this is the kind of thing we need to do and if we're being too naive about it, it can look kind of tricky because naively what you'd do is you'd write this as as we've explained is some filtered co limit over copies of P, so that's kind of over convergence, and then you'd take you'd first calculate P tensor $R$ and then you'd pass to the filtered co limit but actually it's not so easy to unwind what $P \otimes R$ is, in particular it's not so easy to see that it would be concentrated in degree zero, so let's use a trick.

% 1:03:22
Let's not use this approach. 

\textbf{Question:} \note{Can't hear the question}
\textbf{Answer:} That was how we produced this thing in the universal case. Recall the idea was that the this $P$ was some version of functions on the open unit disc and then when you have this $q$ and maybe all of its fractional powers you could scale that open unit disc and get some version of functions on an arbitrary disc and it wasn't the correct one, but when you make it over convergent it doesn't matter, it'll be by some kind of sandwiching argument? 

% 1:03:55
Alright, so the trick to calculate is some general category theory fact.
So the lemma is so. If $(\mathcal{C}, \otimes)$ is symmetric monoidal and's say infinity category, it's not too relevant, but then, so if you have a tower, so $X_3 X_2 \to X_1$, in $C$, where each map is \underline{trace class}. So, so $X \to Y$ is trace class means it comes from a map X, so from the unit to x dual tensor y or X dual, I'm not assuming $X$ is dualizable, this is just the internal H from $X$ to one, uh, sorry, yes, us closed, thank you, m. There's probably a way to, well, never mind, um.

Where are we, then, for all Y in C we can calculate the co-limit over N of X and dual tensor y. So we pass to the, we have a tower here, we pass to the Dual thing, which gives a sequence, and we take the co-limit over that sequence. Then this is the same thing as co-limit over N of the internal H from xn to y.

So, this is Elementary, um. I'll leave it, I'll leave it just like that without giving the proof. You just have two systems, and you make two in systems, and you make maps backwards that go up a step using the Trace class hypothesis, okay? So, again, you, how do you know there is a, the limit makes sense?

Okay, well, this, this is actually an equality of IND objects, so I mean, it's an equality of end objects. Does it, doesn't it, doesn't matter, okay? As an in object, and you have to know what is the end for, this is end for, for C in in C, which makes sense is an Infinity.

In particular, if C has co-limits, you can remove the quotation marks. If C has co-limits, and tensor product commutes with co-limits, which is the case in our examples, then you can remove the, I mean, you can.

What we're going to do is we're going to recognize, so if you take y equals the unit, then, you know, and, that this object here, um, coim of the internal H. Ask a technical question, yes, about the definition, it's nice. What do you mean by it comes from a map? It's that the, is given by this, ah, right, so if you're given a map like this, then you can tensor it with X, you get a map from X to X tensor x dual tensor Y, and X tensor x dual has an evaluation map to the unit, so you then get a map from X to y.

So we're going to, we'll, so we, will recognize, C as a co-limit over now P, du, and apply this with Trace class transition Maps. So once we do that, then we reduce to, reduce to some, to looking at null sequences in R again, and then just some filtered co-limit of some space of null sequences in R, and you can actually modify this to the thing where you require these to form a null sequence, and they wouldn't be the same at each term, but it's quite easy to be, they're the same when you take the filtered colum again, any two versions of the unit dis are kind of the same after you make them overon convergent, and then the calc, the calculation is very easy, once you once you do this, and for this, uh, uh, for that, we can use S Duality on P1, it happens to be over the liquid base, but it, I mean, the gaseous base, but it doesn't much matter, um, so St Duality on P1, this, if you, so, and the, and the six funter formalism, so, um, sidity on P1
The Dual on the other side, but we can do it with a trick more or less because, so this gives also, so this, this over con, this by the over convergence, you get that the the single Dual of the end object, sorry, we can then we can write this, this is a pro object, we can, we get, we get, we can get this, we can view this over convergent thing as an end object, and it's dual will be a pro object, and it will be the pro object given by this thing where you increase the radius as well. But then now we can view that Pro object as an inverse limit of the over convergent things with the the non-strict inequalities over there, and then, and then use The Duality result in this direction on each of those.

There's some trick, trick with over convergence. This implies that if you take the the Dual of this, so if you view this as a pro object, because it's an inverse limit of the things where we have a greater than, then the Dual of that Pro object is the end object we're interested in. And that gives a, that gives an expression exactly like this, that this guy is a co-limit of du of P's. We still need the trace class claim, but I claim that that also holds for soft reasons.

There's a general topology fact that if X is a topological space and Z and Z' are closed subsets such that there exists an open U which lies in between them, so two closed subsets which are separated by an open subset, then the map from the push forward of the constant sheaf, the Restriction map, is Trace class in the derived category of sheaves on X, with values in D of Z, which is not the category of X in general because it's right.

So then on the level of just this closed interval from 0 to plus infinity, then any of these transition maps will be Trace class for this reason, and then you can pull back, that's a symmetric monoidal functor, you get that Trace class, you get a trace class map in the derived category of P1, but it lives on A1, and then there's another trick to see that its image under the forgetful functor to the base is also Trace class. So there's pull back to A1, use another trick, and the conclusion is that the Restriction map, say from any of these over convergent guys, is Trace class. And then also for pres, then again by sandwiching different versions of the discs and changing the Radia, and using the trace class Maps as a two-sided ideal in all maps, then you get the presentation like this, which let you calculate, do what by hand? You want to write over functions on this disc, I mean the declanse is to use pH pieces, you can just system and see that transition has really dis, that there's, I believe you can do that. Certainly, I remember doing that in the complex case, and I assume it works over the gashes base too, but I thought it was nice to be able to do it without doing any calculations.

Okay, you still need to know that the way you presented is the way you scene presented. Yes, that's true, that's true. The remark was I wasn't being very careful here about writing what the filtered systems are and all this. You have to show that it is there.

Okay, so what was a bit of a digression. So what were we doing? We were calculating the normed analytic ring structure, and the conclusion was that the norm on Spec R gashes is given by usual over convergent functions on dis over r. So then you need to, so to see what did, what were we trying to show? We were trying to show that for this Norm here, this Universal Norm that we produced by fixing the norm of Pi, that automatically the norm of every other element is correctly bounded. So to show Norm of f contained in zero f for all f, it suffices to show it translates into, oh no, now we have bad notation, it's not clear a priority that it is finite, sorry, you have to know even the fineness is not a statement, that's correct. So we have to show that if you take this and you mod out by T minus F, you just get R, this. This is Elementary, and indeed this is Elementary, so the map giving this is of course setting T equals to F, and it's quite Elementary to see that you get the correct short exact sequence of Banach spaces, so that the kernel of this map is.

T minus. $f$ is of the. That mere existence of the $m$ is mere existence of the map. Enough. Oh, that's a really good point. That's a really good point, Peter. Thanks. So what Peter was saying?

So what Peter was saying is that we know a priori that this thing is an idempotent algebra over $A_1$, so over the polynomial ring on one generator. Therefore, when you base change it along here, then you get an idempotent algebra over $R[T]/(T-F)$, which is $R$. So this thing is an idempotent algebra over $R$, and so is $R$ itself. And if you want to show that two idempotent algebras over $R$ are equal, it's enough to just produce an algebra map between them. Thanks a lot, that indeed makes it very. No, just, we already have the unit map. We have a map in both directions indeed, but we already have the unit map.

Let's see. Because when you do it analytically with this, instead of over conversion, just conversion on the closed disc like before, and of course you have a map when $f$ is less than or equal to $C$, you get them up. But to prove division, you still get to some convergence is not okay in the Archimedean case, and so it is, but it's not idempotent. It's not idempotent in that case if you don't do the overconvergent one, if you just do the one without the overconvergence. It's not going to be idempotent, as the argument would not work exactly. So somehow the proof of idempotent is you must have already done work similar to showing that.

In the non-Archimedean case, the thing with rigid algebras does work, and in this case, rigid algebras are important, or not, they are, yes. When you do it with the non-Archimedean, I mean, the everything non-Archimedean, then, why is it? I mean, we basically proved it when we discussed the solid theory, but maybe, if everything is not so, there you can use the solid, then it is, right? So, in any case, the map exists because you can evaluate at Tate's $F$, and as Peter points out, that's enough. Thanks, Peter.

So, the first part of the claim was that this thing, the Berkovich spectrum, this analytic stack which I'm calling the Berkovich spectrum, is pine when you have that assumption. The next claim was that the analytic ring structure is independent of the choice of the norm. Proof continued: we need that $\text{Spec}(R_{\text{gas}})$ is independent of the norm and $\pi$. If you have two topologically isomorphic rings, it depends on the condensation, exactly. Without the condensation, you could have one norm with this one $\pi$, another norm is $\pi'$, exactly.

Let me give the independent description. The claim is that $R_{\text{gas}}$ is the initial analytic ring with a map from $R_{\text{triv}}$ to $R_{\text{gas}}$ such that for all topologically nilpotent units $\pi$ in $R$, the map $\Z_q^\wedge \oplus \pm 1 \to R$ given by $q \mapsto \pi$ factors through $R_{\text{gas}}$. This condition only depends on the topological ring $R$.

One can compare to Uber and get that it's the same. It is the topic of some, you can, but in the Aredian case of course there is this. In the Berkovich Spectrum, there is this condition that I mean, I think if you take the, you could also like try to make a modification of the Berkovich Spectrum, only thinking of $R$ is a topological ring where you ask for these seminorms to be continuous, and then I think if you take that space and then you mod out by this exponentiation action by positive real numbers, then that will be the same as the Berkovich Spectrum for any fixed Norm, satisfying condition star.

No, because when Aredian, the non-Aredian things, yes, you don't because in the $B$ space you don't identify no to its power, to its powers, so I don't see how you, no, but the identification is maybe not so obvious. It's kind of, well, I mean, I'm not, I'm not sure, but so you, because you, you, you want to claim that your, your, your Bëlovy spectrum of course it maps to the space, the Bel space as we said, yes, and but you don't claim here that the map is, because if you want to, claim the map is the same, you have to compare the average spaces, and this looks like a little bit tricky, at least in the away from the non-Archimedean case, we can understand it anywhere, I'm not sure.

I don't know. Let me give the proof of this claim, which is kind of giving an intrinsic description of this Tate's analytic ring structure. So, note that if $\pi$ is topologically nilpotent, then there exists an $n$ such that after passing to some power, you have small Norm, in particular, you have Norm less than one. And let me note that this condition here, this is invariant under replacing $\pi$ by any power, this is actually a remark that Peter made at some point, some point early on, so we can assume that $\pi$ is Norm less than one.

But then, but then, for the universal Norm we built over our gaseous, sorry, well, sorry, I need to fix. Okay, so my claim is going to be, so certainly this $R_\mathrm{gas}$ that we built, we built it so that it satisfies a weaker version of this property, where you only demand it for a fixed $\pi$ satisfying this condition here, and what we need to show is that, let's say, our gas was built to satisfy star just for some fixed $\pi_0$, and now we have to show that it's satisfied for all choices of $\pi$.

But then, so can assume $0 < \pi < 1$, and then that implies that the norm of $\pi$ is in this interval from 0 to 1, which we already showed implies that $\pi$ is gaseous, just from the axioms of a normed analytic ring.

Okay, what is $\pi_{\Z_R}$? For real $\Z$, what is it? It doesn't make sense, because I've only defined this when $R$ is a Banach ring satisfying this condition star. Sorry, the other condition star about the existence of a $\pi$ Norm between zero and one, etc. $\pi_z$, no, fixed $\pi_0$ is a, it's a, the way I built this was I took my Norm and I took a fixed $\pi_0$ satisfying condition star, and then I built my analytic ring and I built my Norm over it, and now I want to check that that thing satisfies this universal property, which means that so it was universally built to satisfy that, just for a fixed one, but then, and to have the correct Norm on there, but and then I want to argue that it automatically all of the other possible $\pi$'s are also gaseous, and we can use the Norm to prove that, because the Norm is such that it, you know, well, such that we have this chain of implications.

Okay, so the last part, the last part is, right, that

With respect to this, but now with this one, it's just some arbitrary map to 0/1. But that implies that with respect to the norm $n$, that if you take $n$ of $\pi'$, this is a map from $\mathrm{Spec}\ \R_{\text{Gauss}}$ to 0/1. And then, it follows that there exists an $\alpha$ such that $n(\pi')^{\alpha}$ is equal to just a constant, the norm of $\pi$, because exponentiation acts simply transitively on 0.

So this is the thing we have to. Let me finish the argument, then we'll address Ofer's first point at the end of the argument. Then, $n$ and $n'$ are two norms on $\mathrm{Spec}\ \R_{\text{Gauss}}$, both with the same value on $\pi'$. By the classification of norms, they must be equal on $\pi$ to some fixed real number between 0 and 1, and we showed that such a norm is uniquely determined when we proved this classification of norms.

Now, to address Ofer's question about whether this map $\alpha$ is pulled back from the Berkovich space. The question is whether the map $\alpha: \mathrm{Spec}\ \R_{\text{Gauss}} \to \R_{>0}$ is pulled back from the Berkovich spectrum. This is true if the norm of $\pi'$ is, and that's true by construction, because the mapping to the Berkovich spectrum was exactly recording the norms of all the elements in $\R$, and in particular we're recording the norm of $\pi'$. So the answer to the question is yes.

I also think it should be true that any two choices of norms on your ring are equivalent up to exponentiation to a constant, in the sense of having a bounded ratio from one side to the power of the other, under some natural topological conditions on the ring.

Also, less than or equal to this, no. Probably, it will be equal to 1/2. Then it will be less than or equal to this. And then, this is one idea, but then if you have got a. But by the way, for with this construction, you'd get only the triangle inequality, maybe up to some constant. Again, no. If I have what I claim is this, this is something that I check. So if you have a non-commutative, forgot now. If you have got a, in general, for uniform spes, they, they have three, three, you need to, they work with three, but if you have an community group, you can do it with two. So I just claim first that if you have, if you have a topologic cian group with topology is defined by sequence of a symmetric neighborhoods of the origin, un, un plus one plus un plus, contain un, then you, you just get a metric by this by imposing that the guys in un know at most one over two to the n, so you have a metric in a generalized sense, it could be plus infinity for something. Then I will change it using the unit, I will correct it using my sud uniformer, but at least I will get, okay, maybe what I'm saying is a bit, maybe I'm thinking too fast with some mistakes, but in any case, I think it will also come out that any two norms are, in some sense. But this I am not sure, any to, okay, maybe we can discuss later. Let me just finish. I have only one more thing to say, and it's quite short.

The last thing is when I talk a bit about just say global globalization, only to say that it's trivial. So we could make a definition, I don't know, I mean, a definition of a Berkovich analytic space, I don't know, is a pair, um, X or triple x s, Pi from, local of opens in X to S, where X is an analytic stack, S is a locally compact Hausdorff space, um, and Pi is a map of locals, such that, um, such that, locally on S for the topology, the open section topology, or the the local section topology, it is isomorphic to spec Burke are, sorry, R Norm, Mr. Norm, and then this canonical map Pi for some Bing R, um, okay. So this is completely trivial now, to globalize, and, so the only point to note is that, working locally on S, you're also automatically working locally on X, and that's because these, by definition of a map of local, is a a cover, and the open cover topology gives a cover in the sense of open covers of analytics stxs here, and those are covers in our Gro and deque topology that we use to Define analytic Stacks, they're even open covers in the sense of the six functor formalism for the what do mean local section to I mean the Gro topology on locally compact Hausdorff spaces, which is generated by open covers, okay. It is isomorphic to isomorphic to this basic object space, so you locally, so recall that these these aine ones are always compact Hausdorff, so you're not going to kind of if you if you say locally in too naive a sense, you're not going to get any examples because you know open covers, open subsets of say R, are usually not compact, right, so but if you do this usual thing of having a compact neighborhood of every point, then it's fine, but I, I, this looks it, I wonder about the derived nature of the of the r when you localize because it seems to me that, of course, you can make this definition, but then you can ask whether, for example, what does mean locally oness, if it is true, if you take D, do you have like, for sufficiently fine, for small open, it is let us say that is, if it is true for some cover, it sufficiently fine one, then probably you have to to pass to to to the ring Associated to some
More sub here that the basic building rings, but then she of over conversion, so it's actually never B she of kind dis between the global Al start with, and no, but still that doesn't I mean that doesn't obstruct the claim that that there's a neighborhood base you know. Yes, I mean ofer's question was about a neighborhood base.

That's that is a good point, and I mean you know you can modify these. You could also you know from instead of instead of these guys you could also pass to inverse limits, so you could starting with these apine guys you could pass to arbitrary inverse limits for example, like so inverse limits in the compact house door space and just filtered Co I mean inverse limits in the category of analytic Stacks, which in the Aline case is just you know Co filtered Co limits of analytic rings, and then these overon convergent things would also count is apine, and then maybe that's a little nicer uh to work with, and there's no harm in in doing that. Um, but say will not be B rings, but they will be condensed rings with certain. They'll still they'll still correspond to analytics an analytic stack with a structure map to a compact house door space and so on. Okay, so you probably instead of B ring you can have a condensed ring with s proper, with with some Norm satisfying some properties and so on. I mean we didn't we didn't try to give the best possible formulation, was just a just wanted to connect to the classical thing. Okay, so that's all. Thank you.

Sorry, can you again with it local section? Oh, yes, so on the on the category of locally compact house door spaces, you can define a Gro topology, where a you know a set of maps I mean it's a set of maps like x i to S forms a covering if uh for every point of s there's an open neighborhood of that point and an index I and a section of the pullback uh uh you know you pull back x i to that open neighborhood you should have a a section there. The map can be arbitrary, the map doesn't have to be an open inclusion, but it's also the same thing as the gro dig topology generated by the the covering families, which are just the usual open covers. So if you look at just the usual open covers and say that you want sheath condition for that, you automatically get sheath condition for anything any any map that has local sections, so that's a so if you want if you, so the sieve will always be the same as the sieve generated by some open cover of s, but it's convenient when you want to talk about the sense in which a locally compact house door space is locally compact house DWF because it's not true in some naive sense, but it's true in this sense.

Okay, other questions. Another question, it seems to me if you take at least naively you take another $p$ as the norm, you're supposed to be a constant one, n Prime of Pi Prime was supposed to be a constant, but n of Pi Prime can vary over the Berkovich Spectrum, but uh, I think with supposed to be a constant because is small than the normal Prim Prim inverse the same was it to be. Actually, no, see the norm Pi Pi Prime was adapted to P Prime was adapted to um, uh, absolute value prime, it wasn't adapted to absolute value, so it's not adapted, so this this n here wasn't such that it satisfies that property with respect to.

So how did we built this n from this absolute value here for which Pi had this property that Norm? So we had in other words, we had the norm of Pi inverse equals Norm of Pi inverse here, right, and then we built this n using this so that the the norm of every element would be bounded by the norm prescribed here, that implies that n of Pi has to be this fixed value, but it doesn't imply anything about n of Pi Prime because Pi Prime doesn't NE Pi Prime doesn't necessarily satisfy this property for this Norm, it only satisfies it for this Norm always.

Okay, thanks.

\end{unfinished}