% !TeX root = ../AnalyticStacks.tex

\section{\ufs What happened so far? (Scholze)}

\url{https://www.youtube.com/watch?v=VMgZSP9sRdo&list=PLx5f8IelFRgGmu6gmL-Kf_Rl_6Mm7juZO}
\renewcommand{\yt}[2]{\href{https://www.youtube.com/watch?v=VMgZSP9sRdo&list=PLx5f8IelFRgGmu6gmL-Kf_Rl_6Mm7juZO&t=#1}{#2}}
\vspace{1em}

\begin{unfinished}{0:00}

Good morning. For the people here, you can fill out the survey there now or anytime this week or next week.

Alright, so where are we in the course? Let's try to take a little bit of a broader view. What happened so far? We started with this idea of condensed sets, the being cured, and so on---condensed mathematics. We tried to make the point that this is a good framework for combining homological algebra and functional analysis, with the goal of using this framework to develop a general, good notion of geometry.

At some point, we wanted a notion of completeness. We went in a certain direction, namely concentrating on solid modules---first over the integers, but then over more general rings of finite type, as discussed last time. This is closely related to a specific framework we discussed, which is extremely closely related to Huber's theory of adic spaces.

There were certainly some conceptual ideas that emerged from this, namely that the notion of completeness shouldn't be something that you define once and for all, but rather is an extra datum that is part of the data of what a commutative ring is in the setup. Completeness is a more relative notion.

So what are we trying to get through in the course? We have this category of analytic rings, and for reasons we already saw a little bit, when you localize, some of the structure might become not well-defined anymore. We might change the precise definition here a little bit to work with derived rings from the start, but I don't want to go into that today.

We want to start from this category of analytic rings and, as in usual algebraic geometry where you start with commutative rings and then build schemes by gluing the spectra of commutative rings, we want to have some procedure where we start with this category of analytic rings and then do some kind of gluing to produce some notion of analytic spaces, or maybe we'll directly go to some category of spectra.

Within this world, we want to find all sorts of examples. We definitely want our theory to accommodate geometry over the real or complex numbers, not just non-Archimedean geometry. So first, we would like to see more examples.

Some key questions are: How do the real numbers fit into this? Can we put a natural structure on the real numbers that is suitable for doing complex or real geometry? And can we also do non-Archimedean geometry using solid modules? What is a meaningful way to combine the two settings?

I want to develop one way of thinking about this, with the following example in mind. Maybe the most prime example of what we should be able to do in some world of geometry is the famous Tate elliptic curve.

It's actually, if you look at Fargues and Fontaine's paper, I think it's the one called "The Geometrization of the Local Langlands Correspondence," there really is also one of the first applications of a Serre's ideas, again discussing like the "tilting" of Deligne-Mumford stacks. This is a generalization to higher dimensional Beilinson-Bernstein varieties.

So, you have the étale spectrum of like $\mathbb{W}_{\mathbf{R}}$ series. Over there, you can look at some of the Fargues-Fontaine curve, which is an analytic space over $\mathbf{R}$. Informally speaking, that's all---there's some coordinate $T$ here, and $T$ should have the property that the absolute value of $T$ is bounded between some $\mathbf{Q}_p$-valued elements. 

And then you look at the part of the multiplicative group where the absolute value of $T$ is bounded by a power of one of those $\mathbf{Q}_p$-valued elements, and then acting on this, you have multiplication by a character. This is actually a totally discontinuous operation because it multiplies the absolute value of $T$ by the absolute value of the $\mathbf{Q}_p$-valued element, which is something between zero and one. So, it's actually a free and totally discontinuous action. And so, in the world of étale spaces, you can really pass to the quotient, and the taking this quotient is the nicest kind of quotient, because otherwise you just get where the quotient is like locally split, and locally the space just looks like this.

And so, you can define the Fargues-Fontaine curve to be this analytic Deligne-Mumford stack, and then you quotient by the $\mathbf{Z}$-action. And I mean, how does it look like? You started with, I don't know, $\mathbb{A}^1_{\mathbf{R}}$, and then by modifying by the $\mathbf{Q}_p$-valued element, it's moving everything towards the origin, but then when you take a quotient, this somehow is the same thing as taking some annulus of radius one of this $\mathbf{Q}_p$-valued element and then identifying the boundary.

And so, using this, you can actually see that this is some---yeah, it's actually proper, so it's proper and separated, and it's also smooth because it's locally just a Deligne-Mumford stack. It's a proper, smooth, connected analytic curve over $\mathbf{R}$. And actually, also, I mean, there's a group structure on the Fargues-Fontaine curve, given by some subgroups. It actually has a group structure.

And basically, in any world of analytic geometry, there should be a theorem that something that's proper, smooth, and one-dimensional is always algebraic, because you can always find an ample divisor by just taking any closed point and then taking the corresponding inverse ideal sheaf. This gives an ample line bundle, and then if you have any kind of version of GAGA or your favorite theorem which should always show that they imply the properness and algebraicity of the thing, and so it's definitely true here.

So, there---actually, so here, there implicitly there is some GAGA for over a nice noetherian base, Fargues and Fontaine's pairs, but also the statement about relative dimension one,

The coefficients of $Q$ to the $n$ in here are just polynomial in $n$. When you do this geometric series, the new coefficients still stay polynomial in $n$. This is a funny observation, because it implies that after just defining something formally as a power series, when you actually write down the equation, you realize that this is something you can specialize to any Archimedean value between 0 and 1.

You can certainly take $\mathbb{C}^\times$ as a complex analytic space, as a complex manifold, and for any complex number between 0 and 1, this picture makes literal sense---you can literally take that portion and you get an actual torus. So you can do that, and the analytic description makes perfect sense.

We would really like to have a way to do geometry so that we can perform this kind of construction of $\mathbb{Z}[Q]$ not just over the full power series algebra, but really over some subalgebra where we put some growth condition on the coefficients, so that we can also later on specialize to more general parts. The precise growth condition that you get here is quite a bit stricter than just the observation that it converges when the absolute value of $Q$ is less than 1.

This convergence is just a subexponential growth of the coefficients. It's pretty clear that geometrically, there should be a direct geometric way of seeing that this power series should converge when $Q$ is less than 1. There's some kind of geometric reason for the coefficients having at most subexponential growth, but it's not so clear how you would see that they actually have at most polynomial growth.

Let me actually talk a little bit about the geometry of the space of continuous valuations on $\mathbb{Z}[[Q]]$. You have this topologically nilpotent unit, and this is an extremely convenient structure to have, because it allows you to compare absolute values of all other functions against the absolute value of $Q$. There is a unique map from this space to the Berkovich spectrum, where the absolute value of $Q$ is sent to some pre-specified element between 0 and 1, say $1/2$. For most of the valuations we care about here, this map will be injective, but there are some rank 2 valuations where there's a little bit of extra information in the geometry that's not remembered by this map.

Next, let's consider a set of all "absolute values", but now they really take values in $\mathbb{R}/\{0\}$ instead of just $\mathbb{R}$. Here, you ask the following things: First, there are some basic properties, like the norm of $1$ is $1$, and the norm is multiplicative. Second, the valuations satisfy a strong triangle inequality, so it considers the usual triangle inequality, but also requires that they are all bounded by the specified norm on $\mathbb{R}$. 

This naturally leads to the product of copies of $\mathbb{R}^{>0}$ enumerated by all elements $x$ in $\mathbb{R}$, endowed with the Subspace topology. Actually, because of this construction, you can always replace $\mathbb{R}^{\ge0}$ here by the interval from $|x|$ to $\infty$, and then an arbitrary compact product of compact $\mathcal{H}$-spaces is still compact $\mathcal{H}$. So this is a nice compact $\mathcal{H}$-space.

The only difference between this space and the $p$-adic spectrum is the possibility of having higher rank things in the $p$-adic spectrum, but these just give rise to some minimal changes in the space. Implicitly, I'm endowing the integers here with a norm where the norm of $0$ is $0$ and the norm of all non-zero elements is $1$, since they all contain a unit for the ring I'm considering.

So I still want to understand a little bit about this geometry. It's more or less the same thing as the Berkovich space of the integers, which has a much richer geometry than the usual spectrum of the integers. Here's a proposition: The norm on $\mathbb{Z}$ I'm considering is the one where the norm of $0$ is $0$ and the norm of any non-zero integer $n$ is $1$, since they all contain a unit. There is also a more natural norm you can put on $\mathbb{Z}$, which is just the absolute value. For each prime number $p$, you have the $p$-adic absolute value, which also satisfies all the required properties, but with a choice of what the absolute value of $p$ is. This gives rise to a "ray" for each prime, and the inverse limit of joining these rays corresponds to maps to complete $p$-adic fields, including the usual rational numbers $\mathbb{Q}$ with the standard absolute value.

All right, let me also already discuss the side variation of this. You can also take the $\mathfrak{B}$-space of the integers with the usual absolute value, so $\lvert n \rvert = |n|$, the positive version of it. 

This has the same pictures. The difference between these two things is just the boundedness condition. Previously, the absolute value of 2 could never be 2 like it would be for the real numbers. But now, the absolute value of 2 can be 2.

So this is actually some of the same picture. There's again this thing, for each $\mathbb{F}_p$ number part, but now there's something extra---there is some kind of Tate interval. This corresponds to the map from $\mathbb{Z}$ to $\mathbb{R}$, and then you take the usual absolute value of $r$ raised to any power $\alpha$, where $\alpha$ goes from 0 up to 1. 

It's better to think of this if you take the usual absolute value, Peter, yes, can I vote that you use $\alpha$ instead of $p$ when talking about raising an absolute value to the $p$ power? Thank you. Just in time, to the $\alpha$. If I parameterize this in terms of some fixed absolute $p$, where $|p| = 1/p$, then this line here corresponds to the $\alpha$ where now $\alpha$ can be anything from 0 to infinity. But for the real numbers, you fix the usual absolute value, then you can raise that to any real power, and it will definitely satisfy the conditions, but actually, if you want the triangle inequality to be satisfied, you realize that this happens only if $\alpha$ is at most 1. So in this sense, this line for the real numbers is actually some stops in the middle compared to the others.

Okay, right. So where are we? We are trying to understand a little bit about the geometry of what the spectrum of $\mathbb{Z}[\frac{1}{q}]$ actually looks like, and it's basically the same as the $\mathfrak{B}$-spectrum, and this is fiber over the $\mathfrak{B}$-spectrum of the integers. Now let's actually try to understand what are all the fibers of this map.

For example, if you take the fiber at a prime $p$ of just $\mathbb{F}_p$, well then it's just one point. Then, as usual, some of the formation of the closed space commutes with some kind of fiber products, and so the rings $\mathbb{F}_p[\frac{1}{q}]$ are just a point, because they're already non-fields, and you fix the value of $q$ to be the half.

In characteristic $p$, the thing has just one fiber, which is this kind of the wrong series $\mathbb{Q}_p$. Maybe I should have said, I mean, this proposition is basically just 0, right, because of the absolute values. But then you can also have this spectrum of the integers, and then for each $p$, you have this half-line of $\mathbb{F}_p$s. If you base change this whole half-line of $\mathbb{Q}_p$s, someone looking at, you're in some sense taking, now this is some kind of punctured open unit disc, and now you're this base change to $\mathbb{Q}_p$ will actually be some kind of punctured open unit $\mathbb{Q}_p$, so this will actually be a punctured disc, and this map here to this line 0 to infinity, depending on how you parameterize it 0 to 1 or whatever, this should be an incarnation of the radii matter, but actually in a slightly funny way, it's say 1 over the $\log$ of the radius or something like this, I won't get it straight.

So there is a whole punctured unit disc here, which has an origin and a boundary. Whenever you fix a specific point on here, then this fiber will be some specific annulus in here, where the absolute value is fixed. And now you can wonder what happens as you move towards the invisible, anyway, but I'm always tempted to try to see the image is this circle here, and then when you move towards

Whereas when you move upwards on this Ray, go towards $\mathbb{Q}$, then you will get other points, and they will get closer to the origin. Okay, so what does this thing actually look like?

There is one special point which is $\mathbb{Q}$ and then there are special points which are $\mathbb{Z}$. And on the way there, you have the punctured open disc. The region in the middle, the punctured open disc, and the slightly mind-bending thing is how the different parts are glued to each other. I mean, the whole thing is a compact Hausdorff space, so it makes sense like to ask when you go in this direction, where you end up.

And so, if you move towards the puncture of this open unit disc, you move towards this point, and you move towards the boundary, we end up towards this point. So this whole space is a space that has $p$-adic regions for each $p$, each $p$-adic region is a function open $\mathbb{Q}_p$, and then they glue to each other in this funny way, where for each one, if you go towards the center, you will end up at the common point, which is is a kind of generic point, and for each one, when you go towards the boundary, you end up at the characteristic."
Now it has some kind of function open over the rational numbers, both $p$-adic places and the archimedean place. And everywhere when you go towards the center, you always end up with the center point of the picture.

The slightly awkward feature of this picture is that I had to now specify the absolute value of $\mathbb{Q}$ in advance, and then as the archimedean part of this picture, this punctured open just stopped at the boundary, although from the perspective of the $p$-adic curve there was no reason for stopping at all.

Right, so then one---but like this is precisely the kind of picture in which we would like to combine non-archimedean geometry. I mean, this space has parts which are literally complex analytic or real $p$-adic analytic, but they sit together in one job. And so one kind of PR version of the question about existence of analytic $\mathbb{R}$-structures is now like $\mathbb{K}^1$ and this guy or more any algebra for the vector---was a natural, and this has been a question that was very much in our minds back when we first found out about the solid theory and then tried to really go further.

It actually turns out that to define this, it's slightly better to work with not just precise the Hahn convergence condition, but functions which converge on some rate, some slightly larger dis. That's more techn. Generally speaking, for any kind of $\mathbb{B}$ offering which has a topological unit in the usual way, you can produce this kind of liquid analytic, and the resulting theory will be extremely close to $\mathbb{B}$ which theory, and this is something that we want to discuss at some point today.

However, I want to also talk about something else, something that we only found out a few weeks ago. Once you have this liquid and structure over $\mathbb{Z}$, maybe greater than a half, you can literally repeat the construction of the $p$-adic curve that I did in the beginning over this ring. And then someone show that the $p$-adic curve is definitely defined over this ring, and then okay, in the end you could also make a half larger and larger and would get that it's defined over the whole open unit dis, but you would not get this way the strange Tooumi draws bound on the coefficients.

There is actually a description of the "liquid" structure of the three modules. Yes, let $S = \bigoplus_{i=1}^n \mathbb{Z}$ as usual. Then in this ring, we take $\mathcal{H}(Q) = \mathbb{R}$, and this $S$ defines the three "liquid" modules on $S$ over this ring. The hard part is to prove that this actually gives an "analytic" ring structure defined as follows.

Just like $\mathbb{R}$ is the union of all intervals $[a,\infty)$ for $a > 0$, also the "free" modules are this union over intervals $[a,\infty)$. But then also, as in the free discrete case, we have this union over all size values, and this time I maybe want to index them by some real numbers $c > 0$. But then once you fix the radius and size, you're just taking an inverse limit of such free modules of the part where the norm is at most $c$. 

A similar situation where you can show that if you bound the norm here and give this a certain kind of topology, this actually becomes a compact $\mathcal{H}$ thing, the limit is still compact. This naturally suggests itself when you try to define some notion of "complete" modules over this ring, because the $p$-adic norm on this ring is defined just this way for one element, and then of course if you have a free module, you're just summing the absolute values.

Let me actually just give a concrete example. In the beginning of my lecture today, I was asking two questions: first, is there a natural structure on the reals, and second, is there one which allows you to combine things. Now I'm starting to answer them in the opposite order. 

First I said that there's something that combines them. Let me now also give the answer to the first question: what are the "analytic" ring structures on the reals? You can specialize from $\mathbb{Z}[t^{1/2}, t^{-1/2}]$ to $\mathbb{R}$ by sending $t$ to some number $T$ here less than $2^{1/2}$. This defines a point of the $\mathcal{S}p$ space that maps to the $\mathcal{B}$ space of $\mathbb{Z}$, and so must actually correspond to some power of the usual absolute value on the reals. 

So inside the $\mathcal{B}$ space of $\mathbb{Z}$, you have this half-interval for the reals, where $\alpha$ from 0 to 1 corresponds to the absolute value on $\mathbb{R}$ to the $\alpha$ power. And now this map is actually realizing some isomorphism between these two things, where the value of $T$ is mapped to some real number $R$, and this can be made explicit: $T = R^{2}$.

Okay, so the value of $T$ determines some $\alpha$, and I could have just told you the formula $\alpha = \log_2(T)$. But this would seem slightly curious---what does it mean? The meaning is that you have a point in the $\mathcal{S}p$ space that maps to the $\mathcal{B}$ space of the integers, and you get some point there. This is the $\alpha$. 

So we have a structure here, and we get one here. But the "complete" modules

The norm $\|x\|_\alpha$ is a sum of $|x_i|^\alpha$. Note here that $0 < \alpha < 1$, so this is not one of the usual kinds of norms that you would usually consider in real functional analysis. This is not locally convex. Usually when you put $L^p$ norms, the $p$ lies between 1 and $\infty$, but here we're going to the left and using this non-locally convex norm.

This norm does satisfy the triangle inequality, but it doesn't satisfy the usual scaling by real numbers. You raise it to the $\alpha$ power, but usually you'd ask that it satisfies the scaling with respect to the usual real numbers, which this does not do.

However, if you just look at the unit balls in a 2-dimensional vector space, they have a very peculiar shape, something like that. So you might be tempted to think that this is a stupid way to put a ring structure on the reals, and you should do something else. But this does work.

I guess that the completed modules on $\mathbb{R}$ should be the bounded measures on $\mathbb{R}$. One way to define them is as the dual to continuous functions, but that does not work. The way to describe them would be to do a similar construction but with one more parameter $\alpha$. This would be closely related to the usual theory of solid modules, in about as clean a way as the usual theory of linearly complete modules.

We were really hoping that this $\alpha$-norm structure on the reals would be impressive, but it isn't. However, it's related to some interesting things in classical functional analysis. One thing is that the category of complete modules is not stable under extensions, which was realized around 1918. But once you allow some locally convex vector spaces into the picture, the theory works again, though you're forced to use strictly convex norms.

I should also discuss one other thing quickly. If we specialize this construction to the $p$-adic numbers, you might expect that we're just trying to extend over the "missing part" at the real numbers, so the $p$-adic part wouldn't change much. But that's not true---you actually get new liquid structures on $\mathbb{Q}_p$, with an $\alpha$ parameter that can be anything between 0 and $\infty$. These $\alpha$-liquid modules on $\mathbb{Q}_p$ can be described in a similar way to the real case.

You need them because the Slo Theory already works with you. So I want to ask a small question just to clarify possible confusion. You have the $L_\beta$ for different $\beta$, which are again defined in the same way, just by some powers of the $\beta$, without taking one over $\beta$. This is a slight mismatch. What I know is that if I literally specialize the definition of $A$ to $\infty$, I would be summing the supremum norms, which is not what I'm doing when I do this $\infty$ here. The supremum here is the actual...there's a little bit of a mismatch in notation between here and here, but I think it's okay.

But what is the when you take the union over $\beta < \alpha$? I think usually it was a filtered union. That is, is it the case, because here you have to be slightly careful that the way you index the constants, some should change when you increase $\alpha$. So why the limit, the union over $\beta$, what are the inequalities between the $L_\beta$ here, the Tate unit over all $\mathcal{C}$? And then, is it literally the case that these spaces get larger and larger as you increase $\beta$?

Okay, think of it like this: in some sense, you can think of a whole series of series that go from 0 all the way to $\infty$, where you put some norm here. At this end, you have solid modules, and then at each point $\alpha$ here, you have the $\alpha$-liquid ones. In terms of the class of modules, being solid is a much stricter condition than being $\alpha$-liquid. So the class of modules that you allow here becomes larger and larger as you make $\alpha$ closer to 0. And as you go to $\alpha = 0$, you have all condensed objects, because if you naively put some kind of $L_0$ norm there, meaning that there's only a bounded number of coefficients that are non-zero, and then you put some bound, this actually recovers a free condensed module without any competition by variant of what I said at the beginning about the integers.

Okay, I guess it's time for the new thing. Right, so back then we were trying to look, find natural candidates for how what the free compact modules could be, and then it was some hard matter of proving that they actually define the existence of being an antintic brain. But in this course, we have a different mindset. It is that to the $\mathcal{F}$ and the structure, we find something natural, the morphisms of this projective object $P$ that you want to be morphisms. The free outno sequence should come after completion. And because in life being groups, this $P$ has these very strong properties, like being compact and internally projective, this will always define a drink structure, so it has become extremely easy to produce analytical instrues.

Here's an example: it turns out that to define the adic curve, you only really need two things. The first is that $P$ should be topological, it's new, and should be unit clear, and both of these have clear meaning in terms of the underlying cont ring. I mean, so being new means that you have a map from basically this guy and up with the natural ring structure, sending 1 to $\mathcal{C}$. But then you need some completeness for your modules, you need to be able to some certain sequences. And really the only thing you need is that $1 - Q \cdot \text{shift}$ acting on this projective $P$ is an isomorphism. And it's clear that there is a universal example of such an antic ring. I mean, you just take the free generated by a unit, that's just a condensed ring, and then this condition puts some ring structure on this ring also, which is good to much. The key is that there is an initial such example. Existence is easy, the hard part is a description. Hey, so it's a pair of a triangle, and you can actually describe a triangle, and well, let me describe the underlying ring. This is precisely those sums such that the PO and thus, if you like, the claim is that we can do analytic geometry as usual over such rings. And so you can just repeat the construction to the curve, and then you will see for geometric reasons,

Already, the free complete guy---a triangle, for example, would be a free non-complete guy, but a triangle is a free complete guy. It is the union over all $a > 0$ of something. So, as usual, this is a limit of a sum of something. 

I mean, of the part of the free module, let's say, where the coefficient of $Q^n$ has $\ell^1$-norm at most $n + n^k$. In the future lecture, I'll give a more precise description. But basically, you're asking for a similar description as before, but now you're asking for some polynomial growth conditions on the coefficients.

The condition here is that the coefficient of $Q^n$ can grow at most like a polynomial in $n$ of degree $k$, but I also have to allow the presence of negative coefficients. Anyway, then the limit as $n$ goes to infinity has $\ell^1$-norm at most $n + n^k$.

If you want the ring to be the thing where you have coefficients that grow at most polynomially, and you think about a way to encode that on the modules, then this is what you would be doing. Peter, do you really mean $\ell^0$-norm or $\ell^1$-norm? Because it's about a free module on $\mathbb{Z}$, so let's say $\ell^1$-norm.

Within doing analytic geometry over this space, you will get a geometric way to construct the Tate curve really over the kind of smallest ring, which is actually the sum of something. This also means that if you specialize back to $p$-adic numbers, then the Ganga series is very close, but not quite at zero. It's a fun exercise to take this description of the free modules and base change it to $\mathbb{Q}_p$.

Let me start now.

\end{unfinished}
