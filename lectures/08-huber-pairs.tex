% !TeX root = ../AnalyticStacks.tex

\section{\ufs Huber pairs and analytic rings (Scholze)}

\url{https://www.youtube.com/watch?v=dIwBTJNN7a0&list=PLx5f8IelFRgGmu6gmL-Kf_Rl_6Mm7juZO}
\renewcommand{\yt}[2]{\href{https://www.youtube.com/watch?v=dIwBTJNN7a0&list=PLx5f8IelFRgGmu6gmL-Kf_Rl_6Mm7juZO&t=#1}{#2}}
\vspace{1em}

\begin{unfinished}{0:00}
  Right, so last time Dustin started talking about the relative solid theory and the relation to adic spaces. I want to kind of continue with that.

Okay, so I guess I want to talk about the relation between these basic objects that appear in Huber's work, that are called Huber pairs, and the kind of basic objects that appear in analytic geometry.

Motivation: we've seen several examples of a pair of a "present" notion of "complete" and the "same". The first one we discussed was the integers ($\Z$) and solid $\Z[1/p]$-modules inside of all condensed $\Z$-modules. But then last time, Dustin discussed the example where you take the polynomial algebra over $\Z$ in $T$, and then within all condensed $\Z$-modules, you can somehow just take the ones that are complete in the sense of this $T$. So you're taking modules over this algebra inside of solid abelian groups.

But then Dustin argued that it's actually quite natural to also isolate a stronger condition, giving the solid $\Z[T]$-modules. Geometrically speaking, this corresponds maybe to some kind of line, and this corresponds to the unit disk. In these cases, the condensed thing was actually just the classical thing.

But this notion of completeness is still interesting. You could also take, I don't know, $\mathbb{F}_p$ or $\Q_p$ or whatever, and then take solid $\mathbb{F}_p$-modules or solid $\Q_p$-modules. In those cases, these are actually just the ones where the underlying condensed group is solid. There's no meaningful way to strengthen this.

Okay, so the notion of an analytic ring captures this situation.

On the other hand, if you learn the adic stuff, then you run into this definition. I mean, the basic objects there are these Huber pairs. So let me recall what these are. These are Huber's definition, although of course Huber used different names.

A Huber ring is a topological ring $A$ with an open subring $A^+$ and an ideal $I \subset A^+$ such that there exists some finitely generated ideal $J \subset I$... Because now it's not clear what "ideal" means, because there are two rings in place, $J$ is an ideal of $A^+$, a subset $A_0 \subset A$ that has the same $I$-adic topology.

Let me give examples in just a second. Let me just finish the definitions.

And then Huber has this notion of a ring of integral elements. This is an open and integrally closed subring $A^+ \subset A$ containing $A^{\circ}$, the power-bounded elements. Second...

And third, a Huber pair is the pair $(A^+, A)$ satisfying these conditions.

Okay, so what are the examples to keep in mind? Let me first give some stupid examples:

Any discrete ring $A$ is Huber, so in this case you can take $A^+ = A$ and $I = 0$. Any ring that itself is an adic ring for a finitely generated ideal $I$ is Huber, so in this case you take $A^+ = A$ and $I = I$.

Maybe actually interesting is when you have something like a Banach algebra. So for example, $\Q_p$ is Huber, or any non-archimedean field is Huber. In this case, you take $A^+ = \Z_p$ and $I$ generated by $p$, which really is only an ideal in $\Z_p$ and not in $\Q_p$.

So basically, the idea is that Huber rings are basically certain kinds of localizations or something like this of such adic rings with some adic topology.

And so also whenever you have any

Kind of Banach algebra over $\Q_p$ or some other non-archimedean local field. There's also always Huber rank. As a zero, you can take the unit ball in your Banach algebra, and as the ideal, you can take the one that comes from a uniformizer.

Remark: The completion of any Huber ring is again a Huber ring. We will generally only consider complete examples now. In the classical sense, allow all convergent sequences mod $\mathfrak{m}^n$.

If you start with a so-called ring of definition, one which has the $I$-topology for some finitely generated ideal $I$, then it will be the case that the completion of $A^+$ will be an open subring of the completion of $A$. This is just the definition of a Huber pair.

At least when one uses Huber rings and Huber pairs to do adic spaces, then the adic spaces associated with a Huber ring and its completion are definitionally the same. In this sense, non-complete Huber rings have at most a technical role.

We will only consider complete ones. So from the following, whenever I say Huber ring, Huber pair, and so on, I always assume that the underlying Huber ring is complete. Okay?

And so, last time, Dustin discussed some notions of topologically nilpotent elements, power-bounded elements, and so on. This can also be defined here.

If you have a Huber ring, then you can define topologically nilpotent elements and power-bounded elements in $A$. This is the set of all $f \in A$ such that $f^n \to 0$ as $n \to \infty$, which once $A$ is assumed to be complete, is really just a condition.

And these are all the elements such that the set of its powers is bounded. It's actually equivalent to saying that it's contained in some ideal of definition, some ring of definition.

Actually, a different way to think about this set: First of all, you have these canonically defined objects, $A^+$ and $A^{\circ\circ}$. But on the other hand, we used these things in the definition that there is some ring of definition and some ideal of definition of the same. These will always consist of power-bounded elements, and the ideal consists of topologically nilpotent elements.

For example, in the case of $\Z_p$, $\Z_p$ is actually the $A^+$, and $(p)$ is actually the $A^{\circ\circ}$. In this case, these inclusions are equalities. Of course, this can't in general be true, because you could also take as ideal of definition the ideal generated by $p^2$, but then you don't have all topologically nilpotent elements.

But in some sense, the $A^+$ gets, in fact, the collection of all such $A^+$'s, and the collection of all such $I$'s. They form filtered collections, and $A^+$ is actually the colimit of all possible such rings of definition, and similarly, $A^{\circ\circ}$ is the union of all ideals of definition.

Alright, this was part one of the definition, together with a little bit of discussion. Part two was---oh no, sorry, there is no part two. I wanted to define, I wanted to say that if I also have a ring of power-bounded elements, I can define an $A^+$, but it's kind of weak because---sorry, okay.

So when you learn Huber's theory, at first, I think it's extremely hard to appreciate the significance of this ring of integral elements $A^+$. It is somewhat necessary to set up the theory, but it's kind of hard to feel why it's necessary.

But it actually turns out that the theory that we develop using the condensed mathematics gives you a very good understanding of what it actually does. Namely, precisely---here's an example.

I actually have several possible examples. For instance, you could take $A = \Q_p\<T\>$, and in this case, $A^+$ is just $\Z_p\langle T \rangle$. For example, you could take just 

So maybe I should give this definition of analytic rings. By the way, sorry, maybe I can make one notation remark. Huber uses the single letter $A$ to denote the whole pair consisting of some topological ring and a ring of integral elements. We will follow him.

Also, when I discuss analytic rings, I want to use single letters to denote my analytic rings. But then they will have an underlying condensed ring, that's $\underline{A}$, and we needed some symbol to denote that underlying condensed ring. We didn't come up with anything good, so we chose to just follow Huber's lead. Okay.

So let's say $A^\triangle$ now some lightning bolt, and then I want to say what is an analytic ring structure on this thing. So what is an allowed class $\mathcal{C}$ of complete $A$-modules? Okay, here's the definition. It's equivalent to the one that Dustin gave in the first lecture, but presented from a slightly more elementary perspective.

Okay, so it's a full subcategory $\mathcal{C}$ of $\mathcal{D}(A^\triangle)$, the category of condensed $A^\triangle$-modules, together with an $A$-module structure. That's the data I just said. Now I will make a lot of conditions on this, but those are conditions that we had already seen before, twice. We stated that solid $A^\triangle$-modules have a lot of nice properties, and it was a long, long list. Sometimes, because we don't want to state this list all the time, we make this definition.

So first of all, $\mathcal{C}$ should be stable under kernels and cokernels. But it's also stable, in fact, under all limits and colimits. All extensions, so if you have an extension of two things in $\mathcal{C}$, it should also be in $\mathcal{C}$.

Then there is a Tor-amplitude condition you want $\mathcal{C}$ to check. It's also stable under all $X \mapsto X^{\oplus I}$ for some set $I$. And $\mathcal{C}$ contains $A^\triangle$ itself.

[Dustin:] So can I ask a question? Does it imply, maybe one can prove from this in some way, for example, that $\mathcal{C}$ is a Grothendieck category? The condition that I allude to is the existence of a set of generators. Is it automatic under these conditions?

[Peter:] Yes, yes. Did you hear his answer?

[Dustin:] Yes, he said yes. And does it imply that the Ext groups in the subcategory are the same as the Ext groups in the full category?

[Peter:] No. I don't know what he said, but the answer is yes. I know the answer is yes, but you didn't hear... I mean, in this presentation, the derived category might not be in degree zero. So if you really phrase it at the derived level, you have to be slightly careful when you say that, right? Because it might not be the case that the thing I will define, the derived category of $\mathcal{C}$-modules, is the derived category of $\underline{A}$-modules. It's not. Dustin, do you hear what I say?

[Dustin:] Yes, I hear what you say. You said that in the category, the thing I will define as a certain triangulated category or stable $\infty$-category as a full subcategory of condensed $\underline{A}$-modules with some properties, which is the correct one. But in general it will not be the same thing as the derived category of $\underline{A}$-modules.

[Peter:] Yes, this is the heart of a t-structure. This is those whose homological groups are in this. So it doesn't imply that the Ext's are the same.

[Dustin:] It does not. And sometimes it does, sometimes it is true.

[Peter:] But like in most practical cases it will end up being true. In full generality, no. Okay. So I can proceed?

[Several people:] Proceed.

[Peter:] Now Dustin put me...made me confused. So I want to claim that there is automatically a left adjoint to the inclusion.
So the claim is that the left adjoint---which I will write as sending a module $M$ to its base change from $A_\infty$ to $A$---mod $A$ is just purely notational for now, but I will think of it as the modules over this analytic ring $A$.

This base change functor has kernel, the $\otimes$-ideal in $\mathrm{Mod}(A_\infty)$, and $\mathrm{Mod}(A)$ acquires a unique symmetric monoidal structure making the base change a symmetric monoidal functor.

Let's sketch the proof. We already discussed the existence of the left adjoint, which is formal nonsense. If it's not, just make it part of the definition. The question is about this kernel being a $\otimes$-ideal.

So what does it mean to be a $\otimes$-ideal? The left adjoint $F$ definitely preserves colimits. To show it's a $\otimes$-ideal, we have to show that if we have something $M$ in the kernel and $N$ is anything, then $M \otimes N$ is still in the kernel.

Assume a module $M \in \mathrm{Mod}(A_\infty)$ such that $F(M) = 0$, meaning it has no maps to any $A$-module. We want to show that for all $N \in \mathrm{Mod}(A_\infty)$, we have $F(M \otimes N) = 0$. 

This means showing that for all $L \in \mathrm{Mod}(A)$, $\mathrm{Hom}(F(M \otimes N), L) = 0$. By definition of $F$ being a left adjoint, this is equivalent to showing $\mathrm{Hom}(M \otimes N, L) = 0$.

Using the Hom-tensor adjunction, this is the same as $\mathrm{Hom}(M, [N,L])$, where $[N,L]$ is the internal Hom. But we assumed $\mathrm{Mod}(A)$ is stable under all limits, in particular internal Homs. So $[N,L] \in \mathrm{Mod}(A)$. Then again $\mathrm{Hom}(M, [N,L]) = 0$ because $F(M) = 0$ by assumption.

The symmetric monoidal structure on $\mathrm{Mod}(A)$ has to be given by taking the tensor product in $\mathrm{Mod}(A_\infty)$, seeing this as a colimit in modules, and then completing again. The question is whether this makes the base change functor symmetric monoidal.

To check this, for all $M,N \in \mathrm{Mod}(A_\infty)$, we can either first tensor $M$ and $N$ and then apply $F$, or we can first apply $F$ to both of them and then tensor in $\mathrm{Mod}(A)$. This has to give the same result.

I think a better statement is that if a map in $\mathrm{Mod}(A_\infty)$ becomes an isomorphism under localization, then tensoring it with anything else also makes it an isomorphism. This follows from the same type of argument, by mapping into the subcategory and using the internal Hom. I did this in the solid case, and it's the same argument here.

Then the point is that, for example, $f$ is from $M$ to its completion, which becomes an item of localization because an important operation. And so if I-- this within some-- the same-- stage it.

So this is some structure you automatically have on a triangulated category $C$ of modules, some kind of localization of condens-- the underlying ring. And requires in terms of product. And now we pass-- D-- let not just say that a structure on an underlying lightens string a triangle.

Then undine the of a modules, the full sub-subset for all $Z$ of all-- let me still just call them $N$, so compx of modules. So group-- let me think homologically, all the homo groups line. Define for Jesus.

And okay, so here-- here's already the warning: there is a natural comparison function from the of mod $A$, but it's not always. And in essentially all, I mean basically, yeah, all cases I'm aware, it will come out to be an equivalence, but it's just not a general effect.

But yeah, so the good thing we are to focus on is the thing that we simply call $D$. And so the previous proposition has an analog on dou.

So E of A triangle triang-- so I mean, probably in one or two lectures we will probably switch to the infinity categorical language, where we would say "stable infinity category" instead. For now, it's not really requir--, so let me just TR more classical terms.

Stable under all-- so again, in stable infinity case we could say "stable under all limits and colimits", but general limits and colimits are not well behaved work Str categories. But you can say something equol-- and stable under all s-- and for product, which are well behaved to what they're supposed to do.

I'm trying to say, right, the inclusion again has left on, that I will call the dve $S$, the product. And again, this has a property that if you have something, it becomes an isomorphism here. Then if you T-- it with anything else, it's still the same. And because this is now a triang c--, you can actually phrase this equ-- in terms of the co--. So if you have something in the kernel of this, then you tend up with something sa-- the.

And then again, this the T-- here. So if one wants to to do the previous type of argument for this fact, then one lends into the question of whether-- of course there is internal or in the full derived category by unbounded and so on. But is the question is whether if you have internal oh-- from anything to something in $D(A)$, then it lies in $D(A)$, right? And this is not-- because of unboundedness, I don't-- of course you have a bounded complex, you have a spectral sequence. I mean, you still have to to to work with that. Here you have unbounding in both directions, I can see in one direction you have IND light conding is still repete. So the derived category is left complete and so I think you can control control the question. Okay, let let me do this in a second when I come to the pro s-- the product maying face chang--.

All right. So see, you have-- I don't know, $M'$ to $M$, $M'$, $M'_1$, $M'$, and there a modules. And let's assume two of them, and by shifting it doesn't matter which to, are $D(A)$. Then we want to show that $M$ is also to show. But for this we just look at long sequence. So we have $H(M')$ have $H$, and these are $A$. So if I have some quent here and have some cur-- kernel right? And this is a kernel of the thing which a mod is a qu-- of a modules. So this of both a modules and then this one is an extension, right?

Here we use stability on the kernels and cions, and then we use stability on extends. No I think standing here actually realize okay. So and the directed to the $H$, and not standing realize that possibility the product, countable ones, they definitely reduce to.

And then the uncountable ones? That was okay when we were working in the full condensed setting and preparing the lecture. I overlooked that there might be an argument to do here.

Dustin, should I just assume that there is a claim on the level of the categories? I'm sorry, I was busy with the chat. What's going on? Why is it stable under all products?

Why is what stable under all products? The subcategory of complete ones? Oh, all products! Instead of just countable products, right? Oh, all products exist, but is it exact in your category? Yeah, this is a problem. $\infty$-limits and all products are not exact. Yeah, this is a problem.

Okay, we'll have to think about this. It's not an actual issue in some sense, but I screwed up the definition. So we should ask for the existence of left adjoints. I mean, Dustin did in the first lecture and I just threw it out when I prepared the lecture. For existence of left adjoints. The definition, I definitely want the admitted left adjoints.

Um, sorry. All right, so now I made this next thing actually part of the definition that exists.

So, let's say $M$ is complete. And then, is any condensed... No, sorry. What I want to show is $N$ is in $D^-(A)$. And then there anything... Then you have to show that for all $L$ in $D(A)$, $\underline{\mathrm{Hom}}(L,N)$ are complete.

And first of all, because $\infty$-sites are what's known as replete, this means it's closed under countable limits of surjections. This notion was introduced in my paper with SAG, but on the pro-side. And one thing we saw there is that this implies that any such... Sorry, for all $K$, I didn't use the letter $K$, right? So for any $K$ which is, for example, a condensed abelian group or module, $K$ is isomorphic to the derived limit of its truncations in degrees at most $n$. Some kind of limit, usually of abelian groups.

I mean, it's somewhat true, right? When you truncate up to some degree and then just take a limit of these things, you're somewhat stabilizing to the correct answer. In general, that's an issue because you're taking a countable limit here. In general, countable limits need not exist. But under this assumption, you can control them.

So this means that I can certainly assume that $L$ is bounded here, right? What I need... So first of all, and to show this, I can again use the adjunction. And I assume that $M$ has sheaf completion. So it suffices to show that the internal hom in condensed abelian groups from $\underline{L}$ to $\underline{M}$ is... Sorry, it's not complete, because then you can rewrite this as a hom from $L$ into this guy. But I assume that $M$ has trivial completion, so it doesn't have to do anything.

Okay, so this I want to reduce to the level where I kind of had the statement that if $L$ is in $D^-(A)$ and $M$ is in $D^+(A)$, then all the internal homs are in $D^+(A)$. The issue though, as already pointed out, is that here we need to ask this condition for all possibly unbounded complexes. That's why I mentioned this fact. So this at least allows us to assume that $L$ is in $D^-(A)$.

So I can assume $L$ is in $D^{\leq 0}(A)$. I usually put this going to the right. Because also all truncations, they are still in $D(A)$. But because the condition was just on the other hand, $N$ can be written as a colimit of the truncations to the left. I mean, this is always true, that there's a colimit of truncations $\tau_{\geq -n} N$. And $\tau_{\geq -n} N$ is in $D^{\geq -n}(A)$.

This is much easier because colimits forgetful to direct sums are always good. And similarly, you can pull the colimit into a limit. And because we know
Okay, I think that's fine. Once you have that, the existence of the tensor product is just the same formal diagram chase that I didn't execute previously, but did earlier.

Another thing I should have mentioned as part of the general theory, but didn't, is that $D^{\\leq 0}(A)$ has a natural t-structure, making it a stable $\\infty$-category. The left adjoint is not generally t-exact, as we've already seen that solidification could turn something unbounded on the left. Still, this left adjoint preserves connective objects.

A t-structure is where you have a notion of truncation of complexes, a notion of complexes which live in certain non-negative degrees or certain non-positive degrees, and they satisfy all the usual properties. We definitionally made this a triangulated subcategory which is stable under all the different functors.

So this inclusion is t-exact, and it's a completely general fact that if you have a left adjoint to a t-exact functor, at least it preserves the connective part. Let me check whether this maps to anything which is concentrated on the right, but this is a left adjoint, so you can compute the $\\mathrm{Mor}$ in the larger category. But then this is still in this category.

In particular, you can talk about the heart. The heart is also definitionally just $A$. If you take this and pass to the heart, this is $A$. If you take the tensor product and pass to the heart, in this sense the derived and abelian level are compatible.

Then there's the other question: if you start here and just animate all these constructions to $\\infty$-categories, do you recover those constructions? This is just not true in general. In general, you don't even recover $D(A)$. Even if you do, there are separate questions about whether you recover the correct functors, and again, not in general. I think if you do recover the correct categories, you also recover the correct functors by functoriality. But the tensor product is a bit subtle. Again, in practice it is true that $C(A)$ is just the animation, and all these functors are correct.

With that out of the way, I'm almost done with my lecture, unfortunately.

Okay, back to the comparison. When we had Huber rings, we had these topologically nilpotent elements and so on. Dustin already gave a variant of this. First of all, Huber rings themselves are Huber pairs $(A, A^+)$. These condense to rings of course. All this is actually fully faithful.

Actually, I should denote these as $A^\solid$ and $A^{\solid+}$ where the $\solid$ means solid. Last time, Dustin already gave a definition that for a solid ring $A^\solid$, we can define subsets $A^\blacksquare \circ$ and $A^{\blacksquare \circ\circ}$ of the underlying condensed ring.

Let me recall, $A^\blacksquare \circ$ was the set of elements in the underlying ring such that the corresponding map $\Z[T] \to A^\blacksquare$ factors through $\Z\langle T \rangle$. There was some discussion about how much structure you need to check here. The condition was that it factors as condensed rings. It's actually enough to check that it factors as condensed modules over $\Z$.

Then $A^{\solid\circ\circ}$ was defined as the set of elements $f$ such that there is a sequence $(f_n)_{n \geq 0}$ with $f_0 = f$, $f_{n+1}^p \in f_n A^\blacksquare \circ$ for all $n$, and $f_n \to 0$. This comes together with a smallness condition that $f$ times the shifted sequence makes it $I$-adically Cauchy.

If you apply this to the case where this solid ring $A^\solid$ arose from a Huber ring, then this is precisely the set of topologically nilpotent elements, and those define the $\circ\\circ$-elements. When you regard 

Is precisely the same thing as $H \subseteq A$ being power-bounded. Dustin showed last time that this is always an integral statement.

This here is always a what's the definition? Yeah, sorry. Given $f$, I can again---let me write again why this means I can "$a$" and then the condition is, I'm already speaking of modules over $C$, modules over the closed unit disc, as we motivated last time. This means that $\lvert f(a) \rvert$ should be at most $1$. So it should be also power-bounded. Okay.

But now, I can also make the point: Assume $A$ is an analytic ring structure on a solid $R$-algebra. Then I can also define an $A^+$. I realized I didn't define this, so let me do this in just a second. Such that the map from $D(T) \to A$ is the same as always.

This map induces a map of analytic rings, from $\mathbf{V}$ (the corresponding solid module). Yeah, that's precise.

So something that I should have said previously but forgot. Let $\phi\colon R \to S$ be a map of condensed rings, $M$ an $S$-module, and $N$ an $R$-module. Then a map of modules $f\colon M \to \phi_* N$ (where $\phi_*$ denotes restriction of scalars) is equivalent to a map of condensed $S$-modules $\phi^* M \to N$ (where $\phi^*$ denotes base change). In this case, you can pass to the left adjoints on the level of derived categories, because you can check it on the level of modules.

Once you pass left adjoints, the left adjoint to restriction of scalars (i.e., what I term the base change functor) becomes the left adjoint here. So you get it also left adjoint here, which is base change. If you want, you can compute it by first base changing as condensed modules and then completing it. You also get a derived functor.

Okay, so the claim is that first of all, once you have such an analytic ring, you can get data as in Huber. I will immediately check that this actually automatically satisfies his list of conditions that he puts on his ring of integral elements.

Conversely (I'm not sure if I have time, I hope I can say it), whenever you have a ring of integral elements in Huber's theory, you can actually produce an analytic ring which is in some sense the initial one.

Okay, so right. First of all, I can also rewrite this power-boundedness condition. It's all those $s$ such that $1 - sT^*$ (which is an endomorphism) maps $D(T)$ into $A^+$.

Changing notation: $P$ was this "$P$" called---it's always the spectrum based on the normed $R$-algebra. And we characterized being solid over $\Z[[T]]$ by, well, being solid (but this we already asked), and that $1-sT^*$ is a morphism on this projective generator.

So what's actually to ask is that if I look at this thing here (an object in $\mathcal{D}(A)$), then this is a "why". If I admit such a map, then this already happens here---like, already here, $1-sT^*$ becomes an automorphism of this object (definitionally). But on the other hand, because this precisely characterizes the solid modules here, you can also show the converse.

Like, if I want to show that I have such $A^+$, I need to show that all complete $A^+$-modules here restrict to complete $A$-modules here. But being complete precisely meant that if I tensor $1-sT^*$ into there, it becomes an isomorphism. And so just translates.

So basically, whenever you have any element $s$ of your underlying ring, you can ask this condition that $1-sT^*$ becomes a morphism of $P$. And this will define for you an analytic ring structure by general theory. Yeah, basically whenever you have an endomorphism of your complete projective object $P$, declaring that this should be an "as morphism" for all $s$ in the ring structure---so you can take
Modules... But then probably it's equivalent to what you have written, if you think about it, but in the derived way. I just want to confirm that the two versions are equivalent. Yes, right? Because you can actually detect analytic $R$-structures on the level of modules. It's enough to check it on the underlying level, the ind-level. That is also true, yeah. Okay, so it's equivalent.

I'm... Uh, yeah, so with derived $I$ would be more confident, but I think the argument just sketched goes through. So here... Right, so the point is that this subset $A_L$ automatically satisfies the conditions. It always contains the topologically nilpotent elements, which are always an open subset. $\{u\in R\mid \sum_{i\geq 0} a_i u^i$ converges$\}$ in particular, this is open. When $R$ comes from $\Q_p\langle T \rangle$, it's always containing the power-bounded elements, and it's always integrally closed.

So why is that? Well, if $f\in R\langle T \rangle$ is an element, then we actually get a power series... That's almost... Yeah, if I have a module that's actually $\Z_p[[T]]$-module in $\Solid_A$, that's automatically $p$-torsion-free, actually. I mean, this proof is exactly the same as last time, just because all I used was arguments about modules being solid over one ring implying they're solid over another, and so on. Yes, let me just state it again. Okay.

So these are actually all there. But this means that whenever I have a module here over $A$, it in particular becomes a module here, which is solid. Everything is solid. So it must be an underlying solid ring.

If $A^+$ is integrally closed, then I get a map from here, $\{a\in A\mid \forall M\in A\text{-}\mathrm{Mod}, M \text{ solid} \implies aM \subseteq M\}$, to $A$. But in particular this means that $A^+$, the underlying condensed ring which we always assume is a complete $\Z_p$-module, by restriction... So if $f$ is an element in $A$, right, so this group scheme inclusions $A^+\to A$... And therefore, something that's the same argument. So in fact, yeah, the argument that Dustin gave there was already talking not just about a triangle, but about any module. And so if you just run his argument, to see this is what he actually...

Okay, so... Right. Therefore, if $A$ is any triangle, I have the solid analytic $R$-structure $\mathcal{A}_1$ which lives over $\Z_p\langle T \rangle$ and consider this morphism of rings of integral elements. I just gave you a recipe here that was taking some such analytic ring structure here and produced a ring of integral elements $A^+$ in here.

And it's actually functorial. So you can actually show that $f:A\to B$, one arrow is contained in the other. Yeah, I mean, if you have... And you get an inclusion of the analytic rings, and this actually has a left adjoint, I assume.

So whenever I have a ring $A$, so I can produce an $\mathcal{A}_1$-ring structure on the solid $A$-modules. Yes, so $\mathrm{Solid}_A$ admits an $\mathcal{A}_1$-ring structure. It's unique because the ring of endomorphisms is unique, and then it's just a condition. It's just a condition that all the $A$-modules are actually solid.

A different way to phrase this is to ask that $1-\varphi$ is an automorphism of $A^+\otimes\Q_p$. It doesn't matter.

All right, so I wanted to say that it has a left adjoint. So if I have a morphism of pairs $(A,A^+)\to (B,B^+)$, then I can send this to the ring $B$ associated where mod $B^+$ by definition, all those

Usually there's just one or two elements, something where you really have to check. All the same thing as those modules, and only for those two elements.

So yeah, to connect to the beginning, for example, like $Z[Z]$-pairs, these are just $Z[Z]$-modules. If I take $Z$, that's the pair. So if I only put $Z$ here, then I'm only asking that it's $Z$-solid over $Z$. So I take all $Z[T]$-modules and $Z$-solid $Z$-modules. But then when I take this, it becomes $R_0$ for $A^+$. So let me just state one last proposition.

So when you start with $H$, and then go to $A^+$, then if I go back, this actually matches back to $A^+$. So if I write, I have an analytic ring and then I can take its plus ring. So this is actually a plus equivalence, the left adjoint functor, from $R_0$ to $A^+$.

So all brings, but they're all---and yeah, I mean, I'm really still quite struck by how closely the theory of solid analytic rings really matches Huber's classical theories. If you restrict to analytic rings where you only allow yourself to put conditions that one minus some element times the shift operator on $T$ becomes an isomorphism, then you're precisely getting those analytic ring structures that are induced by rings of integral elements in Huber's sense. So which is kind of very strong aposteriori motivation for this definition.

All right, I should stop.

\end{unfinished}