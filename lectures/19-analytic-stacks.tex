% !TeX root = ../AnalyticStacks.tex

\section{\ufs Analytic stacks (Scholze)}

\url{https://www.youtube.com/watch?v=T9XhPCI8828&list=PLx5f8IelFRgGmu6gmL-Kf_Rl_6Mm7juZO}
\renewcommand{\yt}[2]{\href{https://www.youtube.com/watch?v=T9XhPCI8828&list=PLx5f8IelFRgGmu6gmL-Kf_Rl_6Mm7juZO&t=#1}{#2}}
\vspace{1em}

\begin{unfinished}{0:00}
So today, we will finally define what analytic stacks are. It's not so difficult. Recall that we have this category of analytic R-spaces. This was actually, I think, presentable. That's been proven - it has all co-limits, and it's generated by a set worth of the usual analytic rings and spaces.

We can make the following definition. I don't expect this to be accessible, but the word "accessible" is just there to deal with some set-theoretic issues. From analytic rings or Fr??chet analytic spaces, toward... Recall, the word "accessible" here, there will be a condition in just a second. This means that it commutes with filtered colimits for sufficiently large U. It's such that, sorry, let me rephrase the conditions as follows.

We have some open cover U of A, and then some hypercover U of A, for which the colimit of the nerve of this cover is isomorphic to A. Also, X of A is isomorphic to the spectrum of the colimit of the sheaves on this nerve. By the way, I write "unspec" where Dustin just wrote "spec" - I don't think we've settled on a final notation, just not to get it confused with the usual spec, let me write "unspec".

This is some form of descent that's somewhere strictly between... Can I ask just a minor technical question concerning the notion of hypercover? In usual algebraic geometry, do you hear me? Yes, I hear you.

So let's consider the case in usual algebraic geometry when you have, for example, an fpqc cover or something like this. There could be two meanings of fpqc hypercover. One meaning is that the map from X0 to X and the map from Xn to the coskeleton are fpqc. The other meaning is that there are coverings for the fpqc topology, which is H. Sometimes it causes some concern, what is the meaning. 

In our context, you use the strict meaning, I suppose. That everything, I mean, you... Yes, so I want all maps to be fpqc, but then there is no difference between, if it's refined by a stable cover, then it's already a cover because of the S condition. Ah, okay, so here it is satisfied that if some, if A is dominated by it's already fpqc cover, so okay, thank you.

Right, and I also need to say that commuting finite products, so commuting products including the empty product, geometrically just means that X evaluated on a disjoint union is the product.

So this condition is, I mean, up to issues, and it's something that's somewhere between 2 and 3. For Čech sheaves, you would only consider some Čech nerves of covers, and those would, by definition of what a Čech cover is, always satisfy this condition, so for Čech nerves of Čech covers, you always have this condition, so it's always a Čech sheaf. But for Čech hypersheaves, you would ask this condition for all Čech hypercovers, whether or not they satisfy this condition, but we definitely want our derived category to satisfy, to be, some abs property, so we need to restrict to classifying the Čech for which this holds.

Let me just state this as a remark and not prove it. Sheafification of an accessible pre-sheaf is successful. This is the analog of the theorem of Waterhouse. This means that you can actually form colimits in this category, because sheafification forms the colimit in the category of sheaves. But then you need again to enforce the strong sheaf condition, so you need to sheafify, and you can check that it preserves the set-theoretic assumptions.

Examples or remarks: For any... You can look at the functor which takes B to the maps from A to B. I have a small technical question concerning the sheaf-ification, which I believe you mean, sheaf-ifying it to satisfy this precise condition, that is not the Čech sheaf and not the Čech hypersheaf, but this intermediate condition. Yes, but then in order to construct this...



Unification---it seems that you need to know that the pullback of this kind of hypercover satisfying the condition on the $\infty$-category is also such a hypercover. But again, this follows from the same argument. So this is actually equivalent to $D(A)$ being the colimit along the Lourish maps of the $VA_\bullet$ in the $\infty$-category of presentable $\infty$-categories. This is a condition that base changes, because the $\infty$-categories base change well.

You use all of this---this was discussed in the previous talk. This is all magic about $\infty$-categories. Then you use it, it is the tensor product when you base change.

Okay, so first of all you have all the fine objects. For any affinoid ring $A$, you can consider the object $\mathrm{Spf}(A)$ which takes any $B$ to the homomorphisms from $A$ to $B$. This already satisfies all the required properties, like commuting with products.

Now you have a hypercover, some $B_\bullet$, and then you want to map to $B$. In particular, you want the limit of the $B_\bullet$ to be $B$. So you map to all the $B_\bullet$ and you also map to $B$ itself, because it's the limit.

This brings up the analytic rings. This fully faithfully embeds the analytic rings into the $\infty$-stacks. The accessibility condition is precisely that you're a small colimit of objects in the essential image. This is similar to how we think about condensed sets.

So what is the $\infty$-category of analytic $\infty$-stacks? For any analytic ring, there is an object which is the analytic spectrum. All the others are built by some gluing procedure, by some colimit of these fine objects. The hypercover condition tells you the ways in which you're allowed to glue.

As will become clear later, we want this very general topology because it means that analytic spaces that seem completely different can actually be the same object, just represented in different ways as a colimit of fine pieces. Often there is a geometric picture that they should secretly be the same, and to say they're really the same, we need to use rather strong topologies.




Rings, and then there's a way to extend them to the full class, but some left extension. And then, I think this way can probably show that...okay, okay. 

All right, so there is a fun from like derived schemes. The derived schemes, they also of course embed into like the so-called $\infty$-topos. We can go to taking particular, like any Spec $\operatorname{Spec} A$ for a ring $A$, and take a condensed thing, which is really just the and, um, and all condensed modules. And if you want, you can also, as Dustin discussed, you can basically put the fpqc topos in here, except you have to do this funny countability assumption, only one fpqc.

So basically, like any fpqc stack can also be mapped, and there's really not much to show here, right? I mean, you definitely just have to spin on rings, and then you just have to show that whenever you have a $\infty$-topos cover, it goes to such a strict cover, but that's basically by definition. And again, you could also use this funny thing between sheaves and hypersheaves here in the $\mathcal{C}$-topology here if you wanted to.

Right, uh, so ah, right, maybe I can mention that the pro-étale site is actually fully faithful into the condensed $\infty$-topos view. It's a full back where you actually uses the $\mathrm{Tan}^{*}$ functor. Do you do you need $QC$? I think you can get rid of this because you can certainly reduce to the quasi-compact case, and then there's this argument of offer that you can write as a limit of quasi-separated things. You can how do you reduce to the quasi-compact? Well, you can certainly, if you want to understand Homs from $X$ to $Y$, you can cover $X$ by, but you can $X$ compact, but then comp-wise they can definitely comp. Oh yeah, okay, there other argument that you can make the compact so can.

Alright, so you have some algebraic geometry sitting in there. Uh, next you have Stein spaces sitting in there, which well, these are good from the fromology one, so these are a plus. Like when you talk about Stein spaces, you always assume that this is Stein, and so what I'm saying now, let me also assume this because otherwise I would run to some longer discussion about exactly what I want to do.

And so I can just restrict this to the one plus of the analytic ring $A^{+}$ solid. So note that this way we get a different fun, get two different functions. So there are actually two ways to embed schemes into analytic spaces. You can take a Spec $A$, you can also match this to $\underline{\operatorname{Spec} \mathcal{O}_{A}}$. And then for there's a question in the chat, Peter, it says for derived schemes you're mapping the trivial analytic ring structure, exactly here I'm currently using the trivial analytic ring structure.

Right, no. And so you can either now take this further and look at $A$-modules and solid $\Z$-modules, or you can look at relatively solid $A$-modules. And so now there, and if you wanted to, you could put it right here suitably formulated and put it then also put it derived here. So then there are like three functors from schemes or derived schemes to analytic stacks,

Of course, this immediately suggests a generalization: for any analytic ring $A$, we can consider the "relative" scheme over $\Spec A$, which can be viewed as analytic spaces over $A$. To do this, we need to consider the "condensed" version of $A$, which has an underlying discrete ring. We can then look at schemes over this condensed ring.

For example, we might have the complex numbers with their usual ring structure, and then consider the usual schemes over the complex numbers as a special case of this framework, by just taking the condensed version of the complex numbers.

Now, you mentioned something slightly puzzling about "derived Huber pairs". This refers to a derived version of the notion of a Huber pair, where we have a derived ring with a $\pi_0$ part and all the higher $\pi_i$ parts. I don't want to go into the details of this here, but it is something that can be defined.

In terms of the fullness of the functor, we know that it is definitely fully faithful on the $\F_p$-linear case, because then it reduces to just maps of rings, and we know that Huber rings are fully faithful for analytic rings.

For the general case, we know it's fully faithful under some mild coherence assumptions. But we don't have a full fully faithfulness result, basically because we don't have a good version of an "analytic Grothendieck topology".

There's also the fact that, in contrast to the scheme case, the "adicity" functor doesn't commute with pullbacks. So you have to be a bit careful there. But if you restrict to the "Tate-adic" or "adic" case, then it behaves as expected.

There's also an interesting feature on the right-hand side: if you take a fiber product of $f$-analytic spaces, it's always just the usual tensor product of the corresponding rings. This is not true in the "adic" case, where you can have a fiber product of adic spaces that is not itself adic anymore. There's some subtlety there that can be explained in a nice way using some derived techniques, but I don't want to get into that here.

You can also do this "non-Archimedean geometry" over the real and complex numbers. For example, you can have complex analytic spaces mapping to analytic spaces over the Gaussian complex numbers. The main issue here is that complex analysts usually don't tell you what an "analytic space" is, but they could. Any complex analytic space can be written as a union of what we call "Stein" subsets, which behave very much like affine objects.

What is a Stein subset? For example, it could be the vanishing locus of some ideal inside a polydisc of complex numbers of absolute value at most 1. The algebra of functions we put on such a Stein set is the algebra of holomorphic functions defined in some neighborhood. It turns out that this Stein algebra is excellent, and if the Stein set is actually a manifold, it's regular and has all the nice properties you could hope for.

So when talking about coherent sheaves on a complex analytic space, you can really just talk about finitely generated modules over these Stein rings, for each Stein subset. There is a small caveat, which is that you could have an ideal such that the number of connected components of its zero locus is infinite, and this would prevent the Stein ring from being Noetherian. But for Stein sets that are actually the Riesz closure of a polydisc, the Stein algebra is Noetherian.

Politics is not arbitrary or compact. Okay, I remember that it was for another compact. 

Right, but if you look at complex geometry, there is a very close analog of the notion of an étale subset. Everything really has an algebra, and everything is very similar to how you do rigid geometry. Similarly to the Cech complex, which is actually a cosimplicial object over the algebra of continuous functions on the spectrum of that algebra, endowed with the Gauß-Hecke complex norm. This has a natural topology, like uniform convergence on compact subsets. So this also has a natural topology. You could also define direct and ind settings, whatever. It's a so-called dual nuclear Fréchet space.

Again, this is fully faithful on the analytic structure. You view it as an algebra in the Gauß-Hecke theory, and then you just take the induced analytic structure. So you just check on the underlying ring, which defines a condensed ring that is actually an algebra over the complex numbers. Because it's dual nuclear, it's actually Gauß-Hecke, and you can just induce up the Gauß-Hecke analytic structure from the complex numbers to here.

Not every dual nuclear space is Gauß-Hecke, but these are actually even nuclearly Gauß-Hecke. What is projective? It means some locally closed immersion into projective space, so maybe like an open subset of a closed subset.

Before I go to the next example, there's actually nothing special about using complex manifolds. You can do similar definitions for real analytic spaces, smooth manifolds, or even $C^0$ topological spaces. You take the ring of continuous functions and do the same construction. In each of those cases, you can decide whether you want real or complex-valued functions, and they give you two different functors where one is just the base change of the other.

Now comes a more involved example. This is showing that all the usual theories of algebraic geometry and geometry that are known can be incorporated into this framework. You could also directly define some notion of stacks. You can imagine many possible notions of a complex analytic stack by taking sites with complex analytic spaces and defining a stack over that topology, maybe just open covers.

But now comes the final example, relating this back to condensed sets. Since everything has become a stack, let's take condensed non-étale sheaves. These actually map into the framework we've been discussing. This works even for just schemes, without needing to go analytic. There is something condensed on the right, namely the analytic rings that were condensed rings, but we're not really using this condensed structure here - it comes from the speed objects.

Right, so if $S$ was a finite set, it would just be a finite disjoint union of copies of $\operatorname{Spec} \Z$. In general, as $S$ is the limit of finite sets, this is the limit of these finite schemes. And so then you can access further to the $\mathcal{I}$. 

And here's the reason that we chose this really funny version of between chiefs and hyper chiefs. So, by definition, they are hyper chiefs of a certain profin-topos, the one that we always use. And so to get this, you have to show that if you have any hyper cover here of a profin set by profin sets, then it goes to something for which you enforce the $S$ here. It definitely goes to a hyper cover, and so then there's a small thing you have to check that it actually is the same condition on the $\mathcal{D}$-category.

Basically, the argument that factorially flat maps satisfy the $S$-condition also proves that countably presented sheaves satisfy it. So I should say that here, the lightness condition is again important, because it's always true that if you have a stative map of profin sets, then if you look at the corresponding map of continuous functions, it's always flat for any profinite space. But we need a descendible map for our business, and so we only know that countably presented factorially that maps are descendible, which is another reason that we have this lightness condition.

All right, so here's a remark. There are different ways of embedding schemes into $\infty$-stacks, and now there are also different ways of embedding something like topological manifolds into $\infty$-stacks or the complex or real numbers. Either you can do the thing where you use the algebra of continuous real-valued functions to do this, or you can treat the topological manifold purely as a topological space or factorially into condensed sets and then go from condensed sets to $\infty$-stacks. This gives you a different thing, which is actually defined over the integers, right, because this $\infty$-stack just has integer-valued functions. These are completely different incarnations of a topological manifold, but there's actually again a map between them.

So let's consider the following example. You can take the two-sphere as a topological space and then treat it as a topological manifold, and this gives you an $\infty$-space which, in the compact case, it really is just the $\mathrm{UN}$ of the algebra of continuous functions. And let me work everywhere over the complex coefficients, which category of, let's say, $\mathcal{C}$-sheaves. What else could I do? I could take $S^2$ and treat it as a real analytic manifold and again build an analytic space over the guess, the complex numbers. Or, you could think about all other possibilities, like $C^\infty$, real analytic, and so now you would have here the algebra of--what's a good notation for real analytic functions? $\Omega$, yeah, $C^\Omega$ sometimes it's called. And I don't know, think about all possible other algebras like $C^\infty$ functions, $C^k$ functions, between they would all also give $\infty$-spaces.

But then, you can also like $S^2$ is like one version of $\P^1$ of the complex numbers, so you can also treat this as a complex space. And well, then you want to get something with the guess, the complex numbers. This is not $\mathcal{O}_\F$ anymore, right? I, there are not so many global homomorphic functions here. It is what it is, it's not $\mathcal{O}_\F$, but it's some glued from two copies of $\mathcal{O}$ over the disc, over the holomorphic functions on the disc, right? So you can cover this by two discs, glued along the over-convergent holomorphic functions on the circle. And similarly, this would be glued, but from two copies of $\mathcal{O}$ over the disc, glued along the over-convergent holomorphic functions on the circle.

Okay, so clearly, like making it more



Take again just $S^2$ and $S^3$ just as a condensed set, and go to the space of functions on this complex. So here, I'm using the functor from Lecture 6. Basically, what happens is that here you want to take the locally constant functions on $S^2$. Well, there are none on $S^2$, but if you secretly think of $S^2$ as being the quotient of a profinite set by a profinite relation, which you can always do by this condensed perspective, then on those profinite coverings you do have locally constant functions, and then you can do this gluing. 

Secretly, to evaluate what this is, you have to remember that secretly $S^2$ is a quotient of a profinite set by a profinite relation, and then you can find locally constant functions. Here you have algebraic functions, here you have holomorphic functions, here you have real-analytic functions, and here you have continuous functions---all of them give you different incarnations of what the two-sphere might be.

There's already one really funny thing, which is something that Dustin already mentioned a couple of lectures ago, that one incarnation of the GAGA principle is that it's actually an isomorphism of analytical spaces. So this is one kind of GAGA statement, which is now not just a statement about some coherent sheaves or some derived category of coherent sheaves or whatever, but really before you pass to linear algebra, on the level of spaces, there is an isomorphism.

I mean, this is an instance where you see how the way you build these things is completely different---one is built from just continuous functions, one from holomorphic functions, and the strong work topology that we impose is precisely there to ensure that you can have the possibility of such interesting results.


The C of modules, of course, is just the category of modules of the algebra C. A vector space with an action of the continuous functions. But as here, there are just sheaves of C-vector spaces on the topological space.

One can show that if X is maybe locally compact, then the sheaf has a question: Peter, why do we consider the gas analytic ring structure on the left?

The reason is that otherwise, this is unclear. For example, if you want functions from complex spaces towards analytics, you need the intersections to match up. If you have an intersection of compact line subsets, then the intersections should again be something compact Stein. And you want this to be mirrored on the side of analytics text in our picture. Which means that if you compute the corresponding completed tensor product of analytic rings of functions on D and the other disc, and then glue them together, you should get functions over convergent functions on S1. And this is a computation that comes out correctly in the gas analytic ring structure, but definitely doesn't come out for the usual analytic ring structure, because then you would just take the usual tensor algebraic tensor product, and this is just nonsense. So you need to complete the tensor product, and then it comes out right.

So for X-dimensional schemes, then you can look at the category of Tex, where A is any, and B changes to the spectrum of A. And this is just sheaves on X, sheaves on just X, some reasonable topological space. Those values, so yeah, so you can think of general sheaves as being some kind of functions on an associated space.

I realized before that I should have said that the functors that take any spectrum A towards the A-modules, or also towards presentable $\infty$-stacks, are linear over T, satisfy the send conditions, and hence induce functors on all things analytic. So for any stack, you can define the direct image and descent, and also you can define what is like a sheaf of categories over X. So here it is again the intermediate condition, but for this functor version, between the center type for any. This will actually both of these things will actually be part of some general six-functor formalism, and at some point we will talk more about that. And basically, all design criteria for the GR topology was actually precisely that, that for any analytic d, we definitely want to be able to talk about the C-sheaves. And at some point we also realize it's probably good to enforce that we can also talk about sheaves of categories, and then we are basically taking the strongest possible GR topology where this is true.

What is Perfil? That is the thing which is gotten by descent from the association Spec A goes to the category of A-modules in PERL, which is a 2-category, yes, a 2-category, but we don't have to care, we can just see it as an $\infty$-1 by neglecting the non.

Right, okay, so this is some general examples that maybe we want to cover in more detail in the remaining lectures, but I did want to come back to the point where we started discussing analytics, which was the construction of the T of the curve, and discuss this in a little more detail now, because now we have the language of talking about this.

So let me recall what we want to do. We have this universal ring, denoted Brr, endowed with the topological unit 2, such that when you take this usual Guy, which is a free Guy on a normal sequence, and you tensor it up to the string, then the operator that's 1 minus Q* shift, so shift is the endomorphism of P that just shifts all the integers on S, and this was this ring structure we introduced some weeks ago. And where an on-computation was that you could actually compute what's the underlying string of this was, and it was this algebra of Laurent series in Q, which has a certain funny to normal condition on their coefficients. So the underlying condensed ring is this algebra, but maybe the underlying ring of the underlying condensed ring, because now I'm just telling you a set, maybe.


Did I tell you the current condensed structure on this? Okay, so this is some funny ring of formal power series, integral series, with rather strong growth of the coefficients. And then we recall that the goal was to finalize the analytic curve over $\mathbb{A}^1$ or really some $\mathbb{G}_m^\triangle$.

This would be a scheme such that if I take the incarnation of $\mathbb{E}_Q$ as a scheme over $\mathbb{A}^1$, this can be written as a quotient of a $\mathbb{G}_m$ over $\mathbb{A}^1$ under multiplication by $Q$ as an operation. And I already a couple lectures back gave the outline of how I thought it should go.

The first step is to see that there is a certain kind of norm, and I will talk about this more in just a second, which goes from the $\P^1$ array, but really the $\P^1$ incarnated as a scheme over $\mathbb{A}^1$, toward the infinity. Intuitively, this tells you how large a point is. Having a point here is basically having a point here, so maybe nothing, think from a field or something like this, and then an element of that residue field. So there's something like saying that on any residue field, there would be a norm $\Z \to \Z$, like the absolute value, but now this map here is really meant to be a map of analytic stacks.

The left-hand side is now an analytic stack. We can take $\P^1$, make it into an analytic stack by this general functor that gives it a trivial ring structure, but then we can base change it to $\mathbb{A}^1$. And maybe here I should write that I mean, so here you're incarnating this by treating it as a condensed set, and we could or could not base change it to $\mathbb{A}^1$. Yeah, we can. You have to choose some value for the absolute value of $Q$, like say $q$. So it should stand for some $q$, $q$ is a section of this, and it's multiplicative in fact, is a unique such thing with the properties I will mention. 

So it should be multiplicative in a sense I will make precise in just a second. So I won't talk about these norms second. Let's say, to make it unique, I should precisely specify where $q$ goes. Meaning, like $q$ is actually a section from the spectrum of $\mathbb{A}^1$ back into this, and this should go to a constant value, something like half. And then you can define the analytic $\mathbb{G}_m$ array as a subset of the $\P^1$ array as the preimage of an open subset $\Z\setminus\{\infty\}$. And then you still have $q$ acting there, and then can take the quotient by $\Z$. And then, by an argument I already sketched last time, this is basically a projective curve. And then you have to prove the zero-dimensional algebraization theorem that would apply to all projective curves with a section, at least, that it's algebraic, so it is in the image of different schemes over the underlying ring.

Towards the end, I mean, you need to make the line, it's not clear even geometrically, but here it's okay. We discussed this, you need some GAGA or exact GAGA, and something which is again where is a soft, soft. Well, in our approach to GAGA, you don't really get SAGA, which is much more general than GAGA, but it doesn't really tell you anything like Riemann-Roch for something you don't a priori know is algebraic. I mean, so that's kind of a separate issue.

Right, so one key notion that comes into this is a notion of norm that we want to have here. So let me actually define this in general. It's a slightly awkward notion of a norm $\mathcal{A}$ that is not really a norm on the underlying set of $\mathcal{A}$ or anything like that, but it's rather


One other thing I want to mention is about the notion of "norms" for elements. We will also follow something, and here's how I want to phrase that. So inside the $\P^1$, I can first look at the locus $\mathbb{A}^1$ where I didn't really have just functions, but then I can look at the locus where the function is "near-potent", and that is some locus given by the zero locus of a certain polynomial. And then you join the two variable, so in other words, it's again just this $\P^1$ treated as an algebra. This maps to the algebra $\mathrm{UNP}(a)$, which is just another ring. Instead of taking the norm for some $s$, I want to say that this composite factors over the closed interval $[0, 1]$, not the half-open one. It turns out that the better notion is this one, for reasons I can't fully explain yet, but it's the one that behaves well.

I should also say that such a factorization here turns out to be really just a condition, because there's really a monomorphism, and similarly for these other conditions like zero goes to zero, I want to say that it factors over the subset $[0, \infty]$ again, that's just a condition.

Alright, so these are some first properties. But then when you have a "norm", you usually ask for some version of multiplicativity, and also some behavior with respect to addition. It turns out we just forget about the addition part, but we keep the multiplicativity condition. However, I'm not working on $\mathbb{A}^1$ but $\P^1$, and the reason I work with $\P^1$ is that even on $\mathbb{A}^1$ I would want to allow some functions that could have infinite norm, and if I'm allowing infinity on the target of this map, I might as well allow it on the source as well.

Now for multiplicativity, I have to be slightly careful about handling the zero times infinity cases. One way to do this is as follows: I consider the locus $X$ inside $\P^1 \times \P^1 \times \P^1$ where $XY = Z$, and similarly there is a smooth surface $\mathrm{xar}$ inside $\P^1 \times \P^1$. On the open part, you can look at the incarnation on the level of schemes of this closure, and one incarnation just on the real numbers, and then you have $X$ incarnated as a schematic morphism to $\P^1$, incarnated as a choice, and this matches to $\Z_{\infty}^3$, and inside here you have this $\mathrm{xar}$.

I wonder if it would be the same as asking first that the map giving the norm is equivariant for inversion on both sides, and then just asking for multiplicativity when you restrict to $\mathbb{A}^1$. I think so, yeah, that was kind of the definition I thought we were using, but maybe...

Yes, but even on $\mathbb{A}^1$ you have to be careful, because $\mathbb{A}^1$ can map to $[0, \infty]$, it could map to $\infty$. Oh, I should have said that I should have said the pre-image of $0$ instead of $\mathbb{A}$

Might the norm of two be unbounded? Usually, the absolute value of two is less than or equal to two, which implies a triangle inequality. However, there could be an example where the absolute value of two is infinite. 

On this Gaussian analytic base, I will try to construct such a norm in the remaining minutes. In particular, we can then look at what the norm of two is. It's some function from the analytic spectrum of a Gaussian towards the extended integers, and it's surjective. 

The method I already stated on the board is that there is a unique norm on the Gaussian space, taking some number between zero and one. It doesn't matter which one, because you can always rescale the target by some exponential and still be alright.

Let's say x is some kind of finite-dimensional locally compact space, and y is then I claim there's some kind of Tannakian duality describing what maps from y to x are. It turns out that to give such a map from y to x, it's just enough to give a functor from sheaves on x towards sheaves on y, with some property. But actually, you don't really need to specify the values of all.

Let me first see. This is the same thing as the functor F from the category of abelian groups on x towards the category of abelian groups on y, such that these functors are compatible with all pushouts and pullbacks, and they should be linear over the base field. The condition is that locally on y, the following happens: on x you have lots of important algebras, namely for any closed subset of x you can take the constant sheaf. This should be connected, and the image under the functor F should remain connected. In general, giving such a map, you can always do it locally, so the condition you have to enforce is just some strict local condition.

It turns out that this is the condition that locally on y, the pullback or the sheaf locally for the topology remains connective. And note that the full category of sheaves on x is actually generated by these guys on the closed subsets. So to define the standard functor, you really only have to declare these important algebras. Describing the map from y to some such guy here is completely determined by specifying, for each closed subset of x, an important algebra, and if they're all already connected, then you're good to go. The only thing you have to check somehow is that the Tor product behavior of these important algebras exactly matches the intersection behavior of the closed subsets of x, and that's what it means to be a tensor.

This is actually a set, even though a priori it's an analytic object. To realize this space over the appropriate category, we need to find some important object. Typically, you have to find something that should correspond to the pre-image of an interval from zero to some number $R$. If you know where these pre-images should go, then due to the compatibility on the involution, you also know where intervals going from somewhere to infinity should go. Everything else can then be written as some kind of colimits and intersections of those.

Really, to describe the SC norm, you just have to say what is the pre-image of the interval $[0, R]$, in other words, what's the locus where the norm is at most $R$. This should be the analytic spectrum of a certain ring, generated by sums of $n \cdot T^n$ that converge for $R' > R$. 

To produce such an object, you can start with a sequence in $A$ and multiply it by suitable powers of $Q$ to make it satisfy the desired condition. This can be described as a colimit of $T$ subject to a condition on $R$, which can be explicitly written down in terms of the Gauss algebra. The key idea is that the absolute value of $Q$ should be $1/2$.

If you form certain tensor products of such algebras centered at 0 and $\infty$, the tensor products will behave in the way you'd hope, matching the intersection behavior of intervals from 0 to $\infty$. This is a computation that relies on working over the Gauss algebra.

I should probably stop here, as I'm running over time. Let me know if you have any other questions!

Theories, yeah? So my question is, can you define an analytic space theory using your language? Because if I wanted to put an additional structure in an analytic space, perhaps I want a general definition for all analytic spaces - put the structure there. That's exactly what we're doing; that's what an analytic stack is. So you can say, "I have an analytic space theory, and this reduces to each known case."

Well, I - are such analytic space? Analytic space theory? I don't know what - yeah, I have to know what you mean by that. I mean, like something that would generalize all the... Yeah, that's what we're - that's what we're trying to do with this concept of analytic stack. Yes, you shouldn't be scared, it shouldn't be put off by the fact that we change the name from "space" to "stack" - that's basically a technicality, but that's the goal.

So you have a definition, like "an analytic space is an analytic stack that..." Well, you could try. Yeah, that's my question - like, do you have a definition, like "an analytic space is an analytic stack satisfying some axioms"? Do you have... Well, we have too many - maybe. Well, what I mean is, like, do you have the ultimate one, kind of like the umbrella one? From - we we tried, but we could never...

I mean, you could just say that the functor of points takes values in sets. No, F-ones are not okay, sorry. What if you ask that it takes values in sets? And even fine analytic spaces are not - not fine. Well, yeah, but the classical ones are. So anyway, the classical, the F-line, like, takes a ring to its underlying... Because the test category consists of derived things, you can... Yeah, yeah, that's true, that's true. No, but anyway, you don't have a definition of analytic... But I mean, you could ask that it's the counit of F along monomials. That's the - sorry, what's the condition? You could ask that it's the counit of F along monomials.

Okay, okay, thanks.

\end{unfinished}