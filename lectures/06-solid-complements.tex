% !TeX root = ../AnalyticStacks.tex

\section{\ufs Complements on solid modules (Scholze)}

\url{https://www.youtube.com/watch?v=KKzt6C9ggWA&list=PLx5f8IelFRgGmu6gmL-Kf_Rl_6Mm7juZO}
\renewcommand{\yt}[2]{\href{https://www.youtube.com/watch?v=KKzt6C9ggWA&list=PLx5f8IelFRgGmu6gmL-Kf_Rl_6Mm7juZO&t=#1}{#2}}
\vspace{1em}

\begin{unfinished}{0:00}
So today I want to, before Dustin takes over on Wednesday again, give some complements on solid modules.

Sorry, just briefly recall the definition. So $P$ has always denoted the projective object which is, as a free $\Z$-module, $\Z^{\mathbb{N}}$, the $\mathbb{N}$-sequences. And then we consider this endomorphism of $P$ which is the identity minus the shift $n$ which takes the generator $n$ to $n+1$.

Then the new definition of a solid module was, though entirely analogous to the old definition, but on $P$-modules. So a $P$-module $M$ is solid if it is, by $f$, an internal $\mathrm{Hom}$ morphism, where intuitively speaking this is a map from $P$ to $n$, the space of $\mathbb{N}$-sequences in $M$. So you take an $\mathbb{N}$-sequence in $M$ and map it to the sequence of its differentials. Intuitively speaking, that this is an isomorphism means that conversely, if you have any null sequence in $M$, then you can sum it. Because if you think about it, then this $m_0$ must be the sum of all--

So this looks like some definition of derived complete modules, like $p$-adically derived complete. I suppose there is some mathematical relation to derived completeness. Let me talk a little bit about that.

I mean, I think on the face of it, it's not related, but okay, there will be some derived completeness today.

And so then the theorem we proved last time was that this category of solid $P$-modules is stable under all filtered colimits (let me just say limits or colimits), under tensor products, and also under internal $\mathrm{Hom}$, making the left adjoint to this inclusion an "absolute solidification", making this left adjoint a symmetric monoidal functor. And it is generated by a single compact projective object. This projective generator is $P$ itself, and on this generator the solidification functor is the identity.

And maybe the other thing I should say is that one can compute, for the solidification of all these free $P$-modules on a profinite set $S$, they are just the limit of the free $P$-modules on the finite quotients of $S$.

Okay, so I think this basically summarizes everything from last time.

Maybe the first thing--okay, then let me first discuss the derived enhancement of the situation.

So let's assume that instead of just considering $P$-modules, you really have a derived $P$-module complex. Then actually the definition still makes sense, and we take that as a definition. So we say that a complex of $P$-modules $M$ is solid if, again, the internal $\mathrm{Hom}$ from $P$ to $M$ is an isomorphism. But again, because the internal $\mathrm{Hom}$ from $P$ is actually exact, this is actually equivalent to asking that for all $i \in \Z$, the internal $\mathrm{Hom}$ from $P$ to $h^i(M)$ is an isomorphism. So this is equivalent to asking that all the homologies of $M$ are also solid.

And again, this is also true in our previous discussion of $P$ being $\Z$-modules, and is in fact a general property for what we call an "analytic ring structure". That you can check completeness at the level of homology groups--classically this is a little subtle to prove, but here again it's an immediate consequence of the internal projectivity of $P$.

Okay, so then some corollaries of this. The class of solid complexes is itself a triangulated subcategory, or if you would work in the $\infty$-categorical setting, it would be a stable $\infty$-subcategory, and stable under all direct sums and products, so all limits and colimits.

And one can again show it's also stable under $R\mathrm{Hom}$, again from the same proof as last time. And then if you just repeat this discussion of tensor products now at the level of derived categories
Not completely automatic, this uses that these projective generators here stay---they somehow have expected vanishing also in light condensed abelian groups. Or, rather, that's the solidification actually here, okay.

Is it the case that if you have any complex, possibly unbounded, whose terms are direct sums of those generators, these generators, then taking solidification term-by-term is a good derived functor? That it gives you the---yes, that's right. Solidification preserves, like, on the level of---maybe forget about the thing.

So then you have this whole subcategory in the derived category of solid objects. And again, by adjunction, you will actually---it will have a left adjoint, some kind of derived solidification functor, which again will commute with all colimits back into $D(Cond(Ab))$. And then this, on the generators, we kind of determined what it does last time, that it becomes this thing, even in the derived setting.

Yeah, okay. Also, on the level of derived categories, we have a full inclusion, really. So it's as nice as it could be. And again, by some adjoint functor theorem, this has a left adjoint. Let me write this as "derived solidification $L$-box".

So in the adjoint functor theorem, usually there is some set-theoretical condition, because you want---I'm thinking of everything as $\infty$-categories, and then everything here is a presentable $\infty$-category. And then $L$ has this general adjoint functor theorem there. I think there are, for triangulated categories, theorems of Neeman and others which would also justify this, I believe. But I mean, they often assume actual compact generation, which is not quite true in this case, so I'm not sure. But the $\infty$-categorical version is definitely sufficient.

Okay, so for before, we wouldn't need the abstract theorem, because we can just prove the existence by hand in this case. Because on a class of generators, we can---we have identified that these guys already know that they exist. But the abstract theorem is good for both left adjoints and right adjoints, assuming that there are enough limits and colimits, right?

I mean, so the notion of presentable $\infty$-category is one which has all colimits and which is generated by a set of objects. And in this situation, every functor that preserves all colimits has a right adjoint, and every functor that preserves all limits and all sufficiently filtered colimits, so it's accessible---also has a left adjoint. And there's a unique symmetric monoidal structure making $A \mapsto R$. And actually, something even better is true, that this is literally the left derived functor of the thing you have between abelian categories. And similarly, you can also---I mean, the derived tensor product is also obtained by deriving the thing you already have on the abelian level.

But here, how do you know that the solid category is presentable in---before you proved the existence? Yeah, I think there are general theorems that if you just ask locality for such a map, there are some automatic presentability theorems, okay.

So once we pass to the derived setting, I can state things from last time. That really, the derived solidification of $\Z[S]$ is still $\underline\Z[S]$. So it is---under solidification, there are no higher extensions. And so also, the derived solidification of $\Z_p$ is this perfect $\Z_p$. And the derived tensor product of $\Z[S]$ with $\Z_p$ is the condensed ring $\underline{\Z_p}[S]$, okay.

So you might think that nothing derived ever appears in practice, but not so. Here's a very funny proposition. Let's say again that $X$ is a CW complex.

So recall that in this situation, we previously had this thing that the $X$-valued points of $\underline\Z$ are the singular chains, right? But now we can also interpret the homology of $X$. And it turns out that the singular homology of $X$ is exactly the homology of the derived solidification, more precisely.

I mean, there's some complex computing singular homology, and this is really isomorphic to the derived solidification. In particular, the $0$-th soli

In general, if you take the union of circles of all dimensions, you get something which is concentrated in degree zero, but solidification goes arbitrarily far to the left.

So again, there's the formal reduction to the case of complexes---just assuming I keep, which, and so for a finite CW complex, how do we actually compute this? Assume $X$ is a compact and finite CW complex. We know this homology is finitely generated in each degree.

I may miss the point, but you said that the derived solidification of $\mathcal{G}(X)$ is a singular complex, right? So it is a quasi-isomorphism? Ah, quasi-isomorphism, sorry. I see, sorry, it's only a quasi-isomorphism. Yeah, that does make sense. Sorry, I see.

Okay, so how do we compute the solidification? We use those things. We take some surjection from some profinite set onto $X$, like the constant map, and then we get all fiber products. This gives a resolution of this condensed sheaf by the constant sheaves, and then the constant sheaves on the fiber products, which are again some profinite sets, and then all the fiber products.

Okay, so that's the resolution of this guy. To form the derived solidification, we can apply the solidification to all the terms here. But for these terms, we know that it's concentrated in degree zero. We know that the derived solidification is computed by this, let me not underline.

Now, this might seem a bit hard to compute, but we know that each term here is actually just $\mathrm{Hom}_{\mathrm{cts}}(S, \Z)$, the continuous functions from $S$ into $\Z$. That's something I discussed at the end of the last lecture. I mean, this formula that this limit of the $S_n$ can be translated into this formula. So these are some kind of measures on $X$, and similarly for the others.

This actually means that all the terms in this complex are isomorphic to their double dual. The continuous functions from $S$ to $\Z$ is the dual of this one, and then I dualize again. The solidification is actually just $\mathrm{Hom}$, and I could also put the $\mathbb{R}$ here, it doesn't change in those cases.

So this actually means that the same formula will be true for this guy, if you think about it. Because it's a resolution, this means that also this derived solidification here, one way to compute it is that it's isomorphic to $R\Gamma(X, \mathbb{R})$.

But now we're in business, because we know that this guy here is isomorphic to the singular cochain complex. Because we know that the singular cohomology for finite CW complexes is finite in each degree, it's taking the dual of singular cohomology against singular homology.

Okay, and so you get that the derived solidification of $X$ is the dual of singular homology. This means that to some extent, this passage from condensed abelian groups or solid abelian groups is like passing to homotopy types, a little bit like, at least for CW complexes, it contracts the interval and then identifies two homotopy equivalent CW complexes. But on totally disconnected things, it's much finer information. You could still invert homotopy equivalences, but not invert weak homotopy equivalences.

Is it the case for locally contractible spaces, in the sense that you explained? Yes, so whatever I said about CW complexes also works for a locally contractible space. Actually, Zhouhan pointed out to me that the paracompactness assumption is actually not required in comparing singular cohomology and sheaf cohomology. No, they are not required. There are some recent papers to that effect. But I don't care in some sense right now.

Okay, so now let me go back and try to describe the category of solid abelian groups a little better. We want to understand the structure of solid abelian groups better.

So there is a compact projective generator. There's definitely a notion of finitely generated objects, and these are precisely the $C(S)$ for $S$ profinite. And then there's a notion of finitely presented objects.

These are the cokernel of maps, and it turns out that the modules behave like modules in the following sense. They are presented objects, form themselves in a category, and the only critical thing is stability under kernels. But once you have that, it's automatically stable under kernels, cokernels, and really only the kernels are something to mention.

The whole category is just the ind-category of those. Also, any finitely presented object actually embeds into a cokernel of an injective map. This is actually something slightly better than what [name] announced in the first lecture. He claimed that the finitely presented objects have a resolution of length two, but actually length one is good enough.

There's a resolution that just stops. Okay, so there are some finite objects in your category which are exactly the cokernels of injective maps, and then everything is a kernel of a map from one of those.

The key step for this is the following. Trying to understand all the finitely presented ones, okay, you have a kernel. You can always find a surjection onto M, and then there's some kernel. The kernel, you still know, is at least a finitely generated submodule of this. So then it would be good to know that all the finitely generated submodules of this are actually such a product. This is what I claim now.

Zero? No, sorry. Let's say product of copies of Z. Generically it could be a product, of course it could be zero.

This definitely implies this statement here, but it also implies, like, okay, so objects are always stable under cokernels and extensions. Things like they're stable under kernels. But then, I mean, this also always reduces to identifying the finitely generated submodules of your generating objects. And if those are all finitely presented, then you're in good shape.

So the theorem is really easy to deduce from---okay, so let's prove it.

We know that it's finitely generated, so we know that there is some surjective map from a product of copies of Z onto M which injects back into [word unclear].

By the way, this is again the result. Previously, a lot of what I was talking about, about the general properties of the categories being Grothendieck abelian, they basically extend to the full condensed setting, not just the line one. This theorem is again one which only holds in the line one setting. And because we're again using some countability in just a second.

Okay, so we have some here. Let's call it g.

We know that the maps from a product back to Z, they are given by direct sum. So we actually know that this is dual to a map in the other direction from a direct sum of these. Basically, our task is to show that whenever we have such a map of a direct sum of these, and when we dualize, then this image here is itself a product.

For the proof, I will use the following fact, which I'm not sure if it's that well-known.

Let N be a countable group that maps into a direct sum of copies of Z. I mean, it's just an abstract---yeah, discrete, no condensed sets---maps into Z.

Maybe, okay, let me leave this as an exercise. I know it's definitely false if you drop the countability assumption, because if you take an arbitrary product of copies of an abelian group, it's definitely not a free abelian group.

But one thing this implies is that if I look at the image of such a map, then this is definitely countable, and it embeds in the direct sum. This also embeds into the direct sum, which implies that the image of the map is free.

For this, you don't need this [previous fact]. It's just the image. Okay, I will use this [previous fact] again in a second, so I wanted to mention it.

The image is free, but then you have a surjection here from this direct sum onto another free group. So this splits; in particular, there's a kernel here, but this must actually split back. So the image of the map splits as a direct summand.

I mean, maybe actually this [previous discussion] is completely irrelevant. Maybe I should focus more on the question at hand. But let me just try to understand a little bit about the structure of such maps and what it implies on the dual side.

So it kind of splits, but this means that you can basically replace this by the other direct factor, which is also [sentence cut off].

Freyd and replace $H$ by the corresponding thing, because then this corresponding product will split into two copies, and this $N$ will just embed into one of them. So we can definitely assume that $H$ is indexed.

Okay, so now we have an injection. I'm going to direct-sum these, and now we have some quotient here. The next thing is to understand the structure of the quotient.

There is a certain quotient of $Q$, which is the image when you embed it into a double---sorry, let me just write it as a product over all possible maps from $Q$ into $\Z$. So you can look at all possible maps that $Q$ maps to a free group, or just to $\Z$, and take the corresponding maps to $\Z$. Then there will be some quotient here, and by the fact that this guy is actually free, this means free or in particular projective, so it splits as a direct summand.

Also, if you look at the torsion of this $M$, then it's easy to see from this that this $Q'$ will have no more maps to $\Z$. Because if this had a map to $\Z$, then because this is a direct sum, there's another map from $Q$ to $\Z$, which would make $Q$ okay. So this means that this has this quotient $\overline{Q}$ which is free, and so this splits back here, but then also back here if you want. So then you can also get rid of that summand.

Okay, so without loss of generality, you can replace $Q$ by $Q'$, and then there are no more maps to $\Z$. Those are without loss of generality. $Q$ has no more maps to $\Z$.

A different way to think about what I'm doing here is to observe that the category of countably generated free abelian groups has core kernels, and this $\overline{Q}$ would be the core kernel.

Okay, so where are we in the proof? We're trying to show that whenever we have a map from a direct sum of $\Z$, then once we dualize, the image of this map is itself a product of copies of $\Z$. Now we split off the kernel, we split off part of the core kernel, and now we're down to a situation where we have this map and $Q$ has no more maps to $\Z$. We're trying to understand what happens if we dualize.

So we have here a product of copies of $\Z$. This map here is our old map $D$. Okay, sorry, this is precisely the internal $\operatorname{Hom}$ from $Q$ to $\Z$, and then the image would be actually $X_1$.

Now, we ensured that the $\operatorname{Hom}$ from $Q$ to $\Z$ is zero, but actually it turns out that the internal $\operatorname{Hom}$ from $Q$ to $\Z$ is zero. The points of $\operatorname{Hom}$ from $Q$ to $\Z$ join into $\Z$, which by using the adjunction in a different order, is the $\operatorname{Hom}$ from $Q$ into the continuous homomorphisms from $\Z$ to $\Z$, which is a discrete guy and where this guy is solid. 

But $Q$ has no maps to $\Z$ and just no maps to any abelian group. Okay, and so what does this mean? This means that the image of $D$ is actually just this product of $\Z$s. So some of the reductions we made in the beginning precisely ensured that actually our map $D$ became injective.

Okay, I guess strictly speaking when I did some of these reductions, it could have happened that an infinite direct sum became a finite direct sum. But just take the direct sum with an infinite thing, whatever.

Okay, so that's good.

Right, so actually call
The light condensed rings are true again, which is also kind of the reason that we never saw any obstruction to this in actual mathematical practice.

We need for all solid $\Z$-modules, the product is solid. To see this in generality, we actually have a resolution of length one. But now, if I tensor this with a product of copies of $\Z$, then this map stays injective. I mean, actually, what you really see is that if you take the derived tensor product of this, you actually just get a product of copies of $\Z$, because you can just write this as a finitely presented guy.

So one thing that's really nice about this whole tensor product is that it really gives a lot of---I mean, many, many computations come out right, sometimes in non-obvious ways. Just like maybe previously, a lot of Ext computations came out right in non-obvious ways.

Let me do some tensor computations. Often, you maybe care about things like: you start with some $N$ being a $\Z$-module and $G$ being a group, and then you form the $p$-adic completion of them. In such situations, one often uses the completed tensor product, which is like the completion of the usual tensor product, if you do some kind of formal geometry or something.

Here's a theorem that is actually what the solid tensor product does. Actually, in full generality, for any abelian group, it's better to replace this by the solid derived $p$-adic completion, which is the derived $p$-adic completion operation. Here, this derived thing just means a complex represented by multiplication by $p$, where this sits in degree zero and this sits in homological degree minus one or homological degree one. Almost always, this agrees with the usual $p$-adic completion in degree zero, but in complete generality, that's a better operation.

Generally, it makes sense to talk about derived completions---I mean, these things are the derived complete modules. Here's a proposition, and because we're doing this, it's slightly better to from the start work in the derived category. Let me assume I have things which are concentrated in non-negative degrees, so things go to the left.

Say I have two of them, $M$ and $N$, and they are $p$-adically complete. So there are isomorphic to the derived $p$-adic completion of $M$ tensored with $\Z/p^n$ over $n$. Obviously, it's actually also a condition you can check on homotopy groups.

Then also their solid tensor product is the derived $p$-adic completion of their usual tensor product. Let me prove this in a second, but let me just note one corollary, which is, for example, that if you take an infinite direct sum of copies of $\Z$, take the $p$-adic completion of that, and then take the tensor product of that with such a $p$-adic completion, you just get the similar $p$-adic completion.

This means that in such situations of more or less formal geometry, for derived $p$-adically complete things, the solid tensor product does what you would expect it to do. There's really nothing special about the number $p$ here---I mean, you could work with any base ring and solid module structures over that ring, and then any element of that ring.

So this $p$-adic completion of the free abelian group, you consider it as a solid abelian group, or as a condensed abelian group. It is solid because the class of solid modules is stable under all limits, and so on. You start with something solid like $\Z$, take a direct sum---it's solid. Take the $p$-adic completion---it's solid. Take a limit, and so on.

Remark: there's nothing special about $p$. For any ring $R$ and any elements $f_1, \ldots, f_n \in R$, and solid $R$-modules $M$, $N$ which are derived $(f_1, \ldots, f_n)$-complete, then the derived tensor product of $M$ and $N$ is the derived $(f_1, \ldots, f_n)$-completion of their usual tensor product.

I should mention this is actually a corollary of the preceding proposition. We know there's definitely a map from here to here, because the soli
To check whether we can actually check after reduction mod $p$. Because for derived completed things, this can be checked modulo $p$. But modulo $p$, both sides just become the tensor product, and you're just taking the usual tensor product of FP's and product, a direct sum of FP's, and a direct sum of FP's.

Okay, so let me give a sketch. First of all, maybe in this situation you can actually work with $\Z_p$ instead, because the derived completion of $\Z_p$ with respect to $p$ is just $\Z_p$. That's because you can take $\Z[[T]]$, $\Z$ power series in $T$, and then you quotient by $T-p$. And mod $p$, you get here $\mathbb{F}_p[[T]]$, but also when you take this power series in two variables and quotient out by $p$, then you get $\Z_p[[T]]$ quotient by $p\Z_p[[T]]$. So this actually means that the derived category of solid $\Z_p$-modules is automatically a full subcategory of solid $\Z$-modules. So being a $\Z_p$-module is not a structure on a solid, it's just a condition. And if everything's complete, then everything becomes a $\Z_p$-module.

So $M \otimes N$ is actually here, and everything is taking place in the subcategory. Okay, and let's actually---let me do the case which is maybe the most destructive case, where $M$ is really just, take a direct sum of copies of $\Z_p$, but you complete the direct sum. Then in order to compute what the solid tensor product does, we have to write this in terms of our generating objects. And this is actually a slightly non-trivial exercise, because there is actually a rather large collection of compact projective objects.

So what's happening here is that this actually is a colimit over all functions from $\mathbb{N}$ to $\mathbb{N}$ which go to infinity, so they only take finitely many values below some constant, of the product over $n$ of $\Z_p$ to the $f(n)$. So what's happening here? First of all, why is there a map? So whenever you have a map $f$ which goes to infinity, you can take the product of these copies of $\Z_p$ to the $f(n)$, and this actually maps to the completed direct sum, because modulo each power of $p$, almost all of the terms in this product will go to zero. That's because $f$ is going to infinity.

And quite obviously, this is also injective. But then you still have to show that it's a surjection. So you have to show that whenever you have a map from a compact object to this completed direct sum, it actually factors over one of these terms. But if you have a map into this completion, then pick any test object $S$ and a map from $S$ to this copies of $\Z_p$, which is by the definition the limit. And so we have, I don't know, $g$ here, and so we have a collection of $m_{g,n}$ here. And then each $m_{g,n}$ comes from some direct sum over integers at most some $a$ of $\Z_p$, right? Because this is a compact object, and maps out of a compact object are finitely generated.

So okay, so let me go on. So there are just a bunch of elements where you get something mod $p$, and then there is maybe a larger bunch where you get something nonzero mod $p^2$, and then there's something even larger where you get something nonzero mod $p^3$. But then you can just find some slowly, very slowly increasing function $f$ which factors over this product. Right, so what's my diagram here? These are my $m$, these are my $m_n$, and okay, this is $f(1)$, this is $f(2)$.

So this means that this tensor product of $M$ and $N$, each of those, is a product of $\Z_p$

And now this has an obvious map, so $f$ on it, over all functions. Let me call them $H$, which are now functions on both coordinates, which just go to infinity. Yeah, $H$ is a function of two variables, and then you put the function of $H$ and $n$ here, where this is completed.

Now, at first you might think that this will surely not work out, because here you're allowed to quantify over all arbitrary slowly increasing functions of two variables, whereas here you're just getting those that are sums of functions of one variable individually. But then there's just---maybe at first slightly surprising but not that hard to prove combinatorially---that whenever you have such a slowly increasing function $H$, you can always find one that's even slower increasing, which is a sum of two functions of individual coordinates. I'll leave it to you as an exercise to figure that out. But that's what I mean, that the softer approach, like for non-obvious reasons, gives you the correct awesome.

And so, yeah, general argument is saying you can reduce to the case that $n$ and $m$ are just some completed direct sums of copies of the generators. And then, although the same argument---you actually have to be slightly careful because a priori, you can also have the case that $m$ and $n$ are completions of a direct sum which is over an uncountable index set. And then you still want that. But then there's actually an argument that the uncountable case just, by more formal arguments, reduces to the countable case. Because these derived completions, they always commute with, I mean, you can always reduce uncountable things to countable things. Just anyway.

Right, so this is a nice computation. There are some even further computations that come out, which really appear when you do some solid functional analysis. So let me work over $\Q_p$ for simplicity, but most of what I'm saying works over any field.

So then again, the solid $p$-adic Banach spaces, or even the derived category, it's just the full subcategory of $p$-modules, and those where $p$ is invertible. And it's the full subcategory of abelian groups, and if you form tensor products and direct sums, it's in the subcategory.

And this category itself, it now has a compact projective generator. Well, here it will be a product of $\Z_p$'s, and here you have to invert $p$, take $(\Z_p)^\vee$. This is a slightly curious kind of topological vector space, or like a condensed vector space, but it actually comes from a topological vector space. So it looks like one where, somehow, the unit ball is this thing, where the unit ball is compact. But actually, there's this general thing that you cannot have Banach spaces or normed vector spaces of infinite dimension where the unit ball is compact.

So it actually turns out that if you were trying to endow this with a norm in the kind of obvious way, where this would become the unit ball, then the norm map would not be a continuous map. And so, yeah, it's not really a normed vector space, it is what it is.

So these things have appeared a little bit also in the classical functional analysis literature. Maybe first Lefschetz called them Smith spaces. And she was studying the analog of those things over the real numbers. They can also consider some type of topological vector spaces, which are well-defined unions gotten by scaling out some compact convex set.

The more usual objects we consider are Banach spaces, and at least separable examples of those are exactly the ones that we considered previously. You take countable copies of $\Q_p$, complete them, and then these things are actually in reality Smith spaces. So either you put "separable" and "light" on those sides, or you... So "separable" on the Banach side means you can take a countable direct sum, and "light" means on the other side, you take a countable product. And this is just the obvious: you take a Smith space and dualize it, but you can also take Banach spaces like so. Yeah, so the generator that we use here, the $(\Z_p)^\vee$, is sometimes dual to Banach spaces. Okay, we're basing everything on them.

Here's again a curious remark, and maybe people will
Just the internal dual, that's true and kind of obvious. But if you go the other way, you can wonder: if you take a b-space and take the dual of that, is it just the usual dual? And the answer is, it depends on your model of the set theory. In the same way as this came up before, it fails under the continuum hypothesis, but under this principle $*$ that I mentioned (that is consistent), it's true. Because a b-space is a special type of pro-space---a limit of direct sums over $\mathbb{N}$ of $\Q_p$-mod-$p^{t(n)} \Z_p$. And so this duality between pro-things and ind-things is precisely the thing that's controlled by principle $*$.

Yeah, the more I think about it, the more I'm actually tempted to just use a set theory model where principle $*$ is true. Because you have the choice of either these higher Ext groups being some junk that really has no meaning at all, or them being zero.

Okay, but other than that---this question was never really relevant for what Dustin and I have been trying to do. Sometimes you run into this thing. I mean, it's definitely like, if you do some kind of computations now within this kind of solid functional analysis, you're doing lots of homs, lots of tensor products, lots of homs, and so on. At some point, surely you will take some RHom of a b-space against something in $\mathcal{D}$ somewhere. And then you can decide whether you work in a model where it's just some junk, or you work in this model where it's what it's supposed to be.

So, what I really wanted to say is, you also have fréchet spaces, which are countable limits of banach spaces along compact maps. These really pop up a lot. And there is a notion---a standard notion---of a completed tensor product for them. In particular, it's the one that's compatible with limits. So on b-spaces it's the usual projective tensor product, and when you have fréchet spaces (in such a countable limit), then it's the corresponding limit of completed tensor products of b-spaces. A general thing that usually happens when you define such tensor products in functional analysis is that you're trying to make them compatible with limits. For example, in this case, which is not related, because our solid tensor product was defined to be compatible with colimits. And usually there are very few functors that are compatible with both limits and colimits---certainly the tensor product is not compatible with all limits. But fortunately, it so turns out that in this situation, it does give the correct answer. So here's a proposition:

If $C$, $M$, $W$ are fréchet spaces (let's say considered as topological spaces for the moment), then you can pass to the condensed world, take the solid tensor product (or even the derived solid tensor product), and it turns out that this is really just the completed tensor product of fréchet spaces. And then pass back. In particular, it's degree zero.

For example, if you take a product of copies of $\Z_p$ and take another product of copies of $\Z_p$, it becomes the bi-infinite tensor product $\bigotimes_\Z^{\infty} \Z_p$. Which might seem like it's a standard thing on compact projective generators, but it's absolutely not. Because these things are very, very far from being compact projective---only these kind of unit balls in there are compact projective guys, but this is a huge space.

Okay, and so let me again just give a proof sketch, or an example. The general proof is, in some sense, combining the proof I did there with those points, with the one I will do here. Bhargav and I kind of tried a minimalistic approach to combining these, but Sasha actually has a very fancy way to combine them that he needed for some categorical stuff.

Okay, so how do we write this product of $\Z_p$'s? It's again such a huge projective limit. And this time, you're taking such unit balls, but where you make the denominators larger and larger. So you take $\varprojlim_f \prod_{n \in \mathbb{N}} p^{-f(n)} \Z_p$, where again $f$ is some function from 

$\Z_p$, and again, this is next to similar Schwartz functions which are functions of both variables going to infinity very fast.

Again, you might naively think that surely if I have two variables, I can build functions that are so fast-increasing that I can never dominate them by something that's a sum of functions in both variables. But it turns out, again by basically the same argument as there is, you can always dominate any such function by a function, alright? That's fun.

I should maybe say, this is some filtered colimit here, but the precise structure of this index poset---this is actually where all the subtlety appears. Here we kind of don't actually have to understand what this looks like, because this colimit you can just prove. But for these $X$ computations, you really have to be starting with such a description in the case of $B$ space and then do that. And so then this huge colimit becomes a huge derived limit, and then you really have to grapple with the structure of this kind of poset of functions that grow arbitrarily fast. This is something that very much depends on your model of set theory, particularly on the cardinality of your set theory. Alright.

Yeah, I think that's basically all I really wanted to say. But if there are questions, we have a few minutes.

Q: For functional analysis in some very large Banach spaces, does it make a difference if you work in liquid condensed sets or all condensed sets?

A: It doesn't make a difference, because all Banach spaces are liquid. All Fréchet spaces are liquid, no matter whether they're separable or not. It's just the Preditorials that aren't factored.

Q: Is the relation between the homology of the $C$ complex and solidification kind of an accident, or should I take it as something deep? Can I expect that we will have another example, in condensed spectra or something like that, for other cases?

A: Well, instead of taking the de Rham $C$ being groups, you could everywhere work in a category of spectra. And then you could also... I mean, maybe a good point to point out the following: There's a spectrum called $ku$, which I like to think of as taking the algebraic K-theory of the complex numbers, but you take into account the topology that the $GL_n(\mathbb{C})$ has. This makes it homotopy-theoretic. Classically, this is a little bit hard to phrase what this is supposed to mean. I believe there are ways to do it anyway.

One way you can do it in our formalism is you can take the K-theory of the complex numbers as a condensed spectrum. Because the complex numbers, this is a condensed ring. And so this actually means that the K-theory series kind of has some automatic structure as a condensed spectrum. Because to give a condensed spectrum, I have to give a function where for each profinite set $S$, I have to produce a spectrum. But I can just take the K-theory of the continuous functions from $S$ to $\mathbb{C}$.

So far, this is not at all true, because here the $\pi_1$, for example, will be $\mathbb{C}^*$ as a condensed being group. But now you can take $K(\mathbb{C})$ and solidify. Just like we had solid sheaves of solid being groups and sheaves of condensed being groups, you also have solid spectra inside of all condensed spectra. And this solidification does exactly what you wanted to do. It somehow contracts all the $GL_n(\mathbb{C})$ that this was built out of, topologically, essentially by the proposition I mentioned in the lecture today.

And then you see that this is actually exactly the... I mean, you could do the same thing really not just for the complex numbers, but for any algebra over the complex numbers, or even a category over the complex numbers. And then take the K-theory as a condensed spectrum and solidify. This actually gives a construction of what's known as a semi-topological K-theory. So solidification is what makes semi-topological K-theory.

If $A$ is any kind of $\mathbb{C}$-algebra with a topology (I don't care about the specific topology), then you can similarly take the K-theory of $A$ and solidify. This recovers something that's known as the semi-topological K
Think: can you get a similar result for $KO$ or $KR$? Sure, just write $\mathbb{R}$. I mean, all the obvious variations. So in particular, $KO$ is $K(C^*(\mathbb{R}))$. There's something $KU^{top}$ or whatever it is, and then maybe related to the...

So here the $K$ is the periodic (or the full) spectra. I mean, in this notation, the $K$-theory spectrum... It turns out to be connective for the inputs that I'm using, maybe not quite obviously so, but it's true. And so there is a connective $KU$. It is much more subtle to recover the periodic $KU$. I mean, you can of course just invert Bott at the end, but...

And sometimes, in the $p$-adic case, the stuff that Dustin Clausen and I developed, which is a different way to approach this kind of stuff, is aimed at defining some kind of version of $C^*(\mathbb{C})$, but now an analytic ring, where the answer certainly has homotopy in negative degrees.

Ian: Is it somehow related to the description of the small $KU$ or small $K(\mathbb{R})$ as some configuration space-like description given by Segal?

I don't know about the... Sorry, there's some description for small $K(\mathbb{R})$ and for $KU$ using a thing given by Segal. I think this statement is much more naive. I mean, this statement is really just one way of encoding the intuition that... Or maybe see, but in some sense this is false. Something that's true before group completion, namely that if you take the infinity groupoid classifying space of $\mathbb{C}^\times$ and then pass to homotopy types here, then $KU$ is a group completion of that, and basically solidification factors over passing to homotopy types. So this is Segal's...

But they have, I don't think, any configuration spaces.

Audience member: To what extent does the solid tensor product combine two topologies? Say we have $M$ and $N$, and $M$ is $x$-complete and $N$ is $y$-complete. There was a thing that I mentioned: if you take power series $\Z[[T]]$ and then $\Z[[U]]$, then $\Z[[T]] \otimes \Z[[U]]$ becomes the $(U,T)$-adic one. So there it does do this thing.

In this more general condensed setting, it doesn't do it. So when you $-\otimes_A-$ and you have this $x$-completeness statement, now when you complete the ring $A$, this one is no longer the derived completion. Does it make a difference for the proof?

Yeah, so sorry, I actually realized that probably the most general version for solidification might not be expected to be true. I was getting confused about this. Well, I mean, if you want to compute some product of a solid thing or whatever, one way is to kind of resolve by the absolute one. And then you're also taking a product over $A$, and so to show that this here is the $xy$-completed one, it's enough to know that all the other ones are. And this way you can kind of reduce to the case that the ring $A$ is actually just $\Z[[x]]$ here. So then the derived $x$-completeness of this one reduces to the one of $\Z[[x]] \otimes \Z[[y]]$ here. But $A$ itself is derived $x$-complete. I mean, if it's basically $x$-complete, it's also derived $x$-complete. So you can kind of reduce this statement to the case that the ring is really just the join $\Z[[x]]$.

Audience member: What did you mean by "passing to homotopy type" in $B(GL_n(\mathbb{C}))$? So $\GL_n(\C)$ is viewed as a topological space, and therefore as a condensed set or abelian group?

Yes. So when I define $K(\mathbb{C})$, one way to do that is to take this classifying space
\end{unfinished}