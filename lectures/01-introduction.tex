% !TeX root = ../AnalyticStacks.tex

\section{\ufs Introduction (Clausen)}

\url{https://www.youtube.com/watch?v=YxSZ1mTIpaA&list=PLx5f8IelFRgGmu6gmL-Kf_Rl_6Mm7juZO}
\renewcommand{\yt}[2]{\href{https://www.youtube.com/watch?v=YxSZ1mTIpaA&list=PLx5f8IelFRgGmu6gmL-Kf_Rl_6Mm7juZO&t=#1}{#2}}
\vspace{1em}



\subsection{Motivation}

Thank you everyone for your patience and welcome. This is going to be a course about analytic geometry. The title is "Analytic Stacks", and we're going to be trying to explain the foundations for analytic geometry that we've been trying to set up for the past few years. 

In this first lecture, I want to give an introduction to the first half of the course (because otherwise, I'd be talking about too many concepts in one single lecture). Let me set the stage by giving some \textbf{motivation}.

Classically, there are several different theories of analytic geometry. I'll just list the ones that I may be familiar with. 

\begin{enumerate}
    \item The gold standard is the usual theory of complex analytic spaces. In the smooth case, these are the complex manifolds. So, these are things you get by gluing together open subsets of $\C^n$ along biholomorphisms (maps that locally admit a power series expansion). There's the nonsmooth case as well, where you also locally allow the zero locus of some finite collection of analytic functions on those open subsets.


    \item There's a generalization of this which was presented in Serre's book on Lie groups and Lie algebras \citeme{}, which is the locally analytic manifolds (or generalization of the smooth case at least). Consider a complete normed field $(K, |\cdot|)$ which can be archimedean or non-archimedean. Then you glue open subsets of some $K^d$ along locally analytic isomorphisms. That means we obtain locally around every point you have a convergent power series expansion with coefficients in the field $K$. 
\end{enumerate}
 



In (2) when $K = \C$, it does just recover the smooth case of (1), and that's a nice reasonable theory. When $K = \R$, it's a version of the theory of real manifolds, which is also a nice reasonable theory. 

But when $K$ is non-archimedean, it's not particularly geometrically rich unless you have some extra structure like a group structure or something like this, which gives it more geometry. The reason for this is as follows 



In the non-archimedean case, for example, when $K=\Q_p$, there's the structure is not rich enough because the topology on $\Q_p$ (or $\Q_p^d$) is totally disconnected.

\begin{remark}
    Every unit ball will break up into $p$ many other unit balls, and those break up into $p$ many unit balls. For example, in Serre's book \citeme{}, you can find a discussion of the classification of compact $p$-adic manifolds, and it's just a very simple combinatorial classification. There's the dimension and another invariant. There's just really not much going on there.
\end{remark}
    
\begin{enumerate}
    \setcounter{enumi}{2} 
    
    \item  This and some examples coming from uniformization of elliptic curves led John Tate \citeme{} to introduce \textbf{rigid analytic geometry}, which is over a non-archimedean field. So, it's a geometrically rich theory which works in the non-archimedean case. 
\end{enumerate}



In contrast to the theories above, you don't kind of really think of it in terms of specifying some ways of gluing open subsets of $K^{d}$, or you don't even necessarily think about it in topological terms at all. You more actually think about it in algebraic terms. Instead of focusing on a local model, which in the classical case might be something like an open polydisk, you instead concentrate on a class of locally allowed functions. So, there's a turn here. 

\begin{itemize}
    \item Focus on local rings of functions instead of local topology and let the ring of functions tell you what the topology is supposed to be.

    \item The local model that Tate uses are not functions convergent on an open disk, but functions convergent on a closed disk, which is something that makes sense to use in the non-archimedean context. The local rings of functions are well quotients of functions convergent on a closed polydisk, and those are the so-called \textbf{Tate algebras}.
\end{itemize}

The manner in which you're allowed to glue these local models to get global rigid analytic spaces is halfway between algebraic geometry and usual analytic geometry.

%It's not clear if Tate algebra is just the functions convergent on the open disk or quotient of. Oh, okay, it's a point about terminology. Yeah, it's not. Sometimes, in some references, they, but I'm not sure if it's a mistake or another. I mean, because I'm not, because depends on the reference, right? Okay, that's in the two references they say, but all the other one avoid algebra. Okay, let's say it's an apoid algebra then. Yeah, I don't. Okay. 

\begin{enumerate}
    \setcounter{enumi}{3} 

    \item This was generalized by Huber to the theory of \textbf{adic spaces}, and this is a generalization of rigid analytic geometry. Note that rigid analytic geometry takes place over a base field, and a very loose analogy would be that adic spaces are to rigid analytic spaces as general schemes are to varieties over a field. So it is useful generalization where you don't have to have a fixed base field.
\end{enumerate}

\begin{center}
\begin{tabular}{ |c|c| } 
 \hline
 Varities over a field $k$ & Schemes \\ 
 Rigid analytic spaces & Adic Spaces \\ 

 \hline
\end{tabular}
\end{center}


Again, you don't think necessarily of your local models topologically, kind of think of them as being algebraically specified in terms of a ring of functions. But here also there's a new twist. 

\begin{itemize}
    \item  You still have local model as some ring of functions called $A$, but you also include extra data of a certain subring $A^+ \subset A $, and what $A^{+}$ does to $A$ is you attach some space of valuations and then $A^+$ will single out those valuations which view this subring $A^+$ as consisting of integral elements.
\end{itemize}

It's actually a very nice extra flexibility you have in Huber's theory that you can consider different choices of $A^+$ on a given appropriate topological ring $A$. We'll see from a different perspective what this choice of $A^+$ is really doing later in this lecture.

%So, maybe I'll go over here. So, when you say local ring of functions, yeah, do you mean really a sheaf on that open set? I just, no, I just mean it's something like you have, you specify a certain kind of ring, and then you say that's my functions on my basic geometric model, you know, like for you have to have transition right when you go from one open set and intersect with other open, yeah, so I what I didn't discuss gluing, so that's that you need to say that kind of thing when you discuss gluing. We'll touch on it a little bit later, but I'm actually going to, for this lecture, I'm mostly going to stick to this sort of aoid context where you just look at a single A, but of course, that does need to be discussed and it will be discussed, yeah.


 Then we proceed to Berkovich's theory. The rings of functions you see both in the context of rigid analytic spaces and in the more general context of adic spaces. The basic examples are things like you have a ring and it's complete with respect to a finitely generated ideal and you give it the inverse limit topology where all the quotients are discrete and you are also allowed to invert something, provided that the inversion is inverting everything you completed along.

 So, the most basic example of these things is you can take $\Z[T]^{\wedge}_{p}[1/p]$, the ring of polynomials in one variable, and you can complete it with respect to the $p$-adic topology and then you can invert $p$ and that's the functions on the closed unit disc in the $\A^1_{\Q_p}$. So, we complete along something and then invert it, and these are also examples of $p$-adic Banach rings, and what Berkovich does is he says let's just work with arbitrary Banach rings to start with.
 
\begin{enumerate}
    \setcounter{enumi}{4} 
    \item \textbf{Berkovich Theory}: The local models are given by Banach rings $(R, |\cdot|)$, note while the theories (3) and (4) are confined to the non-archimedean case, this theory over here actually allows archimedean phenomena as well because the real numbers and the complex numbers also count as Banach rings.
\end{enumerate}

The global theory here is not quite as smoothly functioning as in the case of $(3)$ and $(4)$. To the ring $(R, |.|)$, Berkovich attaches the space of multiplicative seminorms $\mathcal{M}(R, |.|)$, which is some compact Hausdorff space, and  the kind of gluing you're allowed to do is in some sense organized by this Berkovich space and I don't necessarily want to get into too much detail about that right now. But for example, you can see the complex analytic spaces as a special case of this, and you can also see rigid analytic geometry in some sense as a special case of this. So, this is my very, very brief review of classical theories of analytic geometry.


%and please feel free to ask questions if you have questions. Um, some certainties lost over like the shiness in the Uber Theory, correct in has this different gluing like in one you can glue along op get to get to get things like he has to work only over a field condition, then something quite artificial to glue it like you do in but in his language, in some sense it just transports yes to his language with some extra things which are difficult to remember, some conditions under which the theories are equivalent, yes, exactly, exactly like the notion allows you to understand, yes, exactly, exactly, yeah, so yeah, so the relation with original geometry is indeed artificial, it's kind of not the correct, okay, but anyway, um, so what do I want to say now? I want to exp... Okay, so

We have all these theories, that's great, but now what again was the motivation behind coming up with a new theory? Is  just going to be point $(6)$ on the list? Well, not exactly.

\begin{question}
    Why introduce a new theory?
\end{question}

Well, all from $(1) - (5)$ are theories of analytic geometry, and kind of the relationships between them are more or less well known and one can formulate comparisons and the web of these things is kind of well understood, in spite of the subtleties sometimes involved in the comparisons, it's fairly well understood.

But so far, there's no common framework in which you can put all of these examples, they all have their own flavors, and while you can formulate comparisons, it's not that those comparisons are taking place in some larger category that you consider, it's kind of done by hand in every situation, formulating the comparisons between these things.

\begin{answer}
    
        We want to accommodate all examples in a single theory, so that's one thing, um, you'd like a general theory of analytic spaces which can be specialized to whichever context you might uh be interested in.

        
        The second reason would be, out of all the above theories the only rich theory allowing both archimedean and non-archimedean geometry is Berkovich's but the gluing is not so well worked outand in particular, the gluing was investigated by Berkovich he basically restricted to the non-archimedean case almost from the start and then more general gluings were investigated by Piono \citeme{} for example but in that they always do the same thing they fix a Banach ring as the base ring and then they define an affine space of some dimension over that base ring and then they glue along some kind of subsets of a affine space that they that they pick out that's sort of how the gluing works in Berkovich's theory. The Banach rings they take as their base rings often have restrictive hypotheses on them\footnote{so its not really understanding how you can glue two Banach rings together to get some more global object and indeed the things you're gluing along in these a affine spaces are oftentimes not even controlled by Banach rings themselves they're just some other kind of objects so the nature of the gluing is constrained to a finite type situation and it's a little artificial, maybe not necessarily artificial but it doesn't quite fit the mold when you think of how you go from like affine schemes to general schemes for example just by gluing those local models in some naive way.} such as finite type.

        
         Even individually in their own context in which they're supposed to operate uh these theories are less flexible than for example the theory of schemes and one major reason has to do with issues of descent.

\end{answer}

\begin{unfinished}{19:54}

Uh, so for example, one of the main constructions when you have a scheme is the category of quasi-coherent sheaves. Um, and that's a big fancy name, but it's really something simple. When you have a commutative ring, you look at the category of modules over that ring, and then it just glues to a general scheme, and that's what a quasi-coherent sheaf is. Um, but you don't have that in analytic geometry in any of the classical theories, and the reason is, okay, I said you always have some local ring, um, which is describing the local geometry, so to speak, and of course, to that ring, you can certainly assign the category of all R-modules, but then it doesn't glue. So, it's just not the case that if you have, you know, in any of the allowed gluings that people choose, it's just not the case that an R-module on two open subsets or closed subsets or anything that have gluing data on the intersection globalizes to a general R-module. The view point was in analytic that coherent sheaves are the basic tool, yes, this is the classical person, yes, and I was going to say this, um, so you only can glue, uh, so maybe finite type or finally presented modules, and this does give rise to the theory of coherent sheaves, which is a beautiful and extremely useful theory. It's one of the main tools you have in analytic geometry, but it's still constrained by the finiteness hypotheses that come into it. So, for example, if you have a map of analytic spaces, you can't consider like the push-forward of the structure sheath as a coherent sheath, but that should be that's one of the main examples of quasi-coherent sheaths you like to play with in algebraic geometry. I mean, unless the map is finite or something, um, so the theory of coherent sheaths is really nice and it works well in analytic geometry and basically all of these contexts, but it's still not as general and flexible as we're used to from algebraic geometry with quasi-coherent she which have no inherent finesse conditions, um, so, so these are all kind of, you could say, theoretical reasons why we might want to new theory, um, but there's also potentially a practical reason, so this is much more speculative, um, so coming from the Langlands program, so, uh, far Fargan Schulza, they famously geometrized a local Lang lens and that led to kind of a clarification of the local Lang lens program, and what was this geometric geometrization based on, it was based on replacing QP, uh, by some more exotic object, uh, far Fanten curve or really you have to let the curve vary in families in some sense you could say far Fen curves, um, and this was produced in the language of attic spaces so it was attic space over QP not at all of finite type, um, so quite a somewhat exotic Beast which thankfully this theory of attic spaces existed to accommodate it, um, and um, again quite speculatively one might hope that not just the local uh Langlands program but the global Lang L program can also be geometrized, um, very okay Peter's always very optimistic but I'm always uh of global Lang length but this would involve replacing say Q or Z by some family of exotic analytic spaces, um, and whatever such a thing is it's going to have to have both archimedean and non-archimedean aspects, for example, there should also be a version over the real numbers which I believe Peter is is working out, um, and it also is not going to be finite type in any sense so there is simply no existing theory which could possibly give the language to describe such an object if such an object even exists, yeah, but it's good to have a a theory a precise theory to to guide exploration of of the possibility of such exotic things, um, so that's another motivation um, so that is the end of my um motivation section, uh, so now's a good time for questions if people have them, so you you are going to introduce this B SP I mean theory what what is are more general theory encompasses all of this yeah I'm going to we're going to introduce a new theory and explain the relation to the previous theories, yeah, yeah, so that yeah that's good to say yeah so our our goal is in this course is to introduce a new theory of analytic geometry and to explain the relation with the previous theories, yeah, so you will let not only the basic analytic ring but also some analytic spaces analytic yeah yeah but today I'm only going to give an introduction to the apoid situations kind of analytic Rings because it'll already be enough, uh, already be enough there, yes, why is the theory of BL spaces insufficient, oh, it doesn't have any archimedean, it doesn't, there are no, it's non-archimedean by Design, yeah, yeah, Al, some I remember forgot now how it was called some kind of spectrum that they combines and forgot the name of this someone consider but they didn't develop it so uhhuh yeah there could be other theories than the ones I

 listed I just listed the ones that I knew were well studied and yeah, okay, so then let me move on uh, so continuing the introduction um, I actually had a question, yes please, so um, in the previous uh, sort of theories like uh, I think like at the birk ofage setting there were like Rich uh, theories of chology itology and stuff that were Vari that yeah but I mean um yeah sure bur deeply studied atal kology in in setting of burkich spaces yeah okay and okay um so the next section is called condensed math so the I mean the cond condensed math yes uh the the F this this issue here the issue with the scent um it can be attributed to the fact that these in all these different theories these local rings that you have describing the local models they're not just abstract rings they're topological rings and for many purposes for example uh for uh well yeah so the local models classically are in fact topological rings and it's important to remember the topology so for example okay an algebraic geometry the uh polinomial ring in two variables which is functions on Aline and two space is the tensor product of polinomial ring in one variable and polinomial ring in one variable okay but in say rigid analytic geometry if you take the T algebra in dimension one and tensor it with a tate algebra in dimension one again you're going to get some crazy thing because you took an algebraic tensor product and you forgot the ptic topology but if you do a a Pally complete tensor product you get the ring of functions the correct geometric ring of functions the two variable case so you need to in performing constructions such as tensor products which are basically calculating fiber products geometrically you need to remember the topology that much is clear um so and that's at the basis for the reason why you don't have a naive theory of quasi-coherent sheaves and you have

Quasi-coherent sheaves and you have naive problems with gluing outside the finite type case. I mean, finally generated, uh, case, but topological rings, uh, and topological modules over them, which would be the kind of natural thing to do if you're thinking about quasi-coherent sheaves, are not suitable, uh, for a general theory, and the basic reason there's, there's many ways of saying it, but the basic reason is that, uh, you know, if you form this category, it's, it's not going to be a billion, the category of topological modules over a topological ring, and you can dress it up however you like, you can make it much more specific or whatever, it's just not going to be a billion, and the phenomenon is that if you have a dense inclusion of modules which happens all the time when you have infinite-dimensional things, uh, then it's going to be both an epimorphism and a monomorphism, uh, generally speaking, uh, in your reasonable categories, but it's not going to be an isomorphism, so you separated, yeah, yeah, I would have to say separated to make that literally a true claim, so you have a non-strict map, then the image is two topologies, and this is not yeah, CEG world, yes, yes, yes, yes, um, right, so, uh, so what we do is we kind of go very back to the start, um, so we, uh, go back to basics and Define a replacement for the category of topological spaces, and we do that in such a way that it's then very easy to pile algebraic structure on top of those things and talk about the analogues of topological rings and topological modules over them and that such that we will get an aelion category, uh, in the end, um, and the basic idea is one that is very old and I'm not sure, so certainly it was, uh, certainly, you know, Gro deck used this idea many times and I don't but I don't, I think it might even be older than Gr, topological space is kind of funny because you have second-order data, you have a set of points and then you have a set of subsets of that, and that's fundamentally what makes it difficult to mix with uh, algebraic structures, so instead you'd want to stick with kind of first-order data just points and the basic idea is you single out uh, a collection of nice let's say test spaces, uh, s, and then instead of uh encoding a topological Space X as traditionally so a set and some set of open subsets um, we just record the data uh of what should be continuous Maps uh, from your test space to that topological space, and kind of atati the structure and properties you see in that situation so we were only we we're going to choose a nice collection of test spaces and then we're going to say we only we only care about a topological space in so far as uh, the phenomena are seen by maps from these nice test objects, um, so you still have X set I put quotation marks around this so so I I'll become a little more formal in a second and then you can ask your question, no, but traditionally like they wanted to doop Theory with then so one way was to have ACC a set and to fix a collection of S to X some people some work with this yes, yeah, yeah, exactly substitute usual on you can Define homotopy and ology and do something yes EX exactly yes, yeah, um, so yeah, maybe spanner is a person and yeah, I don't know, yeah, um, so formally, uh, so formally, uh, our test spaces, uh, will be profinite sets so-called profinite sets, uh, which is which is the same thing as a totally disconnected uh, compact house door spaces, um, it's also just the same thing as inverse limits of finite sets where the finite sets have a discrete topology and the inverse limit has the inverse limit topology, um, and so this is what we we uh used in a previous iteration of this kind of course, um, so the first course Peter taught on condensed math was uh with this class of test spaces but it does cause some troubles because uh uh, it's a it's a large category so there's no if there's no Cardinal bound on the profinite sets then when you encode all of this data you're encoding more than a set's worth of data and it it does cause some technical troubles we more or less worked around them but so we're actually going to take a a slight variant of this for the purposes of this course um and we'll explain in more detail in the first few lectures I think um why we make this precise choice but so so a we'll say a light profinite set uh is a accountable inverse limit of finite sets it's also the same thing as requiring it to be metrizable um and then so uh a light condensed set uh is a sheath of sets on the category of light uh, profinite sets uh with respect to the groi topology uh, and now I'll explain the groi topology so so it covers our finite collections uh uh of jointly surjective Maps continuous Maps yeah um um so finite dint unions are covers and a surjection gives a cover so this is like theology because you could add you can add also okay a covering seeve is one which contains a finite collection of jointly surjective Maps Okay um so okay this is we're using the language of gr toies here uh to be sure and possibly not everyone is familiar with uh the language of Gro to and kind of the general how you play with categories of sheaves and so on let me make it more explicit uh so more explicitly uh a light condensed set is a functor okay I'm not going to avoid the language of categories and functors uh so light profinite set up to the category of sets uh such that so the first thing is is that well X of empty set equals Point second thing is that X of a disjoint Union of two things is the product uh and the third thing is that if you have a surjection t to S then uh XS is The Equalizer of uh XT and then the two different pullback Maps you have to the fiber product um and and the example example is any topological Space X gives a condensed set where the funter of points is just given by The Continuous maps from s to X so it's easy to see well the the functor

iality is just that you if you have a map from T to S and a map from s to X you get a map from T to X and it's easy to verify all of these properties uh for continuous Maps out to an arbitrary topological space so groi topology I mean you can just unwind it to this but um actually it's kind of a bit of a bit elusory the elementary nature of this definition.

Because um we are going to fairly seriously make use of the theory of Gro IND topology and sheaves in this course, it's just not worth it to avoid that theory, especially since it comes up both here in the definition of light condensed set and also later in the way in which you GL glue uh the apine case of analytic spaces to General analytic spaces. Um, it's not going to be we're just going to use the theory, so if you're not familiar with the theory of Gro deque topologies and sheaves I suggest and you want to follow this course I suggest you read up on it, um, okay, yes, question, so what is a morphism between two uh, two profile sets like Prof set, yeah, just a continuous map, require some filtration it's just a continuous map but it's also a map of pro systems if you're thinking of it as a accountable inverse limit it's equivalent proem it's yeah, yeah, it's the same same thing yes, do we know what are the points in the category of FL content sets points in the c ah oh oh in the topos thetic sense yeah, I think so yeah they gean um debatable so we we'll discuss more such things in the in the coming lectures um isance of topological space in is that skul no not not for General topologic IAL spes but for a large class of topological spaces it is yeah um right so so some there's one thing that's clear right I started with a general idea or you take some collection of test spaces and then you okay then you say what you axiomatize the properties of maps from test spaces to a given topological space and you arrive at this axiomatics but okay it's actually not so simple because there's many possible choices of test spaces and beyond that there's many possible choices of which properties you want to put in your act which properties you see mapping out from those test spaces to x that you put in the axioms uh for your general objects like condensed sets actually there are other properties Satisfied by X of s when X is a topological space that I didn't put in the axiomatics um so there's kind of a little bit of a delicate balance here and we will discuss more about how we arrived at precisely this this Choice both of the test category of light profite sets and this for now I just want to make a couple of remarks which will maybe give a a sense for why we make this precise definition so so first of all uh two maybe the two most important examples of light profinite sets are the point uh and then there's this uh the one point compactification of the natural numbers um and with this you kind of get the underlying set so if you take X of s then you think of that as the underlying set of your condensed set and with this you get kind of a notion of a convergent sequences uh in your your condens set you get a set of conver set of convergent sequences again it's abstract so it's not literally well in the case of a topological space it literally is the set of convergent sequences so also those things were considered in this topology of oh yes yes yes that's correct yeah and but they had this they use this projective things projective covers which are not light that's correct so this will be discussed this will be discussed in in in due time yeah so but yeah so yeah it's true bot and schultza had this definition but that doesn't mean that it was necessarily the correct thing to do for the purposes I'm about to discuss but yeah it turned out it was um okay and then but then um uh so then another remark is that allowing all surjections uh to count as covers gives a nice simplification of the structure of the category and in particular it gives some good homological algebra properties when you pass to light condensed ailan groups um this you have lot of flexibility of working locally when you allow arbitrary surjections to count as covers but on the other hand uh restricting the topology or requiring uh the topology the gro de topology to be finitary uh gives good categorical compactness properties for light profite sets uh sitting by the on meding inside all uh light condensed sets and and moreover and even and even for uh all metrizable compact house D spaces so for example uh the unit interval famously is a has a surjection from the canra set given by decimal expansions uh and this is a light profinite set and the fact that you have a finitary Gro deque topology and on the other hand this this guy is covered by this guy which is one of the basic test objects it means that the compactness of these compact hous door spaces which kind of we know from General topology actually translates into a nice categorical compactness property inside this larger category um and for this it's actually important to have a these larger light profinite sets than just the sets of convergent sequences okay um let's see okay questions Prof at light it's countable no no that's not countable I mean you you could ask the collection of clo and subsets to be countable so count yeah or yeah something like or it's the same as metrizable for sure so yeah no no they adjust every point of the accountable neighborhood B is weaker than the second count yeah some way yeah yeah you need you need a accountable basis for the topology in total yeah so anyway okay yeah it is the same thing as opposite category of the countable yes I guess yeah exactly yeah it's the same saying that the set of clo and subsets is is countable yeah okay further questions um so I remind you that this is this is just an introduction we will go into much more detail in the in the coming lectures okay so so what do we have now we have uh oh maybe I give myself a Blackboard um so what do you have so far when now I've explained the light Prof finite sets and so um we're going to move on to analytic rings and I'll start with a point which is that these okay from now on sorry so I'm going to drop the light okay so just so I don't have to write it and say it all the time from now on condensed set means light condensed set and profite set means light profite set I always have this cabil uh hypothesis floating around um so condensed sets uh gives rise to the notion of condensed ring and condensed module over condensed ring and it's really uh if you're familiar with the gr topologies and so on it's completely immediate so it's just a sheath of rings and then a sheath of module over that sheath of rings on this uh on this site here it's also just a ring object in this category

 and a module object over the ring that ring ring object in this category yeah a question Zoom to write a little bit bigger oh a question to write a little bit bigger that sounds like more of a comment or a request okay uh I will do my best and please hassle me again if I don't live up to it yeah

Um they didn't ask me to write more clearly. Well anyway, Tech, sorry, the formulas more clearly, thanks. Yeah, okay, um. And that's all well and good and you might think naively, okay, so now we have a category of condensed rings, for example. Why can't we just use that as our uh local models for our analytic geometry kind of by analogy with schemes, where schemes are based on discret H. By the way, ring means commutative ring. Um, schemes, you start with discrete rings and then you figure out a way to glue them and then you get schemes. Um, but it's not enough just condensed rings to get a good theory of analytic geometry.

But no, please, cont string is the same object where SE is replaced by a. Yes, that's correct. So, uh, but there's no additional structure on ring, just a abstract ring, right? But condensed ring, it's just an abstract ring with those properties. No, no, but no, no, no, it's a collection of abstract Rings, one for each s, yeah, yeah, okay, right, yeah, but it should satisfy all those properties and Xs cross XT, what does it become, the tensent product of the Rings? No, the cartesian product of the Rings, yes, I see, okay, um, yeah, U where was I H, but are not enough uh to give a good geometry. And the basic reason is as follows, so maybe I should say condens ringings alone. So, the category of condensed rings has pushouts given by relative tensor products, just like in classical commutative Rings, um, and those relative tensor products are what geometrically speaking should calculating fiber products for you, um, and they're the things that I said should correspond to completed tensor products, right, um, but if you have condensed strings, a and b, over a condensed string K, and you form this relative tensor product in this category, you can ask well, what is the underlying ring of this and it turns out it's not actually hard to see from the nature of the grot topology that this is the same as the abstract tensor product of the underlying ring rings of all the individual things. So, this condensed ring here is just, just gives a condensed structure just on the yeah, yeah, just gives a an well non-trivial to be sure but just gives a condensed structure on the on the abstract tensor product, so it's not in particular is not giving a completed tensor product the completion procedure does change the underlying set, right, so this is not not yet doing the correct thing, uh, okay, so to fix this uh, we put additional structure on a condensed string, um, so we record some class of modules, I mean condensed modules, uh, which are to be considered as complete in some sense, complete. So, the basic um, yeah, and that will uh, oh, I forgot to write larger, oh, that would give the notion of analytic ring.

So, an analytic ring will be a condensed ring together with some extra structure which will tell you which of the which of the condensed modules over that condensed ring you should consider as complete with respect to the theory that's being described by the analytic ring, um, but before I make the definition more precise uh, I have to scare more people away. I already said you should know Gro de apologies, get ready, um, so I kind of I have to say more precisely what I mean by a ring, so but I'm going to scare you but then I'm going to say well you shouldn't be too scared, uh, so why is this why is this a question I just I already told you ring means commutative ring right, but um, You didn't say no, I I did you need to pay better attention um, so what kind of ring um, okay so experience in algebraic geometry shows that the generally correct notion of a fiber product of schemes is actually the derived fiber product which on the apine level corresponds to derived relative tensor product of rings. Now the reason more people don't do it that way is because uh, it it's a technical hassle to talk about these things, these derived tensor products and derived rings and so on, but actually that that's not true anymore we have Jacob L's works it's not a technical hassle anymore you just have to you just have to do it and it's no problem so we're going to do it because it's the the correct uh it gives you the correct relative tensor products with giving a good theory in general.

Now that being said basically all of the basic examples that we discuss almost all of the basic examples that come up will not have any derived structure they'll just be ordinary rings so you can comfortably follow the course even if you're not very familiar with derived rings but you should bear in mind that for you know for the for the general claims that we're making it won't necessarily be true if you imagine everything to be an ordinary ring although in examples many things will indeed be Ord Ary Rings um, but now we go down a rabbit hole because once you decide to work with some notion of derived Rings there's actually several inequivalent choices of what you could mean by that um, so so should be derived uh, but which kind there's there's two basic options and that's kind of uh, e infinity algebras and what some people call animated commutative rings. I'm one of those people uh, which

Are the things that are presented by simplicial commutative rings, and then there's also the choice of whether you want it to want to allow negative homotopy. In both cases, um, we won't allow negative homotopy, and we're going to, for the purposes of this course, we'll choose this one here. It's more directly tied to classical algebraic geometry. When you start with ordinary schemes and you take derived tensor products, the things you get always have this extra structure, so it makes sense to remember that and not think about these more general things. But actually, the whole theory that we're developing works perfectly fine in any of the different variants, and in fact, it's even less technical to set up in this seemingly more complicated setting here, for reasons which I think we'll get into. Um, but so algebra in the sense of Spectra exact, there's also that choice. Yeah, you could do you mean infinity algebras over Z or over the spher Spectrum. Yeah, um, so okay, so formally then, so just to get it on the so formally, uh, a light animated ring, oh, sorry, condensed animated ring, is a hypersheaf of animated rings on the uh, site of light profinite sets, but again, I'm not saying light. Um, okay, and now I'm going to make another convention that I'm probably just going to say ring when I mean animated ring, and if I'm want to stress that it actually just lives in degree zero, I'll say, uh, classical maybe or static. Um, static being kind of the opposite of animated, um, right, and that should hope also help those of you who are not familiar with the theory to just pretend that everything is an ordinary ring because that's that's pretty much okay. Um, all right, um, and the basic invariant for us of such a condensed animated ring is its derived category, so the, oh, I need to write bigger of such, uh, such an R is its full derived category. Uh, uh, is this the in L in somewh, yeah, so yeah, you look at just at hypersheaves of modules over this, you know, up unbounded modules over this sheaf of rings, are yeah, justo just a sec, do simpli and coal or does it do it, I don't know, coal no if you want to, you have simplicial modules of simplicial ring, this is not enough to have unbounded in the, no yeah, you don't I mean yeah, you don't really set it up like that, you don't talk about simplicial modules over simplicial ring because then that would be the connective part, you could do that and then just say you kind of formally add in the negative things by by filtered coits or something, I mean by yeah, you do it by I mean the way he does it is he forgets the infinity algebras and then that's just a commutative algebra object in D of Z, and then that has a natural notion of module in the infinity category Theory, so so I mean the theory of modules factors over the underlying infinity algebra and that's just and this is in in which in the in in higher algebra maybe or or S AG probably discussed in more detail spectral algebraic geometry, so but there is no uh, I'm going to say something to help Orient yeah, so if R is static so again that means it's just an ordinary T string uh, then this is the usual or the infinity category enhancement of the usual uh, unbounded derived category uh, of the aelon category of condensed R modules again in the in the totally naive sense of you have a sheaf of rings and you take a sheaf of modules over that sheaf of rings yes, can I ask you to comment on specifically why you want hypers sheaves everywhere it's because we want things like convergence of posnov Tower and we know we can prove that in the world of hypers sheaves but we can't prove that in the world of sheaves so we don't know that sheaves and hypers sheaves are the same thing and also we can always prove everything is hypers Chief and we can always I mean hyper covers never give us more trouble in practice than ordinary covers okay so we're not losing anything by requiring that yes just have a questiony like yes like we can Define schemes in general like just Co limits of representable sheets uhhuh what happens if we do the same here like we take like we glue representable shav over condensed strings no with just condensed strings you're never going to get a good theory you need you need this extra structure yeah I mean if you take sheet over condens strink Sal doesn't I mean the same reason will hold yeah I mean this will be this will be your pullback in pre- sheeps and it's just not the right thing so yeah uh other questions what oh yes hi Matthew yes is not admitted a static animated ring is a condensed anim in the new sense seems like you need some Vanishing of chology because was she condition in just a one categorical s it works I'm not going to get into it right now we'll talk after yeah yeah this is not the time for such a technical question apologies but don't worry if it's correct yeah um okay so uh okay so now I can give the the formal definition um wait did you define are theyre uh yes yes I don't want to say discreet because we also have this condensed stuff and then you discreet could mean yeah so that's the reason for changing the terminology yeah okay so what M no for no no right exactly exactly um okay so the definition is so an analytic ring uh is a pair R uh and then I'm going to use funny notation uh uh where this triangle thing uh is a condensed ring and uh the derived category of the analytic ring is supposed to be a full subcategory of the derived category of this condensed ring which is sort of its envelope um is such that and then we're going to demand some rather strong closure properties remember the idea was this was supposed to be singling out a collection of complete modules um and the first condition is that uh this full subcategory is closed under so inside this ambient category here is closed under all uh limits and colimits uh the second property is that if let's say n Li in here and M lies in here uh then kind of the internal Ram uh from M to n still lies in the smaller thing um the third condition uh is kind of technical so I said we uh we wanted our rings to be connective so no negative homotopy in some sense we also want to require that our analytic Rings be connective and we say it like this so if uh if uh I'd say

This uh denotes the left ad joint to the inclusion. Um, so again that's some kind of completion functor. Then uh this completion uh sends uh the connective subcategory here to the connective subcategory again so it preserves connective objects. I should have said I'm sorry I meant to remind over here so I said if R is static this is the usual ual unbounded derived category of this ailan category. In particular, it has a t structure, has a notion of connective objects and anti-c connective objects but even in general for a general R you still have a t structure but it's not the derived category of its hard anymore. When your ring is not static, say which kind of structure you have in just in terms of the vanishing of of chology, yeah, exactly, and you use homological notation. I always use homological notation. Yes, yes. Do we need object uh H why do we require this? There's some statements uh for some statements it's convenient to have a kind of reduction from a general animated ring to the static case, namely, it's Pi 0 and for this kind of reduction, it's um, it's important to have this kind of control on connectivity. So once you assume the limits and C this is under usual categories or Infinity categories, the same that is well if it's a triangulated subcategory closed under products and direct sums then yeah then to have the int you need. Sometimes some Cal condition is it automatic yes yes so like being generated by a set yeah it's automatic it's automatic yeah Qui one question yes uh with condens ring you mean condens animated ring yeah I I made that convention maybe only in words but yes exactly uhhuh so from now on a ring is a static ring and an animated ring is a ring um okay uh over let's let another one have a chance first uh yes the analog of the ICL inside that indeed it is indeed it is yes uh over no just to make sure so one and two implies that the left agent exists right okay um ah right I forgot to say what a map is so a map of analytic rings sorry I I I I I need to hear what you said so the cond ring is a animated for always anim anim yeah uhh is just a map of condensed rings such that so it's just a condition uh if M lies in D of s uh then the Restriction of scalers of M uh should lie in D of R so so uh along R to S yeah I wrote it a little funny but I hope I hope the meaning is clear so if you have an object in D of s triangle uh which happens to lie in D of s then when you restrict to an our triang our triangle module it should line D of R um okay so I'll make some remarks so uh there's always a t structure on D of R and in fact it's quite naive so the connective part is just uh the intersection of the connective part for the enveloping ring um and same with the anti-c connected part so you can check everything in this potentially more familiar category here so it's actually zg graded not just positively graded the modules are z-graded yes the ring is positively graded that's what I thought okay but the modules you allowed to be zrad indeed indeed um and in particular you get an aelan category so D of R the heart of the T structure which again is just a d of R intersect this um and actually this ailan category also determines the the analytic ring structure so I can also so I can also say that uh D of R is giving an analytic ring structure on our triangle and um there's actually an equivalent axiomatics uh just at the aelan level so I could instead say that I give a a anim a condensed ring and an aelan subcategory of you know the heart of D of r or D of R triangle satisfying certain axioms yes sorry um just maybe kindy of slow it down but for the definition an analytic R yeah I'm just trying to like understand why like uh why in what sense is it analytic like like why yeah so that's something I can't answer right now it'll come when I when we discuss examples and so but the motivation was the simple thing I said that we want relative tensor products to be completed and okay uh yes Robert can I drop hard triangle if I want and just remember the category as a I forgot an axium sorry you just reminded me that because the answer to your question is no because of this axium um sorry I forgot to require that the the unit should be complete so yeah the ring itself should be complete then why can't I drop the Top our triangle entirely well you yeah and then you need some extra structure on D of r i remember as a condensed category yeah yeah that's still not enough because I'm doing the animated context and not the infinity context but if I remember yeah then that's enough yeah yeah yeah yep um okay so that's another perspective is you were just giving an Aon category of complete modules there's also a third perspective which is useful so so to to understand

What an analytic ring structure is, this is very abstract, and it's talking about big categories and stuff. But for a light profite set, I wasn't, I said I wasn't going to say light S. We can consider the free S, the free module. So it's denoted R bracket S. And what is it by definition? You take the free module over the condensed ring, and then you just complete it. So put in there via the left joint. And these generate D of R greater than or equal to zero under co-limits. So those are kind of your basic basic building blocks, your basic generating objects. 

And again, there's another equivalent axiomatics which takes as the second data in the pair, not the category full subcategory or Dr, but just the collection of free modules on profinite sets, objects in D of R triangle. So to gain intuition about what this thing can kind of look like, it's useful to think. So this is not in the heart in general. In general, it's not in the heart. In basically all examples, it is, but in the general theory, it's not. 

So think, so intuition, so this R triangle bracket S is kind of, well, it's just R linear combinations of points in S, kind of completely intuitively, finite R linear combinations of points in S. But you can think of it as a space of R linear combinations of derac measures. Hey Robert, I did something for you. And then, then this RS is some completion. So that's a bigger space of measures. 

So again, an analytic ring structure can also be thought in terms of as specifying some space of measures on a profinite set. And what is the role of this space of measures? So an M, let's say in the heart for simplicity, lies in Dr heart if and only if for all maps F from our triangle S to M of our triangle modules, there exists a unique extension along R bracket S. 

Or in other words, if F is kind of a function from your profinite set to your module, which is some kind of linear algebra object with a topology, and mu is one of these kinds of measures that you're allowing, then we get a well-defined integral of this function along the integral over S of the function. I don't know D mu or whatever you want. 

You can pair them to get a value in the target module. So this is explaining some sense in which this is this behaves like a completeness condition. It's complete enough that you can do non-trivial integrals against certain classes of measures, which you specify as part of the data. You could say yes, is your D the full subcategory? Yes. And is it compa... no, even this one isn't. They're all presentable, but not, yeah, yeah. 

And that's because we did this light restriction, yeah. So we'll have to get into that. Yes, how is the light restriction? Is it Omega one compactly generated? It is, it is Omega one compactly generated. Yes. Is it dualizable? No, I don't think so. Yeah, yes. Well, I wasn't claiming there exists a unique extension. I was claiming this condition is equivalent to this condition. 

So were you saying why if this lies here, does there exist this unique extension? Yeah, so that's because basically just by left ad jointness, but it comes from unraveling the definitions. What is the question? Was which one is is only one comply generated? The big one or the small one? Neither. Oh no, both of them, sorry, both. Both. I marred the Omega one, yeah. Both are, oh, both, yes, okay. 

So I don't think I'm going to have time to get to examples, which is rather unfortunate, but I don't know, maybe Peter will, I don't know, we'll see what Peter plans for Friday. Be nice to talk about some examples. Um, but instead, I think I'll probably finish, well, yeah, I'll probably finish with a discussion of co-limits in the category of analytic rings, and in particular, I want to talk about pushouts because this is the crucial thing which is supposed to give completed tensor products which correspond to geometrically good fiber products. 

So where am I? Here, so your anima structure on... sorry, anity, oh yeah. So one of the things we prove is that it comes for free, okay, yeah. I mean you have, I mean this is a, I mean no make no mistake in the maps. You have a map of animated rings from the r triangle to S triangle, but then it turns out whatever the linear algebra operations you might expect to like symmetric powers and so on will sort of automatically go through. 

Yeah, it's not obvious by any means. So your definition of you back left that, yeah, that's right, that. So that's the most important functor, is the left ad joint to the thing that was in the definition, that's a very good point, that's yeah. So there's some things I'm not mentioning like that D of R is actually symmetric monoidal and this completion functor is a symmetric monoidal functor and these pullbacks, the left ad joints we were talking about, are symmetric monoidal. 

So these are the things that um and preserve. Subat, well by definition it's defined to be the left adjoint to this restricted functor, yeah, or you take the but but is not true that if you do the left joint on the level of the these envelopes that it necessarily preserves the category, you have to complete at the end. 

So again it's kind of a completed tensor product in this base change here, um, thanks for the question, yeah. Okay so so co-limits in analytic rings, um

 so so so um filtered co-limits or more generally sifted co-limits. I'm sorry, oh, the usual notion. So based on finiteness, I mean yes, oh Alf not filtered, yeah. I thought it might be important because you have, I mean indeed one could imagine it might be important. But when I say this then there's no discrepancy, so it's a, I mean there's no ambiguity, so um so if you have filtered co-limit of RI, then the underlying animated R is just the filtered co-limit of the underlying things, um, and also the free modules are are similarly described is just the filtered cimit of the the free modules on the um so that's rather rather naive, um. 

And what's kind of left is pushouts and this is more interesting so pushouts so if again we have maps of analytic rings I'll call them K A and B um I'll write the push out as a relative tensor product just because um then the derived category of the pushout can be more or less immediately described so I'm not talking about the first point in the data but just the second Point um uh so abstractly this category will be the same thing will

Actually, be a full subcategory of. You take the push out in condensed rings, and then it's the full subcategory such that the underlying a triangle module lies in DA and the underlying B triangle module lies in DB.

But, but this caution. A triangle tensor K triangle B triangle D A Tor K B is not an analytic ring. So, it satisfies one through three but not four. So, it's almost an analytic ring. The only thing is that the unit object, the underlying ring, is not complete. But then you fix that by applying a completion procedure to fix this. You can still prove there's a left adjoint to DA Tor K B sitting inside DA triangle tensor K triangle B triangle.

So, I think you're using D in two ways here. Simp, you mean the category of A is a ring. This when you're writing it, it's also it's also part of the definition of anal. That's true, that's true. So, let me make a remark to reconcile this that there's a trivial example of an analytic ring structure on any condensed ring, which is that you take D of A to be equal to all D of A triangle. And with this interpretation, the notations are completely consistent. So, you could call that analytic ring. You could call that analytic ring a triangle. It's just the um. So, every condensed ring can be viewed as an analytic ring with kind of trivial analytic ring structure or maximal analytic ring structure. Everything is complete. And with that in mind, then there's actually no conflict in the a different thing you could have done here, which is take so you have a an antic R. So, it has a subcategory is part the data. You could take DA tensor DP as the categor DK. And that would be a different thing then doing this. Oh no, actually, it's the same. Yeah, it's the same.

Yeah, so um okay. What next? Um, yeah, question. Oh, question. Oh, I'm going to call on the person who raised his hand first. Sor yeah, that one just the same as the tensor product of the categories. Oh, that was already asked and answered. Yeah, over okay, is it, do you change the ring when you apply completion? You change, you change the, because the simpli R is D anyway, you apply the completion in the category D and then you must get another simpli ring. Yes, we have to prove that it's not obvious but yes. So, there's a so yeah, so then when you apply that completion procedure, the category stays the same but then this becomes completed. And also, I should make a remark that in complete generality, it can be rather difficult to understand this completion process. You kind of have to iterate applying the completion for A and the completion for B sandwiching them between each other take account do it countably many times take a co-limit like abstractly that's the formula for this completion procedure here. Now in practice, it turns out you can calculate it and this is one of the points uh in practice I don't think I have time to discuss examples today but this completion procedure which replaces this by the true underlying ring of uh the the the pushout in uh analytic Rings it produces the geometrically correct completed tensor products in analytic geometry.

Um okay so maybe um there's a question from Zoom. Yes, is the condition on the fenus still necessary or did you show that it is always satisfying it's always satisfied in this that's one of the reasons for choosing this light condensed set framework it's actually always satisfied it doesn't matter um so maybe okay maybe I'll start to talk about examples so I risk kind of getting cut off in the middle of an explanation but I feel like it's just too dry without well I won't get very far okay let's do let's try let's try um so I want to talk about uh maybe solid analytic Rings um and this will relate to attic spaces or hu pairs um so uh it's kind of going to be a non-archimedian addition so so I mentioned that if you if you have an analytic ring and we haven't talked about how to produce them yet but if you have one um then it's nice to look at these free modules and profinite sets to get an idea about what what's going on what what spaces of measures do you are you actually looking at here and I also said that the basic example of a profinite set well besides the point was this n Union Infinity classifying convergent sequences so but um in this linear case it's it's natural to consider the following so given an analytic ring R it's natural to consider what you could call I guess space of measures on the natural numbers and I don't mean the free module on this discret thing but what I mean is you take uh the free module on this sequence space and then you mod out by Infinity so this in some sense classifies null sequences in our modules oh I have a backboard up there too and it turns out it's not hard to show that um addition on n induces a ring structure on uh on this MRN um and as a ring it kind of sits in between two rather extreme options uh uh so maybe maybe you want to think of this as t to the N for the purpose of this kind of discussion um so it's sitting somewhere in between the polinomial algebra and the power series algebra over your ring R um as you could imagine for something like a space of null sequences right uh or sequences with some gross growth condition I mean it's really dual to null sequences it's maybe some kind of summability condition um so geometrically speaking we have the apine line and we have some version of the formal neighborhood of the origin and then we have something that sits somewhere in between right um and now I'm going to single out a condition which is kind of a non-archimedian condition that morally speaking will mean that this guy uh lies inside the open unit dis of radius one um but formally so let's say definition uh R is solid uh if um if you if you do this you get zero but since you qu by the constant R Infinity is just the ah okay this is the home yeah that's just R and then but you know and then it's mapping into r n Union Infinity by the inclusion of the point Infinity so R infinity is the free yeah in this the free R module on

 this finite yes view as a as a as a condensed object or no no it's not cond what what is that like solid over Z or uh just a sec I mean this is the i' so far I've just said this definition right so you I mean in this sense but

Just a second.

Okay, so there's an interpretation of this which is well if you have this and it's actually equivalent then you get a measure so to speak. So, $T - 1$ has to, you know, multiplication by $T - 1$ has to kill. So, if you have anything here there has to be a pre-image under multiplication by $T$ minus $1$. So, this means you get some measure here such that $T - 1 \times \mu$ is equal to the unit object in this ring $MRN$, which is kind of the sequence $1, 0, 0, \ldots$. And if you think about what this means, thinking about this measure space sitting between polynomial and power series, this corresponds to kind of sum over $n$, $t$ to the $N$, that's at the very least what it maps to in the formal power series ring. But on the other hand, this measure space, as I said, classifies null sequences.

And you know, so and this measure pairs with a null sequence to if you have a null sequence in an $R$ module $M$ and a measure, I said you can pair the two things to get a function. And the way it works is you take your null sequence, you put it as coefficients here. And yeah, you set $t$ equal to one. So, what this, the interpretation of this is that every null sequence, you kind of have to work it out but the interpretation is that every null sequence is summable, which is kind of classic non-archimedean condition. So, solid is kind of one way of saying non-archimedean in this context but it's kind of fun that geometrically you can think of it as constraining the location of something between zero and the whole real line.

Maybe I will state the theorem. Yes, no. Okay, so theorem, there's a question from chat, the multiplication closed in $MR$, multiplication closed, I mean, it's a ring. I don't know what may I have sort $S$ as you construct it. I guess it's ConEd as some cone it's not the best. Okay, okay, yeah, yeah, yeah, yeah, yeah, yeah, yeah, but in the Universal case $Z$ with the trivial analytic ring structure, it lives in degree zero, there's no, I mean and also that's a summand, it's really but it's still not it's not in the heart it is in the Universal case it is and to produce a ring structure I can work in it's a ring yeah in the Universal case it's a ring and then by base change it's a ring in whatever sense you want in whatever other context I okay, yeah, okay, um, right. 

So theorem, so there exists a solid analytic ring, so it's called "zolid" and well, the underlying condensed ring is just the usual integer $Z$ kind of discret topology. And then well, the derived category is something which I'll discuss in more detail such that an analytic ring is solid if and only if there exists necessarily unique map from zolid to $R$ and moreover, you can actually understand this analytic ring very very explicitly. So, there are some nice results on linear algebra in this basic category.

So, the first thing is that you, well, the first thing you want to ask is what are the free modules on profinite sets. So, let's say $S$ is some countable inverse limit of finite sets then this is just the inverse limit of the free module on the finite set which is just a finite direct sum of copies of $Z$. And also, this is abstractly isomorphic to some countable product of copies of $Z$ countably infinite unless of course $S$ is itself a finite set. What does that say next to the there exist is that say unique? It says necessarily unique, yeah, yes. 

So, the second thing is that, oops. All right, so these, so these. These are remember I said that these guys always generate the category so in this case these products generate the category but moreover these guys here are Compact and projective generators of the well let's say the of the heart um but they live in degree zero so another thing you have is that the derived category here is just the $D$ usual derived category of its heart so it's enough to talk about the bilan category um and also these are flat with respect to the tensor product the completed tensor product uh which I'm kind of mentioned exists um so here's another point where we use the lightness because Sasha fimale proved that this is not hold if you increase the cardinalities on the profinite sets um and moreover you can calculate tensor products rather easily so tensor product of this with this uh over $Z$ solid is just have the infinite distributive law so to speak um that makes for very easy calculation and uh sorry so just asking the the regular is not enough it is really yeah, it's really accountable really yeah, yeah um right so the and the the collection of finitely presented objects in $D$ of $Z$ heart uh which is it generates it under filtered Co limits um is a bilan and closed under extensions and every every finitely presented $M$ has a resolution a free resolution you could say by product of copies of $Z$'s of length at most two meaning a complex where there's three non-trivial terms and two non-trivial maps uh so this kind of gives you a very good hold on calculations in this category very very explicit um so note that you can interpret this sort of as saying that uh so zolid behaves like a regular ring of Dimension two uh so $Z$ is a regular ring of Dimension one we somehow picked up an extra dimmension and that can be attributed to the nonous dorf phenomena that you see in uh in solid ailon groups um but in in all things told you get a very good handle on this category so I think that's the the only example I have time to discuss and thank you for your attention last actually there's only isable there's only one uh right there's a single generator actually this is kind of a general phenomenon because the the free module on the canra set will always will always generate yeah, yes when when when we taking the pushup uh there's a completion fter AI gives you a DED category object why is it that it's a ring yeah that's something you have to prove and

 it's not obvious thank you yeah mhm yes Chad ask question if there will be lecture notes and video recordings video recordings yes lecture notes we're kind of trying to write a book at the same time as we give these lectures and it's not clear to what extent we'll be releasing things sequentially or all at once at some point so in
 Uh, uh, uh. I mean, keep me motivated to. No, I'm sorry. . What can I say? Yeah, I don't know. Or do you like Reman Rock? Of course, I like Reman Rock. Okay then, it proves the most general possible Reman Rock theorems in analytic geometry. Yeah, so he crucially uses these derived categories and the fluidity of the formalism and so on.

Yes, what's the notion of the right-hand triangle? Um, Peter told me Huber uses it. Uh, had to find something. Yes, is there technical advantage to using a light condens set instead of the general condens? I mean for this antic. Yes, advantage, but for just topology. Oh, for general topology, well, I don't, yeah, I mean, there are some more subtle properties that tend only to hold for general topology maybe, not, but once you start talking about topological groups and so on maybe, yeah, yeah, nice. The same question but backward, yes, go ahead.

I just strictly prefer the light setup to the other one, unboundedness and yeah. Is there any case where I would actually want to go back to that one? Well, it's, at the very least, psychologically comforting when you have a strong limit cardinal that you get like compactly generated derived categories and so on. Turns out it's not so important necessarily in practice, but it's kind of maybe quite important psychologically. But then okay, that's just a larger cardinality bound why you would go all the way, it's just to avoid choosing a cardinality bound and so you can say all compact Hausdorff spaces are profinite I mean or condensed sets but there's no real reason necessarily.

What is dist use Infinity algebras? What are those? The structure, can we compare? But I don't, I don't understand the question but what is this term infinite? Oh, e-Infinity, ah, okay, uh, uh, everything works the same except it's a bit easier if you use e-Infinity algebras, yeah, the people who are comfortable with the infinity algebras are not laughing, why, why not, why not even better? Why not E1? I what is the commutativity, oh yeah, so you could, you could do a version of this theory of course with E1 and E2 but but there's something very special about e infinity or animated commutative which is that co-products are the same as relative tensor products that's very nice and moving to realms where that's broken can be a real pain.

Okay, so that's it, thank you.
\end{unfinished}