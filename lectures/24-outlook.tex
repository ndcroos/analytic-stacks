% !TeX root = ../AnalyticStacks.tex

\section{\ufs Outlook (Scholze)}

\url{https://www.youtube.com/watch?v=YKw1XaueLJY&list=PLx5f8IelFRgGmu6gmL-Kf_Rl_6Mm7juZO}
\renewcommand{\yt}[2]{\href{https://www.youtube.com/watch?v=YKw1XaueLJY&list=PLx5f8IelFRgGmu6gmL-Kf_Rl_6Mm7juZO&t=#1}{#2}}
\vspace{1em}

\begin{unfinished}{0:00}
  Okay, so welcome to the last lecture. Today, I want to give some kind of outlook. With Dustin's lecture on Wednesday, we kind of finished what we promised in the first lecture. So today, I want to talk about some directions one could go in with the kind of machinery we developed.

Some years ago, I did a lot of pic geometry, and I always wanted to have a way to do this not just periodically but also with real numbers and over spy. But it was always clear to me that I really needed a completely new language to talk about these things. As I said already in my first lecture, this is the reason that I was really putting a lot of effort into this project. Finally, I have the feeling that we basically have now the language that we always wanted, and then now, it is a sensible question to just try to really use it to do a lot of things.

It appeared to me that there was this original goal that we maybe had in mind for what the series should do, but on the other, it's also good to look in other areas of mathematics to see how the theory might be useful. I'm not really competent, but I still want to give some vague ideas that I think might be worth looking at.

Okay, so here are some possible directions, and I will start from the most well-developed to the most speculative. First, we do have now a general theory of analytic sheaves, not just conditions imposed on the modules in analytic geometry in all flavors of analytic geometry. This unified theory is not just a formal thing but a full six functor formalism of six functor Street, which lets you play a lot. In particular, some things we kind of looked at using this formalism is that it's actually a non-trivial application of this general theory of sheaves without any Mysteron or otherwise hypothesis. For example, even for B spaces of finite type over the integers, we had to really work a lot to define things, but in our formalism, they just come with the structure, period.

There are also all sorts of Gaga theorems that you can reprove, but you can also prove various new sorts of Gaga theorems. There are various results about the sands of vector bundles, and for example, there's this famous paper of Greenfeld about infinite-dimensional vector bundles, where he proves some nice results, and I think our techniques could be useful for proving yet another variant of that kind of result.

We also know complex geometry, and we discussed things like off for complex manifolds, and such things are kind of one kind of approach using this formalism. One thing which we in some sense still haven't quite figured out but are quite optimistic that in principle could be done is to prove the Sing index theorem using our technology.

Related to these last points, they are, of course, very closely related to the notion of CAS theory. And one thing that was kind of missing for a while is the notion of the CAS of analytic spaces, which is defined first by Toal and then for general schemes by Thomason, and it really uses that you have a well-behaved category of coherent sheaves on.
It and then maybe actually stable. Infinity category of per complexes and then you can define the case theory of that.

But going to analytic geometry, there was the issue that there is not a good enough category of modules of which you could then take, apply these categorical techniques and get some kind of case theory.

You can actually do it; you actually don't use exactly those CL shields as there, but some variant of nuclear modules. But this has not been analyzed to some extent, so this definitely uses the work of Sasha Eimo to define Cas of dualizable categories. And then if your analytic spaces are actually Ed space, this was worked out in the Ph.D. thesis of Andf.

We had some problems at first to define the complex numbers, but also some ideas how to do that. The Cas here might actually also help for this. There are actually also some other relations to the work of Sasim, so he has these very strong results about the category of localizing motives, proving that it's a rigid category, particularly dualizable itself. Using this, you can actually define certain refined variants of Clic hology, topological cyclology, and so on, that are actually not taking values in some kind of complete category as usually when you take as one6 points you get modules over some power series ring which are complete, but instead you can go to nuclear modules in our sense again.

Let me mention maybe that if it's okay, Dustin has a joint project with Brund, where they use some of this nuclear Cas theory to settle some like questions from homotopy theory. Oh yeah, but we figured out how to avoid it actually. Ah, okay, yeah, too bad.

I was a bit disappointed too, but all right, so that maybe where I'm kind of coming from, and where there had a lot of applications, and where really a lot of work has already been done is in the area of like thetical modation and so on.

There was a spetic hology, which was defined from former schemes, and one question that people had was how to really define it for rigid generic fibers. This then very much is related to analytic stats in our sense, so in particular, there is no what's called the analytic R st, defined by the work of Gu. It gives you a way to talk about six fun formalism on what's classically known as like dcat modules, so there's a certain completion of the ring of differential operators, giving some kind of differential operators of infinite order, defined by Aov and One SL. They suggested that when you work in analytic geometry, you should really look at these modes over these dcat modules and try to get a six formalism for that, but as usual, you run into something of function analysis issues doing so.

Using our analytic geometry, Ro G was able to write down, by specializing the six fun formul to specific St, a very general six fun formul for these dK modules.

This is related to D modules; there's also in the pic world the analog of like hi hi bundles, and some kind of Simpson correspondence, the first incarnation of which goes back to Dinger and Fings. These can also be interpreted as in terms of a stack, and so that's there so-called Analytics St.

This you can find in particular, there like, okay, maybe not yet using our technology, just essentially work of Aner and H, who recently obtained really strong results on the P correspondence.

This work on these hotate staks is also very closely related to what's known as geometric 10, which originally arose from Sensi, which is something about P go representations, and which has been used to very great effect by Lou Pun and also Ralo for applications to P points. Also, last Friday, Vin Pon gave a talk about his joint work with boxer, Kary, and G, where they proved modularity of Genus 2 curves of a Q in many cases, and some of the key technical parts of this proof is really using this kind of technology.

What this also is pointing to is really that there should be some version of theal for loc representations, in terms of some kind of geometric Lance of the center, so some version of what I did in my paper was long thought. I mentioned here something that's in some sense combining or just different the and St St, so is also from antic citation.

Thing. This is a proposal for something in this direction. I propose this by Helman.

By and by, in this direction. So, I mean this is of course maybe what my original interest is, that there was this work with SPK on consideration of local Langlands, and I would like to formulate all incarnations of local Langlands in such terms and eventually then also Global Langlands.

This is just to a large extent, I mean, at least work in progress, and this is probably already very speculative, but this is something that's very actively investigated.

Actually, a lot of progress on this was made during this house trimester that happened last summer here in Bonn, and in particular, we discussed a lot about these things. Then, at some point, we realized that now that we understand the Brau story really well and that we understand the kind of correct geometric language to phrase these things in, we can just basically one-on-one translate all the ingredients in the Brau world to the real world. So, there's a real analog of virtually everything that's happening in the Brau world.

So, there's also an analytic stack, and that's actually a funny version that is actually isomorphic to the Brau stack, as it so happens in this case. Where this is some incarnation of a real Hodge correspondence, you can also define an analytic P-space, which in this case actually maybe there's some kind of non-trivial JP. Okay, and again, there's some analytics stack that encodes the vector BCE, and it encodes some kind of periodic variations of F-structures, and so there's also an analytic stack including variations of twist structures.

Of course, this ties in very well with all the work of Simpson in this world, and then Maukie developed and really developed the Ser variations of F-structures, which are generalizations of variations of Hodge structures. There's a certain action of new1 on these things, and you want objects, and then it seems to be possible to synthesize everything and get also a formulation of real local Langlands correspondence for real Lie groups, for some kind of locally analytic representations.

Good. Can you produce the weight filtration too in the analytics stack? It's a very good question. Let me comment on this in a second. Not yet, but there's some kind of geometric Langlands on twist1, which is a kind of real analog of Brau.

I gave a talk in Muenster three months ago, where I was outlining the general form that we should take. I will give some three lectures actually in Princeton a month from now, and I'll say a bit more about how this is supposed to work.

Actually, part of this, it's maybe a small thing, but maybe not. Usually, when you talk about representations of G of R, you run into all sorts of functional analysis issues, and usually, you replace them by more algebraic notions of Banach modules by passing to the finite vectors and some compact subgroup. But then the theory somewhat less invariant because you need to choose this K, and it becomes more algebraic.

But in the real world, there's really not much of an issue of really encoding representations of a real group. We don't need those, but you have many representations which have the same Harish-Chandra modules because you can use different functions.

What we actually do is we will look at the real group. It's a real analytic, it's a group object in real analytic manifolds, and so you can do the kind of real analytic incarnation in the analytic stack. And so, this is a group object in the analytic stack, and you can take the classifying space of this. Let's say the stack to guess this complex number is not that, real. I mean, as usual, like the classifying space, there are something like representations of the group, but to realize them representations, you need to go back to the point where, and there's a functor. So, there's a projection from the point to here, and you can take either P or P upper star, and for any of these usual G-K modules, there's a canonical object in here, and then the P upper star, should I mean, this is something we...

The P should produce a minimal globalization. The P Street should produce a maximal globalization. So, what C you take, you take C with the gas. Yeah, you can take the gas as it's enough. And so, you have the.

Of course, this is closely related to some results on existence of analytic vectors in representations. It should; otherwise, you would not do, do you use such? I mean, the fact there are enough analytic vectors. Let me not try to say anything precise about this relation here, because it's something I still need to think much more about.

So, there was this question about weight structures. And I believe it's related to the following.

Now, we run into the speculative realm. Really, so for these things, I'm pretty confident that some version of this will work out. This is at this point more of a speculation, but I have a very strong belief in it.

Classically, in all sorts of questions about function analysis and complex geometry and everything, you often really need to put metrics on something. If you really want to prove the H decomposition, at some point you need to do some L2 stuff, put metrics on stuff, and so on. This is something that we cannot yet incorporate or what we cannot yet translate into our world everything that's related to metrics at this point. But I believe there is a clear way to look, namely, you should look at this extended version F that we had. I will connect this in a second to this question about weight structures.

So, we have this Berkovich space, and maybe the way to do it is WR. So, you have some kind of central point related to the real numbers, and then there was a rate for Q2, and then at the end of this, you have kind of F2, and then there was a rate for G3 at the end of the G, F3, all the other primes, and then there was also the Prim corresponding to the reals.

Usually, like in the Berkovich space, this kind of ends in the middle because you ask for a triangle inequality. But we don't have to do that, and so there is no point at infinity here. And there's this point at infinity, and our geometry kind of tells you that this point must be related to metric. And to some extent, you can already see that when you work with some fragments of it, but also by analogy here, like if you work topologically, then extending over this point precisely means that you put some kind of vector bundle over this thing, like a Z2 mod over the integral topological numbers. And so, extending over bundle places is definitely putting some metric on things, and it's very sensible to think that whatever exactly happens here, it must have something to do with putting metrics on some kind of real stuff.

Some question we have in our mind is whether there is a way to prove the Hodge decomposition, like for complex K-manifolds or something like this, using some geometry that will involve this extra point here. In this abstract language, this is kind of difficult for me to think about, but we actually have a very good analogy. Again, I mean, we can use this analogy between three and four. Periodically, if you work over this kind of part of this picture, then over this part of the Berkovich space, like over the open part, you can define a complex space in the sense called QP, locally analytic. This corresponds to the union over P, where Z is really just the analytic spectrum of the locally Euclidean function from DP to GP, so ones that are locally developable into power series expansion. And this kind of comes up very naturally in all these investigations there, and we know that this analytic space, which lives over the open part because it naturally has K coefficients, this has a canonical extension. What was the F2 point? And so, this is very much related to a theory of locally analytic power series representations, I mean, these are the coefficients of the groups of some PP, at least some fragments of which you can find somewhere in the literature.

Periodically, there is some kind of, yeah, so this canonical is actually a little bit subtle to write down, it has some divided powers. But it exists and is very important for this P story. And one thing this suggests, and which I don't yet know how to do, is that if you look at the real part of the picture, real, and then close on infinity, then...
Over here again, you also have the real numbers, like as a real analytic space. So, the thing that's covered by, yeah, so there are as a real analytic manifold, and again, you locally, the thing with the functions are the real analytic functions. So, the ones that are locally developable and locally be developed to power series expansion, something that match to the base, and I mean, it match to the to the open part of the base, but like each, each $\phi$ over a point is this one, and it suggests that this should this should extend canonically over the function, yeah, canonically over close Point Infinity.

In a way, I don't yet know how to really think about, but this also suggests that if you have some, like I mean, if this could be done, and this would be some kind of ring object, then also this real analytic group would flips over like, oh yes, this should also should also expend.

So, here, you put, you use the $\mathcal{G}$ at all points, you use the $\mathcal{G}$ structure or use $\mathcal{L}$ for different, no, we don't. I will come to $\mathcal{L}$ structure later today for now, it's not needed for anything. So, you just use the same, the same at all points of the, yeah, so acting on this again, you have the rescaling action. So, there's a lot of different copies of the real numbers now, sorry for that. You, a raling action of the, every single $\mathcal{V}$, but but, and it maps to the $\mathcal{B}$ in in your sense, to the $\mathcal{B}$ space, I mean, to the analytic stack of the. So, this, yeah, so this, this maps to the space of norms, and it's an open Subspace of the space of norms, and over there, you have this $\mathcal{S}^\mathcal{T}$ which is like real of locally analytic, analytic thing, which lives up over the open part over the open Ray, but then there should be a way to canonically extent.

So, but properly speaking, you mean $\mathcal{R}\mathcal{L}\mathcal{A}$ cross $\mathcal{O}(1)$ right, or $\mathcal{Z}$. And so, I expect that whatever kind of group that is, in maybe representations of this have some kind of metric structure attached to them. I don't really know like that, if you want extend representation over the open part of the punct, I would expect this something to putting a metric on it, but I don't know. These are some objects that also exist, that I don't yet know.

All right, maybe let me mention, ah, I mean, also, I mean, this some like, and you can also just just try to understand like, we can look at analytics text over the space of Longs that really map to the close Point infinity and try to understand what the kind of geometry is there, and this is some very peculiar geometry where you, it's still about some kind of real complex manifolds or something like that, but you are able to localize much much more, you're able to really zoom in finely into your space, and so I think it's very interesting to try to investigate what geometry this point looks like. I can all agree, so we can zoom in and some, it's a very different.

So, I mean, you have to, for example, you have to use $\mathcal{R}\mathcal{E}$ sphere, and then the bounded part of the $\mathcal{R}\mathcal{E}$ sphere is actually just what seems to be just the Close unit disc, the bounded part is some kind of weird overconvergent, minimally over convergent neighborhood of the of the close unit disc. It's essentially just the $\mathcal{C}\mathcal{L}$, so any real number that's bigger than one is an unbounded function on this thing.

So, I think it will take a while to figure out how, why do you picture the sphere? Sorry, Peter, why do you picture the sphere? I mean, actually, I mean, we were always looking at this, this normal like $\mathcal{T}^1$ right, $\mathcal{T}^1_\infty$, yeah, and and the 
Antincs

So, like any fine guy, I mean, we first associate just the drive category, but then we could also associate module categories over it. But then, as I was explaining last Friday, a series of present infinity n SP due to Stanage, um, and so you can just so you can do two PRL just continue forever. And I mean, so you can define n PRL on X for any X, and you cannot just define it, but again, there's some you have all the kind of six fun on it, and so you can just play with it.

The existence of this is not a speculation, but what comes next is definitely a speculation.

Okay, so I have a strange project with stars and Z where they're computing some Quantum CH Sim Varian, and I don't know what they are, but they seem to get certain power that seem to be very related to the kind of periodic structures I'm seeing. And for a long time, I'm trying to make sense of whatever they're doing.

But now, recently, I realized that I'm able to send about higher categories, and that well, the C hypothesis tells you that this kind of to fi series I should really just be certain higher categories and then try to understand which one they should be. And you can essentially say it, but it doesn't quite work.

So, here's what's called Quantum transus or something. This, I must be aware that I'm saying words that I don't understand.

Okay, so, so, so, to the extent that I understand anything, um, you have maybe start with a complete group and maybe Forlani simple or something like this, relevant. And then you fit what's called level. I mean, so there's a funny computation that if you look at maybe I don't know, let's see simply connected, so then the first interesting formology group of G is a third formology group, which I mean a simple assumption z, um, and yeah, so let's fix the level, which is just an element.

And so, here G is just considered a topological space, but then you also have kind specifying space of G. Um, I mean, g map to be Cub, so this, if you want the net from G cling space of z, um, this actually d loops it's map of groups automatically obstructing, so m from BG to B4 to the z z. Um, and then, but you can also restrict that to G as a real analytic thing.

Right, because in general, like I had to think that whenever you have a real analytic manifold, it maps to the incarnation of m, like is a condensed set kind of incarnation, and this basically what I'm using here. Uh, and so, but then there, like, okay, so then there's also like the exponential sequence analytic.

Yeah, thanks. Um, have exponential sequence, so everything's living here over, let's say, guess is complex numbers, um, and then a composite map to be to the four of GA venes because for here chology has trivial and so this actually lifts to be cubed of the analytic GM and of course the analytic GM fus.

So now, this sounds really like the enact. All right, but so, so, so, what does it mean to give a map to be cubed here? Well, this is too hard for me to think about, but um, so map to BGM, well, that's just the line model, right? So, BGM, that's just a classifying space for line bundles. Then, b square GM, uh, this is something giving you some algebra or something, so this giving you a Twist of the category of modules. And then the Cub GM, this means that you give a Twist of the category of categories is over over it.

Yeah, so this map called Alpha from BG a real GRP QM specifies a Twist of like PRLG, I some, let's call it just L or invertible, it's invertible object is a 2PR.

This, all right, okay. So, have this, uh, and you have the projection just to the point, or the point is like the guess complex numbers. And I think what people do is like, okay, so they, they want to look at like some family of G tsers like bundles with a flat connection or non-flat connection, whatever, um, and but sometimes Twisted by this Alpha. And so, I mean, this is somehow more or less governed by pulling back this L under Alpha, but then they want to integrate over the space of all G ches. So
Somewhere in a three-category, so now we have one. So the question is, is this streetable, and as any expert would immediately tell you, this, there's no chance of working.

Because the space of conformal blocks, I believe it's called, you must put some holomorphicity constraint, and I'm kind of not doing that here. But I mean, in some sense, it's not so far, so, but it is true, duable, I think that's, that's, you don't need much, okay. So I didn't carefully check it, but I believe it is to dualizable, and to check that it would be S-realizable, and for S-realizability, you would need the following things:

You would need that $\pi$ is kind of from proper and smooth, so the diagonal of $\pi$ is proper, smooth, and the diagonal diag of P is proper, smooth. So there are six conditions to check, and only one of them fails. Okay, so more concretely, this is like the point, this one reduces to the group being proper and smooth, and this means that the inclusion of the point into this, so for all of these, you would need that they're proper, that they're smooth, any guesses for which of the six fails? Actually, the one you probably would expect the least is the smoothness of this map, the smoothness of the second, yeah, only.

Just a digression, the map to be, you said that it lifts to a map from B4Z to B3GM analytic, but you want to BGR locally analytic instead of BG, so the obstruction is something having to do with topology with coefficients in something like continuous topology, but it is in your language, so I'm not sure what it corresponds to, but it's just the étale topology of the stack, basically, the topology of the stack, but it is not true for BG itself, it lifts, or I mean, here would be the topology of kind of condensed set ST, which would actually be the étale topology of this funny ST is singular topology. So if you would go here to be for those complex numbers, you would just T the, this is a singular commod with a complex number, so this one wouldn't lift. You need to go to, the obstruction is has to do with maps to, to right? Yeah, so it's some kind of, I mean, up to this analytic, it's basically coherent, but on this classifying ST as a condensed set, the coherent topology is kind of singular topology. Ah, okay, okay, and it's because it's a compact group that it vanishes on the go to the lotic, yeah, but the compact is much more crucial for these kind of problem sols we can hear.

Okay, so this doesn't word, but in some sense, I feel like it was so close to working, so maybe you just need to tweak this, this here a little bit, and as I said in five, there's actually a canonical candidate where you kind of go to the Point Infinity, maybe that one works, very naive question, uh, doesn't word, how replacing by fiber, fiber of the extension first, and it seems weird that this should put the kind of correct holomorphicity constraint into the picture, but I think the formalism might just work out to do that, and there's also some structure you're not using with the like, like, um, like I think this map to B3GM that you build, it should really have a connection, I mean, like, I think that there, you know, could kind of be going to deLine topology and weight too.

Uh, yeah, so, so maybe this is not yet completely correct. I just want to point out that I mean, you can just play with these things, and you can hope that using if you would actually understand what you're really trying supposed to do, uh, you could write down something that would actually produce something special. So I mean, usually what you would try to do this, you would run to all sorts of issues that you always want to do functional analysis and high categories, and I mean, people manage to do a lot of things, but here you can just very naively try.

Alright, so this brings me to the last thing I want to talk about a little, and this one I actually want to go into a little bit more details, and actually, I mean, this was very fancy hyro, and now it becomes a bit more concrete again. 

So this is about a theory of where fundamental proces now, it is we don't see very well, it's condensed, yes, cond, it'll resolve.

So, yeah, so
The kind of theory where our situation where our series should be useful, and so I just try to see to what extent it is useful. Okay, so let's consider a fiber-like model situation that we may be interested in the following: Consider a fiber product of many M1, M2, next of let's say compact, oriented compact M, M1, M2. Feel free to assume that these Ms are closed immersions. I don't think that's relevant, but I mean, so yeah, consider, but which is of expected Dimension zero, in other words, like d M1 plus d M2. So then, if the transverse intersection turned out to be transverse, this X would just be a finite set of points, oriented points actually, and so you could count the number of intersection points.

This intersection is transverse and finite, and actually also oriented, so any point knows whether it should be counted as plus one or minus one. So you get a well-defined counted sign count of the elements of X. This is a variant under so long as you stay as intersection stays transverse. Okay, so then that's the setup. For, always a question, the question is, can you Define this sign count of X in general, I mean, without transverse intersection, purely intrinsically on X? Of course, every connected component will have a well-defined number, which is the part of the intersection number coming from this connected component.

Sure, but even these local ones you need to Define, right? The good situation, I know, mean you have two circles meeting transversely, and then this plus one, this minus one, intersection zero. But now, this might be generated, I, you might have two circles that I don't know, some situation like this where this might be, I know, this, this, might situation, it can be infinitely conic and Stu, I don't know, I mean, this might be tangent to infinite order somewhere, might be the same for, then cross, I don't know, all sorts of really funny behavior.

Okay, so if instead of compact manifolds, you have some kind of smooth projective varieties, then in this case, it's known how to do this, but also like the intersection cannot be as bad as for smooth manifolds, where the intersection, basically, as a topological space, has basically no structure whatsoever.

Okay, so let me first State Z, yes, compute the F product in our C, so in particular, X is some kind of deriv, and this time, it's actually somewhat critical to do this not over the gases real numbers, but over the liquid real numbers for some choice of id structure. This depends on a parameter that doesn't actually matter for this application. Yeah, so for everything else we did in our course, this kind of liquid structure that we once produced was very much effort, not so relevant, but here, I think it actually really is relevant for reasons I will explain in a second.

So what's the, so yeah, so we can Define kind of a notion of we can look at analytics texts in our setups that just have the property that locally, it is possible to write them such an intersection, and then, one can can only produce a virtual fundamental clause on. Okay, so but, there are some symplectic experts present at NP, in particular, Nate Botman and K Barheim, and I kind of discussed this a little bit with them already, and I want to continue those discussions. But in particular, they made me aware that, like, even in this kind of simple space, it's kind of surprising that this should work.

So, let me say why this is surprising. Let's consider the case where those M1 and M2 are Cur, there's my picture over there. And then, just assume you're in a local situation where you kind of have two things meeting at just one point locally. In the best possible scenario, this intersection is transverse, you just get a point count, I don't know, plus or minus one, depending on your orientations. I don't know, then the next worse scenario is if it's a square function, then okay, this to be zero, and if it's like a cubic function, there should be again, first one, and so on. And so, this tells you that if it's Vanishing to finite order, then basically, it's clear that we're okay, right, because we just need to remember the vanishing order of that function, and this will tell you what the this function crosses the line or it doesn't.

No, but now let's consider a situation. As you can have $\mathcal{C}^\infty$ manifolds, but there are more. Tangent to infinite order, something like $x - 1, x^2$, but then you can have a very similar function which is $s(x) \cdot x$. And so what I'm claiming is that the locus where this function is zero, whatever that means, determines whether this function crosses a line or not. The claim is: the vanishing locus of these functions, just the vanishing locus, determines, I mean distinguishes, these two cases in particular, determines whether the function crosses a line or not in which category in our category.

Okay, so let me make it slightly more explicit. That, like, in the classical geometric picture, it's quite unclear what kind of structure you need to give the anything. I mean, the intersection point which knows it, it must, in some sense, must know more than all the derivatives of this function, because it will never be able to double from the derivatives, but it knows much, much less than the germ of this function. It really only knows eventually. So, classically, we would maybe try to use the whole germ of this function in order to remember whether it crossed or not.

Right, so what is the locus of say some $\mathcal{C}^\infty$ function from $\mathbb{R}$ to $\mathbb{R}$? Let's assume it really just has an isolated zero, $Z$. So this vanishing locus, in general, for this, is just the analytic spectrum of the $\mathcal{C}^\infty$ functions from the module generated by $f$, where $f$ is this thing here. I mean, it is a liquid, special type of condensed $\mathbb{R}$, with the analytic ring structure. Analytic ring structure doesn't matter. It is a very funny one, though. I mean, if $f$ would only manage to find out to order then, this would just be the usual polynomial algebra. Finish in, and so this is some nice algebra, but, if $f$ vanishes to infinite order, then this is some funny non-separated thing, and so technically, you would have trouble and this with a topology, but in the condensed world, I mean, it has a natural condensed structure.

So, here's a proposition that tells you that the vanishing locus somewhat knows about this. If you have two functions, $f$ and $g$, and there exists an isomorphism of liquid algebras between their vanishing loci, you mean, analytic rings with the liquid... Well, I mean, I don't need to put the analytic structure on them, because it's just induced one, so I can really just look at them as liquid algebras. Also, I mean, $f$ is still a non-zero divisor here under this assumption I made, so this really still concentrating degree zero, just not Hausdorff. So, I don't have to say anything.

So, assume that these are isomorphic. The condensed set of $\mathcal{C}^\infty(\mathbb{R}, \mathbb{R})$ means that you have to define smooth functions from $\mathbb{R} \times$ a profinite set to $\mathbb{R}$. You do it in the usual way, yes, the $\varprojlim$ construction, just completely internally in the condensed world, and it will produce the right condensed structure. Or, just remember that it's a $\mathfrak{C}$ space, topological thing, just pass to condensed after verfying those, give you the right condensed liquid thing. It doesn't change the condensed, liquid is just a condition. I mean, just say condensed.

Okay, so assume this, then, actually, $f$ is $g$ times $u$ for some is an invertible function. And so, in particular, being an invertible function from $\mathbb{R}$ to $\mathbb{R}$ must either be everywhere positive or everywhere negative. And so, multiplying by such a unit cannot change whether the function crosses the line or not.

I believe like on nuclear nuclear fet spasis the liquid tensor product is doing the expected thing and so it actually is producing the C functions on the product, but it's not true for the overall G there would be some smaller thing. But concretely, what does it mean? A liquid is a set, I mean R is just the usual condensed ring, yeah, real numbers, but then you put different ring structure there which is much closer to asking that like the building blocks are some kind locally convex things. They're not quite locally convex, they're just $\alpha$ locally convex versus $\alpha$ might be slightly less than one, but you're very close to locally convex setting. And so when you form this T products, it basically allowing all locally convex combinations, and at least if you kind of have nuclear transition maps, then the precise kind of complexity you need actually it doesn't matter, one limit works okay.

So this is the deriv T of product, it would be works. So this is a non-computation that comes out right in the liquid World, and so from this you can deduce that if you take $C^\infty$ functions from $\mathbb{R}$ to $\mathbb{R}$, that this is actually isomorphic over $\mathbb{R}$. 

Uh, is a usual coordinate, yeah, so maybe isomorphism should be condensed our algebra with the coordinate in the proposition, otherwise I don't believe if you don't have the coordinate, so I mean I said that both of them have the isolated zero at zero, that's what I use, but I don't use anything more than that. Let's see what he's doing, but he's not.

All right, so let's analyzation. So let's assume we just have an abstract ASM. So for both of these, they admit unique map to the real numbers, because you can classify all the maps from $C^\infty(\mathbb{R})$ to $\mathbb{R}$, they just given by real numbers, a variation, that's some real number and only one of them, one of them f is equal to zero. So what is definitely true is that they have this unique evaluation to the real numbers, and this must commute, because in those cases.

All right, and now I have $\mathbb{R}$ joint key equal to zero, nothing to here, why is the usual coordinate. Right, I mean maybe I should $C^\infty(\mathbb{R}, \mathbb{R})$ here. And so I get there, and so where does this T go? Well, I mean, let's assume G is not just also Vanishing first order, then like I mean you have this unique map, and you also have to Unique first integral neighborhood, so this must be a function that it's not managing to first order here. Do you actually care about this? Maybe I don't care.

All right, so I have this map, but okay, I also have this compos map, and of course it has a quotient here, and if I want, I can also lift to here, right, because I just need to lift my element T from here to here and I can do that. Yeah, so you get it f is equal to G * unit up to a change of coordinate, so it's G times the change of coordinate compos a coordinate change times the unit change.

Okay, let me to this algebra here, just I mean let call this here A and this here B. So one I want to claim that this map from $\mathbb{R}$ joint key to B is a usual, there exists the unique extension which is just a condition. Okay, and uh, so is a module same, and there is a unique extension in which sense, in the sense abstract Rings or I mean this sense here, right, so it can be at most one extension, okay, saying. And I claim it exists, and for this I can, it's enough to check that it exists here, but I can classify all M from joint T to here, and they all do extend, right, okay, they do extend by just comp by comp. All right, and you really just need the extension is necessary, you need to just need to prove existence, but this map is given by some $C^\infty$ function from $\mathbb{R}$ to $\mathbb{R}$, but then any other $C^\infty$ function from $\mathbb{R}$ to $\mathbb{R}$ can just be evaluated
At least, this property about F crossing the line is only deep crossing the line because so there is n's presentation, not saying it right, but let me make a point here. So, there are some existing theorems, a series of derived manifolds, but they proceed very differently, and in particular, they consider some kind of algebra of functions where, by definition, whenever you have a function, you can kind of compose with any $C^\infty$ function from $\mathbb{R}$ to $\mathbb{R}$. This is some kind of data I put on the Rings, but this observation is telling you that these kind of funny derived potions that we're getting, they just have the property that whenever you have a function, you can always canonically extend to $C^\infty$ function. 

So, they acquire the structure that for any function, you can compose with a smooth function, but you don't need to put it in the beginning. Try to see if can Solage was trying to save. This is the canonical put that you use that probably, and I mean those things can be classified; these are just given by usual new function from $\mathbb{R}$ to $\mathbb{R}$. And so then, okay, so has a ization, and so maybe you have to put a reparameterization in, then just makes a similar, sorry, I want to say the same thing, it's faithful embedding. 

H, so this just pullback by right, so yeah, okay, maybe up, okay, so right, but so some of the thing that makes this work is precisely this thing that you do get these unique extensions to $C^\infty$ function, because otherwise, you cannot control the kind of $C^\infty$ from here to the $C^\infty$ section from here, you wouldn't be able to compare.

Alright, so let me now state some more objects. Let's say, as locally compact Hausdorff space, and let's assume also it's a five-dimensional, which is always satisfied when it's intersection-like that, plus a cheve of animated all together. So, that is locally isomorphic, but not with a given data, so this is just a condition, intersection. 

Then, one can define a family, depending on S, of smooth manifolds, or just constant intersection of smooth manifolds. The intersection of smooth manifolds is constant locally or is it varying continuously? No, I mean locally, you can write open subsets of S can be written as the fin where the that's so coming from the point, yeah, ah okay, S itself is okay, S, S, this is okay. Then, one can define an expected Dimension, some C, it's a locally constant function, this integer coefficient, and an orientation, $\Omega_s$, so this is a zero persistent on, right?

And, okay, so let me denote by $\pi$ from S to $\star$ just the projection of locally contact po spaces and virtual fundamental T of like S equ structure sheet Al, also this depends on S, all of these depend on S, which is a global section on S of the dualizing complex. This makes sense, yes, this makes a lot of sense, and is because in particular, if the expected dimension is zero, as is compact, and choose an orientation, oh my God, fix, I mean, assume exist and fixed one, then you can define the Fundamental class, Global fundamental class, as some of the integral of the virtual fundamental class, Global, I know the count, signed count of SOS as the integral of the virtual fundamental class over S.

Okay, so I mean under G equal to zero in this orientation, this goes away, and then you just, and as compact, you have a trace map that goes back to the integers. Alright, so let me sketch the construction, and so basically, this is just the direct analog of the algebraic geometric construction. I think this was pioneered in the work of, and, then many people worked on this, and I'm not an expert on this at all. The paper where I learned it from, some kind of algebra, I mean, lination I know by a, but the ideas, I think, will maybe. 

So, what they define first of all is this, what they, I think, called the intrinsic normal cone, and so this realizes.

That describes some of the fibers. Some of this corresponds to the cotangent complex. Of like over R, in our world, I mean, first of all, you can somehow define such a derived intersection of like smooth things. But also, if you compute what this kind of cotangent complex does, that can like abstractly be defined for $C^\infty$ functions, basically for the same reason that the T-products came out right. The Čech complex also comes out right for $C^\infty$ functions on a manifold. And then, for such a derived T-part, of course, you get an S. So this is actually a perfect complex.

What's the superscript on the L? Here, some kind, so actually I just want to define this here as a locally compact space. Let's say it's a condensed, some kind of stacky version, where some tangent directions give you stack directions. I mean, it's not really, you don't need to go all the way only to group, you don't need high, I don't need high.

For the L itself, it is concentrated in the okay, okay, yeah, sorry, perfect amplitude, amplitude 01. So, the already did it. Okay, for a compact space or a pro-finite set, if you want, T. Well, first of all, from T to S, you can just regard this as a map from the space of continuous functions from T to R to some, let me call it, unexpected SOS. So that you can, by locally the spectrum of this algebra, you can build an analytic stack in all sense. And then, if you evaluate this on just the continuous functions, you just get maps from T to the underlying set. But now, I want to define what are the maps from T to this intrinsic pH.

This can actually be defined as a map from a funny algebra. You take this guy, and you model it by zero. In other words, this is a tensor product of the continuous functions from T to R, over R, join Epsilon, where the degree of Epsilon is equal to 1 and some particular square to zero. But where the structure shift comes in, okay, it's so, here it doesn't really matter because you always factor over the continuous functions, but here's the R-structure that is focusing that to something. There is, okay, so in the first thing, you can replace it by the kind zero, and even by the homotopy class, because here, not, and just by the, you can analyze what does it take to lift the map from here to here, and it's just something in terms of the cotangent complex. And so, you can realize that the fibers here have some kind of geometric incarnation. You also have the cotangent complex, so all the tangent directions, they give you kind of stacky directions. And, yeah, and if there's some kind of actual obstruction, this is the non-smooth part, and there's actually also some vector bundle directions.

Yeah, some particular kind of smooth map, or can understand it. And so, and yeah, right, so there's a Čech complex, and like the perfect complex, so it has a kind of dimension, which is the difference between the vector bundle dimensions. And so, this already gives you the expected dimension. Just look at the rank of this guy. And this kind of situation here, this geometry here, will also produce the orientation data for you. Think about it.

So, actually, what you really need to produce is the global section, an element in this intrinsic normal cor, with values in the data you really need to produce. And now, there's this really funny thing that let's take R as a condensed set here down here, and then one can write down something which everywhere except at zero, it's just a point, but in this F at zero, you get this funny fullness. There's some kind of condensed form which lives over the real numbers and which everywhere except at zero is just a point, but then as you degenerate to the point, suddenly, there's very sticky thing, and you see this condensed thing here.

Corresponds to, but first of all, let me use G for safety because I'm not sure if I've used F before. Corresponds to a map from, like this should map to R. So, first of all, correspond to a continuous map from T to R.

But once you have that, you can look at maps from the continuous function from T2 R modul G, where, of course, if G is the zero maps, then this is just recovering what I told you before. If G is non-zero, well, then if equ equ by an invertible function, this is just zero, so you're not getting this is no data. So, you guess just get a point, but then if G is a function that's somewhere invertible, somewhere zero, then suddenly you have this kind of thing where is producing some kind of doing these things together.

And now, it's just some simple game with six fun just to produce a class because, basically, generically, you just have a canonical section here, it's just a point, and then you just degenerate that section.

Let me try to do it. So, consider the sheath F which is form. So, this guy here is streetable, and that's the only way I currently know how to check it is to use it. I assume that a der section of smth manifolds, and for manifolds, it's true, and then it's disable on fiber products. In principle, you could probably produce this canonic virtual fundamental class with much small assumptions. The only thing you really need to that this map here isable.

But then, you can Define this thing which is a sheet on followe, and then, okay, so you have an open subset J which is the cone where from zero, and then you have I which is exclusion from the cone, and then you have some excision sequence for F. Here, you have which is just I the of the p stre, and then, in particular, you get a boundary map here, and the boundary map just gives you the CLA you want. And that's always a degree shift, I'm getting confused about, but it did work out.

Because the ididentification the I F when you, yeah's a shift by one appearing here because I'm pulled back from, uh, yeah, so there's a shift by plus or minus one here, minus one because I take Z here on the real line and not pulled back from the point. And so, then you get a m from the X zero of R Z Z towards the H, and okay, this, and then you take a class here which is zero on one connected component and one on the other, and the image is what you're looking for.

By the way, is it true that this pie cone def is not just triable but smooth? No, I don't think so. I would have guessed okay, I don't think so. I so this cone of s, this is some crazy thing right? I mean, it's some kind of fiberwise, it's some kind of vector bule like thingy, but over this crazy s, I mean, this s is completely nasty in general. So, if this map was smooth and also like the F was smooth, but this has just no way of being smooth, I don't think.

Okay, so I don't think this thing that appears here is in vertical at all, but you can still can still degenerate the topological fundamental class on a point towards this cone using a little bit of six fun, and then, yeah, this Con difference from the original space just by some kind of fiber BS, which you can analyze and which give you both the dimension shift and the orientation I'm over time.

I want, I don't know if it obstruction. The well, the TR obstruction series isn't that just that there this quention complex that behaves right? Okay, that's just completely infil this six. Any further questions?
Right. So, it stays a triangle on $St$. Okay, okay, okay. Okay, thanks, Peter. Yeah, all right.

\end{unfinished}