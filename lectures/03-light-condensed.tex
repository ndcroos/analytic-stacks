% !TeX root = ../AnalyticStacks.tex

\section{\ufs Light condensed sets II (Scholze)}

\url{https://www.youtube.com/watch?v=me1KNo3WJHE&list=PLx5f8IelFRgGmu6gmL-Kf_Rl_6Mm7juZO}
\renewcommand{\yt}[2]{\href{https://www.youtube.com/watch?v=me1KNo3WJHE&list=PLx5f8IelFRgGmu6gmL-Kf_Rl_6Mm7juZO&t=#1}{#2}}
\vspace{1em}

\begin{unfinished}{0:00}

Okay, so let me recall a little bit where we were last time. Last time, we had this category of light profinite sets. There are several ways to think about this: either as sequential limits of finite sets, or as metrizable totally disconnected compact Hausdorff spaces, or as countable Boolean algebras. 

\begin{align*} 

= \{ \text{metr top. dis. comp. Hausdorff spaces} \} 
= \{ \text{countable Boolean alg. \}^{\op}

\end{align*}

% 1:29
We equip this with a Grothendieck topology where covers are generated by the following two families: 
\begin{itemize
\item finite disjoint unions
\item \emph{all} surjective maps.
\end{itemize}

% 1:58
As I kind of stated last time, this has a very important consequence that I want to stress again: \emph{sequential limits of covers are still covers!} This property will actually be extremely crucial in order to have a good resulting theory of something like topological abelian groups that we want to use to have good homological algebra properties. This is something I maybe want to stress today a little bit, where this appears.

% Grothendieck topology covers
% - finite disjoint unions
% - \underline{all} serjective mapq
% (\leadsto sequential limits of covers are still covers

% 2:39
Then we had this category of light condensed sets, which were these sheaves for the Grothendieck topology on this category of profinite sets. 

\begin{align*} 
 \Cond\Set^{\light} = \Shv() %todo eq


% 3:12
In general, whenever you have sheaves on some site and the generating site embeds (well, at least if it's a subcanonical site), you have a Yoneda embedding. 
% 3:26
You have the light profinite sets sitting inside there. 
This sends any profinite set $S$ to the functor that takes any $T$ to the maps from $T$ to $S$, maps in profinite sets. 

\begin{align*}
 S \mapsto (T \mapsto Map(T,S) = \Cont(T,S) % todo
\end{align*}

If you prefer to think more concretely in terms of compact Hausdorff spaces, this is continuous maps.

One important property, which is also completely general, is that the image generates under colimits. You can think of general condensed sets as being built out of profinite sets by some kind of gluing procedure.

Then we also compared this to the category of topological spaces. In particular, we can embed this here as certain compact Hausdorff spaces. In fact, more generally, for any topological space $A$, you can build such a thing. I think I called a topological space $A$ last time $\underline{A}$, which is given by the same procedure. You send any $S$ to the continuous maps  $S \to A$. So this diagram commutes.

% 5:23
As I also said last time, I kind of want to improve on something I said last time, this $A \mapsto \underline{A}$ has a left adjoint. It takes any condensed set $X$ to, well, if you have a condensed set, in particular you have the value on a point, which we think of as the underlying set of this condensed set. Then we were equipping this with a certain topology.

\begin{align*}
X \mapsto X(*)_{top} % todo
\end{align*}

% 6:14
Here we're taking the disjoint union over all possible maps from the Cantor set into $X$ of the resulting map from the Cantor set onto $X(*)_{top}$. The reason for the Cantor set being that it's any profinite set that admits a section from the Cantor set. So it's enough to allow the Cantor set here

\begin{align*}
\text{Cantor set) \mapsto X(*)_{top %todo disjoint union
% quotient topology
\end{align*}

% 6:43
But something which I kind of forgot about and that Yosuke Morita reminded me after my lecture is that actually, you can describe this quotient topology differently. Because convergence of the Cantor set can be detected by sequences. Or more precisely, if you map all possible convergent sequences to the Cantor set, it's still a quotient map.

% 7:06
So actually, you can also consider all, let's call this $\beta$, all convergent sequences in $X$ of just the convergent sequence. This is also a quotient map in topological spaces.

\begin{align*}
\end{align*}

So if you want, Cantor set can actually be thought of as the colimit over all, say for example, countable closed subsets. 

\begin{align*}
\text{Cantor set) \cong \colim(Z) %todo
\end{align*}

So for example, finite unions of convergent sequences. This is in $\mathbf{Top}$. It will not be true in condensed sets, and I will discuss this in a second. But it is true in $\Top$.

% 8:15
This means that equivalently, you can describe this corresponding topological space as just the one where you just remember what the convergent sequences are. That's enough to determine this topology. In particular, this means that this funny notion that I was talking about of being metrizably compactly generated is actually just the same thing as being sequential. So that continuity can be checked by checking whether convergent sequences go to convergent sequences.

\begin{align*}
\text{"metrizably compactly generated"} = \text{"sequential"} 
\end{align*}

And so the thing I said last time, someone, more succinctly, is saying that sequential topological spaces embed into light condensed sets. Okay.

Okay, so from this perspective it seems that allowing the Cantor set didn't really help at all. But now I want to say why we allow the countable set. So is it the same or more general than first countable topological spaces? I forgot all my general topology. Does somebody know, is sequential the same as first countable? It could be, discrete on huge, but if the greatness of first countable... Because in the previous version of the theory it was first countable, I forgot, I think first countable with all points closed. But now we don't need all points closed because we don't have the set theory, right?

I mean, why allow the countable set? We're frozen, I'm frozen. I hope I'm un-frozen.

So the reason is that in any topos, for example, whenever you have sheaves on any site, there is an extrinsic notion of being compact and of being Hausdorff. So these are the following notions that I want to recall. Frozen again, frozen again, why? Oh my God.

So this general topos notion is what's called quasicompact. If any cover admits a finite subcover, here this general notion would amount to saying that there exists a surjection from the countable set onto $X$, or $X$ is empty. Yes, thanks.

And then there is this intrinsic notion of being quasiseparated, of being Hausdorff. And this notion was introduced by Grothendieck in SGA4. And of course, Grothendieck was using algebraic geometry lingo, so instead of Hausdorff he said separated. And because it's just a general topos notion, and not the specific separated notion in algebraic geometry, he called this quasiseparated.

So in general this would say that for all quasicompact, well maybe I'm assuming here something about enough quasicompact objects, but there are enough quasicompact objects. And for all quasicompact $Y$ and $Z$ mapping to $X$, so we hope that the technical problems got fixed, let's see, the fiber product between the two is quasicompact.

And again, let me simplify this here again, because one can always surject onto any quasicompact guy by something by the countable set. For all two maps from the countable set onto $X$, the fiber product, I should have chosen notation for the countable set, is still quasicompact.

So this I'm basically trying to say that, sorry, need not be surjective, just a map, that if you map a countable set into $X$, then it may not be injective. So this might factor over some quotient of the countable set. But you're somehow declaring here that it's the quotient by a closed equivalence relation on the countable set, which is one way to express some kind of Hausdorffness.

Okay, so in any topos you can talk about these quasicompact and the quasiseparated objects. And at least if your topology is finitary, the quasicompact guys are exactly those where like a finite union of the generating objects maps onto it.

So if we only allowed, for example, the convergent sequence as our basic guy in the test category, and the quasicompact objects would all be quotients of this guy, or at least be countable. So in particular, the countable set would not at all be quasicompact. And if you want to access the countable set, it would be precisely by such a colimit, that would be this huge colimit of smaller objects.

And this would mean that the intuition for what a compact object is, like the countable set is very much a compact object and you want that, this would be destroyed if you passed to such a topos.

And in fact, I mean, with our notion of what the generating objects are, you in fact get this really nice proposition that, by the way, quasicompact I very often abbreviate to "QC". I mean, in principle I hate abbreviations, but okay, this is QC and this is QS. And if both of them are assumed, then these are called "QQS".

So you can wonder, what are the compact Hausdorff, so to say, light condensed sets?

So this means that the topos, abstract topos theoretic notion of quasicompactness really very closely mirrors the idea of what a compact space is. And similarly, the general topos theoretic notion of what Hausdorff means mirrors precisely what you think Hausdorff should mean. And so this forces our Grothendieck topology to be finitary, because otherwise the basic objects wouldn't be quasicompact. And it forces us to include the countable set in the formalism.

Then you can also wonder more generally, if it drops the compactness hypothesis but still requires a Hausdorff condition, what are those things? So what are the quasiseparated condensed sets? Those also can be described in more standard terms.

So we know that the metrizable compact Hausdorff spaces are allowed. And in general, you should think of these guys as being rising unions of quasicompact subsets. In fact, they are exactly the ind-category, where all the transition maps are closed immersions of metrizable compact spaces. Let me write injections---I really mean all maps are injective, in which case they're automatically closed immersions of metrizable compact spaces. In the ind-category. So last time I was talking about the pro-category, which were formal inverse systems, and these are formal directed systems. So there is a category, where the objects are functors from a filtered poset towards metrizable compact Hausdorff spaces with all the maps injective. And functors, I mean maps, are the usual thing on ind-objects.

So you could also drop the Hausdorff here and then drop this here. And let me just point out that within this category you have what topologists call the compactly generated spaces. Probably meaning sequential. But note that they are not the same, because in this category, as I said previously, you can write the countable set as a huge colimit of countable closed subsets, but not in the category of quasiseparated condensed sets.

I mean, these are both condensed sets. This is the prototypical example of an ind-system of metrizable compact Hausdorff spaces. This is also an ind-system where you just have one term, it's actually some quasicompact quasiseparated guy. But as condensed sets, they are very different. Because this is really just a formal colimit, and this is just one object. In particular, this guy is quasicompact and this one is not.

But actually, if you ask for ind-systems that are countable, then on such things, these all come from here. So some of the difference between these two categories only comes when the colimit category is pretty large, as it is in this case.

So I don't understand the subtlety. Is there a condition on---so if you take the condensed set which is the direct limit of the injections of the countable unions of images of the sequence space, does this give something by the equivalence of categories which is in this category MCG? It's not an equivalence of categories, right? This is just a full inclusion.

What is included in that? Ah okay, this one is included. Okay, I did not realize that this is what you were saying. 

All right. So let me just say something that's implicit. We said that something very nice, like the interval, is definitely a nice metrizable compact Hausdorff space. We said it should be quasicompact as a condensed set. And I said that quasicompact means that there must be a surjection from the countable set. In fact, that is true. You can find a surjection from the countable set, and in fact some kind of canonical one.

I mean, if you have a sequence $a_0, a_1, a_2, \ldots$ of either zeros or ones, you can send this to the number in binary $0.a_0 a_1 a_2 \ldots$, the binary expansion, which is an element in the interval. And any point in the interval admits such a representation. So you get the surjective map.

But we definitely need the countable set, or something as large as a countable set, to surject onto here. And so you can then recover this whole guy as a certain closed equivalence relation here, where the equivalence relation is precisely this nasty thing that $0.111\ldots = 1.000\ldots$. All right
Okay, so we're mainly interested in this formalism of condensed sets as a framework for doing some kind of algebra where all the objects have something like a topology.

To get this started, let's talk about condensed abelian groups. There are actually two ways to think about this, which I didn't say last time. Either these are abelian group objects in condensed sets, or these are sheaves on this category of condensed sets with values in abelian groups. Let me just write in symbols: sheaves on $\mathsf{Cond}$ with values in $\mathsf{Ab}$. Similar remarks apply to any kind of algebraic structure. If you have rings, for example, you can view them as ring objects in condensed sets or as sheaves of rings.

From the general theory of sheaves and Grothendieck topologies, we know that this is an abelian category, in fact a Grothendieck abelian category. So it has filtered colimits, etc. It also has a tensor product. Let me describe a little bit what the properties are.

The unit object is just the condensed set $\underline{\Z}$ associated to the integers as a discrete set. If you have two condensed abelian groups $M$ and $N$, then $M\otimes N$ is the sheafification of the presheaf that sends a condensed set $S$ to the tensor product $M(S)\otimes N(S)$. This is just a functor from condensed sets to abelian groups, a "presheaf", and then you can always sheafify.

There's a further property that I want to stress, which is also completely general. If you have a condensed abelian group, then you can forget its abelian group structure and just have an underlying condensed set. But this has a left adjoint, a kind of free construction. A "free condensed abelian group"---let me drop the "condensed" when I say that now---takes any condensed set $X$ to the free abelian group on $X$. Here, as in general for these colimit-type constructions like tensor products and left adjoints, you always have to sheafify. So it's the sheafification of the presheaf that takes any $S$ to the free abelian group on $X(S)$.

There is already some structure here that you don't often think about in topological abelian groups. You don't really often take the free group on a topological space. Here it exists, and it's actually a completely fundamental structure.

The idea is that this free abelian group on $X$ is some topological abelian group. What is its underlying group? It's just the free abelian group on the underlying set $X(\ast)$. So if you have any kind of topological space, you can just take its free abelian group. Of course, this machinery will then put some kind of topology on the free abelian group.

Let me actually discuss this in an example. Sheafification won't change the value on the point, because any cover of the point is split, so the sheaf condition is kind of vacuous on the point.

Here's an example, which is kind of related to the introduction I gave last time. We can take the real numbers $\underline{\R}$. Very soon I will forget to write the underline all the time, because implicitly everything has become a condensed set. But okay, so we have the real numbers as a condensed set, and then we can take the free condensed abelian group on that.

What kind of object is this? It's sums, if you want, of real numbers $x$ weighted by integers $n_x$, where the $n_x$ are almost all zero. You might think of these points $x$ as measures, and then these are finite sums of measures.

But now it also has some topology, where you kind of remember that these $x$'s are allowed to move continuously. But then, if you have $x + y$ in general, that's a non-zero element. However, when $x$ and $y$ become the same, then suddenly this collapses.

It's maybe not so clear how you would actually describe this topologically. If you wanted to describe this as a topological abelian group, you would have to declare what the open subsets are, and I think that's a little bit tricky to visualize.

But you can actually say what it is as a condensed set. First

Again, the $C_0(\Z)$ itself can be written as a rising union over all integers $n$ of subsets where you are only allowing sums $\sum_{i\in I} n_{x_i}[x_i]$ where the sum of the absolute values of the $n_{x_i}$'s is at most $n$. Everything is contained in something like this, as condensed sets.

So whenever you have a profinite set mapping into here, it will actually factor over one of these subsets. And these guys, they are compact Hausdorff and metrizable. It's a kind of fun exercise to figure out how to describe such a compact Hausdorff space.

I claim that whenever you have any compact Hausdorff space and you look at finite integral sums of points of them where the sum of the coefficients is at most $n$, there is a canonical compact Hausdorff topology on that. Classically, this takes a little bit of thinking. But in this formalism, this free construction just produces it for you automatically.

The subtle part is that there are some kind of non-trivial identifications you have to make. In general, $x+y$ is a non-zero element, but when they become equal, you have to collapse this to zero. To make this true, you actually have to use that in our Grothendieck topology, we didn't just allow finite disjoint unions but also effective epimorphisms. Otherwise, this wouldn't come out right.

Also, by general nonsense, if you start with something that already had a group structure, then if you pass to the free ring, this now has a ring structure where the multiplication comes from the addition. In particular, this is actually a condensed ring completely naturally. This is kind of related to the question I had in my first lecture, like how do you draw an element in all its real powers? It's just done by this construction.

These are general features. Now I want to mention a few features that are quite specific to this light condensed setting. The first thing is that countable products are exact. This might seem like an operation you're maybe not so often doing, but something you are certainly very often doing when you do some kind of functional analysis is to take sequential limits of surjective maps, and they are still surjective.

For example, in functional analysis, maybe you have some kind of Fréchet space and it's a sequential limit of Banach spaces. Maybe in that case it's not even surjective, but it will also come out right. You definitely want sequential limits to behave nicely. For example, if all the transition maps are surjective, you definitely want the limit to still be surjective. As I will argue in a second, this more or less forces you to go where your basic objects that define your site must be some kind of totally disconnected things.

You have these, and there is this other property that I will explain in a second. You have the sequence space $\Z^\N$, it's a profinite set or light condensed set, and you take the free ring on that. This turns out to be internally projective.

Recall that one way to say what projective means is that in any abelian category, you can define Ext groups. It means that $\Ext^i(P,-)=0$ for $i>0$. Internally projective makes sense when your category also has a tensor product, because if you have a tensor product, then you can define an internal Ext, and internally projective means that the internal $\Ext^i(P,-)=0$ for $i>0$.

The first two properties are actually things that are better in all condensed abelian groups, because all products are exact. Well okay, the third property is also solely still true. However, the third property is something that's only true in the light setting.

Let me just define the internal Ext. Basically, if you have a tensor product, then you can also ask for an internal Hom, which is some kind of adjoint of the tensor product. Similarly, the internal Ext will be some version of the same thing on Ext groups.

Okay, so let me prove this actually. Let me first note that one reduces to showing that taking products of surjections is surjective. The only thing that's not true for general products is that a product of injective maps is always injective and so on. The only thing that's not clear is that the countable product of surjections

For all $M$, you can take the product over all $n$ at most $M$ of $M_n$, and then the product over $N$ bigger than $M$ of $N_n$, and this surjects onto the product of the $N_n$'s. Now because a finite product is the same thing as a finite direct sum, they always preserve surjective maps. They're always exact, and so you can always, like, for finitely many coordinates, kind of do the lift. But then this guy here, I mean, this map is the limit now over $M$ of these things.

And so if we're asking whether a countable limit of surjections is still surjective, let's assume you have such a diagram. We have $M_0$, $M_1$, $M_2$, and $M_\infty$ is the limit of these, which certainly maps to $M_0$. We ask ourselves whether this is surjective. So what does it mean to be surjective in the sense of sheaves?

This means that whenever you have one of your objects generating your site, so any light profinite set, and a map from here, then, well, if it was a surjection as presheaves, you should immediately be able to lift that to here. But in fact, it's enough to find a surjective map from some---let's call it $S_\infty$---and a lift to here. So does there exist some other light profinite set surjecting onto $S$ and a lift to $M_\infty$? That's the question. That's what surjectivity on coverings amounts to.

But let's just see what happens. We definitely, well, right, so this is this map to $M_0$. But we know, because this map is surjective, we know that there is a light profinite set and a lift to here. And then again, because this map is surjective, there exists some further light profinite set and a lift to here.

And so now you can just take $S$ to be the limit of this diagram. So inductively, you construct light profinite sets with a map to here, and then take the countable limit of these surjective maps. As I said last time, and I think I recalled today, countable limits of surjections are still surjections in light profinite sets. Hence, this guy is still allowed as a cover in our Grothendieck topology.

So in the definition of your $X$ group in the sheafification, you drop the underline and the left $p$. So $\Z_p$ $s$, you mean $\Z_p$ $s$ underline? Let me drop the underlines. Okay, so $S$ is for me a light profinite set, and they sit inside of light condensed sets. So if you feel better, make an underline. And like, the Yoneda embedding has no decoration for me, it's just the same thing. But yeah, also in three, like, this threefold guy, it's like, maybe I should also underline $10$ here.

Okay, so before I go on to the proof of three, let me reflect a little bit on what happened here. This finishes one and two, and let me come back to that in a second.

So here, the critical thing---we definitely, for doing good homological algebra, like a functional analysis kind of homological algebra, you definitely want property two. But critical for two was exactly this property that countable limits of surjective maps are still surjective, that limits of covers in your Grothendieck topology are covers.

And I claim that this basically forces you to use totally disconnected spaces as building blocks. Why? You might also have the idea, and I think people are doing that, that if you want some kind of nice category of something like topological real vector spaces, you might work in this kind of smooth setting where you take your defining site to be like smooth manifolds, and the Grothendieck topology just the usual one of open covers of smooth manifolds.

But in that case, this property that countable limits of covers are covers is just not true. Because if you have, like, maybe just the real line, then you can cover it by two intervals. Each one of them you can again cover by two intervals, and then keep doing that. But then the intervals shr

Disconnected compact Hausdorff spaces. And initially, we took all of them. We realized it's slightly better to restrict to symmetrizable ones and to have all surjective maps as covers. Because you can actually show that any surjective map of topological spaces can be written as a sequential limit of actually open covers. So if you want open covers and their sequential limits, you need all of them.

Actually, I should maybe say that this is not some kind of ahistorical comment. This pro-étale topology was, in fact, first---I mean, so this stuff about condensed sets, this comes from something called the pro-étale topology, like originally in Bhatt-Scholze for schemes. And there, the wish was precisely that limits of surjective maps should still be surjective. Because this was---we wanted to have certain sheaves which were naturally certain inverse limits, and we want them to be well-behaved. And for this reason, we allowed these countable limits, or all limits of covers, to still be covers. So this is really the origin of this whole theory.

All right, so this was a small interlude. Excuse me. So if you have a surjective map of topological spaces, for example, you take the $\N \cup \{\infty\}$, two copies of $\N \cup \{\infty\}$. So all of the two sequences will converge to infinity. And then this is covered by two copies of $\N \cup \{\infty\}$, and they intersect at infinity.

You made a statement to the effect that--- not all transition maps are surjective. But actually something slightly stronger is true. I don't need all the transition maps to be surjective. I only need that the maps down to $S$ are surjective, and the limit is still okay. And in that sense, you can realize $S$ as okay in this other sense.

Okay, something slightly weaker would be---I'm not sure. I would have to think how much difference it makes.

Okay, so I want to prove this thing that this is internally projective. But before I do that, let me make a warning that this is a phenomenon that's only true once you pass to groups. So $\N \cup \{\infty\}$ is not at all projective in condensed sets.

And I actually kind of expected Gabber wanted to go there with this remark. Because this is precisely the example that I need to do. So you can have a surjective map of topological spaces and the convergent sequence downstairs, so that there does not exist a lift. And one example for this would be to just take this to be the convergent sequence itself, and this to be like, I don't know, $2\N \cup \{\infty\}$ disjoint union $2\N+1 \cup \{\infty\}$. Some breakup of $\{\infty\}$ as like the limit of the even guys and the odd guys.

Then like on the integers, you only have one possible lift. But the even guards converge to something else than the odd guards. But miraculously, once you pass to groups---it so happens that if these were condensed abelian groups, in fact you don't---then any convergent sequence can be lifted.

All right, let me actually use a slightly different guy. So let $M$ be the free group on the null sequence. So you can take $\N \cup \{\infty\}$ and mod out by $\{\infty\}$, which is actually a direct factor, right? Because you have $\{\infty\}$ mapping here and then projecting to a point. So this splits, actually has a direct summand.

And so, and of course, the integers themselves, they are projective. That's okay. So the question is really whether this other part---so this classifies null sequences mapping out of $M$, is the same thing as giving a null sequence in the other guy. Or in other words, it's a free condensed abelian group on a null sequence. And free group on a convergent sequence, but then the limit point should be zero.

So if we want that this guy is internally projective---and let me actually focus on the projectivity, and then just
The property that $\infty$ goes to $0$... Because this is a surjective map of condensed groups, this means that there is some surjective map from a profinite set that lifts to $\N$. That's just what the surjectivity means.

Now how do covers of $\N \cup \{\infty\}$ look like? $\N \cup \{\infty\}$ is like you have a discrete set and then it accumulates towards infinity. Now, in the pullback, you have some complicated $S$ upstairs here. But you can always make that smaller because, for each of the discrete points here, you can just pick any lift. I don't care which one, and then all these lifts together with keeping everything at $\infty$, this is a closed subspace. So there exists $S'$, a closed subspace, so that over the integers, it's just always a point.

You can assume that this $S$ here is somehow a different compactification of $\N$. Without loss of generality, we can always make $S$ smaller here, as long as it's surjective. So we can assume that $S$, the part over any finite thing, is just a point. Okay, but there are many compactifications of the natural numbers. For example, this one, so we can't expect that we can directly split that.

But now we actually use a property of profinite sets that I mentioned last time. Okay, so let $S_\infty$ be the fiber over $\infty$. This might be profinite, so there's a certain subspace in this, but everything here is profinite. In particular, $S_\infty$ is profinite. I stress it here because here it's really the critical point where profiniteness is used. This means that it's injective in profinite sets. In all profinite sets, it's also non-empty. And so there exists a retraction $r$ from $S$ to $S_\infty$.

Okay, so let me call this inclusion here $i$. Maybe this was my original map $f$ and this was a map $g$. Now we can just write down the lift. Consider the following map from $S$ to $\N^\sim$, which is the map $g$ that we had. But then from it, you subtract what you would get if you first use the retraction and then re-embed into $S$ and then apply $g$.

So if you look at $S_\infty$, then on $S_\infty$, these two maps are just the same because this was a retraction. Am I frozen again? So on $S_\infty$, it's actually just the constant map $0$. Was that a question? Am I frozen again? Do you see me now?

But $S$ surjects onto $\N \cup \{\infty\}$ and this map to $\infty$. Because this is actually a surjective map here, you can show that this is a pushout in profinite condensed sets. That's basically a version of this gluing condition that I gave. If I want to give a continuous map from here to somewhere, it's enough to give one here and here which agree on the overlap. But it's definitely enough to give one here, and then the question is just when does it descend down to $\N \cup \{\infty\}$? This means that on fibers it must be constant, but there's only one fiber in some sense where there's something to check, which is the fiber at infinity. And there we precisely assume that it all just comes from a point.

But now you're precisely in this situation. You have a map from $S$, you have the zero map, and they agree on $S_\infty$. So this means that actually this whole map factors over a map from $\N \cup \{\infty\}$ to here. Let me call that map $f$.

And now we basically have all the structure we need, and we just need to check that we actually produce the lift. So what do we have now? Now we have a map from $\N \cup \{\infty\}$ to $\N^\sim$ which vanishes on $\infty$. So in other words
Because this was a retraction, and so this means that, because of this pushout property, you really get an exact sequence. You have a map from the convergent sequence modulo infinity to $N$.

Now the question is whether we've actually lifted $f$, and the claim is that we did. To check that we did, note that this guy here is still surjected on by the free guy on $S$. So to see that this map here is $f$, it's enough to check for the composite.

The point is that if I take this thing that I used here as a correction term and I project that down to $N$, then this map from $S$ to $N$ vanishes on all of $S_\infty$. This term projects to zero in $N$, because its composite map from $S$ and $N$ vanishes on $S_\infty$. So the correction term indeed vanishes, and $g$ was a lift, so we're fine.

For internal projectivity, let me just not do it. It's essentially the same argument, you just have an auxiliary profinite set floating around that you also need to cover a bit. Again, it's critical that sequential limits of surjections are surjective.

Here's just one remark about the comparison to all condensed abelian groups. This is something that I already said a bit earlier, but just to say it more explicitly:

In all condensed abelian groups, products are exact and you have projective generators $\Z[S]$ where $S$ is what's called an extremely disconnected set. For example, it's a Stone-Čech compactification of some discrete set. These are huge things, they have cardinality $2^{2^{\aleph_0}}$.

One thing that's better there is that you really have projective generators. You don't have projective generators in light condensed groups. You have this one guy which is the free guy on a null sequence, but the free guy on a countable set cannot be covered by a projective object. This is slightly unfortunate, but actually not that bad in practice.

The good thing is that in $\CondAb^\light$, you have this really explicit projective object which is the free guy on the $\N$ sequence. Whereas here, these projective guys exist by the axiom of choice, but they are completely inexplicit and their precise structure depends on which model of set theory you're working with.

But one thing I want to point out is that none of those projective generators there are internally projective. Basically, if you take a product of two Stone-Čech compactifications, and $I$ and $J$ are infinite, this product is never projective. That's actually not so easy to see, but we can prove it. This is a rather severe technical issue in the category of all condensed abelian groups. It's extremely nice that you have these projective guys, but for many arguments you would really like to know that they are internally projective, and they are not. That's bad.

In light condensed groups, at least we have this one really nice internally projective object. That's good.

The other thing I wanted to point out explicitly is that the free guy on the $\N$ sequence is not projective in all condensed abelian groups. The light ones embed into the category of all condensed abelian groups, and you could ask if it's still projective there. But there, precisely the Stone-Čech nonsense gives the obstruction. When some universal compactification of the integers, the biggest one, is the Stone-Čech compactification, this rejects. You can show that there does not exist a splitting here.

Maybe another thing I should say is that this is also related to certain questions about Banach spaces that people have studied in literature. There, the only known injective Banach spaces—there's some kind of duality that what used to be projective in this condensed stuff will become injective in the category of Banach spaces—are continuous functions from some $S$ into $\R$, where $S$ is one of these extremely disconnected guys which I could have also allowed. Basically, such a Stone-Čech compactification or a retract of it.

It's known that all injective Banach spaces are retracts of continuous functions on a Stone-Čech compactification.

What about injectives? In your case, it is for light condensed abelian groups. Is it enough injective? For all condensed abelian groups, there are no nonzero injectives. In all condensed abelian groups, there are no non-zero injective objects.

They exist for set-theoretic reasons. For like general nonsense reasons in light condensed abelian groups. But I don't think you can write any of them down.

And what corresponds to this free abelian group sequence is the Banach space of null sequences. This is not injective as a Banach space. But it is what's known as separably injective, where you only test this injectivity against separable Banach spaces. This is very much related to the thing that this guy is not projective in condensed abelian groups, but it is in light condensed abelian groups.

% 1:24:37
Actually, I think I made this realization that it is projective in light condensed abelian groups after looking up this proof that this guy is separably injective. It was an open question in the Banach space literature whether the continuous functions on this guy is a separably injected Banach space. In the notes on complex geometry, Kan kind of proves that it's not separably injective. 
% 1:25:12
In particular, this guy also doesn't behave like a projective object, even when you test against light condensed abelian groups.

One thing to take away here is that you might think that all this totally disconnected nonsense shouldn't really appear when you do functional analysis over $\R$. But actually, people that studied Banach spaces intensively, they are very much studying continuous functions on totally disconnected things.

% 1:26:01 
Let me finish by talking about cohomology. 

\textbf{Question:} You also said something, I forgot in which context, about the axioms AB5 and AB6 of abelian categories.

\textbf{Answer:} In Grothendieck's Tohoku paper, you can find a lot of axioms that an abelian category might or might not satisfy. AB5 is the question of whether all products are exact. This is true in all condensed abelian groups. It's not true in light ones, but at least the countable ones are okay. When AB5 is satisfied, you can ask about AB6, which is a certain question about the commutation of infinite products with filtered colimits. This always sounds confusing, but you can actually make a statement that's true.

% 1:26:47 
This property, AB6, is always true if you have enough projective generators. In particular, it's true in condensed abelian groups. If you restrict this question to countable products, then it's also true in light condensed abelian groups. 
% 1:27:07
So AB6 holds for countable products in $\Cond\Ab^\light$. One way to see this is that you have this fully faithful embedding of light condensed abelian groups into all condensed abelian groups. This commutes with all colimits and countable limits. In particular, these countable products are some of the correct products there, the ones that you would also compute in condensed abelian groups. But you can also just check it by hand.

\subsection{Cohomology}
In the last bit, I want to talk about cohomology. When you study topological spaces, you probably also care about the cohomology, in particular for manifolds where you would want that to be the singular cohomology. When you think about a topological space, we already have an idea of what the cohomology should be, just as we already had an idea of what a complex thing should be like, and so forth. 
% 1:28:25
But on the other hand, whenever you work in a topos, that topos somehow comes with its preconceived notions of not only what compact objects are, but also what cohomology is.

% 1:28:38
If $X$ is any light condensed set and $M$ is any abelian group (it could even be a condensed abelian group, but let's restrict to discrete ones for the moment), you can define the cohomology of $X$ with coefficients in $M$. This is just something that is there whenever you work in the topos.

One way to define the cohomology of $X$ with integral coefficients, in terms of the formalism that I now introduced: You take R Hom in the category of $X$-groups in light condensed abelian groups (actually they would also be the same as in all condensed abelian groups, oh well, I'm doing light stuff so let me stick.

% Definition
H(X, M) = \Ext_{\Cond\Ab^\light} (\Z [X], M)

So if say $X$ is a CW complex, this thing is $X_\bullet$ or it's underlined. It's exactly the singular cohomology. 

% 1:31:07
% Theorem
if $X$ CW
$H^i (X,M) =~ H_sing^i (X, M)$ todo
is exactly the singular homology 
% 1:31:30
This might seem a little bit weird because we didn't really put any geometry into the definition of these condensed sets, right? We were just using totally disconnected things, and suddenly using totally disconnected things, we are still able to probe whether that circle is not contractible. But it comes out right.

% 1:32:03 
Let me make one remark about this and then stop. So as I said, this is here the $\Z$ of these $X_\bullet$ groups, and I told you how to think about this, right? So this was the thing where you take finite sums of $X$-points of $X$ valued by integers. 
And so this is actually---there's a Dold-Thom theorem which basically tells you that this guy is something like a model for the homology of $X$ once you, once you pass, once you treat this up to homotopy equivalence. 
% https://en.wikipedia.org/wiki/Dold%E2%80%93Thom_theorem
% https://ncatlab.org/nlab/show/Dold-Thom+theorem
This kind of guy is a model for the homology of $X$. For us the interval in condensed sets is like an actual interval, it's not a point. So we didn't yet pass to homotopy, but one way in which you're doing that is by taking $X$'s out of it, because an interval cannot map to something, this being discrete. And so this dualizing, I mean this is like homology, so the dual should be like cohomology, which fits the picture. Let me just stop here.

% 1:33:51
Other questions? So we can also consider the internal $\Ext$-functor. In this case, this would just unravel to an adjunction to also replacing $M$ by the continuous functions from $S$ to $M$, which is still an abelian group, so it doesn't really do that much more. 
% 1:34:19
Maybe one comment to make is that, and maybe I won't talk much about it, but for all sorts of things like continuous group cohomology and so on, sometimes it's not quite clear, like, what is a continuous representation of a group, what is the right notion of continuous group cohomology? Because often this is just represented by some explicit cochain complex but if you just treat your topological groups as condensed abelian groups then it's clear what an action on a condensed abelian group should be, and it's clear what cohomology should be. There's some kind of general topos-theoretical answer to what it should be, and this always gives you the expected answer. It's not always the same thing as the thing computed by continuous cochains, but when it's not, it's a better answer.

% 1:35:09
\textbf{Question:} So here, when you have a CW complex, if you have more generally a local system, I think it will give a sheaf on condensed sets over $\underline{X}$, and then you can say that the usual cohomology, it's like singular sheaf cohomology. It satisfies cohomological descent for at least for compact Hausdorff spaces, it satisfies cohomological descent of surjections of compact Hausdorff spaces. So to compare the two sides, it looks like using the theory of cohomological descent to essentially to do the, should give, to do it with local systems. 

\textbf{Answer:} You can use any coefficient system, really, on $\underline{X}$. It's a really robust result.

But if you, for example, work in this topos of sequential spaces, where you only allow $\N \cup \{\infty\}$ as a generating object, then if you would try to compute this cohomology of, like, the interval, then you would first have to express your interval in terms of your generating object. So we would write this as this huge colimit of all these small countable sub-closed subsets in the interval, and then this $C$

\end{unfinished}