% !TeX root = ../AnalyticStacks.tex

\section{\ufs Normed analytic rings (Clausen)}

\url{https://www.youtube.com/watch?v=wk_wInYTasQ&list=PLx5f8IelFRgGmu6gmL-Kf_Rl_6Mm7juZO}
\renewcommand{\yt}[2]{\href{https://www.youtube.com/watch?v=wk_wInYTasQ&list=PLx5f8IelFRgGmu6gmL-Kf_Rl_6Mm7juZO&t=#1}{#2}}
\vspace{1em}

\begin{unfinished}{0:00}

Okay, so now continuing the discussion, I want to continue the discussion from last time. I want to start by maybe going over a little bit of what Peter said, perhaps adding some details or some explanations. The main topic is normed analytic rings. 

We had this, and now we've officially defined this framework of analytic stacks that we're working in. We saw that there was an embedding of --- well, at least a functor, we saw that there was a functor to analytic stacks, and it's induced by these things that are generated by profinite sets. So, say $T$, and to one of those you assign---I'm just going to write it, I know this is probably overloaded notation, but you just have this discrete ring of continuous functions on your profinite set with values in the integers. It's a discrete ring, and you can view it with the maximal analytic ring structure, so uncompleted. You look at all condensed modules over this discrete ring or derived condensed modules over that discrete ring, and that's the analytic ring that you associate. And then its spectrum is an analytic stack.

The thing that makes this functor nice is that this assignment here sends hypercoverings to Čech hypercoverings, satisfying Čech descent or descent. Apparently, we don't need to prove that these are equivalent, Čech descent and $C$-descent. This is something that Peter referred to in some explanation. The idea is that when you have a category, it's equivalent to the inverse limit by Čech or the direct limit by $C$, and those are purely formal facts about pro-limits, proved by Lurie, that don't need anything on the shape of the diagram or whatever.

So these were the things that defined the notion of analytic stacks, but it also preserves finite limits. So, for example, pullbacks---well, let me say finite limits. If you take a pullback of profinite sets, that's just a filtered inverse limit of pullbacks of finite sets, and then it goes to a filtered colimit of the same situation for finite sets. This reduces to finite sets, and for finite sets, it's kind of completely clear. And those two facts kind of imply that this functor here, the induced functor, is cocontinuous and finite-limit preserving, which is nice.

Now, I want to mention that there's a paper on the arXiv by Rogozhin, and he studied the analog of this with analytic stacks replaced by fpqc $\mathcal{C}$-sheaves, in usual sheaves, fpqc sheaves in usual algebraic geometry over a field. He wrote a nice article, which he recently posted on the arXiv, which goes into some detail about properties of---well, I mean he's working in a slightly different setting, but much of it is the same. I would recommend reading that.

Okay, and then the other example that Peter discussed was an example of a condensed set. Let's start with this setting. Let's say $K$ is a compact Hausdorff topological space. How does one get the associated analytic stack from this?

Well, the underlying object is a light condensed set, but there is no well-defined notion of an underlying topological space of an analytic stack. You can map this to topological spaces, but the analytic stack itself is a generalization of affine schemes, and there's no direct way to extract a topological space from it. Once we introduce normed analytic rings, we'll see that there is a nice way of extracting an underlying topological space, but for arbitrary analytic stacks, it's not so straightforward.

Underlying topological spaces for things over that normed analytic ring, let me continue the story, having unsatisfactorily not addressed the question. So, if you have a compact, Hausdorff, metrizable, and finite dimensional space $K$, this is the same as saying that $K$ is embedded as some closed subset of a finite product of copies of the closed unit interval. These things are very easy to imagine.

Then, we can view this as a condensed set, in particular, a condensed analytic space. And then, we get an associated analytic stack. Peter made a claim about the functor of points for this analytic space associated to $K$.

There are two claims. The first is that the most important invariant of an analytic stack is its derived category. We saw that étale descent implies that the derived category of a stack is well-defined by pullback. The derived category of the analytic stack associated to $K$ is just the derived category of sheaves on $K$ with values in derived abelian groups. More generally, if you base change this stack to an analytic ring $R$, you get the derived category $D(R)$.

The second claim is that if you want to map, say, $\text{Spec } R$ for an analytic ring $R$ to this $K$, this is equivalent to giving a symmetric monoidal, cocontinuous functor from sheaves on the condensed abelian group $K$ to the derived category $D(R)$, such that, étale-locally on $\text{Spec } R$, the functor sends the connective part of the sheaves on $K$ into the connective part of $D(R)$.

The finite dimensionality of $K$ is not a crucial assumption here. You can define the analytic stack for any $K$ without dimension restrictions. I will sketch the argument for this.

The key point is that both the left-hand side and the right-hand side are sets, not higher structures. In the world of analytic stacks, everything is implicitly a condensed object, which could introduce higher homotopy, but the pullback-preserving property of the functor ensures that it is actually a monomorphism in condensed analytic spaces. This means that maps can agree in at most one way, so we have a set, not a higher structure.

On the right-hand side, the space of functors is a set because it is $D(R)$-linear, so it is just the space of cocontinuous functors from sheaves on $K$ to $D(R)$, generated by the representable ones. This is determined by where the functor sends the free things on these representables, and then passing to the complementary closed subsets, it is determined by where it sends the constant sheaves on those closed subsets, which should go to idempotent algebras in the target $D(R)$, subject to some natural conditions.


Construction. There's a, if you have a union of closed subsets, you can kind of use a Mayer-Vietoris to express the constant sheaf on the union in terms of the constant sheaf on the two pieces and the intersection. And that gives kind of an algebraic formula for what the value on the union should be, which you can write down just at the level of idempotent algebras. And you ask that that union goes to that algebraic construction you have here.

And then, and then you still want this condition here, but that's just another condition.

So this looks like a sheaf theory. This looks like the direct category sheaf on K, yeah. Yes, this uses the finite dimensionality, yes, that does. 

And then the equivalence with sending various pieces---the equivalence of giving something on the closed subspaces, is it does it use? This doesn't use, no. So when you work with shifts or so far, I haven't used finite dimensionality. If you take this in the sense of Leray, then everything I say works generally, so for an arbitrary topological space K, but certainly for a compact Hausdorff space.

Okay, right. And so the point is that idempotent algebras form a, this is actually just a poset. Um, so there's a priori, it's an infinity category, but one checks that it's just a poset. So a map, if it exists, is unique, you can uniquely write it down. So then we just have, the right hand side is just you have, you just have to give a map of posets to specify all of this---a map of posets satisfying some simple conditions to specify such a symmetric monoidal blah blah blah blah blah functor and so on.

The next thing to note is that, okay, we should now---now I'm claiming these two sets are in bijection, and I should first write a map. And the map, the map in this direction is obviously going to send F to $F^*$, where $F^*$ is---um, um. So well a priori, D(K), well D(K) is this thing, but we can restrict to the full subcategory where you require the values to be discrete, or I could say $D^{con}(Z)$-linear functors from sheaves on K with values in $D^{con}(Z)$. And then I, so but I should, that gives one of these things, but I should maybe explain why the uh, that's a derived category of condensed abelian groups, light, sorry, light, yeah. Okay, so you have a uh, uh, so F goes to, uh, spec, ah, by definition, spec out. Okay, is H a map from the first? You have to give a map on the ring, but this is part of this functor, the map on rings is is included in KN, the no, so what is it? Your F from Spec R to K means that you have some shriek hypercover of this, uh, and a map of that shriek hypercover to the hypercover of this by profinite sets. Oh, because K is not, okay, yeah, and uh, okay. 

And if is totally disconnected, then it is the same as a map, if K is totally disconnected, yeah, so maybe, yeah, if K is profinite, then the left hand side is the same as a map of rings from continuous functions on K with values in Z to just the underlying ring of this, an analytic ring that's by construction. Um, that those are the same. 

Okay, and this actually explains why this condition is satisfied, I mean for arbitrary K, because for arbitrary K you can surject from a profinite set T, and then this is a surjection, so by definition there will be some sheaf cover Spec R where you factor through a map to T, but once you factor through a map to T, then your $F^*$ is just being induced by this functor, but this is just a filtered limit of, uh, of, uh, yeah, filtered colimit of C(like K,Z)'s where this is finite, and um, well what do I want to say, the, um, so the, the, so well what what I want to say maybe, yeah, I don't know if that's the right remark to make. So in the case when K is profinite, then, so you want to prove that the image of every conn



Profinite. So, in this case, the connectivity is automatic. And if you want to go backwards, given such a symmetric monoidal functor or such an association with idempotent algebras, it's enough to go backwards, assuming this condition, because you can work shriek-locally, since both sides are shriek-sheaves or satisfy shriek-descent. 

Conversely, if you're given such a symmetric monoidal functor, on the connective level, what you can do is take a hypercover by profinite sets. Each of these will automatically be profinite sets because they'll be closed subsets of the product of two profinite sets. Then you get a corresponding diagram of commutative algebras in the derived category of $\mathbb{R}$-algebras. You just take $\pi_0 \mathcal{R}\Gamma$ of the constant sheaf on $\pi_1$, $\pi_2$, and so on. These things are all going to be connective, and in fact, you have a formula that gives that this is the Čech nerve of just the first map.

Now, you can apply your symmetric monoidal functor $F$, so you can take $F$ of this $\pi_0 \mathcal{R}\Gamma \mathbb{Z}$. This is a commutative algebra object, concentrated in degree $\mathbb{Z}$. And then you can say it's actually an animated commutative algebra. Then you take the analytic ring defined by the $F$ of $\pi_0 \mathcal{R}\Gamma \mathbb{Z}$ modules, and just the induced analytic ring structure. You can check that this procedure induces a map from the Čech nerve of this $\mathbb{R}'$ mapping to $\mathbb{R}$, such that the underlying $\mathbb{Z}$-algebra maps to the constant sheaf $\mathbb{Z}$, and the complement is the cofiber.

Okay, so where are we? Ah, yes, so then what we get is a map from the Čech nerve of Spec $R'$ mapping to Spec $R$, to the Čech nerve of $t_0$ mapping to Decay. And then we'll have produced a map by descent. We have produced a map from Spec $R$ to $K$. 

So as long as we know that this is a cover, okay? But note that this is a proper map by construction. We built it as something where the, so to speak, the completeness condition isn't changing when you go from $R$ to $R'$. You're just extending the ring. So by the criterion we had for hypersheaves, it's enough to see that the unit object, which is our triangle, lies in the category generated by the image of the forgetful functor $D(R') \to D(R)$. And for that, by applying $F$, it's enough to see, using the lower shriek, which is the same as the lower star, so it's really just a forgetful functor, that the image, this means the image closing by cones and retracts and finitary operations, yeah.

Okay, then applying $F$, it's enough to see that the constant sheaf on $K$ is in the subcategory generated by the lower star functor, so sheaves of $t$ on $T$ values in $\mathbf{D}(Z)$. Okay, or in Ila's language, we need to see that the lower star of the structure sheaf on $t_0$ is descendible.

Now, remember that we could choose this arbitrarily, so it's actually enough for me to produce a cover by a profinite set for which I can check this descendability. We can reduce, by pullbacks, to $K = \{0,1\}^n$, and then I'll discuss just the case $n=1$ because it's simpler to write down.

You can also reduce to this case, yes, that's true. This means when you take the fiber, there's some... We didn't get into much details about how to make these sorts of arguments, but Ila gave some toolkits which are nice. So then in this case, you take the usual kind of Cantor set. So I'm going to produce my cover of the closed unit interval by the Cantor set. What I can do is take the two halves of the closed interval, and their disjoint union is a space mapping to the unit interval. And then we can do the same thing again, the base to the power of two, and so on. 

Then we have a sequence of spaces mapping to the unit interval, where in the inverse limit, you actually just get the Cantor set, which is then forming a cover of the closed unit interval. So what does this mean on the level of these pushforwards? These pushforwards from the Cantor set will then be the sequential colimit of the pushforwards from each of these.

Now, there's a general fact about this descendability: if you have a sequential colimit, it's enough to establish descendability with uniform exponent of nilpotence for each of the finite ones. And for each of the finite ones, you have descendability basically because you have a Mayer-Vietoris cover for the closed subsets, and the reason you get a uniform bound on the exponent is because in each case, there's only double intersections that you need to be concerned about and no triple intersections. So in the end, you get something like exponent of $n+2$ for each of these individual things, and then 3 or 4 for the colimit or something like this.

Okay, now let's come back to the question about $E_\infty$ versus animated commutative...
About $E_\infty$ versus animated commutative algebra. My apologies, let's actually move on. I want to make a remark at the end of this.

Maybe it's good to say that at least in this example, we can do the $H_e$ structure by hand, because you only have to do it for the sequential part. Yes, that was the argument I was going to suggest, and then I was thinking in my head about why for idempotent algebras it's automatic, and it wasn't quite clear to me. But I think you can do it by hand if you know the categorical structure well. We should think more carefully about exactly how to do this and report back next time.

But let's move on. I want to make a small remark before we move on.

Remark: By similar arguments, or basically the same arguments, if you have a functor like this, it's equivalent, a priori, to saying that you have this connectivity estimate on Spec $R$ locally. But it's equivalent to ask that over a proper cover, a cover by proper maps, you get the connectivity. So you don't actually have to, when you're working over a ring $R$, give you a ring structure on $R$ to get this connectivity.

Could you say that one more time, Peter? 

Yes, then you can use this to define a new notion of complete modules, which are modules over the... Yes, I see. Indeed, yeah, yeah. But are all of the intermediate guys connective as well? Yeah, I guess they are. Well, okay, yeah, so maybe connectivity isn't even important there anyway.

Alright, oh yes, and another remark is that this will actually be a little bit useful now. I've not... Sorry, oh, we're over here in the setting of number 2 in the theorem here. Okay, so let's say you want to produce a map like this by this procedure. You want to produce a symmetric monoidal functor, and then you have to check this annoying condition that on some étale cover of Spec $R$, you have this connectivity condition. I'm saying, or, let's... No, I'm making a different claim. Sorry, that... Let's say you have a map like this, then you know that étale locally, you get this. But in fact, what we see in the proof is that after a proper map, even after just a proper map from a proper cover of Spec $R$, you can ensure this condition. This comes from after proving the equivalence, yeah, yeah, yeah, not a priori, exactly.

Another remark is that if you have $Z$ closed inside $K$, or this is actually more general, but, and then $U$ is the complementary open, then they are also each other's complements in analytic stacks. So these two do determine each other, the associated analytic stacks do determine each other in the naive way. I.e., if you're given $Z$ and you want to know $U$ as an analytic stack, its functor of points is you just map to $K$ such that when you pull back to $Z$, you get the empty analytic stack. That's the same thing as mapping to $U$, and vice versa, so you can check on profinite sets where it's quite elementary.

Okay, so now let's get to... Sorry, why did I do that? How else did you want me to write them? Only what it means doesn't mean anything. It's just because the... Well, I wanted to write, for example, I wanted to write this one, this one right above this one, because I'm saying the these two end points map to the same point down here, so I wanted to stack them vertically like that. Maybe that's the reason, does that make sense?

Yeah, yeah.

Okay, so now we get to a topic that I think is fun, which Peter introduced last time: this kind of norms on analytic rings. So let's say that $R$ is an analytic ring. Definition: a norm on $R$ is a map of analytic stacks from the algebraic $\mathbb{P}^1$ over $R$, which is something you can build over any analytic ring by just base change from the trivial case with algebraic geometry, to the closed interval from $0$ to $\infty$, which is a condensed set and thereby an analytic stack. Let's call this map $n$. And then I'm going to give some conditions, which are going to be different from the ones that Peter gave last lecture. So a priori,

Right. The first condition is that, when you restrict the norm function on $\mathbb{P}^1_R$ to $\mathbb{A}^1_R$, you get a norm function on $\mathbb{A}^1_R$. But then what does it mean when you have an element of $R$? Do you get a real number or an element in $[0, \infty]$? No, you do not get an element in $[0, \infty]$; you get a map.

Note that if you are given an $f$ in $R$, that induces a section from $\mathbb{P}^1_R$ to $\mathrm{Spec}(R)$. Then you can compose that with the norm map to $[0, \infty]$, and what you get is a map from $\mathrm{Spec}(R)$ to $[0, \infty]$. Exactly.

Now, suppose you start from an obstruction. We said there are several ways to view it as an analytic object, and for each of those, you have a notion of $G$. On the other hand, you have a geometry, like in Berkovich or like this. Can you say what the relations are? We will discuss these things later, but I want to get the basic definitions and results in place first.

Yes, so a norm for an element of the ring, you don't get a real number; you get a map from $\mathrm{Spec}(R)$ to the non-negative real numbers plus infinity. You can think of this as consisting of a family of residue fields, and for each residue field, you get a real number, but they could be varying with the residue fields. Relatedly, if you have a norm on $R$ and you have a map from $R$ to $R_P$, you get a norm on $R_P$ just by composition. So it's really a geometric thing; it's something you can pull back, and it still persists. That's important to realize.

Okay, right, so that's a prelude, and then the first condition is that the norm of zero, which is a map from $\mathrm{Spec}(R)$ to $[0, \infty]$, should factor through the terminal map to the terminal analytic stack, which is also the analytic stack associated to the condensed abelian group which is the singleton point, which is a subset of here. So this is a condition that the norm of zero is the constant function zero. By the way, in this geometric perspective, there's sometimes a question of whether the norm of one should be equal to one or zero for the zero ring, but this is avoided here because when you have the zero ring, the norm is both one and zero, because then this is the empty set, and the map factors both through zero and through one. And so in this geometrical perspective on norms, there's no way to get messed up.

Okay, wait, sorry, over here, there's no---we're not saying anything about what happens to $\mathbb{A}^1_R$, just let me finish.

The second condition is that the following diagram commutes: $\mathbb{P}^1_R$ maps to zero in the second and third conditions. I'm going to try to say the norm is multiplicative. The first thing I'm going to say is that if you have inversion, so here we have $\lambda$ goes to $\lambda^{-1}$, which exchanges zero and infinity, we also have let's call the coordinate in $\mathbb{P}^1$ $T$, and we also have $T$ goes to $T^{-1}$, exchanging zero and infinity, and I want this diagram to commute. By the way, the maps from anything to $[0, \infty]$ of a space like $k$ or $\mathbb{A}$ is just a set, yes, and even with semi-compact ones, this was said before.

Okay, so conditions one and two. Now note that one and two imply that if you take the infinity section, so, let's say, the norm of infinity, this factors through infinity. Okay, now before I---yeah, and now, yeah, okay, maybe three. Set, I don't---it's one, right? Ah, does this already imply that?

Infinity. Then we have that if you take A$^1_{\mathbb{R}}$ analytic cross A$^1_{\mathbb{R}}$ analytic, and then you have the Norm. Oops, oh I guess this is just $\mathbb{R}$ but okay, I'll continue to write it as $\infty$. Infinity. Then here we can take the product, so and we still get something in the region from zero to $\infty$ which is contained in $\mathbb{Z}_{\infty}$ closed.

On the other hand, this we can map to $\mathbb{P}^1_{\mathbb{R}}$ cross $\mathbb{P}^1$ via multiplication. TS goes to St, or TS, this is the multiplication $M$ on $\mathbb{P}^1$ is not defined in general. TS, the norm inverse. Why, I'm going to justify this afterwards, I mean, so yeah, so actually, well, yeah, that I'll justify why this map is well-defined at the end.

Right, so this map should commute, so that's saying that the norm is multiplicative, but as Peter is pointing out, one needs to justify that such a map indeed exists. Let me do that now. 

So the claim is that A$^1_{\mathbb{R}^n}$ is a subset, well, a submonomorphism admits a monomorphism to $\mathbb{P}^1$ by definition. Another thing that by definition admits a monomorphism to $\mathbb{P}^1$ is the algebraic affine line over $\mathbb{P}^1$, and the claim is that this one's contained in this one. A$^1_{\mathbb{R}}$ is just the polynomial ring in one variable over $\mathbb{R}$, without any special structure in the category. It's an affine analytic stack, and it's given by keeping the same class of complete modules you already had in $\mathbb{R}$ and just adding the polynomial variable as operators.

The point is that we want to produce a map from A$^1_{\mathbb{R}^n}$ to $\mathbb{P}^1$. We have to check it on the two charts of $\mathbb{P}^1$, one of which is already A$^1_{\mathbb{R}}$, and the other one is the other A$^1_{\mathbb{R}}$ to some inverse operator.

So here's what I'm going to say. We know that the Norm of $\infty$ is equal to $\infty$. This implies that the $\infty$ section of $\mathbb{P}^1$ is an algebra over the Norm upper star of the structure sheaf of $\infty$. One way to think about it is that, in the set of schemes where the scheme is endowed with the trivial analytic structure, there is a general statement about this.

Let me try doing it the way Peter was suggesting. Let's abstract a bit, let's move $\infty$ to zero and abstract a bit, do it for $\mathbb{P}^1_{\mathbb{Z}}$. It means you can this $\mathbb{P}^1_{\mathbb{Z}}$. Suppose given an analytic stack over A$^1$ such that if you pull back to the zero section, you get the empty set, so it misses the zero section. Then the claim is that this $\mathcal{F}$ factors through $\mathbb{G}_{\mathbb{m}}$ over $\mathbb{R}$.

The reason for this is you can think of maps like this in terms of symmetric monoidal functors. If $X$ is $D(\mathbb{A})$, then you can think in terms of the corresponding pullback functor from $D(A^1_{\mathbb{R}})$ to $D(\mathbb{A})$. And what do we know about this pullback functor? We know that it kills the structure sheaf of the origin. But then it's just a purely algebraic fact that if you kill the structure sheaf of the origin, then you factor through inverting the parameter. So the structure sheaf of the origin is just the structure sheaf on $\mathbb{A}^1$



"Yeah, so thanks Peter. I think that's a much nicer way of saying it." Everyone, okay. So, that's the claim, and that implies that this map is well-defined, because certainly the multiplication is well-defined on $\mathbb{A}^1$. But maybe then I could put the $\mathbb{A}^1$ here a priori, and a priori have only the $\mathbb{N}$ going here. But then a posteriori, if I require this diagram to commute, then it follows that this actually lands inside $\mathbb{A}^1_\text{analytic}$, because by definition that was the pre-image, because this map factors through $\infty$.

Okay, all right. I want to get to something fun. "Yes, please, no, no, please, please." The fact that we denoted $\mathcal{I}$ and the norms here should somehow correspond to norms in analytic ification. What? Yeah, so I'm going to give some of the motivation at the end, but let me finish with the axiomatics. 

I remember last time, I think Roch said that Gaga was really like, he noted down Gaga as an isomorphism of stacks. And somehow that was some sort of analytic ification. So does that also correspond to a norm then? This question, I suggest you keep for later.

Okay, yeah, let me finish with the axioms. 1, 2, 3. Ah, so four. Okay, so axiom 4. Now, maybe now's the time to say a bit about motivation. So, what if you have an analytic ring? What we're going to try to do is we're going to try to say if you have an analytic ring, you want to try to build some geometry over that ring. But it's hard if you're just given an analytic ring and you don't know anything more about it or you don't have any extra structure on it---it's kind of hard to build analytic geometry over it. I mean, basically all you can do is you can do this trick of importing algebraic geometry for an arbitrary analytic ring, that's more or less all you know how to do.

What we're going to be doing, and what---well, one measure that you have some good analytic geometry is that you have some nice subsets of the affine line. And nice in the context that we're discussing here means, for example, sheafable, so that the six functor formalism works, and then it kind of really feels like you're doing geometry in some more or less traditional sense. 

But again, on a general analytic ring, you don't know how to write down any interesting sheafable subsets of the affine line, so you have to give yourself some of them. And that's the point of this notion of normed analytic ring---we're giving ourselves basically discs of certain, of some arbitrary radius inside the affine line. And they will turn out to define sheafable subsets of the affine line, and then we can get started on doing geometry that resembles some sort of traditional analytic geometry based on open discs or closed discs or what have you. But you have to have this extra structure on your base before you can get started on that game. If you don't have a notion of a norm on your ring, you can't start talking about closed discs and open discs of certain radius. So that's what we're doing right now.

But we already have some sort of things that kind of seem to function as a---we've already seen certain versions of the unit disc. For example, we spent a lot of time discussing this

Measure concentrated at $1$. Okay, you use the multiplication on $\mathbb{N}$ or you use the addition on $\mathbb{N}$ to give the multiplication on $\mathbb{PR}$, yeah.

So, right, so in particular you get a factoring like this. This is generally true that you have a diagram like this, where this is the canonical map. In most examples, this $\mathbb{PR} \to \mathbb{R}$ mapping is a monomorphism, an injection. So in most examples, this lives in degree zero and this map is an injection. This is some kind of sequence space.

What is the condition that the sequence satisfies? Well, in the abelian category, which is the heart of this discussion, in most examples this lives in the heart, and this also in most examples lives in the heart. I mean, I guess maybe I should be using this notation, but I'm being a little bit sloppy here.

Yeah, and then this is just an injection, so to speak. So, in most examples, $\mathbb{PR}$ is like the set of sequences $r_0, r_1, \dots$ satisfying some summability condition, such that if you termwise multiply by a null sequence, you get a summable sequence. The notion of summability depends on the analytic ring structure.

But at the level of this discussion, you could imagine, for example, $\mathbb{R}$ being the real numbers and summable means usual absolute sum, the absolute values you get a finite number. That's not actually a special case, but it's close enough and it serves the purposes for this discussion.

And this is kind of just by the universal property of $\mathbb{PR}$ that this is the correct interpretation, because $\mathbb{PR}$ is maps out of $\mathbb{PR}$ to an $\mathbb{M}$ in the abelian category. These are supposed to correspond to null sequences in $\mathbb{M}$ by construction.

So, you know, it's the kind of thing which when paired with a null sequence, you get an element in $\mathbb{M}$, and so the idea is this procedure. Well, if you well, I'll let you maybe, maybe me trying to explain it is not as helpful as all that is any analytic ring $\mathbb{R}$, but this is not precise mathematics here.

Yeah, but in this norm business, $\mathbb{R}$ is just a discrete ring. In the norm business, $\mathbb{R}$ is an arbitrary analytic ring, $\mathbb{M}$ is then, $\mathbb{P}$ is a... Okay, so you say that usually it's, say, in an object in the category of $\mathbb{R}$, yeah, usually that's right. And this is the what you call $\mathbb{M}$, $\mathbb{M}_0$. Sure, well, I mean, yeah, I could, yeah, I mean, so I was saying that this is the interpretation you get in practice, where the notion of summability depends on the analytic ring structure, and the way you see that this is the correct interpretation is by thinking about what it means to map $\mathbb{PR}$ to $\mathbb{M}$.

And so, like, for example, the most basic thing was, it would be if you map $\mathbb{PR}$ to $\mathbb{R}^\triangle$, then that's the same thing as giving a null sequence. But then, if you think in terms of what happens when you restrict to here, you have some coefficients in a polynomial. If it terminates, if you have zeros after a while, then what you're doing is you're just summing the null sequence times those things to get an element in $\mathbb{R}^\triangle$, and then you imagine that that summing should make sense for something which is not necessarily eventually zero, and so this is kind of the interpretation you should give that.

Right, and what, so, and so if, for example, $\mathbb{R}$ is $\mathbb{C}$ with the Gauß or Liouville analytic ring structures, what you see is that, if you look at holomorphic functions on the usual ring of holomorphic functions on the closed unit disc, meaning they converge on the closed unit disc, then every one of those satisfies this summ

Okay, what's our goal with all of this analytic geometry? For me personally, one goal since 10 years or so has been---there's all this fancy geometry that has been developed $p$-adically, like perfectoid spaces, prismatic cohomology, and $p$-adic shtukas, and the geometrization of local Langlands. This all works quite beautifully over the $p$-adic numbers.


So, condition four is split into two parts. The first part is that if you have a map from $D$ or $D_r$ to $\mathbb{P}^1$ mapping to $0$ and $\infty$ via the norm, you want this to factor through the map up to $[0,1]$ that corresponds to this, saying that this notion of the unit disc is sandwiched between closed and open. 


This is a stronger condition than just saying you have a map like this, and the nice thing about this stronger condition is that, first of all, you can check it in practice, and second of all, it's just a condition, whereas giving a map like this is a priori structure.

We can think of this norm as non-archimedean or archimedean at will, as we haven't enforced any compatibility of the norm with addition. Last time, this was a cosmetic change from Peter's lecture - he did not mention this axiom, and we're not sure whether it's a consequence of the other axioms. If the base solidifies to zero, like over the real numbers, then this condition can be proved from the other axioms. But for other bases, it's an open question.

So, what is a norm again? If you have a norm on an analytic ring, this gives you, for example, the constant sheaf $\mathbb{Z}_{r}$ on $\mathbb{P}^1_r$, which should be interpreted as the algebra of overconvergent functions on the closed unit disc of radius $r$. The overconvergence is built in, as this is the filtered colimit of the constant sheaves $\mathbb{Z}_{r'}$ for all $r' > r$, under the restriction maps.



Suppose we have a disc of radius $R$. Some version of this disc of radius $R$ would be forced to be the over-convergent one, because of this property. And because it's a pullback functor, so it commutes with colimits.

Okay, so the data of these guys, given the axioms, this determines $n$ because it's very easy to classify the closed subsets of the interval. You only need to know about things less than or equal to something and things bigger than or equal to something, but the axiom about inversion gives you one in terms of the other. So you just have to give these things, subject to some simple conditions, in order to specify a norm.

The next topic is classifying norms. I don't mean like as in classifying spaces or whatever, I mean how to classify norms on a given analytic ring. So, let me start with one lemma.

Given a norm on an analytic ring, you can just check that the usual algebra, geometry, or theory of the complex numbers does satisfy these conditions. So this way you can produce, by hand, such norms over $\mathbb{Q}_p$.

What does "over-convergent" mean? It means that an over-convergent function on a disc of radius $R$ is a function which converges on some disc centered at the same point with larger radius. So it extends to a function on an open neighborhood of the closed disc.

Given a norm, you can look at the locus where the norm is strictly between zero and one. This projects down to $\mathrm{Spec}\ R$, and I'll say that this is a cover in the graded topology we've been considering. Essentially, if we're willing to work locally, we can assume we have an element $Q$ in the underlying ring such that the norm of $Q$ lies strictly between zero and one.

In fact, the stronger claim is that if you take the preimage of 1/2 under the norm, this is a cover in the graded topology. This is not necessarily an affine cover, but it can be refined to one. For simplicity, let's assume that this is affine, so there's an argument for getting the conclusion anyway based on resolving this by a profinite set and using the descent ability that was proved earlier.

Assume that this norm inverse of 1/2 is connective, so it's affine and proper over $\mathbb{A}^1$. By the descent ability criterion, our triangle lies in the algebra generated by this. But I claim our triangle is actually a retract, directly a retract without any cones. The intuition behind this should be clear - this is some version of a Laurent series ring, and you can just pick out the zeroth coefficient of your Laurant series to get a linear map which splits the unit. But we have to make sure it holds in this completely abstract setting.

Setting here, so what you do is you take this structure sheaf of $\mathbb{P}^1$. Then you can take the norm upper star of $\mathbb{Z}$. Okay, so this is a pullback in $\mathcal{D}$ of $\mathbb{P}^1$, just because the constant sheaf on this interval is the pullback of the constant sheaf on $[0,1]$, constant sheaf on $[1/2,1]$ glued along the constant sheaf at the point.

Then we apply push-forward to $\mathcal{D}$ of $\mathbb{R}$. The cohomology of the structure sheaf on $\mathbb{P}^1$ is just $\mathbb{R}$ concentrated in degree $\mathbb{Z}$. So what we get is an $\mathbb{R}$-triangle. I'm going to use the same notation, or just push it forward to the other $\mathbb{A}^1$.

There's a pullback in $\mathcal{D}_{\mathbb{R}}$, but it's also a pushout in $\mathcal{D}_{\mathbb{R}}$. If I want to make a map from here to a certain place, for example to our triangle, I make a map here and a map here and make sure they agree there. So what I do is here I take evaluation at 0 from our triangle, which goes to $\mathbb{R}$ triangle, and here I take evaluation at infinity, which also goes to our triangle. They clearly both give the identity map when you restrict to $\mathbb{R}$ triangle, so this gives the map here which when you restrict back to here is the identity.

That produces the desired retraction. As for the connectivity, there's a proper cover by something affine, and not just proper but also descendible. The unit here is generated by the image of this, and it follows that the unit here is generated by the image here, which provides the desired refinement by an étale cover of affines.

Now let's give ourselves this extra data which we've assured can be gotten locally. Given a norm $\mathbb{P}^1_{\mathbb{R}}$, norm 0 to infinity, and a $q$ in $\mathbb{R}$ such that the claim is that $q$ is given by a map from $\mathbb{Z}[t]$ to our triangle, or to $\mathbb{R}$. This factors uniquely through this gaseous base ring. The proof is that spec $\mathbb{Z}[q]/(q^2 - 1)$ maps to the affine line over $\mathbb{Z}$, and this is a monomorphism. There's a contractible space of factorisations if one exists.

Exists. This is not obvious from the definition of the Gauss base ring, because by definition of the Gauss base ring, we took $\widehat{\mathbb{Z}}_q$ which was just $\mathbb{P}$, and then we inverted $q$ and then we enforced this $q$ being Gauss. And this is not idempotent, so not idempotent. So enforcing the condition of being Gauss, what defines a monomorphism on the level of analytic stacks, because it's just some quotient of categories. But because this is not idempotent, it's really not clear from that description that the composite map all the way to $\mathbb{A}^1$ should be idempotent, because it seems at first you have to choose something which is extra structure, namely a proof that $q$ is topologically nilpotent, so to speak. But you can do it in the other way.

Instead, you can also think of this as $\mathbb{P}$-modules or $\mathbb{P}$-modules. You just take a polynomial ring in one generator, and make that generator Gauss. Oh, and then maybe I have to invert $q$ too. Okay, just so. And then the claim is that $\mathbb{P}$ is idempotent in $\mathcal{D}(\mathbb{Z}_q[\frac{1}{q}])^{\mathrm{Gauss}}$. And this actually, so we want to check that $\mathbb{P}^1 \cong \mathbb{P}$, that's actually a claim that happens after base change to $\mathbb{P}$, so to speak. So it actually reduces to a calculation here, that if you take this $\mathbb{P}$ over $\widehat{\mathbb{Z}}_q[\frac{1}{q}]^{\mathrm{Gauss}}$, now we have both a $q$ variable and then we have a $t$ variable say, coming from the $\mathbb{P}$ here. So that if you take this and you mod out by $t-q$, you just get the underlying ring.

Yes, you should still write Gauss, because completion changes the... Oh, okay, sure, yeah, the underlying ring of the analytic ring structure. And Peter described this free module, and if you use that description, you will find yourself, you're just, you have to check a short exact sequence and you can do it. So that's the uniqueness.

As for the existence, well, let me just say it in words. It's fairly elementary from the definition of the norm. So because our norm is contained in here, this means that we're away from the inverse of 1 at infinity. And that in particular means that we're away from the $\mathcal{D}$ sitting at infinity. If you take this $\mathcal{D}$, the thing that was $\mathrm{Spec}(\mathbb{P})$, and you translate it to infinity on $\mathbb{P}^1$, then that's going to be contained in this locus. But then saying that you're away from the $\mathcal{D}$ sitting at infinity is exactly the same thing as saying that $q$ is Gauss. But then on the other hand, the norm of $q$ is contained in the closed interval $[0,1]$, because of the axiom I'm referring now to the axiom 4 about the placement of $\mathcal{D}$ in this norm. This implies that $q$ comes from $\mathcal{D}$, so $q$ is topologically nilpotent. It comes from $\mathcal{D}$, so $\mathcal{D}$ was this $\mathrm{Spec}(\mathbb{P})$ mapping to the affine line, it therefore maps to $\mathbb{P}^1$, and then you can pull back that map under the automorphism of $\mathbb{P}^1$ which is the inversion map, and you get something abstractly isomorphic to $\mathcal{D}$, but the structure map to $\mathbb{P}^1$ is different.

Okay, I'm almost

The open interval (0, 1). Yes, it's not---oh, thank you. And this is by recording $q$ and the norm of $q$. This map is an isomorphism, i.e., giving a norm on an affinoid ring plus an element of norm strictly between zero and one is the same thing as specifying what the norm of that element should be, which is this function here. And requiring that that $q$ actually come from this $\mathcal{G}as_{K}$-theory that we discussed earlier.

Both sides are "sets", no---almost, well, except for the annoying fact that $\mathbb{A}^{1}$ is not a set. So the norms on $q$: the first thing we said is that the norms really is a set, because there is no---and the $q$ of course is something in some set, no, no, the $q$ is not in a set because this could be derived. So both of these map to $\mathbb{A}^{1}$, and like the fibers of these maps are sets, so they're as much sets as each other. But they're not each individually sets. Ah, so $q$ is a derived---ah, okay. But you still think of $q$ as choosing something in the zero part of the simplicial---yeah, yeah, yeah, yeah, implicitly the higher simplices are there also.

So let me give the proof. After a \'{e}tale cover, we can assume $q$ admits all $n$th roots, compatible $n$th roots for all $n$. This is because the \'{e}tale cover is countable in the fpqc topology. So we're free to work locally. We're free to assume that $q$ has all $n$th roots. Then the claim is that we have a morphism $\mathbb{D}_{q} \to \mathbb{A}^{1}$ given by multiplication by $q$. And then we have $\mathbb{D}_{q}$ here. This makes sense for any $q$ in $\mathbb{G}_{m}$. Note that if $\alpha = q \cdot \beta$ or $q$ is topologically nilpotent, then we get a map from $\mathbb{D}_{\alpha}$ to $\mathbb{D}_{\beta}$ induced by multiplication by $q$. Here we use that $\mathcal{P}$ is a Hopf algebra encoding multiplication.

Right, I may have gotten the ordering wrong. This is some disc of radius $q^{-1}$ or some version of a disc of radius $q^{-1}$. It's not the correct one because it's not a monomorphism and it blows up along the boundary, but then we can fix that. So $\mathbb{D}_{q}$ or $\mathbb{D}_{\alpha}$ gives a disc of radius $\alpha^{-1}$, and the maps above give the inclusions between such discs.

Now we apply this with $\alpha = q^{m/n}$ in the rational numbers, and we get discs of radius $\|q\|^{m}$. We can freely assume the norm of $q$ is constant equal to $1/2$ by rescaling the norm. Then you use these to make the over-convergent versions for an arbitrary real number---you can look at all the rational numbers bigger than it, you have these kind of "fake" discs of that rational radius, and then in the colimit all these problems.


About blowing up along the boundary, it becomes the thing that has to be specified if you're given a norm on $\mathbb{R}$. When you make it over convergent, it will be forced to be equal to the thing that comes from a norm if you have a norm. So, that's more or less an argument why this is a monomorphism. And then, if you want to check that it's a bijection, you just have to produce a norm on here and you do it by following this procedure. You can even adjoin all the roots that you want.

So, you take this "fake disc," you expand it by multiplication by powers of $Q$, and then you build the inclusion maps between those discs. You make them over convergent, and then you check that you have all the axioms. Sorry for going over time, thanks for paying attention.

In this inclusion, we also add an analog and discuss Huber rings of power-bounded elements. If $Q$ is power-bounded, it's not entirely clear in this general setting what the class of $Q$ is for which you get such a map. For example, this argument here doesn't prove that you have an identity map when $Q = 1$, but obviously you can take $Q = 1$.

For roots of unity, we know that the norm is 1. So, if you have a $Q^{1/N}$, its norm has to be the $N$th root of the norm of $Q$. This implies, in particular, that roots of unity go to 1.

Okay, see you on Friday. As for the intuition for not being able to do geometry unless you choose a norm, I don't want to make such a strong claim. But if you want to define a notion of geometry that resembles usual complex analytic geometry, which is based on open and closed discs, then you need something that measures the size of a radius of a disc, and that's what the norm does. You could choose different norms, but then you'd get something basically equivalent to the usual complex analytic geometry.

So what's our goal with all of this analytic geometry? For me personally, one goal since 10 years or so has been---there's all this fancy geometry that has been developed $p$-adically, like perfectoid spaces, prismatic cohomology, and $p$-adic shtukas, and the geometrization of local Langlands. This all works quite beautifully over the $p$-adic numbers.


So actually, one thing we're going to do is we're going to mod out by---at some point, we're going to mod out by these sort of exponential rescalings of the norms. When you do that, you can't talk about a disc of a fixed radius, but many other things still work okay. 

Yes, we discussed this many times. So if you have an affinoid universe, you choose, let's say, the absolute value on $\mathbb{Z}$ and 1, then you get a norm on this. And in particular, in each residue field, you get a non-archimedean norm. Yes, and this is like choosing this collection of norms.

Yes, so it is not at all the same as, so, say, for another field, it is related to the previous national norm, but for a more global object, it is very far from, like, the Benvéniste space on just one, it's quite a different way to think of norms. This is the definition, yes. And so you, and presumably what they said is an equivalence to take Huber rings, the national norms, and your sense is the same as a way to, of course you can, should it be continuous? 

Exactly, continuous on the vector of rescale a given one, so it is a torso under scaling of the fixed one, that's precisely correct. And it sort of, uh, sort of follows from this, uh, okay, this. Okay, so in this example, this is what you get. And of course, then you can, uh, and of course for other analytic Huber, just locally, you can do the, I mean, you can, and for non-analytic Huber, this is something different---well, it's kind of, it's ruled out actually. I mean, you have to be Tate, you have to be analytic. I mean, we were saying Tate instead of analytic in this class, but you have to be Tate by this result if you have a norm. I mean, then locally for discreet guys, you are not having any non-trivial, any of exactly. That's right.

But what we're going to see is that if you mod up by rescalings, then there's a way to---there's a way to extend this. So when you mod up by rescalings on the norms, they form an analytic stack, by the way. So it's a it fits the definition of an analytic stack, sending R or Spec R to the set of norms on R, that's an analytic stack. And you have to put covered by S, yeah, I just did. Okay, okay, yes.

And but, and it is such that, like, discreet Huber things can't, can't map to that stack, but there's an enlargement of the stack which also accommodates solid $\mathbb{Z}$ and anything living over solid $\mathbb{Z}$ and so on, which we'll probably discuss. So there was this thing, like in the theory of diamonds, that you want to get from an analytic---like in the $p$-adic setting, you get a diamond from analytic, but if you have a non-analytic, still you can look at maps of antic, and then get that kind of, there was this non-quasi, I forgot now how, maybe it's not related to this, but somehow looks, I'm not sure what you're referring to actually, because there was a diamond only allowed Tate-Tate objects, so even if you start with a discrete one, I still only remember how Tate things than, okay

\end{unfinished}