% !TeX root = ../AnalyticStacks.tex

\section{\ufs The solid affine line (Clausen)}

\url{https://www.youtube.com/watch?v=fUjn2rGw9SA&list=PLx5f8IelFRgGmu6gmL-Kf_Rl_6Mm7juZO}
\renewcommand{\yt}[2]{\href{https://www.youtube.com/watch?v=fUjn2rGw9SA&list=PLx5f8IelFRgGmu6gmL-Kf_Rl_6Mm7juZO&t=#1}{#2}}
\vspace{1em}

\begin{unfinished}{0:00}

Today we're going to be talking about a little bit of geometry, maybe the solid affine line. So let me start with a recap of what we've seen so far.

We had this category of solid abelian groups, which was a full subcategory of condensed abelian groups. This was some kind of analog of complete non-archimedean topological abelian groups, but it's kind of, from a formal perspective, easier to work with. It's an abelian category. In fact, it's an abelian category closed under limits, colimits, extensions, and many other things besides. Also, we had this left adjoint called solidification. There's a symmetric monoidal structure here, which you think of as a completed tensor product, such that this completion functor is symmetric monoidal.

The definition was in terms of this object $P$, which was kind of the free module on a null sequence. So the definition was that $M$ is solid if this map from null sequences to null sequences given by identity minus shift is an isomorphism. Okay, so that's what we had last time.

Now I want to start doing a little bit of geometry. We're going to be modest and look at the affine line, which is actually the most important case. So about this $P$, it turns out that $P$, even before solidification, is a ring. There are two shifts one can look at, because you can shift one to the left and one to the right. I want the one that's injective.

$P$ is a ring, and in fact, there's a ring map from the polynomial ring in one variable to $P$, and the shift is multiplication by $T$. So that's what you have before solidification. But after solidification, something extremely nice happens.

When we solidify, we already explained what the free module on $P$ is. There's a compact projective generator here, which was a countable product of copies of $\Z$. That's kind of the basic object from which everything is built. You can get that object just by solidifying this basic sequence space $P$. If you take into account the ring structure, what happens when you solidify is you just get the power series ring in one variable. So in the solid context, the universal carrier of a null sequence is just this power series ring.

But moreover, there's a very nice property of this situation. Lemma: If you do the tensoring in the solid world of this power series ring with itself over the polynomial ring, you just get the power series ring again. Maybe the multiplication map is an isomorphism, you could see, even the derived tensor product.

Well, this is quite elementary, because we know how to do tensor products in $\mathrm{Solid}_\Z$. Peter gave lots of examples of calculations in this category, and the tensor product of two of these basic guys is just another one of them. 

Proof: If you do $\Z[[T_1]] \otimes_{\mathrm{Solid}_\Z} \Z[[T_2]]$, that's just $\Z[[T_1, T_2]]$ by the basic calculation of the solid tensor product. Then if you want to get to this, you just mod out by identifying the two different variables $T_1 - T_2$, and the power series ring modulo $T_1 - T_2$ is the power series ring in one variable.

There's maybe another perspective Peter also mentioned.
So, fact: if you take the solid tensor product of two derived complete things, then the result is still derived complete. Well, maybe you have to worry a little bit about this, but morally, if you want to check this is an isomorphism, both sides are derived $T$-complete. You can check it modulo $T$, and modulo $T$ it's really a triviality.

Okay, the derived solid Tor product---is it defined as the derived $T$-product, then you solidify? Or how is it defined, exactly?

Yeah, you can either say it's the left derived functor of the solid tensor product, or you can say that you take the derived tensor product and then derived solidify it. It's all the same.

And do you need to know there are somehow enough flat things to construct the---you don't need to know that. But it is actually true that all of these guys are flat. I don't know, did Peter mention that? I forget.

Um, okay. Yeah, so a priori, you need to resolve both variables, but since you have flat objects, you only need to resolve one.

Okay, so what is the interpretation of this? This corresponds to the affine line, you could say. And then, since this satisfies this idempotent property---sort of like when you invert an element in a ring, you get the same thing---you can think of this as corresponding to some subspace of the affine line. A subspace, as opposed to just a random space with a map to the affine line. And that's the interpretation of this idempotency.

So then you could ask: how should one think of this subspace? What subspace is this? The naive thing to say would be that it's the formal completion of the affine line at the origin. Well, because it's a power series ring. But that naive interpretation is not the correct one.

So $Z[[T]]$ does not correspond to a formal neighborhood of zero in $\mathbb{A}^1$. And the reason is that that interpretation is not stable under base change.

So look at base change, by which I mean this is all implicitly over $\Z$, but we could tensor it to any ring, or any solid ring, and see what pops up. Let me do a little calculation.

Let's base change to a non-archimedean field, say---I'll just take the simplest example, $\Q_p$, and we'll see what pops up. So we take the affine line over $\Q_p$. Oh, sorry, well, maybe I'll just say: we take $\Q_p$ and then we do a solid tensor product with this thing here, and we want to compare this to $\Q_p\langle T\rangle$.

So $\Q_p$ is $\Z_p$ with $p$ inverted, and the solid tensor product commutes with colimits in both variables. So this is the same thing as $\Z_p$ solid tensor product $\Z[[T]]$, and then you invert $p$.

But this $\Z_p$ has a resolution by two copies of---so $\Z\langle U\rangle/(U-p)$, it has a resolution by two copies of this power series ring. So we know how to do these tensor products, and it just does the naive thing of pulling in the limits.

So the result of that is that you get $\Z_p[[T]]$, and then invert $p$ on the outside, which is not the same thing as the formal completion at the origin, namely $\Q_p[[T]]$. It's contained in there, and it's the subset where the coefficients are bounded in $p$-adic norm.

So what is the interpretation then of this ring? This ring is the ring of functions on the open unit disc in $\mathbb{A}^1$ over $\Q_p$.

So if you, maybe after an arbitrary non-archimedean field extension of $\Q_p$, you could ask: when you plug in a value in that field, when will such a series converge? And the answer is, it will converge if and only if the number you plug in has absolute value strictly less than one. 

It's the ring of bounded holomorphic functions on the open unit
Unit disc is you can imagine putting the open unit disc at infinity instead. So then you're looking at the complement of the closed unit disc. The complement of a thing is just as good as describing the thing.

Let's take this guy and put it at infinity instead. The complement---well, this is all just at the level of loose thinking. We can take this power series ring in one variable and tensor it over... but let me call the variable $T^{-1}$ instead, because I want to be thinking of putting it at infinity. Then I tensor over $\Z[T^{-1}]$ with $\Z((T^{-1}))$, so then I'm puncturing at infinity, so that I actually live in the affine line.

So I have a homomorphism from $\Z[T]$... And what is this ring? I'm just taking this power series ring and inverting the variable there. Another way of describing this is as $\Z[[T^{-1}]]$. That's the open unit disc centered at infinity.

If we want to understand the closed unit disc, we should in some sense be kind of localizing away from this complement. Another way of saying that is that we should be killing this object. To get... we need to kill this thing here.

So... I'm just going to make a preliminary note. This $\Z[[T^{-1}]]$ is a module over $\Z[T]$, as I just exhibited, and it also has a very simple resolution.

Note that $\Z((T^{-1}))$---let's disambiguate and call the variable $U$---we can think of that as $\Z[[U]]$ and then invert $U$. One way to invert $U$ is to adjoin the thing that you want to be the inverse, and then enforce the equation that says that they're inverse to each other. When you look at it like this...

I'll rewrite that. We get $\Z((T^{-1}))$ has a two-term resolution, where you have $\Z[[U]][T]$ and $\Z[[U]][T]$. Here you have $U-1$, so it's a two-term resolution. What are these objects? These are just the base changes of our fundamental $p$-... not to $\operatorname{Solid} \Z$ but to $\operatorname{Solid} \Z \{T\}$, and then adjoin a variable $T$. So these are resolutions by the compact projective generator of the category of $\Z\{T\}$-modules in solid abelian groups.

So killing this thing should be the same thing as requesting this map to become an isomorphism. Now this suggests the following, based on an analogy with the definition of $\Solid$ up there.

Definition: Let's say we have one of these guys. Let's say that $M$ is $\Z[T]$-solid if and only if when you take $\underline{\operatorname{Hom}}(P, M)$ and then $\underline{\operatorname{Hom}}(P, M)$, and then you take the map which is given by---so now $U$ corresponds to the shift, and $T$ is the extra thing we have acting on $M$, because $M$ is a $\Z[T]$-module, so we have $(\operatorname{shift} \times T - 1)$---we want this map to be an isomorphism.

Then this note up here is saying that that's the same thing as requesting that if now I should maybe pass to $\mathbf{D}(A_\infty)$... it's the same thing as requesting that there are no $R\operatorname{Hom}$s from this object we're trying to kill into $M$.

Okay, so now the theorem is basically that this definition, the name is well chosen. So this $\Z[T]$-solid theory over $\Z[T]$ is very similar to the $\operatorname{Solid} \Z$ theory over $\Z$

Derived category of the abelian category. Okay, so that's all the kind of formal stuff. But in the solid $\Z$-theory, it was also important to understand the basic compact projective generator, which is always gotten by just solidifying the sequence space.

So let me make a claim about that. If you take the $\Z$ power series---well, maybe I'll just say let's not think of it as a ring; we're thinking of it as a module now. If you take a product of copies of $\Z$, base change it to $\Z[T]$, and then solidify, you get a product of copies of $\Z[T]$.

So maybe a little remark about an interpretation of it. I said "solid $\Z$" was kind of analogous to complete non-archimedean topological abelian groups. Non-archimedean means, say by definition, that there's a basis of neighborhoods of the identity consisting of open subgroups.

So what would be the analogous interpretation of "solid $\Z[T]$"? It would be that you have a complete non-archimedean thing where moreover there's a basis of neighborhoods consisting of $\Z[T]$-submodules. And you can think of---for a basic example of a $\Z[T]$-module which is non-archimedean but does not satisfy that property, you can think of this ring. You cannot find a basis of neighborhoods of zero which are stable under multiplication by $T$, because they're stable under multiplication by $T^{-1}$ instead. But in some sense, the theorem is that if you kill just that one guy, then you've explained the difference between the two notions.

Okay, so in the previous theory without "$[T]$" it was a little bit different. It was not defined using this "$p$". Well okay, it basically was done this way. I mean, maybe we didn't make this explicit and we maybe more talked about just this, but we definitely talked about this. Yeah, but it does make it more clear to think of it that way with the $p$.

Okay, so the proof. Yes, confused about the $\Z[T]$ itself. Is $\Z[T]$ solid? Ah, so it is. That's part of the theorem, because I'm claiming in particular that this is $\Z[T]$-solid and $\Z[T]$ is a retract of this. So it's something that we need to prove and we will prove it. But it's not "solid", what $\Z[T]$ is un-"solid $\Z[T]$". Oh no, it is. As an abelian group, you mean? Yes. No, every discrete abelian group is solid because it's generated under colimits by $\Z$. No, it's an important point. Thank you for bringing it up. $\Z[[T]]$ power series is also solid. $\Z[[T]]$ power series is also solid, yes, because well, it's a limit of things. I mean, you can build it from limits and colimits from $\Z[T]$. So it's all right.

For the proof, well, all of these properties, all except the last. Those are exactly the same. The arguments for all of those things were completely formal, just based on the fact that you have this internally projective object $p$ and you're asking that some endomorphism of internal $\underline{\text{Hom}}(P,M)$ become an isomorphism. That was all that Peter used when he was proving the analog of these claims for solid $\Z$. So the fact that we have a good formal theory is already contained in there. And then in Peter's lecture, the hard part was identifying the free modules, that when you solidify this sequence space $p$, you actually just get a product of copies of $\Z$ fills up the whole thing. Thankfully, that part is actually going to be easier here because we already have solid $\Z$ and basically

Exactly. I mean, you basically---I don't know. Like I said, I sort of tend to think on the derived level from the beginning, but the basic point is that everything has a resolution by these internally projective guys. And then on those, they're the same, and then it's some derived limit of that. And it's just internally projective in $h$ in all condensed, in light condensed, I mean. But it's really not... that's really not necessary either. I mean, this also works in all condensed. I claim it's formal, and I also claim I don't want to get into the detail right now. So yeah, let's maybe discuss after if you still have questions.

Okay, right. So the proof of claim: This follows formally from the fact that this ring $\Z[[T^{-1}]]$ is idempotent over $\Z[T]$, which follows from the very first description I gave of it as the base change of---we check that the power series ring is idempotent and we put that at infinity. And the idempotency is preserved.

So for example, using the idempotency of that, what is this thing here? It's just the homotopy fiber of the inclusion of this into this, and this is the base ring. And you can easily check that idempotency is equivalently to equivalent to the derived idempotency of this object. And that means that if you take this expression and you apply it again, you get the same thing back. So that's kind of an idempotent operation.

And again, the same idempotent will prove to you that if you take this operation and $\mathbb{R}Hom$ from this guy, you get zero. So this thing is $\Z[T]$-solid. And that's not quite everything you need to check, but it's basically everything you need to check. It's an idempotent operation, and $m$ is solid if and only if this map is an isomorphism. So it's just completely formal, and I won't write out all of the details.

Note: This is a formula for the derived solidification of a general $\Z\langle T \rangle$-module. But something nice happens if your $\Z\langle T \rangle$-module is base changed from a solid $\Z$-module.

And next claim is that the functor---the sort of pullback functor from $\mathrm{Solid}(\Z)$ to $\mathrm{Solid}(\Z[T])$, sending a module $M$ to $M\otimes_\Z\Z[T]$ and then you solidify---this functor is $T$-exact, preserves limits and colimits, and it sends $\Z$ to $\Z[T]$.

If we prove this, then in particular we get this claim, because this is that functor applied to product of copies of $\Z$. I claim the functor commutes with products, so then it's enough to understand what happens with $\Z$. But I already also claimed that $\Z$ goes to $\Z[T]$, so we'd be done.

Proof: We just take this formula for the solidification, and we plug in the case where $m$ is also induced. We get that $(M\otimes \Z[T])^{LT,\mathrm{solid}}$ is the same thing as $m\langle T\rangle$ (by that I mean $m[T]/\Z\langle T\rangle$). And now we're computing an $\mathbb{R}Hom$ over $\Z[T]$.

So the way to do that is to disambiguate the two occurrences of $T$ and then equalize them at the end. So that's calculated as $\Z[[U]]$, and a two-term complex (I'll write it vertically so to speak). Then you equalize $T$ and the shift operator on here, which is induced by multiplication by $T$

Multiplication by $T$ is here given by sending $U$ to zero, $U^2$ to $U$, $U^3$ to $U^2$, etc. So if you do that there, then that induces an endomorphism here, and that's the multiplication by $T$ on this resulting thing.

Sorry, I missed what you said. What is the isomorphism to this? This bit of somewhat silly notation is the homotopy fiber of this map, or some kind of shift of a mapping cone, and that's the same as this. It's also the same as this.

Okay, and now we're done. Well, now this functor again---this is internally projective in $\mathrm{Solid}_\Z$, so this functor is $t$-exact and it preserves limits and colimits in $\mathrm{Solid}_\Z$. But limits and colimits in $\mathrm{Solid}_{\Z[T]}$ are calculated on the underlying level, because it's just part of a module category.

The last claim is that $\Z$ goes to just the usual polynomial ring in one variable. What happens when you plug in $\Z$ here? You're taking $R\mathrm{Hom}$s from this product of copies of $\Z$ to $\Z$. It turns into a direct sum of copies of $\Z$. You can check that what you get is just the usual $\Z[T]$, even with the $\Z[T]$-module structure. Really, there's a natural comparison map which you see to be an isomorphism.

Okay, so there was this---someone proved it's not related directly to the theory, but someone proved $R\mathrm{Hom}$ from the infinite product of copies of $\Z$ to $\Z$ is a direct sum. But this is not---here you are doing it in---this is easier, yeah.

So we can also do other examples of such a thing. Let's do another example. That finishes the proof of the theorem, so we now have a grip on this $\mathrm{Solid}_{\Z[T]}$ theory. I want to advance the interpretation that the $\mathrm{Solid}_{\Z[T]}$ theory is like working over $\Z[T]$. Without the solidification, it's like the affine line, and $\Z[T]$ with the solidification is like the closed unit disc. That was kind of the interpretation that I started with.

But let's do an example again in non-archimedean---let's base change to a non-archimedean field and see what happens. So let's take another example above. Let's look at $\Q_p[T]$, so that's functions on the affine line. Now we want to restrict to the closed unit disc in the sense that we've just described, so we take the $T$-solidification of this.

This is an instance of this functor here, and I just said this functor has all the properties in the world. It commutes with colimits, so I can again take the $1/p$ to the outside. Sorry, it also commutes with limits, so I can take the limit over $n$ of $\Z/p^n\Z[T]$, $T$-solidified, and then $1/p$ at the end.

Now we're applying this functor there to $\Z/p^n$, but that's just two copies of $\Z$. So you get the same answer as for $\Z$, it's just discrete. This is just inverse limit over $n$ of $\Z/p^n\Z[T]$, $1/p$, or in other words, it's $\Z\langle T\rangle$, $p$-completed, $1/p$. These are the functions on the closed unit disc, kind of as desired.

Okay, so maybe I'll take a five
The complement, and the complement was the open unit disc, or rather, the open unit disc moved to infinity. So that's this thing, and that corresponds to $\mathcal{D}^\circ = \mathrm{Mod}_{\Z((t))}$.

I think someone asked before, but I forgot that you had the solidification, which is $\Z[T]$. There should be another on the other side. Wait, wait, wait...

So then, this is just the natural inclu---well, it's a forgetful functor. A priori, you're forgetting the extra module structure, but it's in fact a fully faithful inclusion because of the idempotency. The idempotency of this thing means that there's at most one $\Z[T]$-module structure on any $\Z((T^{-1}))$-module structure on an $\mathcal{E}_\Z$-module.

So this is actually like a localization sequence, or what have you. This is the symmetric monoidal quotient of this by this thing, which is kind of an ideal in there. And what kind of adjoints do we have, and which of them are well-behaved, and so on?

Well, we know that this one has a right adjoint, which is given by the inclusion. But it also has a left adjoint. And the left adjoint is given by sending $M$ to, I guess, the fiber. So you take $M$, and then you base change it to infinity, and you take that cone diagram here. So it's the fiber of $M$ mapping to $M[T]$ over $\Z[T]$ with the $\Z((T^{-1}))$.

And this left adjoint is, in a sense, better behaved than the right adjoint, the naive inclusion. By the measurement that it's better behaved, well, they both commute with colimits, but this one satisfies a projection formula with respect to this fundamental functor, which is the symmetric monoidal functor. So this left adjoint is kind of linear over these two symmetric monoidal categories. So this is better.

I'm going to call this one $j^*$, and this one $j_!$, and this one $j_*$, this inclusion here. And on the other hand, for here, we have this functor here, which I'll call $i_*$, the inclusion there. So it has a left adjoint, also $i^*$, which is just the base change functor, which is symmetric monoidal. So that's kind of the more fundamental one from this perspective. But it also has an adjoint $i^!$, which is given by some arimum, but it's less well-behaved again, because here, this one satisfies a projection formula with respect to this one. So this functor is linear over this symmetric monoidal category, just like this functor is linear over this symmetric monoidal category.

"$\ast$ is an adjoint, the inclusion has $!$ on it." 
"It does, but that one we don't talk about."
"Yeah, now this was the question I..."
"Yeah, yeah." Because already this one's not as nice as, I mean, the good adjoint is actually up here.

So, the interpretation that this suggests---this is exactly like if you have a topological space $X$, and then you have $Z$ a closed subset, and then $U$ the complement open, then you get, if you have $\infty$-categories of sheaves, you have exactly the same thing. Where you have the open over here, you have this thing satisfies a projection formula, you happen to have this other guy, but it's not as well-behaved. And then you have $i_*$ from $\mathcal{D}(Z)$, $i^*$, and then $i^!$. So formally speaking, it behaves exactly the same way. And actually, they're both special cases of the same thing, which is: you have a symmetric monoidal category, and you have an idempotent algebra in it, and it generates the whole situation.

In this case, you have this category and this idempotent algebra. Here you have this category, and then the pushforward of the structure sheaf from the closed sub, or pushforward of the constant sheaf from the closed subset. An
You said that the rational opens are closed.

Yeah, for example a distinguished open is just the ring with the function inverted - that's quasi-compact.

The direct image of the structure sheaf is quasi-compact.

Okay, so now... Yeah, by the way, we're going to be guided by this sort of thing in setting up the definition of analytic stack and so on. One of the things we discovered is that when you move to this condensed/solid context, you actually get six functor formalisms in large generality on derived categories of "quasi-coherent sheaves". They really have nice interpretations.

For example, if you think of this thing as being what sits at infinity, then it makes sense that this is extension by zero. You're taking your sections and killing the ones that live near the boundary. It all plays quite nicely, and we showed how to give proofs of things like c-t-ness using these formalisms.

So we take this perspective seriously - the derived categories and functors between them are going to dictate to us what the geometry looks like.

There was another thing I remembered in the break that I forgot to mention. I said that this functor is t-exact and preserves limits and colimits, but I want to caution you that t-solidification from solid Z[T] is not t-exact. It's only off by one. I said it was given by RHom from this object which has a 2-term compact resolution. So it's only off by one from being t-exact.

It sends everything connective to something connective, and everything anti-connective goes to at most a 1-shift of something anti-connective. So t-solidification is not t-exact, but it's very controlled. The first map in the composition is t-exact, but the second is not. It's a bit funny.

Alright, so now I want to motivate what I'm going to do in the rest. Where are we probably going to go? We want to look at solid rings, i.e. commutative algebra objects in this tensor category solid Z. Generalizing our discussion when R is Z[T], we want to see subsets like closed/open unit discs. Of course, when you have a big algebra, you'll get many more such subsets.

We want to organize what you see in a nice way. It's not necessarily the most general thing, but basically we're going to take the things you see over Z[T] and base change them along all possible maps Z[T] → R, i.e. all possible functions in R.

What we'll end up with is the statement that the derived category of solid Z-modules localizes along the valuative spectrum of the underlying discrete commutative ring of this solid ring. You take the valuative spectrum of R viewed as a discrete ring.

The valuative spectrum is a residue field and valuation. The valuation doesn't have to be integral on R, just any valuation. It has a topology which induces the topology on things like Spa, but it's more general - sometimes one gets points which are not in Spa.

This space is similar to the usual spectrum of a commutative ring. It's a spectral space with a basis of quasi-compact opens, and even a particularly nice basis analogous to distinguished affine opens in algebraic geometry. While a distinguished open is parameterized by a single element, here you have to take a bit more data.

The basic opens are the so-called rational opens. You take finitely many functions $f_1,...,f_n, g$ and form $\{x \mid |f_i(x)| \le |g(x)|\text { for all i}\}$. The interpretation is that you invert $g$, so we're inside the distinguished open for g, but then shrink further by requiring $|f_i| \le |g|$.

Right, and so I'm saying that this localizes on this meaning. You have actually a sheaf of categories, a sheaf of symmetric monoidal categories. And what you're going to attach to this thing is a version of the solid theory. So you look at those $M$ in $D(\Mod_R)^\solid$ such that, well, first you want to say that multiplication by $g$ on $M$ is an isomorphism. And second, you want to say that if you take this internal hom from $P$ to $M$, and you take $f_i^* \otimes [-1]$, this should be an isomorphism for all $i$. Or in other words, $f_i/g$. Sorry, so in other words, you want---so you think of all these guys as maps from say $\Spec(R)$, so to speak, to the affine line. Then you want that $g$ lands in the standard codimension one locus, and you want that the $f_i/g$ land inside the closed unit disc.

Okay, so I'm not going to go into details about that. But where should I go now? I don't know, maybe... Here, this $|P|_B$ just refers to the local correspond to... I mean, the only... I don't... You can think of it as a locale, but it's also a topological space. And the points have a nice description and so on.

Okay, so what do I want to do today? Well, or partly do today. So in particular, I'm claiming this category localizes along the space. And in particular, you get a structure sheaf on the space. And what I want to do in the next bit, so goal for rest of lecture, is make this structure sheaf explicit and compare, well maybe probably start to compare, to Huber's theory. So we're eventually going to produce this data by very easy formal means, but it requires some language and setup. So we can't do it yet, but I want to make at least this part of it explicit already at the beginning.

Okay, so let's see. Let me make a... So let's start with this generality. We have a solid ring, and let's take an element $f$ in $R$, or really I should say in the underlying discrete ring of $R$ in case there's ambiguity. And that in particular, well, that gives you a map from $\Z[T]$ to $R$ which sends $T$ to $f$. And then we can sort of see how these loci that we've identified can be, or correspond to, properties of $R$.

So let me make a definition. $f$ is topologically nilpotent if this map, factors, I should say, through the power series ring. And $f$ is power bounded if... Well, I want to say that if this map, well the map, geometrically factors through the closed unit disc. But the way to say that is, if---so let's say if $R$ is actually, so $R$ is an algebra over $\Z[T]$. In particular, it's a module over $\Z[T]$, and we can ask that it be solid.

In the first definition, the factorization is unique. Yes, it's unique if it exists because of this idempotency. So that's an interesting fact, actually. This is also the free module on a null sequence. And so even though there are sort of Hausdorff solid abelian groups that have like non-Hausdorff behavior, still, this limit is unique if it exists.

Okay, so basically in the definition of solid, you're imposing that certain limits exist uniquely, even though you have non-Hausdorff behavior. Okay, so that's the same thing as saying that if you take $Hom(P, R)$, $f^* \tnsr [-1]$, that this is an isomorphism. That's just... Oh, someone's talking. Hello?

Yes, yes, the first condition is saying that it's a null sequence which is the $p^n$ go to zero. So it's literally... Yeah, it's... I mean, I was going to explain the relation with classical definitions, but it's quite immediate for this one that it's the same as the classical definition. So maybe I'll just repeat what Peter said. So this is
Sure, let me fix that for you:

Imprecise earlier, thank you. Yeah, in the sense of $\Z[T]$-modules is enough, or I'm not sure. I don't think so. Thank you for the comment. Yeah, just as rings, okay.

So, all right, here's a lemma giving basic properties. So, we write $R^{\circ}$ (this is again kind of standard notation in Huber's theory) for the set of power-bounded elements, and $R^{\circ \circ}$ for the set of topologically nilpotent elements, which is a subset of $R$. Well, yeah, I'm going to prove that.

The lemma is that $R^{\circ}$, inside this ring here, is an integrally closed subring, and $R^{\circ \circ}$, first of all, is contained in $R^{\circ}$, and it is a radical ideal.

So I'm still confused about this $N$-sequence. Something is an $N$-sequence in the sense that the map extends to another sequence, then you don't know that it factors as a ring map, that this is a ring map? That seems correct to me, yes. Okay, so it doesn't mean topologically nilpotent in your definition, in your sense. But on the other hand, maybe if you have something that comes from a Hausdorff topological ring, for example, if the target is quasi-separated, then independently of asking about the algebra structures, the limit is unique if it exists. Yeah, quasi-separated was this analog of Hausdorff in the condensed setting.

So still, if you start with something Hausdorff, then it is automatically a ring map if it is an $N$-sequence, yes, because of density. Okay, so we are not sure whether "topologically nilpotent" in the $N$-sequence sense is always the same as this other notion. Okay, Hausdorff, yeah, all right.

So, proof. Why is this a subring? Well, I guess maybe the first thing to check is that it has a unit (that's part of what I mean by "subring"). But if you look at the definition of "solid", that's one way of saying it. Put $f=1$ here, okay.

Now, I want to show that if $f$ and $g$ are in there, I'll prove that it's a subring by showing that if you apply any polynomial to $f$ and $g$, then you're still in there. So how can we do this? Maybe there are different ways, but I think the cleverest one is: We can look at the map from the polynomial ring in two generators to $R$ which sends $X$ to $f$ and $Y$ to $g$. Our hypothesis is that $R$ is a solid $\Z[X]$-module and a solid $\Z[Y]$-module, and what we want to conclude is that for any map from the polynomial ring in one generator $T$, $R$ is a solid $\Z[T]$-module, okay.

So first, resolve $R$ by its resolution. It's definitely solid as an abelian group, so we can resolve $R$ by direct sums of compact projective generators. But we know that $R$ is a solid $\Z[X,Y]$-module. So if we solidify with respect to $X$, then what does this turn into? It turns into a direct sum of products of copies of $\Z[X,Y]$ by the properties of derived solidification that I proved earlier. But then it doesn't change $R$, so that's still a resolution of $R$. Then we solidify with respect to $Y$ and we get a direct sum of products of copies of $\Z[X,Y]$ again by the same reasoning. So in total, we see that $R$ can be resolved by these guys, but each of these is clearly solid over $\Z[T]$, because it's a colimit of limits of things which are solid over $\Z[T]$. This is solid over $\Z[T]$ because it's discrete, and then this is a product and that's a direct sum, so in total it's solid over $\Z[T]$.

We conclude that $R$ is solid over $Z[X]$. Oh, I switched to $T$ somehow. Switch to $X$ somehow.

We conclude that this composition $R$ is actually solid as a $Z[T]$-module. Oh, I didn't prove... I forgot to prove it's integrally closed. I'm sorry.

Let's do that. It's again a very similar argument. Let's say that we have an equation of the form $\sum c_i t^i = 0$ where all $c_i$ are power bounded. Again, we just make the universal thing. We have $Z[x_0, \dots, x_{n-1}]$ and over that we have the ring where you adjoin another variable, let's call it $t$, and then you set the equation $\sum x_i t^i = 0$.

We have our solid ring $R$ and by hypothesis we have a map here such that when we compose to here, $R$ becomes a solid module over each of the $x_i$ variables. What we want to show is that when you compose here, it becomes a solid module over this $t$ variable.

I'll just say it quickly in words. You use the same trick. You resolve $R$ first as just a $Z[x_0, \dots, x_{n-1}]$-module in solid $A_{inf}$ groups and you solidify with respect to each of the variables. You find yourself built out of products of copies of this ring, but you're also a module over this, so you can tensor up to this. But that's a finite free module over that ring, so then that will just go inside the products and you find that you're resolved by a direct sum of products of copies of these guys. Because this individually is solid and solid is closed under limits and colimits, you deduce that that's solid as well. The key here is just that this is finite as a module over this, so that tensoring with it you can bring inside the product. Finite free, yeah. Although, well, that's not really necessary. The ring's Noetherian and you can resolve... finite would be enough in fact.

I'll leave the rest to you. It's completely analogous arguments, like why it's an ideal and why it's a radical ideal even. Fun exercises in that style of argument.

So now we can describe this structure sheaf. Suppose we have again a solid ring $R$ and then $g, f_1, \dots, f_n$ in $R^\mathrm{solid}$. The claim is that there exists a universal or an initial solid ring $R^\mathrm{solid}_{g,f_1,\dots,f_n}$ with a map from $R$ such that first of all $g$ becomes invertible in there, and secondly $f_i/g$ is power bounded in $R^\mathrm{solid}_{g,f_1,\dots,f_n}$ for all $i$. That kind of encodes the idea I was talking about, where you want $f_i/g$ to go to the closed unit disc, you want $g^{-1}$ to be invertible.

The proof is you can just construct the guy. You could first invert $g$, but maybe I'll invert $g$ at the end. What you can do is take $R$ and adjoin polynomial variables $x_1, \dots, x_n$ and then solidify with respect to all of them, which recall does something when $R$ is not discrete. Remember we had the example of one variable and $R$ was $\Q_p$, then this gave us the Tate algebra.

Then we can say that these variables are supposed to be $f_i/g$, so maybe $gx_i = f_i$ for all $i$. It doesn't matter at which point you invert $g$, but let's do it at the end. This kind of obviously satisfies the correct universal property. First we freely adjoin solid variables, then we impose the relations which guarantee that kind of thing. There's one small thing to check, which is that after you take this in which these are definitely solid and then you do these operations, you need to see it's still solid. But that's because it's all colimits.

So this is the kind of thing you're looking at. Now I want to make a caution here. This is not
That means what you're producing a priori is a derived sheaf, not an ordinary sheaf. So that's okay, but now I'll say, in most practical cases, almost all practical cases, all $\pi_i = 0$ for $i > 0$. So in practice it doesn't seem to cause trouble, but it's an important thing to keep in mind.

The second warning---even if $R$ is a very nice kind of Huber ring, then this quotient may not be quasi-separated. So it's a kind of an analogue of restricted power series in general. Yes, exactly. This will always be the usual Tate algebra in many generators. That follows basically from the arguments we gave for any sort of Huber ring $R$. But then when you take this quotient by the ideal generated by these elements, it might not be closed. So in principle you could be having a non-Hausdorff quotient here.

So when you say Huber ring, you mean complete Huber---yes, I mean complete Huber ring, thank you. But this theory is defined for complete Huber rings as well, or no? This theory is only defined for complete Huber rings, the theory I'm discussing, because you cannot define the condensed set of any topological thing. You can define it, but it won't be solid unless the thing is complete in general. Ah okay, you want it solid. So you want complete, and then the other guys, when you do this, it is still solid but doesn't come from---it's kind of an analogue of, yes.

Yes, but again, in almost all practical cases, or maybe all practical cases, it is quasi-separated, i.e., Hausdorff.

So what I was trying to aim for is the relation to Huber's theory. I guess I probably won't get there. So if we---I think we'll definitely discuss this in more detail later, but just as a preview: if $R$ is a Huber ring (and don't worry if you don't know what that is, it's just a certain nice kind of topological ring that people use in non-archimedean analysis), and let's say complete---no, just Huber, I'm just doing the rings now, not the pairs. And if the ideal generated by $f_1, \ldots, f_n$ inside $R$ is open, which is the condition that describes rational opens in the space of continuous valuations as opposed to the space of arbitrary valuations, then in this case Huber defines the ring of functions here. And you could ask what the relationship is with this thing that comes from the solid theory.

This is the quasi-separification of this more generally defined thing that we have here. So the solid thing, or maybe for emphasis I should put the $\pi_i = 0$ as well---you can get the Huber thing functorially from this more general thing, in particular from our structure sheaf. But they're not necessarily equal in general. In all practical cases, though, they are equal. That's the general outline of the story.

So in general, if you always use the structure sheaf, it always satisfies the---yes, so it's always a sheaf. Yeah, so maybe that's an important point to mention. There was this little fly in the ointment in Huber's theory, that in the general setting he defined a structure sheaf, except it wasn't a sheaf, it was only a presheaf. In all practical cases it was a sheaf, but still, the general theory was kind of missing something for that reason. That's fixed by this. If you no longer care about things being quasi-separated and non-derived, then you get a good plain old structure sheaf. And you get even more, you get the derived category of quasi-coherent sheaves which localizes. Also you get the possibility of defining all of these things even without this condition being present.

Okay, I think probably I'll stop there. Thank you for your attention.

Yes, do we have any use of discontinuous valuations? Any use? Yes, I know Scholze likes them, so he came up with this notion of adic spaces, and those correspond to certain---yeah, but you don't really use them. I think he used them, but you---oh me? Well look, I just build the theory, I don't---he calls them adic spaces, I don't think he developed the theory, he just kind of did it in an example. So he wanted to exactly include things like this open unit disc.

So this---yeah, things like the 

This non-quasicompact thing, to fit it inside Huber's theory---but Clausen advocates that sometimes it's better to just work with this thing as if it were fine, and that's something that our theory easily accommodates.

Yeah, there are people who did---I forgot the name of this---and constructed some notion of rigid, I forgot the name. Some draw some paper on this, on like a variant of... I forgot the name of the... There was some paper some time ago, but I'm not sure.

Okay, any other questions? Yes?

So every solid $\Z$-algebra... Yes, so you're... Yeah, so this notion of analytic ring, we haven't defined it yet. It kind of organizes a lot of discussion, but indeed, if you have a... If you have a solid... If you have an algebra in solid $\Z$-modules, you get what we call an induced analytic ring structure. So it's just all... You take all, I mean, the module category you have, and that sits inside condensed $\Z$-modules, and that is an analytic ring. Yeah, but in the case of $\Z$, we had something more natural---more natural, it's arguable, it's more complete. There are two things. They exist. They do, I mean, you want them both, I think, and can be extended... Can be extended to any discrete ring, right?

Yes, that's right. Not in general, not... Not general solid $\Z$-algebra. Well, for a general solid $\Z$-algebra, I don't necessarily know of any, like, completely canonical... Well, like, maybe... I mean, one thing you could do is you could take all of the power-bounded elements and force all of them to be solid. That would be kind of the maximally complete thing that you can get via the stuff that we've developed. But there could be further completions of that, as far as I know. I mean, I...

You're welcome. Yes, with the definition of solid analytic rings we used in the very first lecture, if we use this instead of a regular ring here, do we get something similar to like Huber's theory for Huber pairs instead of just regular? Could you repeat the question? I'm not sure I understood.

In the very first lecture, we defined solid analytic rings. Yes, with the ring and the direct category. If we try applying something similar here to analytic rings, what do you mean? Ah, you mean like this discussion of this thing here? Oh, yeah, yeah, yeah. You can do that. But do we recover, like...? Oh, yeah, you recover the theory with the $R^+$ in there as well, yeah. So elements $R^+$, you require solidity. Exactly, not for all of the guys. Exactly. So you can, you can... Yeah, and we will discuss this. You get... You can add that extra flexibility into the picture, yeah. And the fact that $R_{\Z_0}$ is this is automatic, that those are the... They're automatically solid, exactly.

Yes, okay. Yeah, so it actually fits remarkably well with Huber's theory. Like, you take tensor products of analytic... You have to first solidify one another, have to take... Yeah, yeah, usually. Yeah, so here it's still the same. Here you don't have... Well, these solidifications all commute with each other, so there you don't need to... Yeah, they're all given by $R$ mapping out of, out of something, and any two $R$ mapping out of commute with each other.

Yeah, other questions? Okay, well, see you on Friday or next Wednesday or whenever.

\end{unfinished}