% !TeX root = AnalyticStacks.tex

\section{\ufs Normed analytic rings (Clausen)}

\url{https://www.youtube.com/watch?v=wk_wInYTasQ&list=PLx5f8IelFRgGmu6gmL-Kf_Rl_6Mm7juZO}
\renewcommand{\yt}[2]{\href{https://www.youtube.com/watch?v=wk_wInYTasQ&list=PLx5f8IelFRgGmu6gmL-Kf_Rl_6Mm7juZO&t=#1}{#2}}
\vspace{1em}

\begin{unfinished}{0:00}
  e
okay  so  now  um  continuing  the  dis  I  want
to  continue  the  discussion  from  last
time  I  want  to  start  by  um  maybe  going
over  a  little  bit  of  what  Peter  said
perhaps  adding  um  some  details  uh  or
some  explanations  so  the  the  main  topic
is  uh  normed  analytic
Rings  um  so  we  had  this  now  we've
officially  defined  this  framework  of
analytics  stacks  that  we're  working  in
um  and  we  saw  uh  that  there  was  an
embedding
of  uh  well  light  condensed  anama  uh  not
well  at  least  a  funter  we  saw  that  there
was  a  funter  to  analytic
stacks  and  it's  induced
by  so  these  things  are  generated  by
light  profinite  sets  so  say
t
and  to  one  of  those  you  assign  I'm  just
going  to  write  it  I  know  this  is
probably  overloaded  notation  but  you
just  you  have  this  discrete  ring
of  this  is  a  discret
ring  of  continuous  functions  on  your
profinite  set  with  values  in  the
integers  um  it's  a  discrete  ring  um  and
you  can  view  it  with  the  uh  maximal
analytic  ring  structure  so  uncompleted
uh  so  you  look  at  all  condensed  modules
over  this  discrete  ring  or  derived
condensed  modules  over  that  discrete
ring  and  um  that's  the  analytic  ring  uh
that  you  associate  and  then  its  spec  is
an  analytic
stack  um  and  uh  the  thing  that  makes
this  funter  nice  is  that  this  uh  this
assignment  here
um  this  uh  sends  hyper
covers  uh  to  shriek  hyper
covers  uh
satisfying  uh  shriek  Cod  descent  or
descent  uh  so  so  apparently  I  don't  we
Pro  that  those  are  equivalent  sh  descent
C  descent  this  the  talk  oh  um  it  it's
the  exact  same  argument  to  you  mean  in
the  hyper  case
or  okay  I  I  remember  Peter  refer  to
something  like  this  in  some  explanation
so  the  when  you  have  a  you  can  formulate
one  that  the  categor  is  equivalent  to
the  one  to  the  inverse  limit  by  sh  or
the  direct  limit  by  and  those  are  it's  a
purely  formal  fact  about  PRL  so  proved
by  lur  that  those  are  equivalent  so  it
doesn't  need  anything  on  the  shape  of
the  diagram  or  or  whatever  yeah  um  so
it's  yeah  uh  so  these  were  the  things
that  uh  these  were  the  things  which
defined  the  notion  of  analytic
Stacks  um  but  it  also
preserves  finite  limits  so  for  example
pullbacks  well  let  me  say  finite
limits  so  if  you  take
a  pullback  of  light  profinite  sets
that's  a  that's  just  a  a  filtered
inverse  limit  of  pullbacks  of  finite
sets  and  then  it  goes  to  a  filtered
co-limit
of  uh  the  same  situation  for  finite  sets
so  this  reduces  to  finite  sets  and  for
finite  sets  it's  kind  of  completely
clear  um  and  that  so  and  those  two  facts
kind  of  imply  that  this  funter  here  so
that  that  the  induced  funter  here  that
you  get  from  this  uh  so  this  is  a  cimit
preserving  uh  Co  limit  and  finite  limit
preserving  um  which  is
nice
um  so  um  now  I  want  to  mention  uh  that
uh  that  uh  there's  a  paper  on  the
archive  by  Rock  uh  gregoric  uh  so  so  he
studed  is  the  analog  of
this  uh  with  a  analytic
stack  replaced
by  by  uh  fpqc  C
sheaves  um  in  in  usual  sheaves  fpqc
sheaves  in  in  usual  algebraic
geometry  over  a  field  um  and  he  wrote  a
nice  article  which  he  recently  posted  on
the  archive  um  which  goes  into  some  some
detail  about  properties  of  well  I  mean
he's  working  in  a  slightly  different
setting  but  much  of  it  is  much  of  it  is
the  same
um  yeah  so  I  would  recommend  reading
that
uh  okay  um  right  okay  and  then  uh  so  the
the  the  other  example  that  Peter
discussed  was  uh  example  of  a  light
condensed  set  let's  let  me  start  with
this  setting  so  let's  say  uh  K  is  a
compact
how  is
DF  I  think  a  question
the  how  does  Rec  topological  space  from
the  associated  antic  St  is  it  the  set
state  of  point  in  some  suitable
sense  so
the  I  don't  see  that  there  is  a
well-defined  notion  of  underlying
topological  space  of  an  analytic  stack
um  so  us
yes  that's  right  it's  a  you  have  you
have  an  underlying  uh  light  condensed
set
um  but  um  I  don't  see  an  underlying
topological  space  Oh  yeah  I  I  guess  then
you  could  map  this  to  topological  spaces
okay  so  you  have
underlying  so  when  you  have  a
stock  uh  no  the  stock  is  is  something
with  it's  a  generalization  of  ring
spaces  here  but  what  is
well  we're  not  really  using  like  a
diagram  You  cover  by  SP  by  kind  of
formal  spec  of  analytic  rings  then  you
have  the  the  the  world  superal  diagram
things  and  then  how  do  you  get  Space  how
do  you  get
the  you  cannot  take  spec  in  the  no
although  I  think  you  can  consider  all
open  substat  maybe  it  makes  sense  in
your  set  to  consider  open  you  could  you
could  use  monomorphisms  of  analytic
Stacks  to  try  to  there's  different  ways
to  okay  there's  actually  different  ways
to  try  to  extract  an  underlying
topological  space  from  an  analytic  stack
and
um
well  yeah  I'm  not  sure  it's  worth  saying
anything  completely  General  in  the  world
of  arbitrary  analytics  Stacks  but  um
well  once  when  we  introduce  this  notion
of  normed  analytic  ring  we'll  see  that
when  you  have  a  normed  analytic  ring
then  um  there's  a  nice  way  of  extracting
underlying  topological  spaces  for  things
over  that  normed  analytic  ring  let  me
let  me  continue  the  story  having
unsatisfactorily  not  addressed  the
question  um  so  let's  uh  like  so  if  you
have  a  compact  house  dworf
metrizable  and  let  me  add  finite
dimensional  and  um  this  actually  uh  is
the  same  thing  as  saying  that  K  is
embedded  as  some  closed  subset  of  uh
some  finite  product  of  copies  of  the
closed  unit  interval  um  so  these  things
are  very  easy  to  imagine
um  uh  then  well  then  we  can  view  this  as
a  as  a  light  condensed  set  in  particular
a  light  condensed  ana  um  and  then  we  get
an  Associated  uh  analytics
bace  um  and  Peter  made  a  claim  about  uh
the  fun  of  points  for  this  for  such  a
for  the  analytic  space  Associated  to
such  a  k
um
so
um  well  or  maybe  there's  there's  two
claims  so  and  two
claims  so  the  first  is  that  so  first  of
all  what's  the  most  important  invariant
you  have  if  if  you're  given  an  analytic
stack  it's  its  derived  category  so  we
saw  that  shriek  implies  star  descent
which  means  that  the  derived  category  of
a  stack  is  well  defined  by  pullback
upper  star  pullback  um  and  uh  this  um
this  D  of  K  I  mean  the  analytic  stack
Associated  to  K  by  abuse  of
notation  uh  this  is  just  sheaves  on
K  uh  with  values  in  derived  to  bilon
groups
um  what's
that  uh  thank  you  yes  thank
you  and  more  generally  if  you  base
change  this  stack  to  an  analytic  ring
then  you  put  uh  D  of  R
there
um  uh  right  uh  and  the  second  claim  is
that  if  you  want  to  map  say  Spec  R  for
an  analytic  ring
R  to  this  k
um  uh  this  is  equivalent  to  giving  a
symmetric  monoidal
functor  uh  from  sheaves  have  now  not
condensed  aelan
groups  uh  to  D  of
retric  monoidal  cimit
preserving  uh  funter  uh  such  that  uh
shriek
locally  on  r  on  Spec
R  what's  that  oh  thank  you  yes  yes  yes
DZ
linear  yeah
um  such  that  Shri  locally  on  R  we  have
that  uh  F  of  the  connective  part  so  she
K  with  values  and  say  d  of  Z  greater
than  or  equal  to  zero  um  lands  inside  D
of  R  greater  than  equal  to  Z  the
connective  part
there
um  so  how  much  the  fin  where  is  the
final  dimensional  use  is  it  some
technicalities
about  I  don't
know  you  can  Define  the  analytic  stock
for  any  case  without  Dimension  right
right  so  the  finite
dimensionality  um  I  we're  going  to  see
uh  where  the  finite  dimensionality  comes
in  I'm  going  to  sketch  the  the  argument
for  this  um
so  okay  so  well  the  first  remark  I  want
to  make  is  that  both  well  the  left  hand
side  is  a
set  well  maybe  I'll  say  both  are
sets
um  so  we're  in  this  world  of  analytic
Stacks  which  is  based  on  some  sheaves  of
Ana  so  a  priori  everything  is  an  Ana  I
mean  your  K  is  implicitly  some  co-limit
of  things  and  could  be  introducing
higher  homotopy  but  because  of  this
property  that  this  funter  preserves
pullbacks  um
so  well
so  um
so  this  is  a
monomorphism  uh  in  condensed  Ana
in  in  light  condensed  Ana  uh  so  by  the
property  of  pres  and  monomorphism  means
when  you  pull  it  back  along  itself  you
get  itself  again  nothing  fancy  then  you
get  the  same  in  analytic
stacks  and  that  tells  you  that  um  if  you
have  two  maps  can  agree  in  at  most  one
way  so  you  have  a  set  and  not  a
generala  um  and  over  here  the  reason
this  complicated  looking  space  of
functors  is  a  set  and  not  anama  is  that
um  you  know  for  well  it's  D  ofz  linear
so  it's  actually  the  same  thing  as  just
if  you  take  I  don't  know  sheaves  of  an
and  you  just  ask  for  a  CO  limit
preserving  functor  like  this  um  and  then
this  is  generated  by  these  uh  like  the
representable
ones  um  and  uh  but  well  if
we
um
uh  but  uh  so  in  particular  over  here
everything  is  determined  by  what  it  does
to  the  free  things  on  these
representable  guys  um  but  then  we  can
pass  to  the  complimentary  closed  and  we
and  it's  also  determined  by  the  um
so  by  the  um  so  this  this  thing  is
determined  by  uh  where  it
sends  uh  you  know  the  the  the  con  and
chief  on  a  closed
subset  um  and  then  this  should  just  uh
should  be  an  item  potent
algebra  in  the
Target  or  should  go  to  an  item  potent
algebra  under  f  it  should  go  to  an  item
potent  algebra  in  the
Target  um  and
um  and  then  uh  you  just  have  to  have
some  conditions  that  are  satisfied  so
you  just  have  to  specify  a  collection  of
item  potent  algebras  in  the  Target  and
then  some  conditions  have  to  be
satisfied  so  let  me  write  down  here  so
um  so  this  is  what  implies  that  uh  left
hand  side  is  a
set  but  uh  for  in  the  right  hand  side  um
is  just  the  set  of  uh  maps  from  you  have
closed  subsets  of  K  to  item  potent
algebras  in  D  of  R  uh
such  that  um  uh  inter  filtered
intersection  uh  goes  to  filtered  Co
liit  um  and
Union  uh  so
uh  Union  goes  to  well  the  analogous
construction  so  there's  a  if  you  have  a
union  of  closed  subsets  you  can  kind  of
use  a  Myer  viat  Taurus  to  express  the
the  uh  constant  sheath  on  the  Union  in
terms  of  the  constant  sheath  on  the  two
pieces  and  the  intersection  and  that
gives  kind  of  an  algebraic  formula  for
what  the  value  on  the  union  should  be
which  you  can  write  down  just  at  the
level  of  item  potent  algebras  and  you
ask  that  that  that  Union  goes  to  the
that  algebraic  construction  you  have
here
um  and  then  and  then  you  still  want  this
uh  this  condition  here  but  that's  just
another
condition  um
so  this  looks  like
sh  this  looks  like
Theory
K  the  direct  category  she  on  K  yeah  yes
this  uses  the  fin  Dimension  yes  that
does  yeah  okay  and  then
the  the  equivalence  with  sending  various
pieces  the  equivalence  of  giving
something  on  the  on  CL  on  H  on  closed
outpaces  is  it  does  it  use  this  doesn't
use  no  so  when  you  work  switch  one  with
shifts  or  so  far  I  haven't  used  finite
dimensionality  if  you  take  this  in  the
sense  of  lury  then  everything  I  say
Works  uh
generally  so  for  I  think  an  arbitrary
topological  space  k  but  um  certainly  for
a  compact  house  door  space
um
uh  okay  um  right  and  so  the  the  point  is
that  item  poent  algebra  is  form  a  this
is  actually  just  a  post
set  um  so  there's  a  a  prior  it's  an
Infinity  category  but  one  checks  that
that  it's  just  a  postet  so  a  map  if  it
exists  is  UN  you  can  uniquely  write  it
down  um  so  then  we  just  have  a  the  right
hand  side  is  just  you  have  you  just  have
to  give  a  map  of  poets  to  specify  all  of
this  um  a  map  of  postet  satisfying  some
simple  conditions  to  specify  such  a
symmetric  monoidal  blah  blah  blah  blah
blah  functor  and  so  on
um  the  next  thing  to  note  is  that  okay
we  should  now  now  I'm  claiming  these  two
sets  are  in  bje  and  I  should  first  write
a  map  and  the  map  uh  the  map  in  this
direction  is  obviously  going  to  send  F
uh  to  F  upper  star  where  F  upper  star  is
um
um  so  well  a  priori  D  of  K  well  D  of  K
is  this  thing  but  we  can  restrict  to  the
full  subcategory  where  um  you  require
the  values  to  be
discreet  um  or  I  could  say  dcon  Z  linear
functors  from  uh  shees  on  K  with  values
and  dcon
z  um  and  then  I  so  but  I  should  that
gives  one  of  these  things  but  I  should
maybe  explain  why
the  uh  that's  a  derived  category  of
condensed  a  bilan
groups  light  condensed  yeah  light  sorry
light  yeah  okay  so  you  have  a
uh
uh  so  F  goes
to
uh  spec  ah  by  definition  spec  out  okay
is  H  is  a  map
from
the  first  you  have  to  give  a  map  on  the
Ring
of  but  this  is  part  of  this  functor  the
map  on  rings  is  is  included  in  KN
the  no  so  what  is  it
your  F  from  spe  R  to  K  means  that  you
have  a  of  analytic  R  well  it  means  you
have  have  some  say  shriek  hyper  cover  of
this  uh  and  a  map  of  that  shriek  hyper
cover  to  the  hyper  cover  of  this  by
profinite  sets  oh  because  K  is  not  okay
yeah  and
uh
okay
um  and  if  is  totally  disconnected  then
it  is  the  same  as  a  map
of  if  K  is  totally
disconnected  yeah  so  yeah  so  maybe  yeah
if  if  K  is
profinite  then  the  left  hand  side  is  the
same
as  uh  map  of
rings  uh  from  continuous  functions  on  K
with  values  and  Z  to  uh  just  the
underlying  ring  of  this  uh  an  analytic
ring  that's  by  by
construction  um  that  those  are  the
same
um  okay  and  this  actually  explains  why
this  condition  is  satisfied  I  mean  for
arbitrary  K  because  for  arbitrary  K  you
can  Sur  from  a  profinite  set  T  and  then
this  is  a  a  surjection  so  by  definition
there  will  be  some  shet  cover  Spec  R  uh
where  you  factor  through  a  map  to
T  but  once  you  factor  through  a  map  to  T
then
you're  uh  your  F  uper  star  is  just  being
induced  by  this  functor  but  this  is  just
a  filtered
cimit  of
uh  of
um
uh  yeah  filtered  colon  of  C  of  like  K
nz's  where  this  is
finite
um  and
um
well  what  do  I  want  to  say  the
um  so  the  the  so  well  what  what  I  want
to  say  maybe  yeah  I  don't  know  if  that's
the  right  remark  to  make  so  in  the  case
when  K  is  profinite  then  so  you  want  to
prove  that  the  image  of  every  connective
object  is  connective  what  do  you  know
you  know  that  the  image  of  the  unit  is
connective  because  the  unit  has  to  go  to
the  unit  in  D  of  R  which  is  this
underlying  ring  which  is  connective  by
definition  um  but  then  uh  on  any  clopen
subset  we  then  have  to  go  to  something
connective  because  it  will  be  a  retract
of  that  unit  um  but  then  the  clopen
subsets  generate  the  connective  part
under  co-  limits  because  they  generate
all  open  subsets  under  co-  limits  and
then  everything  is  generated  under  that
by  Co  limits  so  everything  will  be
everything  uh  in  this  thing  here  in  the
case  where  this  is  profinite  everything
here  will  be  a  cimit  of  retracts  of  the
unit  objects  and  therefore  we'll  land  in
here  um  in  the  case  where  case
profinite  um
so  in  this
case  the  connectivity  is
automatic  um  so  that's  why  this  is  is  a
well-defined
map
um  and  uh  if  you  want  to  go  backwards  so
given  such  a  symmetric  monial  functor  or
such  an  association  with  itm  potent
algebras  um  it's  enough  to  go  backwards
uh  assuming  this  condition  because  you
can  work  shriek  locally  because  both
sides  are  shriek  sheaves  or  sh  satisfy
shriek  descent  so  to  go
backwards
um  uh
conversely  uh  if  you're  given  such  a
symmetric  monoidal
funter
um  let's
see  on  the  connective  level  um  then  you
want  to  produce  uh  this  map  here  um  what
you  can  do  is  you  can  take  uh  oh  sorry
this  was  okay  uh  you  can  take  this  hyper
cover  by  take  a  hyper  cover  by  profite
sets  or  just  well  I  guess  just  a  cover
is
fine  once  you  start  with  a  surjection
from  a  profinite  set  oh  sorry  what  am  I
doing  uh
uh  um  each  of  these  will  automatically
be  profinite  sets  because  they'll  be
closed  subsets  of  the  product  of  two
profinite
sets
um  uh  then
um
uh
then  uh  then  you  get  a
corresponding  uh  diagram  of
rings  uh  communative
algebras  uh  in  D  of  R  greater  than  or
equal  to  zero  so  you  just  take  so  if
this  is  pi  uh  so  Pi  Z  and  then  this  is
pi  1  and  then  Pi  2  and  so  on  so  you  have
um
Pi  0  lower  star  of  the  constant  Chief  Pi
1  lower  star  of  the  constant
chath  and  so  on  um  have  the  constant
Chief
um  and  these  things  are  all  going  to  be
connective  um  and  in  fact
um  you  have  AA
formula  uh  which  gives  that  this  is  a
this
thing  uh  is  the  I  a  check  the  co-  check
nerve  I  don't  know  of  just  the  first
map  um  so  because  these  maps  are  proper
uh  any  map  of  pro  you  know  compact  house
door  spaces  is  proper  for  using  proper
base  change  you  can  see  that  uh  like  the
whole  the  whole  diagram  is  just
determined  by  the  first  part
um  so
um  uh  and  now  you  can  apply  your
symmetric  monoidal  functor  F  so  you  can
take  uh  F  of  this  pi0  lowest  r  z  and  now
this  is  a  commutative  algebra  object
um  uh  well  first  of  all  I  should  be
saying  r  Pi  lower  St  Z  but  it's  actually
concentrated  in  degree
Z
yeah  yeah
um
um  and  then  in  fact  you  can
say  in  fact
animated  um  so  it's  not  it's  not
difficult  to  produce  the  animated
structure  but  let  me  not  get  into
it  um  and  then  you  take
uh  uh  to  R  Prime  to  be  the  analytic  ring
defined  by  um  f  of  Pi  0  lower  star  Z
modules
uh  and  then
um
um  just  the  induced  analytic  ring
structure  of  you  you  just  take  this  ring
object  in  here  and  look  at  that
underlying  ring  and  just  modules  over
that  ring  in  in  the  thing  you  already
have
um  and  then  you  can
check  uh  that  this  procedure  uh  induces
a  map  from  the  check  nerve  of  this  RP
Prime  mapping  to  R  so  spec  so  spec  RP
Prime  mapping  to
R  to  Spec
R  uh  and  then  by  the  construction  you
get  an
uh  so  is  this  z  underl  z  the  same  Z  I'm
sorry  where  are  we  on  the  top  board  yes
z  underl  z  yeah  and  then  have  Z  right
are  they  the  same  they  are  not  the  same
so  uh  this  one  maps  to  the  constant  Chi
Z  and  uh  that  one's  the  the  the  co  the
the  co  co-  fiber  when  Z  is  the
complimentary  closed  subset  to  the  open
subset  U  so  they're  not  the  same  but
they  determine  each  other  in  a  canonical
way  so  you  said  something
to  so  this  is  actually  a  single  sh  in
any  case  it's  a  some  f  because  it's  def
is  a  symmetric  mon  you  get  some  Comm
object  some  der  like  in  you  say  there  is
canonical  way  to  view  it  is  a  simpl
right  right  and  but  this  is
not  I  think  in  characteristic  zero  it
should  be  yes  in  characteristic  zero
it's  it's  clear
not  because  of  the  abstract  way  you
define  you  have  the  condition  it's  not
clear  that  you  that's  right  yeah  it
requires  a  little  extra  argument  so  you
can  see  for  example
that  yeah
it  requires  an  extra  argument
but  so  probably  the  work  that  you're
doing  with  Maxim  will  make  it  more  clear
because  so  I  so  what  what  uh  what  what
is  being  pro  probably  carried  out  at  the
moment  is  a  kind  of  a  categorical
perspective  on  animated  commutative
rings  so  they  you  can  s  single  out  what
properties  of  the  derived  category  of  an
e  infinity  ring  or  the  C  you  should  just
say  category  of  modules  over  an  e
infinity  ring  what  structure  do  you  need
to  put  on  that  to  promote  the  E  infinity
ring  to  an  animat
uh  thing  and  it's  something  like  derived
symmetric  power
functors  and  um  you  have  them
on
uh  you  have  them  on  I  mean  maybe  there's
an  even  simpler  argument  sorry  I  think
in  this  case  there  might  be  a  simpler
argument  so  you  want  to  produce  a  sorry
uh  let  let's  let's  leave  that  aside  for
the  moment  and  address  it  at  the  very
end  okay  so  yeah  but  yeah  sorry  I
um  okay  so  where  are  we  ah  yes  so  then
what  we
get  is  a  map
from  the  check
nerve  of  Spec  R
Prime  mapping  to  Spec
R  uh  to  the  check
nerve  of  uh  t0  mapping
Decay  and  um  then  we're  going  to  be  then
we'll  have  produced  a  map  by  by  by  kind
of  by  descent  we  have  produced  a  map
from  Spec  R  to  K  so  then  if  we  just  so
we  as  long  as  we  know  that  this  is  a
cover
okay  um  but  note  that  this  is  a  proper
map  by  construction  we  built  it  as
something  where  the  so  to  speak  the
completeness  condition  isn't  changing
when  you  go  from  R  to  RP  Prime  um  you're
just  extending  the  ring  so  by  the
Criterion  we  had  for
shable  uh  things  it's  enough  to
see  uh  that  the  unit  object  which  is  our
triangle  uh  lies  in  the  category
generated  by  the  image  of  a  so  in  the
image  of  the  the  forgetful  functor  uh
sorry
D  of  R  Prime  to  D  of
R  and  for  that  by  applying  F  it's  enough
to
see  using  yeah  the  lower  shriek  but
lower  shriek  is  lower  star  so  it's
really  just  a  forgetful  functor  from  and
the  image  this  means  the  image  and
closing  by  closing  by  say  finite  Yeah  by
cones  and  retracts  and  so  finitary
operations  yeah
okay  um  then  applying  F  it's  enough  to
see  that  the  constant  chief  on  K  is  in
the  subcategory  generated  by  the  a  lower
star  functor  so  H
sorry  sorry  sheaves  of  t  on  T  values  in
DZ  U  so  Pi  lower  star  sheaves  on  K
values  in  D  of
Z
okay  um  or  in  in  in  uh  or  in  ail
Matthew's  language  we  need  to  see  that
uh  Pi  lower  star  of  the  structure  chif
on  t0  is
descendible
right  um  now  remember  that  we  this  was
kind  of  we  could  choose  this  arbitrarily
so  it's  actually  enough  for  me  to
produce  a  cover  by  a  profinite  set  for
which  I  can  check  this  descend  ability
um  so  we  can  reduce  to  by  pullbacks  we
can  reduce  to  k  equals  like  01  to  the
N  um  and  then  I'll  I'll
discuss  just  the  case  n  equals  1  because
it's  simpler  to  write
down
uh
so
um  gu  you  can  also  reduce  to  this  case
what's  that  yes  that's  true  by  by  kith
yeah  you  can  also  reduce  to  this  case
yeah  that's  true
yeah  so  this  means  when  you  take  the
fiber  P
some  so  there's  some  yeah  we  didn't  get
into  much  details  about  how  to  make
these  sorts  of  arguments  but  aille  gave
some  tool  kits  which  are  nice  um  so  then
in  this  case  you  take  the  the  usual  kind
of  caner  set  so  you  have  the  so  I'm
going  to  produce  my  cover  of  the  closed
unit  interval  by  by  uh  by  the  caner  set
so  what  I  but  let  me  write  it  in  a
specific  way  so  I'm  not  going  to  not
going  to  write  so  profinite  said  a
priori  you  want  to  write  it  as  an
inverse  limit  of  finite  sets  but
sometimes  you  can  instead  write  it  as  an
inverse  limit  of  compact  house  door
spaces  and  in  the  limit  you  just  happen
to  get  a  Prof  finite  set  and  that's  the
best  thing  to  do  here  so  what  we  can  do
is  we  can  take  you  know
uh  the  two  halves  of  the  closed  interval
and  and  their  disjoint  Union  is  a  space
mapping  to  the  unit  interval  um  and  then
we  can  do  the  same  thing  again  this  the
the  base  to  the  base  to  exp  of  the  yeah
so  then  that's  a  so  then  we  have  a
sequence  of  spaces  mapping  to  the  unit
interval  where  in  the  inverse  limit  you
actually  just  get  the  caner  set  um  which
is  then  forming  a  cover  of  the  closed
unit  interval  so  what  does  this  mean  on
the  level  of  these  push  forwards  so
these  the  push  forward  from  the  caner
set  will  then  be  the  sequential  Co  limit
of  the  push  forwards  from  each  of
these
um  now  there's  a  general  fact  about  this
descend  ability  that  if  you  have  a
sequential  co-  limit  it's  enough  to
establish  descend  ability  with  uniform
exponent  of  nil  potent  for  each  of  the
finite
ones  um  and  for  each  of  the  finite  ones
you  have  descend  ability  basically
because  you  have  a  my  vus  cover  for  the
closed  subsets  and  there's  the  point  is
the  reason  you  get  a  uniform  bound  on
the  exponent  is  because  in  each  case
there's  only  double  intersections  that
you  need  to  be  concerned  about  and  no
triple  intersections  so  in  the  end  you
get  something  like  exponent  of  n  pot  two
for  each  of  these  individual  things  and
then  three  or  four  for  the  the  co-  limit
or  something  like  this  each  and  the
composition  of  the  individual  thing
is  no  when  you  have  the  individual  in
because  to  adjacent  one  you  have  only
adjacent  and  then  so  when  you  divide  K
you  claim  that  this  is
a  ah  okay  and  then  there  I  understand
that  when  you  go  to  Power  like  01  to
some  power  then  the  the  the  estimate
will  become  where  else  yes  and  therefore
like  for  the  cube  you  cannot  do  it  you
can't  expect  anything  that's  right  yes
so  this  is  where  the  the  finite
dimensionality  really  comes  in  so  it
plays  a  sort  of  a  technical  role  in
being  able  to  ignore  the  difference
between  this  and  the  derived  category  of
shav  of  a  being  groups  but  um  but  uh
this  is  the  this  is  really  where  you
need  finite
dimensionality
okay  uh  okay  now  let's  come  back  to  the
the  question  about  e  Infinity  versus
animated  commutative  so  um  now  we  have
this  we  have  this  object
um  oh  wait  is  this  going  to  work  just  a
sec  oh  maybe  this  isn't  going  to  work  uh
no  let's  not  come  back  to  the  question
about  e  Infinity  versus  animated
commutative  with  my
apologies
um  okay  let's
actually
um  right  oh  I  want  to  oh  so  at  the  end
of  this  I  want  to  make  a
remark
um  I  maybe  it's  good  to  say  that  at
least  in  this  example  of  this  H  he  can
ow  five  by  hand
structure  because  you  only  have  to  do
for  Al  because  sequential
of  yes  that  was  the  argument  I  was  yeah
that's  the  argument  I  was  going  to
suggest  and  then  I  was  thinking  in  my
head  um  about  why  for  item  potent
algebras  it's  automatic  and  it  wasn't
quite  clear  to  me  um  but  yeah  but  I
think  you  can  do  it  by  hand  I  mean  if
you  knew  the  categorical  structure  well
we  should  we  should  think  more  carefully
about  exactly  how  to  do  this
um  and  report  back  next  time  uh  so  okay
um  but  let's  move  on  I  want  to  make  a
small  remark  before  we  move  on
um  remark  so  by  similar  arguments  or
basically  the  same
arguments
um  if  you  have  a  a  map  like  this  um  then
or  sorry  if  our  functor  like  this  it's
equivalent  um  a  prior  you're  saying  that
you  have  this  connectivity  estimate
shriek  locally  on  Spec  R  um  but  um  it's
equivalent  to
ask  uh
that
um
so  uh  over  a  proper  cover  a  cover  by
proper
Maps  you  get  the
connectivity  so  you  don't  actually  have
to  when  you're  wor
there's  over  AR  that  mod  give  you  an
ring
structure  on  on  your  R  could  you  say
that  one  more  time
Peter
don't  connect  algebra  over  any  antic  R
yeah  then  you  can  use  this  to  Def  new
notion  of  complete  modules  which  are
modules  over  the  sky  oh  yes  uhuh  I  what
you  call  thetic  ring
always  I
over  H  that  that  that  seems  to  work
indeed  yeah
yeah
yeah  um  but  are  all  of  the  intermediate
guys  connective  as  well  yeah  I  guess
they  are  well  okay  yeah  so
yeah  um  okay  or  maybe  connectivity  isn't
even  important  there  anyway  all  right  uh
oh  yes  um  and  another  remark  is  that
this  will  actually  be  a  little  bit
useful  now  I've  not  lost  ah  I'm  sorry  oh
we're  over  here  in  the  setting  of  number
two  in  the  the  the  theorem  here  okay  so
so  let's  say  you  want  to  produce  a  map
like  this  by  this  procedure  so  you  want
to  produce  a  symmetric  mordial  funter
and  then  you  have  to  check  this  annoying
condition  that  on  some  shet  cover  of
specr  you  have  this  connectivity
condition  I'm  saying  or  or  let's  yeah  or
I  don't  know  you  have  no  I'm  no  I'm
making  a  different  claim  sorry  that  uh
let's  say  you  have  a  map  like  this  then
you  know  that  shriek  locally  you  get  uh
this  but  in  fact  what  we  see  in  the
proof  is  that  um  after  a  proper  map  even
after  just  a  proper  map  from  a  proper
cover  of  specr  you  you  can  ensure  this
condition  this  comes  from  after  proving
the  equivalent  yeah  yeah  yeah  yeah  yeah
not  a  prior  not  a  priori  exactly
yeah  um  another  remark  um  is  that  if  you
have  Z
closed  uh  closed  inside  K  or  this  is
actually  more  General  but  um  and  then  U
is  the  complimentary
open  uh  then  they  are  also  each  other  as
complements  uh  in  analytic
Stacks  so  these  two  do  determine  each
other  the  associated  analytic  Stacks  do
determine  each  other  in  the  naive  way
I.E  if  you're  given  Z  and  you  want  to
know  U  as  an  analytic  stack  its  functor
of  points  is  you  just  map  to  K  such  that
when  you  pull  back  to  um  to  Z  you  get
the  empty  analytic  stack  that's  the  same
thing  as  mapping  to  U  and  vice
versa  so  that  that  you  can  check  on
profinite
sets  where  it's  um  where  it's  quite
Elementary
um
okay
uh  so  now  let's  get
to  sorry  why  did  do  right  see
toag  um  how  else  did  you  want  me  to
write  them  only  what  what  it
means  doesn't  mean  anything  um  it's  just
because  the  well  I  wanted  to  write  for
example  I  wanted  to  write  this  one  this
one  right  above  this  one  because  I'm
saying  the  these  two  end  points  map  to
the  same  point  down  here  so  I  wanted  to
stack  them  vertically  like  that  maybe
that's  the  maybe  that's  the  reason  does
that  make  sense
yeah  yeah
okay  so  now  we  get  to  a  topic  that  I
think  is  fun  which  Peter  introduced  last
time  this  kind  of  norms  on  analytic
rings  so  let's  say  that  R  is  an  analytic
ring  so  definition  uh  is  that  a  norm  on
R  is  a
map  of  analytics
Stacks  um  from  the  algebraic  P1  over  R
which  is  something  you  can  build  over
any  analytic  ring  by  just  base  change
from
the  trivial  case  with  algebraic  geometry
um  to  the  closed  interval  from  0er  to
Infinity  which  is  a  condensed  set  and
thereby  an  analytic  stack  um  so  let's
call  this  map  n  and  then  such  that  and
then  I'm  going  to  give  some
conditions
um  which  are  going  to  be  different  from
the  ones  that  Peter  gave  last  lecture  so
a  PRI  I'm  adding  an  extra  condition  and
maybe  slightly
rephrasing  um  another  of  his  conditions
and  we  don't  know  whether  they're
equivalent  but  certainly  uh  we  want  all
of  the  properties  I'm  about  to  list  so
let's  take  this  as  the  official
definition  now
um
uh  right
um  so  the  first  condition  is
that  okay  maybe  the  first  thing  to
remark  is  that  such  a  norm  function  on
P1  of  r  well  certainly  you  can  restrict
it  to  A1  of  R  you  get  a  norm  function  on
A1  of
R  um  but  then  what  does  it  mean  when  you
have  an  element  of  R  then  do  you  get  a  a
real  number  or  an  element  in  zero
Infinity  no  you  do  not  get  an  element  in
Zer  Infinity  you  get  a  map  so  so  so  well
let  me  so  note  if  you  if  you're  given  a
if  you're  given  an  F  in
R  then  that  induces  a  section  so  from  P1
of  R  to  Spec  R  uh  then  you  get  the
section  corresponding  to  f  um  and  then
you  can  compose  that  with  a  norm  map  0
infinity  and  what  you  get  is  a  map  from
Spec  R  uh  to  0
Infinity  your  yeah  yeah  exactly  now
suppose  just  to  to  I  don't  know  if  this
is  a  big  to  but  suppose  you  start  from
an  obstruct  we  said  that  there  are
several  ways  to  view  it  as  an  analy
Okay  and  like  you  can  use  the  the  or  no
I  mean  anyway  there  at  least  two  or
three  ways  I  remember  and  then  for  each
of  those  you  have  a  notion  of  G  on  the
other  hand  you  have  a  gome  like  in  Bel
or  like  this  so  can  you  say  what  are  the
relations  so  you  have  and  what  are  yes
we  will  we  will  discuss  uh  these  things
later  but  I  want  to  get  the  basic  uh  the
basic  definitions  in  place  first  basic
results  and
definitions  so  yes  so  a  norm  for  an
element  of  the  ring  you  don't  get  a  real
number  you  get  a  map  from  Spec  R  to  to
the  real  numbers  plus  infinity  so  if  you
you  can  think  of  this  as  consisting  of  a
family  of  residue  fields  and  kind  of  for
each  residue  field  you  get  a  real  number
but  they  could  be  varying  with  the
residue  Fields  so  to  speak  um  relatedly
if  you  have  a  norm  on  R  and  you  have  a
map  from  R  to  RP  Prime  you  get  a  norm  on
RP  Prime
so  just  by  composition  um  so  it's  really
um  it's  not  like  giving  a  assigning  a
norm  to  each  element  in  R  it's  a
geometric  thing  so
it's  yeah  it's  it's  it's  a  geometric
thing  it's  you  can  pull  it  back  and  it
still  persists  so  uh  that's  important  to
important  to
realize  uh  okay  right  so  that  that's  a  a
Prelude  and  then  the  first  condition  uh
is  that  Norm  of
zero  which  I  am  which  I'm  which  is  a  map
from  Spec  R  to  01  or  0  Infinity  uh  this
should  Factor  through
so  Factor  through  the  terminal  map  to
this  is  the  this  is
the  this  is  the  terminal  analytic  St
terminal  analytic  stack  but  it's  also
the  analytic  stack  Associated  to  the
condensed  anama  which  is  the  Singleton
point  which  is  a  subset  of  here  um  so
this  is  a  condition  that  the  norm  is  the
constant  function  zero  the  norm  of  zero
is  the  constant  function  Zero  by  the  way
in  this  geometric  perspective  over
there's  something  you  might  appreciate
there's  sometimes  this  question  of
whether  normal  one  should  be  equal  to
one  or  zero  for  the  zero
ring  nor  what
is  yeah  but  this  is  avoided  here  because
when  you  have  the  zero  ring  then  the
norm  is  both  one  and  zero  so  to  speak
because  then  then  this  is  the  empty  set
and  then  uh  uh  that  map  factors  both
through  zero  and  through  one  and  so  in
the  if  you  take  this  geometrical
perspective  on  Norms  there's  no  you  can
never  get  messed
up  it's  not  the  same  notion  but  I  mean
just  to
show  um
okay
um  okay  wait  sorry
over  here  there's  no  like  we're  not
saying  anything  about  what  happens  to  A1
of
R  just  let  me  finish
um  so  the  second  condition  is  that  the
following  diagram  commutes  so  P1  VAR
maps  to  zero  in  the  second  and  third
conditions  I'm  going  to  try  to  say  the
norm  is
multiplicative  um  the  first  thing  I'm
going  to  say  is  that  if  you  it  commutes
with  inversion  so  so  here  we  have  Lambda
goes  to  Lambda  inverse  which  exchanges
zero  and  infinity
um  but  we  also  have  let's  call  the
coordinate  in  P1  T  and  we  also  have  t
goes  to  T
inverse  um  exchanging  zero  and  infinity
and  I  want  this  diagram  to  commute  so  by
the  way  the  maps  from  anything  to  Z
Infinity  of  a  space  like  k  or  a  is  a  set
in  your  it's  a  set  yes  just  a  set  yeah
yes  and  even  with  sem  compact  okay  this
was  said  before  in  yeah  any  any
condensed  set  will  go  to  a  an  analytic
stack  whose  funter  of  points  is  a  set
yeah
mhm
okay  uh  so  this  one  and  two  so  now  note
that  one  and  two  uh  implies  that  uh  if
you  take  the  infinity
section  so  oh  let's  say  Norm  of  Infinity
uh
uh  so  Spec
R  Norm  of  infinity  0  Infinity  this
factors  through
Infinity
um
um  okay  now  uh  before  I  yeah  and  now
yeah  okay  maybe  three  so  Set  uh  I  don't
uh  it's  one
right
ah  does  this  does  this  already  imply
that  yeah  okay  that's  good  I  was  going
to  make  that  remark  after  the  next  one
but  yeah  I  guess  it's  already  implied
here  okay  cool
um  and  Al  the  norm  of  minus  one  is  well
by  the  same
Al
okay  yeah  thanks
guys
okay  so  three
um  um  so  I'll  Define  given  a  norm  on  an
analytic  ring  I'll  Define  the  associated
analytic  Aline  line  to  be  uh  the  subset
of  P1  where  the  Norms  are  a  real  number
and  not
Infinity
um  so
um  uh  then  we  have  uh  then
require  uh  that  if  you  take
A1  R  analytic  cross  A1  are  analytic  and
then  you  have  Norm
Norm  uh  to
oops  oh  I  guess  this  is  just  R  but  okay
I  I  I'll  continue  to  write  it  as  0
Infinity  um  uh  then  here  we  can  take  the
product
so  and  we  still  get  something  in  the
region  from  zero  to  Infinity  which  is
contained  in  uh  Zer  Infinity
closed
um  uh  then  on  the  other  hand  this  we  can
map  to  p1r  cross  P1  oh  sorry  p1r
period  um  via  multiplication  so
TS  goes  to  St  or
TS  this  is  the
five  ah  the  multiplication  M  on  P1  is
not  defined  in
general
TS  ah  the  norm
inverse
why  I'm  going  to  I  I'll  make  the  yeah  so
Peter  is  making  the  pertinent  remark
which  I'll  justify  uh  afterwards  I  mean
so  yeah  so  actually
yeah
um  well
yeah  that  I  I'll  justify  why  this  map  is
well  defined  at  the  end
um  right  so  this  this  map  should
commute  so  that's  saying  that  the  norm
is  multiplicative  but  as  ofer  is
pointing  out  the  one  needs  to  justify
that  such  a  map  indeed  exists  um  let  me
do  that  now  so
uh
so  so  y
TS  goes  to  TS  is  well
defined  on
a1n  um  so  the  claim
uh  uh  is  that  uh  so  A1  RN  is  a  subset  uh
well  sub  monomorphism  admits  a
monomorphism  to  P1  by  definition  another
thing  that  by  definition  admits  a
monomorphism  to  P1  is  the  algebraic
apine  line  over
P1  and  the  claim  is  that  this  one's
contained  in  this
one  ALB  line
is  is  the  you  take  the  polom  bring  one
variable  yes  over  over  R  yeah  over  R  in
which  so  in  other  words  you  take  for  the
condensed  you  take  just  head  po  long  in
in  a  stupid  way  I  mean  without  any
special  and  the  category  is  the  the  s  i
okay  I  would  say  yeah  so  it's  the  thing
it's  an  it's  an  apine  analytic  stack
this  a1r  and  it's  um  just  given  by  again
keeping  the  same  class  of  complete
modules  you  already  had  in  R  and  just
adding  the  polinomial  variable  as
operators  H  why  is  this  true  yeah
so  um
right
so
uh  so  the  point  is  that
um  so  we  want  to  produce  uh  so  you  have
to  check  it  on  the  two  charts  of  P1  one
of  them  is  already  A1  R  and  the  other
one  is  the  other  A1  R  yeah  to  some  Inver
the  operator  when  you  know  that  you
right  so  so  here's  what  I'm  going  to  say
so  we  know  that  a  norm  of  infinity  is
equal  to
Infinity
um  and  uh  so  what  does  this  imply  so  for
Infinity  we  have
um  this  implies  that  the  uh  infinity
section  of
P1  uh  is  uh  a  module  over  or  an  algebra
over  uh  is  an  alge  albra  over  um  so  Norm
upper  star  of  the  structure  chief  of
infinity  yeah  one  way  to  think  about  imp
anal  section  and  then  there's  General
that  many  and  Serv  close  at  Z  and
analytic  St
that
in  the  set  of
sches  where  scheme  is  endowed  with  the
the  trivial
analytic  okay  so  let  me  try  doing  it  the
way  I  think  it's  basically  equivalent
but  let  me  let  me  try  doing  it  the  way
Peter  was  suggesting  so  let's  let's  uh
let's  abstract  a  bit  let's  move  Infinity
to  zero  and  Abstract  a  bit  do  it  for
like  it's  P1  Z  so  it  means  you  can  this
P1  Z
and  so  suppose  given  an  analytic  stack
over  A1  um  such  that  if  you  pull
back  uh  to  zero
section  uh  you  get  uh  the  empty  set  so
it  misses  the  zero  section  then  the
claim  is  that  this  uh  F  factor  is
through  uh
GM  uh  r
and  it's  part  of  a  more  General  claim
that  Peter  was  making  about  a  a  scheme
and  a  closed  sub  scheme  but  okay  scheme
with  the  with  the  kind  of  trivial
analytic  ring  analytic  kind  of  yeah
let's  say  usual  a  fine  scheme  is  the
kind  of  the  most  trivial  analytic  yes
yes
um  so
um
right
um  right  so  fact  factors  oh  so  and  the
reason  for  this  is  you  can  think  of  maps
like  this  um  well  suffice  it  to  treat
the  C  case  where  X  is  is  itself  apine  um
and  then  Maps  like  this  you  can  think  of
in  terms  of  symmetric  monoidal  functors
so  so  if  x  is  D  of  uh  I  don't  know  a
then  you  can  think  in  terms  of  the
corresponding  pullback  functor  from  D  of
a1r  uh  to  D  of
a  um  and  what  do  we  know  about  this
pullback  functor  we  know  that  it  kills
uh  uh  kills  the  structure
sheath  uh  of  the
origin  but  then  it's  just  a  purely
algebraic  fact  that  um  if  you  kill  the
structure  chief  of  the  origin  then  you
factor  through  inverting  the  the
parameter
so
so  the  structure  sheath  of  the  origin  is
just  structure  sheath  on  A1  modulo  T  the
parameter  T  so  if  you  kill  that  thing
then  you  kill  everything  that's  built
from  that  via  colimits  and  so  you  don't
see  the  difference  between  an  object  and
the  result  of  inverting  t  on  that  object
and  then  that  exactly  gives  the  factors
through  yeah
yeah  so  thanks  Peter  I  think  that's  a
much  nicer  way
of  of  saying
it
um  everyone
good
okay  uh  so  right  so  that's  the  claim  and
that  implies  that  this  map  is  well
defined  because  certainly  uh  the
multiplication  is  well  defined  on
A1  but  um
so  maybe  then  I  could  put  I  put  the  A1
here  a  priori  uh  and  a
priori  have  only  the  N  going  here  but
then  a  posteriori  if  I  require  this
diagram  to  commute  then  it  follows  that
actually  this  uh  this  lands  inside  A1
analytic  because  by  definition  that  was
the  pre-image  because  because  this  ma
factors  through  zero
Infinity
okay  all
right
uh  so  oh
man  all  right  uh  I  want  to  get  to
something  fun  yes  please  no  no  please
please
uh  so  the  fact  that  we  denoted  is  and  so
the  Norms  here  should  somehow  correspond
to  like  correspond  to  Norms  and  analytic
the  notion  of  analy  ification  um  what
yeah  so  I'm  going  to  give  some  of  the
motivation  at  the  end  so  but  let  me
finish  with  the  axiomatics  like  building
on  that  like  so  like  I  remember  last
time  like  I  think  R  SCH  said  like  Gaga
really  was  like  at  the  like  he  he  noted
down  Gaga  as  like  like  as  an  isomorphism
of  stacks  yeah  and  uh  and  someh  that  was
some  sort  of  analy  ification  so  does
that  also  like  correspond  to  a  norm  then
so  this  question  I  suggest  you  uh  keep
it  for  later  yeah  yeah  um  so  uh  yeah  um
let  me  finish  with  the  axioms  uh  1  2  3
ah  so
four  okay  so  axium  4  now  so  maybe  now's
the  a  time  to  say  a  bit  about  motivation
so  the  um  so  what  if  you  have  an
analytic  ring  what  we're  going  to  try  to
do  is  we're  going  to  try  to  say  if  you
have  an  analytic  ring  you  want  to  try  to
build  some  geometry  over  that  ring  um
but  it's  hard  if  you're  just  if  you're
just  given  an  analytic  ring  and  you
don't  know  anything  more  about  it  or  you
don't  have  any  extra  structure  on  it
it's  kind  of  hard  to  build  analytic
geometry  over  it  I  mean  basically  all
you  can  do  is  you  can  do  this  trick  of
importing  algebraic  geometry  for  an
arbitrary  analytic  ring  that's  more  or
less  all  you  know  how  to
do
um  uh  so  what  we're  going  to  be
doing  um  and  what  well  one  what's  one
measure  that  you  have  some  good  uh
analytic  geometry  it's  that  you  have
some  nice  subsets  of  the  Aline  line
um  and  nice  in  the  context  that  we're
discussing  here  means  for  example  shable
so  that  the  six  functor  formulism  works
and  then  it  kind  of  really  feels  like
you're  doing  geometry  um  in  some  more  or
less  traditional  sense
um  so  but  uh  again  on  a  general  analytic
ring  you  you  don't  know  how  to  write
down  any  uh  interesting  shable  subsets
of  the  apine  line  so  you  have  to  give
the  you  have  to  give  yourself  some  of
them  and  that's  going  of  that's  the
point  of  this  notion  of  normed  analytic
ring  we're  giving  ourselves  basically  uh
discs  of  certain  of  of  some  arbitrary
radius  inside  the  Aline  line  um  and  they
will  turn  out  to  Define  shable  subsets
of  the  apine  line  and  then  we  can  um  we
can  get  started  on  doing  geometry  that
resembles  some  sort  of  traditional
analytic  geometry  based  on  open  discs  or
closed  discs  or  what  have  you  but  you
have  to  have  this  extra  structure  on
your  base  before  you  can  get  started  on
that  game  if  you  don't  have  a  notion  of
a  norm  on  your  ring  you  can't  start
talking  about  closed  discs  and  open
discs  of  certain  radius  so  that's  what
we're  doing  right  now  but  um  we  already
have  some  sort  of  things  that  kind  of
seem  to  function  as
a  we  had  we've  already  seen  certain
versions  of  the  unit  disk  so  for  example
we  spent  a  lot  of  time  discussing  this
basic  module
p  uh  which  recall  was  this  uh  free  free
module  on  N  Union  infinity  modulo
infinity  um  and  and  of  course  you  can
base  change  it  to  any
ring  uh  any  analytic  ring  even
um  um  and  so  well  what  what  do  we  know
about  this  guy  well  we  know  it  does  have
a  map  from  R  bracket
T  so
geometrically  uh  so  let  me  just  this  is
going  to  be  not  not  good  notation  but
let's  let's  set  uh  D  equals  spec
P  so  it's  some  conversion  of  a  dis  um
and  then  Dr  will  be  the  base  change  to
R  again  your  was  a  usual  algebra
structure  right  that's  right  yeah  and
we've  we've  also  discussed  the  algebra
structure  on  P  which  makes  this  an
algebra
map
um  so  um  so  this  uh  so  D  does  map  to  the
Aline  line  by  this  uh  by  this  map  here
um  but  as  we've  also  maybe  discussed  uh
so  it's  a  but  but  it's  not  a
monomorphism  so  it's  not  really  a  subset
of  the  aine  line  so  it's  a  it's  a  proper
map  in  our  setting  so  it's  shakable  but
um  it's  not  um  it's  not  a  monomorphism
so  it's  not  really  a  subset  of  the  Aline
line  so  what  does  it  look  like  in
examples
um  uh
so  in  general
so  we  have  well  this  this  map  from  RT  to
PR  uh  this  induces  an
isomorphism  on  mod  t  to  the  N  for  all
n  so  uh  this  this  uh  this  map  which  is
not  a  monomorphism  it's  it  is  an
isomorphism  on  the  formal  neighborhood
of  the  origin  so  at  least  in  some  sense
you  should  expect  close  to  the  origin
this  is  a  monomorphism  and  then  what
happens  is  it  kind  of  blows  up  as  you
move  away  from  or  it  can  potentially
blow  up  as  you  move  away  from  the
origin
what  uh  T  goes  to  like  the  The  Rock
measure  concentrated  at  at
one
um  okay  you  use  the  multiplication  on  N
or  you  use  the  addition  on  N  to  yeah  to
give  the  multiplication  on
PR  yeah  um
so  uh  right  so  in  particular  you  get  a
factoring  like  this  so  this
um  and  so  that's  that's  generally  true
that  you  have  a  diagram  like  this  um
where  this  is  the  canonical  map  um  and
in  in  most
examples  so  this  p  r  mapping  to  r  t  is  a
monomorphism  is  an
injective  so  in  most  examples  this  lives
in  degree  zero  and  this  map  is  an
injection  so  this  is  some  kind  of
sequence
space  and
um  what  is  the
condition  that  the  sequence
satisfies
well
uh  so  in  the  aelon  category  the  which  is
the  heart  of
Dr  so  in  in  in  most  examples  this  lives
in  the  heart  this  also  in  most  examples
lives  in  the  heart  I  mean  I  guess  maybe
I  should  be  using  this  notation  but  I'm
being  a  little  bit  sloppy
here
um  yeah  and  then  this  is  just  an
injection  so  to
speak  um  so  so  in  most  examples  PR  R  is
like  the  set  of  sequences  r0  R1  um
satisfying  some  summability  condition
so  such  that  if  you  termwise  multiply  by
a  null
sequence  uh  you  get  a  summable
sequence  uh  where  the  notion  of
summability  depends  on  the  analytic  ring
structure
but  at  the  level  of  this  discussion  you
could  imagine  for  example  are  being  the
real  numbers  and  summable  means  usual
absolute  some  the  absolute  values  you
get  a  finite  number  that's  um  that's  not
actually  a  special  case  but  it's  it's
close  enough  and  it  it  serves  the
purposes  for  this  this  discussion  um  and
this  is  kind  of  just  by  the  universal
property  of  PR  that  this  is  the  the
correct  interpretation  because  PR  is
Maps  out  of  PR  to  an  to  an  m  in  Dr  these
are  supposed  to  correspond  to  null
sequences  in  M  by
construction  um  so
um  you  know  it's  the  kind  of  thing  which
when  paired  with  a  null  sequence  uh  you
get  an  element  in  m  and  so  the  idea  is
this
procedure  yeah  so  we  null  sequence  and
then  yeah
well  if  you  well  I'll  let  you  maybe
maybe  me  trying  to  explain  it  is  not  as
helpful  as
um  all
that  is  any  analytic  ring  I  but  but  this
is  not  precise  mathematics  here  yeah  but
in  this  Norm  business  r  r  is  just  a
discret  ring  in  this  Norm  well  no  no  in
the  norm  business  R  is  an  arbitrary
analytic  ring  am  m  is  then  p  is
a
okay  so  you  you  say  that  usually  it's
say  in
object  in  the  category  of  our  yeah
usually  that's  right  yeah  yeah  and  this
is  the  what  you  call  M  uh  M  oh  sure  well
I  mean  yeah  I  could  yeah  I  mean  so  so  I
was  I  was  saying  that  this  is  the
interpretation  you  get  in  practice  where
the  notion  of  summability  depends  on  the
analytic  ring  structure  and  the  way  you
see  that  this  is  the  correct
interpretation  is  by  thinking  about  what
it  means  to  map  PR  to  m
and
um  so  like  for  example  the  most  basic
thing  was  it  would  be  if  you  map  PR  to  R
triangle  um  then  that's  the  same  thing
as  giving  a  null  sequence
um  but  then  um  if  you  think  in  terms  of
what  happens  when  you  restrict  to  here
you  have  some  Co  if  you  have  some
coefficients  in  a  polinomial  so  if  it
terminates  if  you  have  zeros  after  while
then  what  you're  doing  is  you're  just
summing  the  null  sequence  times  those
things  to  get  an  element  in  our  triangle
and  then  you  imagine  that  that  Su  this
is  saying  that  that  summing  should  make
sense  for  something  which  is  not
necessarily  eventually  zero  and  so  this
is  kind  of  the  interpretation  you  should
give  that
um  right
uh  uh  and  what  so  so  and  so  if  for
example  R  is  C  with  the  gases  or  liquid
liquid  analytic  rank  structures  what  you
see  is  that
um  you  have  if  you  look  at  holomorphic
functions  um  on  the  usual  usual  ring  of
holomorphic  functions  on  the
uh  closed  unit  dis  meaning  they  overon
Converge  on  the  closed  unit  dis
um
then
um  then  every  one  of  those  satisfies
this  if  you  look  at  the  coefficients  of
the  power  series  it  will  satisfy  this
summability  property  because  your  power
Series  has  a  value  at  one  so  the  you
know  the  thing  the  sequence  of
coefficients  better  be  summable  so
you'll  get  a  map  like  this  um  uh  so  this
summability  condition
is  weaker  than  the  condition  which
defines  these  things  but  it's  a  stronger
than  the  condition  which  defines  a
holomorphic  function  on  the  usual  open
dis  um
so  uh  if  you  think  in  terms  of  your
usual
um  your  usual  complex  analysis  then
that's  where  your  p  is  sitting  it's
sitting  somewhere  between  the  Open  disc
of  radius  one  and  the  closed  disc  of
radius
one
um  and  that's  the  kind  of  behavior  we're
going  to  be  aati  in  condition  number
four  yeah  so  this  is  something  like  a
what
uh  X  exponential  decay  and  this  is
something  like  exponential  growth  and
this  is  something  like  Su  ability  so
kind  of  makes  sense  that  it's  it's  in
between  the
two  um
okay  uh  was  a  bit  of  a  long-  winded
explanation
um  what's  I  haven't  written  it  yet  I've
just  given
motivation
um  but  yeah  so  what  what  you  want  to
think  is  in  these  kinds  of  normed
analytic  ring  settings  this  well  okay
so  then  condition  four
is
uh  so  condition
four  uh  is  split  into  two  parts  well  the
the  first  part  is  that  um  if  you  have
you  have  D  mapping  to  P1  or  Dr  mapping
to  P1  R  uh  mapping  to  0  Infinity  um  via
the  norm  uh  you  want  this  to  factor
through  uh  the  map  up  to
01  that  uh  corresponds  to
this  um  saying  that  this  notion  of  unit
dis  is  sandwiched  between  closed  and
open
um  um  but  you  also  want  that  this  if  you
take  Dr  U  mapping  via  the  norm  map  to  0
infinity  and  if  if  you  pull  back  to  uh
01  uh  the
open  well  a  half  open  segment  from  0  to
one  um
then  uh  you  should  get  an  isomorphism
over  here  so  uh
so  uh  sorry
uh  yeah  so  Norm  inverse  of  01  uh  uh
sorry  ah  shoot  um  I  need  I  need  P1  to  be
in  there  thanks
yeah  oh  sorry
Norm  um  so  then  we
have
um  so  I  want  that
um  yeah  wrot  an  inclusion  there  sorry
sorry  oh  I  wrote  I  didn't  mean  to  write
an  inclusion  thank  you  yeah  thank  you
what  I  want  to  say  is  that  uh  so  this  uh
yeah
well  let's  say  this  map  is  an
isomorphism  over  the  locus  uh  uh  given
by  the  open  unit
dis  so  so  this  is  like  some  sort  of
proper  map  which  blows  up  along  the
boundary  of  the  open  unit  dis  um  but  on
the  interior  of  the  open  unit  disc  it's
an
isomorphism  so  uh  this  is  some
something  stronger  than  saying  that  you
have  a  map  like  this  um  and  the  nice
thing  about  this  stronger  condition  is
first  of  all  you  can  check  it  in
practice  and  second  of  all  it's  just  a
condition  whereas  giving  a  map  like  this
is  a  prior
structure  um  so  can  I  think  of  this  nor
edian  nor  you  can  think  of  it  as  non
archimedian  or  archimedian  at  will  we
haven't  enforced  any  compatibility  of
the  norm  with
addition
so  it's
yeah  last  last  time  yeah  so  this
is  it  is  if  you  look  for  real  numbers
for  example  right  so  the  um  so  this  was
the  so  the  the  cosmetically  I  changed
the  discussion  of  multiplicativity  from
Peter's  lecture  that's  that's  just  a
cosmetic  change  this  is  um  potentially
more  serious  Peter  did  not  mention  this
axium  last  time  and  we're  not  sure
whether  it's  um  a  consequence  of  the
other
aums  but  um  yeah  as  Peter  mentions  if  if
you're  over  a  base  which  solidifies  to
zero  so  for  example  if  you're  over  the
real  numbers  then  then  you  can  prove
this  from  the  other  axioms  but  we're  not
sure  whether  which  solidifies  to
zero  it  means  when  you  base  change  to
solid  if  you  base  change  Spec  R  to  spec
of  solid  z  uh  then  you  should  get  the
empty  analytic  stack  if  if  that
condition  is
satisfied  then  then  this  this  this
condition  follows  from  the  other
conditions  but  so  can  you  give  example
of  saying  that  solidify  to  zero  yes  the
real  number  is  solidify  to  zero  so  any
analytic  any  analytic  ring  where  the
underlying  ring  is  an  algebra  over  the
condensed  ring
R  uh  will'll  have  that
property  so  so  again  so  you  have  the
something  going  to  spec  Z  spec  the  Sol
but  all  the  the  nonar  median  seem  for
are
not  so  over  there  potentially  there  so
as  far  as  we  know  so  far  there  could  be
some  weirdo  Norms  which  don't  satisfy
this  condition  but  yeah  they're  not  any
of  the  usual  ones  the
usual  the  the  ones  that  don't  satisfy
this  unless  it's  some  really  extremely
interesting  new  mathematical  phenomenon
which  seems  unlikely  they're  probably
just  very  pathological  we're  taking  this
as  an  axium  okay  and  then  you  could  as
the  question  whether  it  follows  from  the
other  AXS  and
um  okay
so
um  so  what  is  a  so  what  is  a  norm  again
so  what  what  is  a  norm  giving  you  so  if
you  have  a  norm  on  an  analy  so  these  are
the
axioms
um  so  this  gives  you  so  again  in  terms
of  the  discussion  of  maps  to  uh  compact
house  doorf  spaces  um  this  gives  you  for
every  for  example  uh
for  you  get  Norm  upper  star  of  the
constant  sheath  on  z
r
um  so  for  if  you  take
R  non-  negative  real  number  uh  then  you
get  this  item  potent
algebra  uh  in  uh  D  of
p1r
um  uh  and  this  course  this  uh  the
interpretation  of  this  should  be  overon
convergent  uh
functions  on  the  closed  unit
disk  uh  of  radius
r
and  the  overon  convergence  it  it  has  to
be  overon  convergent  there's  no  other
possibility  because  note  that  the
constant  shei  z0
R  uh  is  the  filtered  Co  limit  of  the
constant  shief  Zer  uh  R  Prime  for  all  R
Prime  bigger  than  R  um  under  the
Restriction  Maps  so  the  overon
convergence  has  to  be  built  in  like  if
you  imagined  that  you  were  sending  this
to  some  some  version  of  functions  on
some  version  of  the  unit  disc  or  unit
disc  sorry  uh  I  mean  disc  of  radius  R
some  version  of  the  disc  of  radius  R
then  that  version  would  be  forced  to  be
the  over  convergent  one  um  because  of
this  property
here
uh  and  because  Co  and  because  yes
because  it's  a  pullback  functor  so  it
commutes  with  co-
limits
um  all  right  um  and  okay  so  and  actually
the  data  of  these  guys  uh  given  the
axioms  this  determines
n  because  it's  very  easy  to  classify  the
closed  subsets  of  the  of  the  of  this
interval  and  um  you  only  need  to  know
about  things  less  than  or  equal  to
something  and  things  bigger  than  or
equal  to  something  but  the  axum  about
inversion  gives  you  one  in  terms  of  the
other  so  you  just  have  to  give  these
things  uh  subject  to  some  some  simple
conditions  in  order  to  specify  a
norm  Okay  so
the  next
topic  is  classifying
norms  and  uh  I  mean  the  the  uh  I  don't
mean  like  as  in  classifying  space  or
whatever  I  mean  how  to  classify  Norms  on
a  given  analytic
ring  uh  yeah  so  let  me  start  with  one
aemma
here
question  that  the
usual  yes  and  yes  there  is
yeah
um  right  so  uh  LMA  so  so  given  a  norm  on
an  analytic  ring  uh  so  a  norm
maybe  just  say  just  said  what  data  you
have  to  give  to  Def  find  nor  and  you  can
just  check  that  the  usual  algebra  either
geometry
orry  solid  the  complex  numbers  with  they
do  satisfy  SE  so  this  way  you  can
produce  by  hand  now  such  Norms  over
QP
right  sorry  but  what  does  over
convergent  mean  uh  it  means  so  an  overon
convergent  function  on  a  disc  of  radius
R  means  a  function  which  converges  on
some  dis  centered  at  the  same  point  with
larger
radius  okay  yeah  so  it  extends  to  a
function  on  an  open  neighborhood  of  the
the  closed
dis
um
right  um  yeah  don't  be  shy  about  asking
such  questions  about  terminology  because
I  know
well  anyway  um  so  right  so  given  a  norm
then  you  can  look  at  the  the  locus  of
um  of  um  the  locus  where  the  norm  is
strictly  between  zero  and  one  uh  and
this  projects  down  to  Spec  R  uh  and  I'll
say  that  this  is  a
cover  in  the  in  the  gro  topology  we've
been
considering
so
IE  if  we're  willing  to  work
locally  uh  we  can
assume  uh  we  have  an  element  you  can
call  Q  uh  in  the  underlying  ring  such
that  uh  Norm  of  Q  uh  is  in  so  to
speak  lies  strictly  between  zero  and
one  so  is  it  the  case  that  when  you  take
any  real
number  yes  just  yes  yes  yeah  yeah  so
actually  so  I'll  make  a  stronger  claim
in  the  proof  so  in
fact  uh  if  you  take  Norm  inverse  of
say2  uh
specr  this  is  a
cover  and  that's  a  stronger  claim  um  is
a  cover  in  the  sh  in  this  gr  top  the
only  gr  de  topology  we've  put  on  on  our
category  okay  but  this  is  the  strong  of
the
is
no  no  it's  a  cover  in  the  gr  de  topology
that  it  can  be  refined  to  yeah  in  fact
this  is  not  necessarily
apine
okay  cover  the  can  be  refined  to  yes  can
be
ref
yeah
um  so  but  that  said  for  let  me  so  um  let
me  for  Simplicity  assume  that  this  is
apine  uh  so  there's  an  argument  for  uh
getting  the  conclusion  anyway  um
based  on  you  again  again  resolving  this
by  a  profinite  set  um  and  using  that
descend  ability  that  was  proved  earlier
so  but  um  let  me  I'm  going  to  skip  over
that  part  of  the  argument  so  for
Simplicity  uh
assume  uh  so  that  this  Norm  upper  star
of  the  constant  sheath  on  1/2  is
connective  um  so  that  this  Norm  inverse
of
12  uh  is
apine  and  it  is  uh  and  it's  actually
proper  over
A1  it's  just  given  by  this  this  algebra
um  and  hence
over  hence  over  Spec
R  so  in  that  situation  this  is  a  both
apine  and  sorry  this  is  apine  and  it's
proper  over  Spec  R  so  by  this  descend
ability  Criterion  we  need  that  uh  our
triangle  lies  in  is  generated  by  this
algebra
so
um  but  what  I
claim  uh  our  triangle  is  actually  a
retract  directly  a  retract  without  any
cones
and  so  implicitly  I'm  viewing  this  over
the  apine  line  instead  of  the  projective
line  and  then  I  mean  I  take  the
forgetful  functor  from  D  of  A1  to  D  of  R
um  when  I  when  I  write  this
here  or  in  other  words  this  is  this  is
descendible  of  index  zero  so  it's  like
The  Descent  is  quite  direct  you  just
take  a
retract
um  then  the  intuition  behind  this  should
be  clear  uh  this  is  some  version  of  a
lant  Series  ring  it's  a  lant  Series  ring
and  variable  t  with  some  convergence
condition  um  and  you  can  just  pick  out
say  the  zeroth  coefficient  of  your
laurant  series  that  gives  a  linear  map
not  an  algebra  map  but  a  linear  map
which  splits  the  unit  um  that's  the  idea
um  but  we  have  to  make  sure  it's  it
holds  in  this  completely  abstract
setting  here  so  what  you  do  is  you  take
uh  you  can  take  this  structure  chief  of
P1
um  uh  and  then  you  can  take  the  uh  Norm
upper  star  of  Z  well  yeah  let
me  well
okay
um  so  this  is  a
pullback  in  D  of
P1
um  just  because  the  constant  sheath  on
this  interval  is  the  the  pullback  of  the
constant  chief  on  zero  one2  uh  constant
chief  on  1/2  one  glued  along  constant
Chief  at  the  point
there  um  and  then  we  apply  uh
uh  push  forward  to
d  uh  to  D  of
R  then  the  chology  of  the  structure  chif
on  P1  is  just  it's  just  it's  just  R
concentrated  in  degree  Z  so  what  we  get
we're  R
triangle  um  and  then  I'm  going  to  use
the  same  notation  or  so  the  same  just
the  these  ones  live  over  A1  or  this  one
doesn't  but  well  whatever  the  just  just
push  it  well  it  lives  on  the  other  A1
but  we  can  just  push  it  forward
um  copy  the  same  guys
over  what's
that  first  do  a  twist  byus  to  get  degree
one  why  do  I  want  chology  in  degree
one  because  you  want  a  map
from  that's  that's  not  the  way  I'm  going
to  do  the  argument  so  um  yeah  so  what
I'm  going  to  say  is  that  um  there's  so
this  is  then  a  pullback  in  Dr  uh  but
it's  also  then  well  pullbacks  are  the
same  thing  as  pushouts  so  it's  a  push
out  in  Dr  um  so  if  I  want  to  make  a  map
from  here  to  to  a  certain  place  for
example  to  our  triangle  I  make  a  map
here  and  I  make  a  map  here  and  I  make
sure  they  agree  there  um  so  what  I  do  is
here  I  take  evaluation  at
Zero  from  our  Tri  uh  which  goes  to  R
triangle  here  I  take  evaluation  at
Infinity  which  also  goes  to  our
triangle  and  then  they  clearly  uh  just
both  both  give  the  identity  map  when  you
restrict  to  R  triangle  so  this  gives  the
map  here  which  when  you  restrict  back  to
here  is  the  identity  um  so  that  uh  that
produces  the  desired
retraction  kind
of
yeah  and  what  about  the  this
connectivity  a  restriction  oh  yeah  so
then  you  what  you  have  to  do  is  you  have
to  see  that  so  there's  then  you  you  have
some  it's  not  if  it's  not  necessarily
apine  you  still  have  some  proper  cover
by  something
apine
um  and  then  um  and  not  not  just  proper
but  descendible  so  so  then  the  unit  here
is  generated  by  the  image  of  this  um  and
it  follows  then  that  the  UN  by  combining
the  two  generation  claims  the  unit  here
is  generated  by  the  image  here  and  that
provides  the  um  that  provides  the
desired  refinement  by  a  street  cover  of
a  finds  so  you  take  any  any  proper  map
to  here  which  makes  the  thing  connective
which  I  I  I  explained  why  exists  then
then  this  map  will  be  a  street  cover
because  because  the  push  forward  will
generate  the  unit  because  the  analogous
claim  is  true  both  here  and  here  ah  and
the  in  the  first  Arrow  it  is  true
because  of  of  the  argument  gave  the
first  half  of  the  class  Yeah  by  this
this  descend  ability  of  Canter  set
mapping  to  the  unit
interval
which  was  the
the  uh  but  n  inverse  12
is  where  is  the  inter  no  ah  it's
a  you  apply  the  you  cover  zero  Infinity
like  you  did  and  you  look  at  what  it
gives  in  this
uh
no  sorry  you  want  to  get  a  fine  things
you  said  that  n  inverse  one  up  is  not  a
fine  in  your  terminology  in  general  yeah
so  how  do  you  make  it  find  by  it's  not  a
pullback  on  the  base  it's  not  it's  not
by  working  locally  on  the  base  it's  by
working  locally  on  on  P1  or  A1  it's  by
working  locally  on  A1  using  the  previous
argument  so  A1  maps  to  zero  Infinity  so
then  there  exists  a  proper  cover  of  A1
by  an  apine  where  the  that  map  factors
for  the  caner
set  and  then  that  will  give  you  the
desired  connectivity  and  also  descend
ability  there
youil  yeah  and  you  have  descend  ability
both  here  and  then  by  that  argument  here
and  then  this  is  aine  n  proper
and
yeah  okay
um  right  so  so  now  we  now  now  let's  give
ourselves  this  extra  data  which  we've
assured  can  be  gotten  locally  so  given  a
norm  uh  p1r  Norm  0  infinity  and  a  q  uh
in  uh  such
that
so  the
claim  is  that
um  so  the  well  we  can  think  of  Q  as  as
as  being  given  by  a  um  um  a  map
from
uh  Z  bracket  t  uh  to  our
triangle  or  to  R  why  not  uh  so  so  T  goes
to
Q  uh  claim  this  factors
uniquely  through  this  gaseous  bace  ring
uh  I  let  yeah  so
well  let  me  not  let  me  not  name  the  the
map  let  me  call  the  variable  Q  here  and
um  so  a
a  variable  which  the  norm  makes  look
like  it's  topologically  nil
potent  um  actually  has  to
um  come  from  this  uh  this  gaseous  base
here  so  the  proof  is  so  let  me  start
with  the  uniqueness  which  has  nothing  to
do  with  Norms  it's  actually  a  claim  that
so  spec  the  claim  is  that  spec  z  q  hat
plus  or  minus  one
gas  uh  which  maps  to  the  aine  line  uh
over  Z
say  uh  this  is  a
monomorphism
um  yeah  so  so  that's  the  that's  the
uniqueness  claim  or  a  strong  form  of  the
uniqueness  claim  there's
a  it's  just  there's  a  contractable  space
of  factorings  if  one
exists  um
uh  this  is  not  obvious  from  the
definition  of  the  gaseous  base  ring
because  by  definition  of  the  gashes  base
ring  we  took  this  uh  zq  hat  which  was
just  n  uh  which  was  just  P  um  and  then
we  inverted  q  and  then  we  enforced  this
Q  being
gaseous  um  and  this  this  is  not  item
potent
so  not  item
potent  so  enforcing  the  condition  of
being  gas  what  defines  a  monomorphism  on
the  level  of  analytic  Stacks  because
it's  just  some  quotient  of  categories
but  because  this  is  not  item  potent  it's
really  not  clear  from  that  description
that  that  the  composite  map  all  the  way
to  A1  should  be  item  potent  because  it
seems  at  first  you  have  to  choose
something  which  is  extra  structure
namely  a  a  proof  that  Q  is  topologically
nil  potent  so  to
speak  um  but  you  can  do  it  in  the  other
way
instead  but
uh
uh  you  can  also  think  of  this  as  modules
over  P  or  P  modules  in  you  just  take  a
polinomial  ring  in  one
generator  uh  and  make  that  generator
gaseous  uh  oh  and  then  maybe  I  have  to
invert  Q  too  well
okay
just
um
so
and  then  the  claim  is
that  is  that  P  is  item
potent  uh  in  in  D  of  Z  Q  Plus  orus  One
gas  um  and  this  actually  so  we  want  to
check  that  P1  or  P  equals  P  that's
actually  a  claim  that  happens  after  base
change  to  P  so  to  speak  so  it  actually
reduces  to  a  calculation  here
uh  that  if  you  take  this  p  over  z  q  hat
Plus  or  Min  -1
gas  um  now  we  have  both  a  q  variable  and
then  we  have  a  t  variable  say  coming
from  the  coming  from  the  P  here  um
so  uh  uh  that  if  you  take  this  and  you
mod  out  by  T  minus  Q  um  you  just  get  the
underlying  ring
so  yes
yes  you  should  still  write  guas  because
completion  changes
the  uh  oh  okay  sure  yeah  yeah  I  mean
yeah  the  underlying  ring  of  the  analytic
ring  structure  and  Peter  described  this
uh
this  free  module  and  if  you  use  that
description  you  will  find  yourself  is
you're  just  you  have  to  check  a  short
exact  sequence  and  you  can  do
it
um  so  that's  the
uniqueness
um  as  for  the
Existence
um
uh  well  let  me  just  say  it  in  words  it's
it's  um  fairly  Elementary  from  the
definition  of  the  norm  so  because  our
Norm  is  contained  in  here  uh  this  means
that  we're  away  from  so
existence  so  Norm  of  Q  contain  in  01
means  we're  away
from  Norm  inverse  of  1
infinity  um  and  that  in  particular  means
that  we're  away
from  uh  the
opposite  the  the  well  let's  say  the  the
D  sitting  at
Infinity  so  if  you  take  this  D  the  thing
that  was  spec  of  p  and  you  translate  it
to  Infinity  on  P1  then  that's  going  to
be  uh  contained  in  this  Locus
um  but  then  saying  that  you're  away  from
the  D  setting  at  Infinity  is  exactly  the
same  thing  as  saying  that  Q  is
gaseous  um  but  then  on  the  other  hand
Norm  of  Q  uh  contained  in  uh  Clos  inter
from  01  because  of  the  axium  I'm  I'm
referring  now  to  the  axium  4  about  the
placement  of  D  in  this  uh  in  under  this
Norm  uh  this  implies  that  uh  Q  uh  comes
from  d
so  uh  Q  is  topologically  nil
potent  so  it  comes  from
D  so  D  was  this  spec  of  P  mapping  to  the
apine  line  it  therefore  maps  to
P1  and  then  you  can  pull  back  that  map
uh  under  the  uh  automorphism  of  P1  which
is  the  inversion  map  and  you  get
something  abstractly  isomorphic  to  D  but
the  structure  map  to  P1  is
different  um
yeah  yeah  yeah
yeah
okay  I'm  almost  done  I  apologize  for
going  a  bit  over
um
so  so  now  let  me  State  the  main  result
about
what  is  comes  from  D
um  comes  from  d  means  that  um  it
promotes  to  a  p
module  so  I  the  I  mean  the  or  the
well  what  I  mean  is  that  Q  so  Q  is  a  no
no  what  I  mean  is  a  so  Q  is  a  section  of
this
projection  um  well  it's  a  section  of
this
projection  and  the  claim  is  that  it  fact
through
D  and  the  reason  is  is  that  it  factors
through  the  norm  inverse  of  this  and  one
of  the  axioms  guarantees  that  that
implies  that  it  factors  through  D  yeah
yeah  have  you  stated  what  the
classification  is  of  I'm  doing  it  right
now  so
theorem  uh  so  for  given  an  analytic  ring
R  um  there  is  a  bje  natural  bje  or  the
the
map  um  from  the  set  of  pairs  well  not
not  really  set  but  okay
n  uh  this  is  a  norm  on
R  and  then  a  a  q  uh  and  in
R  uh  such  that  Norm  of  Q  is  contain  in
01  so  up  to  a  a  cover  of  the  of  Spec  R
um  this  is  you  know  we  can  we  can  assume
we  have  this  data  so  up  to  some  cover
we're  classifying  all
Norms  um  so  but  then  from  each  of  these
I've  just  uh  I've  just
said  uh  you  get  a  map  from  Spec
R  to  to  well  first  of  all  uh  you  get  a
map  to  spec  of  Z  Q  Plus  orus  One
liquid  uh  but  then  you  also  get  a  map  to
the  open  interval  open  interval  from  0
to
one
um  yes  it's  not  oh  thank  you  gasas  yeah
thank  you
uh  and  this  is  by  recording  q  and  the
norm  of
Q
um
uh  this  map  is  an
isomorphism
I.E  giving  a  norm  on  an  analytic  ring
plus  an  element  of  Norm  strictly  between
zero  and  one  is  the  same  thing  as
specifying  what  the  norm  of  that  element
should  be  which  is
which  is  this  function  here  and
requiring  that  that  Q  actually  come  from
this  this  gases  theory  that  we  discussed
earlier  and  both  sides  are  sets  no
almost  well  except  for  the  annoying  fact
that  A1  is  not  a  set
so
a  yeah  so  so  the  the
norms  and  Q  the  first  the  first  thing  we
said  is  a  is  the  set  the  nors  really  is
a  set  because  there  is  no  and  the  Q  of
course  is  something  in  some  set  no  no  no
the  Q  is  not  in  a  set  because  this  could
be
derived  so  so  both  of  these  map  to
A1  and  like  sort  of  the  fibers  of  these
maps  are  sets  so  there  there're  as  much
sets  as  each  other  um  but  they're  not
each  individually
sets  ah  so  Q  is  a
derived
ah  okay  is  but  you  still  think  of  Q  as
as  choosing  something  in  the  Zero  part
of  simpli  uh  Zero  part  yeah  yeah  yeah
yeah  implicitly  the  higher  s  the  higher
maps  are  there  also  yeah  so  it's  not  you
have  to  construct  them  as  well  they  come
automatically  from  the  for  I  don't
know  so  let  me  give  the  proof  um  so  uh
how  non  acate
L  not  now  not  now  uh  so  the  proof  uh  so
um  so  after  a  descendible  cover  after  a
cover  uh  we  can
assume  uh  Q  admits  all  power
admits  and
compatible  nth  roots
uh  for  all  n  so  IE  so  this  is  because
the  uh  this  is  descendible
so  it's  countable  in
fpqc  um  so  we're  free  to  because  both
things  are  sheaves  uh  in  our  topology
both  sides  are  sheaves  in  our  topology
uh  we're  free  to  work  locally  so  we're
free  to  assume  U  that  that  the  Q  has  all
all  nth  Roots
um  then
um  then
um  uh  then  the  claim  is  that  so  then  so
then  so
write  uh
DQ  uh  for  the
pullback  so  we  have  D  going  to  P1  or
let's  say  A1
uh  and  then  we  have  multiplication  by  Q
on  A1  um  and  then  we  have  DQ
here
um  so  for  this  would  this  makes  sense
for  any  q  and
GM
uh
um  note  that
if  if  um  Q  equals
Q  maybe  I  maybe  sorry  I  apologize  I
apologize  maybe  I  say  d  alpha  for  Alpha
and  GM  but  note  that  if  Alpha  equals  Q  *
beta  or  Q  is  uh  top  ically  nil
potent  then  we  get  a
map  uh  from  d
alpha  to  and  I  have  to  get  it  going  the
right  direction  so  d  alpha  to  D  beta  is
one  of  the  two  which  is  also  induced  by
multiplication  by  Q  so  here  use  that
this  p  is  a  hop
algebra  encoding  multiplication
so  um
um  right  uh  I  may  have  gotten  the
ordering
wrong  so  this  is  uh  this  is  some  disc
now
um  this  is  some  disc  of  radius  Q  inverse
or  some  version  of  a  disc  of  radius  Q
inverse  it's  not  the  correct  one  because
it  kind  of  it's  not  a  monomorphism  and
it  blows  up  along  the  boundary  but  um
but  then  we  can  fix  that
um  so  then  so  DQ  so  or  d  alpha  sorry
gives  uh  a  disc  of  radius  a  kind  of  disc
of  radius
Alpha  of  radius  Alpha  inverse
maybe  um  and  the  the  the  maps
above  uh  give  give  the  inclusions
between  such
discs
what  about  this
discuss  I'm
sorry  or  the  identity  so  say  what  do  you
mean  the
identity  you
have  to  A1  right
yeah  a  bu  of
want
consider  yeah  I  think  I  I  think  I'm
saying  the  correct  thing  but  um
yeah  um  okay  but  now  we  apply  this  apply
this  uh  to  like  Alpha  equals  Q  to  the  uh
you  know  m/  n  so  m/  n  uh  in  the  rational
numbers  and  we  get  dis
get  discs  uh  of  R  of
radius  uh  nor  of  Q  to  the  m  ah  sorry  so
sorry  we  can  we  can  assume  sorry  I
should  the  first  thing  I  should  have
said  is  we  can  freely  assume  that
the  that  the  norm  of  Q  is  constant  equal
to  12  by  rescaling  the  norm
by  I  I  feel  like  I'm  not  I  Peter  what
just  happened  this  is  crazy  I  I  I
literally  just  repeated  your  lecture
with  more  detail  and  I  didn't  get  any
farther  I  thought  I  was  going  to  get
farther  um  I'm  sorry  for  that  um  so  we
have  we  have  to  do  a  better  job  with  the
organization  I  don't  know  um  but  okay  so
you  get  discs  of  this  radius  and  a
inclusions  between
them
uh
um  but  then  you  uh  you  do  this  over
convergence  so  uh  then  you  use  these  to
make  the  over  convergent
versions  um  for  an  arbitrary  real  number
you  can  look  at  all  the  rational  numbers
bigger  than  it  you  have  these  kind  of
fake  discs  of  that  rational  radius  and
then  in  the  co-limit  all  these  problems
about  blowing  up  along  the  boundary
disappear  and  it  it  turns  into  the  thing
which  has  to  be
specified  uh  if  you're  given  a  norm  on  R
so  when  you  when  you  make  the  make  is
over  convergent  it  will  be  forced  to  be
equal  to  the  thing  that  comes  from  a
norm  if  you  have  a  norm  so  that's  more
or  less  an  argument  why  this  is  uh  a
monomorphism  and  then  if  you  want  to
check  that  it's  a  bje  you  just  have  to
produce  uh  a  norm  on  here  and  you  do  it
by  following  this  this  procedure  or  Norm
on  here  and  you  can  even  adjoin  all  the
roots  that  you  want  um  and  then  you  do
it  by  following  this  procedure  so  uh
yeah  so  you  take  this  fake  disc  you
translate  it  around  by  multiple  by
powers  of  Q  and  then  you  build  the
inclusion  Maps  between  those  discs  uh  I
don't  mean  translate  I  mean  expand  by
multiplication  you  build  the  inclusions
between  those  discs  and  then  you  make
them  over  convergent  and  then  you  check
that  you  have  all  the  axioms  so  uh  yes
sorry  for  going  over  time  thanks  for
paying  attention  so  in  this  inclusion  of
this
using  we  also  add  an  analog  and  discuss
ruber  rings  of  to  power
wounded  oh  yeah  yes  yes  if  Q  is  power
bounded  yes  does  it  still  follow  that
you  have  an  inclusion  or  not  maybe  you
need  the  to  work  in
more  solid  cont  there  but  no  it's  yeah
it's  a  good  point  um  yeah  it's  not  I
mean  it's  not  entirely  clear  in  this  in
this  general  setting  uh  what  the  class
of  Q  for  which  you  get  um  such  a  map  is
yeah  let  me
see
because  I  mean  for  example  this  argument
here  doesn't  prove  that  you  have  an
identity  map  when  when  when  like  I
wasn't  allowed  to  take  Q  equals  1  here
but  but  obviously  you  can  take  Q  equals
1  so  I  obviously  haven't  made  the  um  the
best  possible  claim  but  um  yeah  but  for
example  if  Q  is  a  root  of
unity  okay  for  roots  of  unity  we  know
that  the  norm  is  one  I  guess  we  we  said
it  from  plusus  one  I  did  do  do  we  know
it  for  did  skip  a  step  that  you're
exactly  pointing  at  which  is  that  let's
say  we've  said  that  Norm  of  Q  which  is  a
map  from  Spec  R  uh  to  0  Infinity  let's
say  that  we've  arranged  that  this  is  1
12  then  the  claim  is  that  if  you  have  a
q1  Over  N  then  its  Norm  is  has  to  be2  to
the  N
yeah  so  one  inclusion  is  obvious  uh  by
just  writing  down  the  diagram  with  the
nth  powers
and  then  you  can  prove  the  compliment
you  can  prove  the  other  inclusion  by
passing  to  the  complimentary  opens  and
running  the  same  argument  there
um  what's
that  oh  yeah  whatever  the  right  thing  to
write  down  is  yeah  thanks  yeah
yeah  I  mean  in  both  real  numbers  you
canally  right  yeah  but  you  have
to  but  I  do  think  you  need  to  invoke
this  fact  that  they're  closed  and  the
open  are  the  complement  of  each  other
to  like  with  with  just  with  writing  down
diagrams  you  only  prove  one  inclusion
but  you  have  the  nend  SP  up  on  P1  I
guess  and  then  you  prove  that  the
diagram  commute  because  you  use  the  both
ala  pces  yeah  and  then  you  you  consider
the  pullbacks  and  get  it  directly  but
I'm  not  sure  this  would
work  does  it  is  it
oh  because  this  is  a  terminal  object  or
I
don't  well  anyway  it's
true  well  okay  you're  probably  right  but
let's  let's
try
okay  Norm
Norm  maybe  you're  right  yeah  I  mean  I
was  I  was  thinking  about  yeah  I  don't
know
yeah  it  keeps  the  one  intersection
upstairs  yeah  it  goes  to
yeah  so  this  over  a
half  but  then  it  take  a  on  the  right  of
a  half  since  is  just  yeah  that's  true
that's  true  yeah  yeah  yeah  yeah  yeah
okay  um  yeah  okay  good  so  there's  no  no
subtlety  at  all  but  you  just  directly
check  that  the  N  through  of  Q  has  to
have  the
norm  uh  given  by  the  nth  root  of  the
norm  of  Q
okay  yeah  that  that  implies  in
particular  that  roots  of  unity  go  to  one
yeah
okay  yeah  see  you  on
Friday
so  what's  the  intuition
for
um  like  not  being  able  to  do  geometry
unless  you  choose  a  n  no  I  I  don't  want
to  make  such  a  strong  claim  but  the  I
mean  if  you  want  to  do  some  geometry
what  would  happen  uh  like  if  you  if  you
didn't  fix  wait  but  can  I  try  to  answer
the  question  cuz  yeah  so  cuz  what  I
meant  was  CU  I  have  to  clarify  what  I
meant  so  that  if  you  want  to  define  a
notion  of  geometry  that  somewhat
resembles  usual  complex  analytic
geometry  which  is  in  the  end  based  on
open  discs  closed  discs  and  so  on  then
you  need  to  have  something  that  measures
the  size  of  a  radius  of  a  disc  and
that's  that's  what  the  norm  does  but  but
isn't  it  kind  of  like  packed  already
when  you  fix  the  base  field  or  like  it
will  change  or  are  there  like  is  there
more  than  one  complex  analytic  geometry
based  on  the  norm  like  basically  I  think
this  is  related  to  the  question  about
the  equivalences  like  equivalent  is  up
to  like  the  geometry  that  will  arise
or  I  mean  you  could  CH  I
mean  you  could  you  could  choose
different  Norms  I  mean  obviously  there's
a  conventional  Norm  on  the  complex
numbers  for  example
but  but  the  fact  that  there's  more  than
one  at  least  in  principle  would  mean
that  it's  it's  not  the  usual  complex
analytic  I  mean  you  get  something
basically  equivalent  so  I  mean  you  could
scale  the  norm  is  something  you  can  do
like  scale  the  norm  by  some  Alpha  it's
you  you  get  basically  the  same  geometry
so  actually  one  thing  we're  going  to  do
is  we're  going  to  mod  out  by  at  some
point  we're  going  to  mod  out  by  these
sort  of  exponential  rescalings  of  the
Norms  um  when  you  do  that  you  can't  talk
about  a  disc  of  a  fixed  radius
but  but  still  many  other  things  work
okay  yeah  we  disc  this  many  times  so  if
you  have  AUD  uni  you  choose  let's  say
absolute  Val  Z  and  one  then  you  get  a
norm  on  this  and  in  particular  in  each
residue  field  you  get  a  norm  a  non
archimedian  Norm  yes  and  this  this  is
like  choosing  this  collection  ofan  yes
yes  so  it  is  not  at  all  the  same  as  like
so  so  let  say  for  another  feel  is
related  to  the  previous  National  Norm
but  for  a  more  global  object  it  is  very
far  from  like  the  bovich  space  on  just
Cho  one  of  quite  different  a  way  to
uniform  think  of  nor  this  is  the  inition
yes  and  so  you  and  and  presumably  what
they  said  is  an  equivalence  to  take  Uber
Rings  the  national  norms  and  your  sense
is  the  same  as  a  way  to  of  course  you
can  should  it  continuous  yeah  yeah
exactly  continu  on  the  back  of  rescale  a
given  one  so  it  is  a  torso  under  scaling
of  the  fix  one  that's  precisely  correct
and  it  it  sort  of  uh  sort  of  follows
from  this  uh  ah  okay  this  okay  so  in
this  example  this  is  what  you  get
exactly  and  of  course  then  you  can  uh  uh
and  of  course  for  other  analytic  Uber
just  locally  you  can  do  the  I  mean  you
can  and  for  nonanalytic  Uber
FS  this  is  something  different  well  it's
kind  of  it's  ruled  out  actually  I  mean
you  have  to  be  Tate  you  have  to  be
analytic  I  mean  we  we  were  saying  Tate
instead  of  analytic  in  this  class  but
you  have  to  be  Tate  by  this  result  if
you  have  a  norm  I  mean  then
locally  for  for  so  for  discreet  guys  you
are  not  having  any  non-rival  any  of
exactly  that's  right  but  what  we're
going  to  see  is  that  if  you  if  you  mod
up  by  rescalings  then  there's  a  way  to
there's  a  way  to  extend  this  so  when  you
mod  up  by  rescalings  on  so  Norms  form  an
analytic  stack  by  the  way
so  like  it's  a  it's  a  it's  a  it  fits  the
definition  of  an  analytic  stack  so
sending  r  or  Spec  R  to  the  set  of  norms
on  R  that's  an  analytic  stack  uh  and  you
have  to  put  covered  by  S  yeah  I  just  did
yeah  okay  okay  okay
yeah  okay  so
yes
and  but  and  uh  it  is  such  that  like
discreet  Huber  things  can't  can't  map  to
that  stack  but
um  but  there's  an  enlargement  of  the
stack  which  also  uh  accommodates  solid  Z
and  anything  living  over  solid  Z  and  so
on  which  we  we'll  probably  discuss  um  so
there  was  this  thing  like  in  the  theory
of  diamonds  and  so  that
that  you  want
to  get  from  an  analy  like  in  the  pic
setting  you  get  a  diamond  from
analytic  but  if  you  have  a  non  analytic
still  you  can  look  at  maps  of  of
antic  a  and  get  and  then  get  that  kind
of  there  was  this
uh  non  quas  I  forgot  now  how  maybe  it's
not  related  to  this  but  somehow  looks  I
I  I  I'm  not  sure  what  you're  referring
to  actually  because  there  was
a  diamond  only  allow  taste  test  object
so  even  if  you  start  with  theen  one  I
still  only  remember  how  tast
things  than  okay  it's
a
not  just  Gally  using  this  of  restricting
nor  could  you  repeat  that  comment  Peter
the  sound  wasn't  so
great
um  yeah  okay  I'm  not  sure  if  you  hear  me
better
absolutely  uh  I  was  saying  that  uh  in  my
work  on  diamonds
I  um  like  I  only  look  at  how  t  ring  so
per  pict  take  ring  snap  to
something  and  even  if  I  start  with  a
discret  ring  I  only  like  remember  these
kinds
of  things  and
so  using  this  notion  of  nor  analytic
ring  uh  one  could  now  similarly  restrict
to  non  analytic  rings  and  look  at  things
only  from  their  perspective  uh  which  may
or  may  not  be  something  want
do  but
it's  kind  of  analogous
right  yeah  so  like  yeah  so  I  guess  in
this  diamond  setting  Peter  proved  some
results  saying  that  even  for  like  a
discrete  ring  or  something  like  the
knowledge  of  how  it  maps  to  perfectoid
Rings  gives  you  maybe  quite  a  bit  of
knowledge  about  the  discrete  ring
there's  a  similar  phenomenon  here  where
actually  yeah  you  can  actually  use
normed  analytic  rings  to  you  can  even
yeah  you  can  basically  recover
everything  just  from  the  funter  of
points  on  normed  analytic  rings  but
yeah
\end{unfinished}