% !TeX root = AnalyticStacks.tex

\section{\ufs Analytic stacks (Scholze)}

\url{https://www.youtube.com/watch?v=T9XhPCI8828&list=PLx5f8IelFRgGmu6gmL-Kf_Rl_6Mm7juZO}
\renewcommand{\yt}[2]{\href{https://www.youtube.com/watch?v=T9XhPCI8828&list=PLx5f8IelFRgGmu6gmL-Kf_Rl_6Mm7juZO&t=#1}{#2}}
\vspace{1em}

\begin{unfinished}{0:00}
  e
so  today
finally  uh  we  will  Define  what  analytic
STS
are
and  in  fact  it's  not  so  difficult
so  recall  that  we  have  this  category  of
analytic
R  uh  so  this  was  actually  is  just  to
mention  at  one  time  it's  actually
presentable  I  think
so  that's  prove  that  it  has  all  Co  in  it
and  actually  it's  generated
by  a  said  worth  of  the  under
all  and  so  we  can  make  the  following
definition  and  I  don't  expect
is
and  accessible
function  so  the  word  accessible  is  just
there  to  deal  with  some  SE  issues
um  from  analytic  ring
or  FN  Analytics  spases
um
toward  so  recall  yeah  so  word  acceptable
here  theis  there  will  be  a  condition  in
just  a  second  so  this  means  that  it
commutes
with  how  I  filter  Co  image  for
sufficiently
large
um  it's  such  a  f  such  that  uh  so  Chief
conditions  um  and  a  preze  way  I  want  to
phrase  the  conditions  as
follows
uh  sorry
uh
um  see  we  have
some  UNP  a  and  then  some  hyper  cover  UNP
a  boards
um  uh  for  which
uh  of
a  y  up  the  streak  is  the  spe  limit  of
the  use  of  all  these
terms
um  uh  also  X  of  a  that's  isomorphically
to
so  by  the
way  I  write  unspec  where  Dustin  just
wrote  spec  I  don't  know  I  don't  think
we've  settled  on  a  final
notations  just  not  to  get  it  confused
with  a  usual  spec  let  me  write  onpc
um  so  uh  this  is  some  form  of  descent
that's  somewh  strictly  between
so  can  I  ask  just  a  minor  technical
question  concerning  the  notion  of  hyper
cover  so  in  usual  algebraic
geometry  uh  do  do  do  you  hear  me  no  is
it  okay  is  the  green  they  hear  I  ask  him
if  he  hears  me  do  you  hear  me  yes  I  hear
you  yes  okay  okay  so  let  us  consider  the
case  in  usual  algebraic  geometry  when
for  example  you  have  like  fpqc  to  or
something  like  this  and  then  you  you
there  could  be  two  meanings  of  fpqc  I
recover  so  one  meaning  is  that  the  map
from  x0  to  X  and  the  map  from  xn  to  co
skeleton  are
fpqc  and  the  other  meaning  is  that  there
are  coverings  for  the  fpqc  topology
which  is  H  so  sometimes  it  is  causes
some  concern  what  what  is  the  meaning  so
in  in  our  context  you  use  the  the  strict
meaning  I  suppose  that  everything  I  mean
you  yes  so  I  I  want  all  mapable  but  then
there  is  no  difference  between  like  if
it's  refined  by  a  stable  cover  then  it's
already  a  cover  because  of  the  S
condition  ah  okay  so  here  it  is
satisfied  that  if  some  if  a  m  is
dominated
by  it's  already  aable  cover  so  okay  okay
thank
you  um  right  and
uh  I  also  need  to  say  that  commut  fin  or
finite  products  so  commuting  products
including  the  empty
product  I  geometrically  just  just  means
that  X  evaluated  on  a  dis  un  is  like  the
product
okay
so  so  this  condition  is  so  so  this
defines  I  mean  up
issues  and  it's
toos  that's  somewh
between
ke  and
three
so  for  Street  sheets  you  would  only
consider  some  a  ches  nerves  of  covers  um
and  those  would  by  definition  of  what  a
street  cover  is  always  satisfies  this
condition  so  for  uh  CH  nerves  a  street
covers  you  always  have  this  condition  so
it's  always  a  streak
sheath  but  and  for  streak  hypers  sheaths
you  would  ask  this  condition  for  all
Street  hper  covers  whether  or  not  they
satisfy  this  condition  but  we  definitely
want  to  our  derived  category  to  satisfy
uh  to  be
some  ABS  property  so  we  we  need  to
restrict  to  classify  the  Cod  for  we
which
weow  all  right
um
and  let  me  just  St  this  as  a  remark
and  not  pro  it  so
uh  yeah  so  sh
ification  of  an  accessible  pre  is
sucessful  so  this  the  analog  of  the
serum  of  water  house
HC  some  particular  this  means  that  you
can  actually  form  Co  limits  in  this
category  because  oper  forms  the  co  limit
in  the  category  of  thees  so  just  the
funs  from  to
Ana
and  this  is  still  an  accessible  fun  um
but  then  you  need  again  need  to  enforce
the  strong  she  condition  so  you  need  to
sheify  and  you  can  check  that  it's
preserves  the  set  fam
condition  all  right
so
examples  or
remark  so  for  any  first  of  all  for
any  you  can  look  at  the  Fun  which
takes
you  can  Define  aspect  a  at  a  factor
which  takes  any  B  to  the  map  from  A  to
B  so  again  I  have  a  small  technical
question  concerning  the  shic  which  I
believe  you  mean  shic  making  it
satisfying  this  precise  condition  that
is  not  not  the  Shri  shift  and  not  sh
hyper  Shi  but  this  intermediate
condition
yes  but  then  in  order  to  construct  this
unification  it  it  you  need  for  the
pullback  it  seems  that  you  need  to  know
like  that  the  pullback  of  this  kind  of
hyper  cover  satisfying  the  condition  on
theive  category  is  in  is  also  this  you
need  to  know  that  the  pullback  of  such  a
thing  is  again  such  a  thing  but  again
this  follows  from  the  same  argument  so
this  actually
equivalent
um  to  D  of  a  being  the  co  limit  along
the  low  street  maps  of  the  of  the  VA
bullets  um  in  in  there  col  is  taken  in
this  word  of  presentable  Infinity
categories  um  and  this  is  a  condition
that  base  changes  because  the  throughout
categories  base  change  in
PRL  and  Bas  change  Always  as
coits  ah  so  you  use  all  of  this  okay  and
this  was  Pro  disc  in  the  previous  talk
this
equuss  I  mean  this  is  all  magic  about
pure  okay  then  you  use  it  it  is  the  tens
product  when  you  bu  change
okay  all  right  um  so  first  of  all  you
have  all  the  fine  objects  so  for  any
antic  ring  a  you  can  consider  onsp  speec
a  which  takes  any  B  to  the  homes  from  A
to
B  um
climing  this  commutes  with  this
satisfies  all  these  properties  actually
already  so  you  don't  have  to  fify  so
certainly  commutes  with  products  right
because  you're  Hing  to  something  here  um
and  then  you  need  to  check  this  uh  this
the  send  condition  here
um  and  now  maybe  I  shouldn't  have  called
this  a  but  so  now  you  have  a  cyper  cover
no  you  have  a  Cy  cover  some  B  some  B
bullets  and  then  you  some  want  to  map  to
B  in  particular  you  want  to  map  like
like  particular  B  is  like  the  limit  of
the  plus  start  again  so  particular  a
would  be  the  limit  of  a  bullets  so  you  m
to  all
the  bullets  and  you  actually  M  to  B  and
also  on  of  the  r  c  to  again  because  it's
limit  this
is
um
so
uh  think  brings
up  and  yeah  I  mean  just  by  basically
only  this  is  a  fully
faceful  uh  fun  so  the  analytic  Rings  up
so  this  embeds  ful  Faithfully  into
analytic
spe  okay  so  obviously  unstack  is  my
notation  for  the
infinity6
um
and  uh  actually  this  accessibility
condition  is  precisely  it's  a  condition
that  you're  a  small  col  liit  of  object
in  the  essential
image
I  me
not  so  why  you're  look  at  our  notes  on
like  condens  sets
and  we  do  play  a  similar  game  there  when
we  have  to  full  category  of  condensed
sets  and  things  so  way  to  think  about
them  e  just  being  small  col  from  Prof
fin  sets  or  being  such  accessible
function  all
right  okay  so  so  what  is  the  SC  of
analytics  SE  well  for  any  analytic  ring
there's  an  object  which  is  some  of  the
analytic  Spectrum  this  all  the  others
are  built  by  some  gluing  procedure  by
some  Co  liit  of  those  the  airine  objects
and  the  way  to  think  about  this  uh  hyper
condition  some  that  tells  you  ways  in
which  uh  which  you're  allowed  to  glueing
so  someone  have  such  a  hypera  satisfying
this  condition  then  the  co  liit  of
this  the  simpli  diagram  there  of  these
unsp  a  bullets  would  be  the  same  thing
as  the  UNP  of
a  no  idea
so  theit  is
over
okay
over  okay  um
and  so  as  will  maybe  become  clear  later
in  the  lecture  today  so  we  want  to
impose  a  we  want  to  have  this  very
general  topology  because  it  means  that
uh  analytic  states  which  are  prior  have
seem  completely  different  because  they
have  completely  different  present  so
they  are  completely  written  in  up
completely  different  ways  as  a  CO  liit
of  fine  guys  uh  often  there  is  like  a
geometric  picture  that  they  should
secretly  be  the  same  object  and  in  order
to  say  that  they're  really  the  same  guys
someh  need  to  shif  find  rather  strong
topologies  makes  them  the  same
thing  all  right  so  okay  so  uh  next
example  that  this  is  in  some  general
bre
geometry  because  of  the  strong  nature  of
the
gr  another
question  oh  sorry  no  no  no  just  the  the
small  in  the  in  number  two  the  small  Co
limit  of  objects  in  which  category  it  is
taken  here  yes  in  analytic  stock  or
in  in
in  PR  no
before  uh  limit  in  before  the  stock
condition  just  it  would  also  be  a  CO  Li
before  the  St  condition  because  actually
sufficiently  filter  Co  limits  preserve
this  uh  this  condition  there  because
there  accountable  limit  condition  only
ever  so  if  you  take  omega  one  filter  col
they  will  always  preserve
this  she
condition
but
uh  no  no  but  you  you  certainly  have  to
take  the  pament  in  here
because  like  even  if  you  GL  just  the  few
F  to  P1  or  something  like
this  like  even  the  P1  wouldn't  be  a  Coit
of  it  sorry  probably  because  sorry  I
know  yeah  I  think  two  things  are
equivalent  but  okay  accessible  should  be
some  condition  preliminary  condition
yeah  so  so  so  a  different  way  to  think
about  this  is  that  you  can  also  like  put
a  cut  of  condition  on  your  analytic
rings  that  there's  some  more  copper
small  for  some
copper  and  then  you  can  just  consider
such  Stacks  to  find  just  a  couple  small
analytic
rings  and
then  there's  a  way  to  extend  them  to
like  the  full  class  but  some  left
extension  and  then  yeah  I  think
this  way  can  probably  show  that  okay
okay
uh  all
right  so  there  is  a
fun  from  like  deriv
schemes  um  the  deriv  schemes  they  also
of  course  embed  into  like  she  called  the
Des  sendable
topology  uh
we  can  go
to  taking  particular  like  any  spec  spec
a  for  us  ring  on  spe
a  Tak  a  condens  thing  which  is  really
just  the  and  um  and  you  all  condensed
modules  and  if  you  want  you  can  also
like  as  Dustin  discussed  you  can
basically  put  the  fpqc  toy  in  here
except  you  have  to  do  this  funny
countability
assumption  only  one
FPC
so  basically  like  any  fpqc
stack  um  can  also  be
mapped  and  there's  really  not  much  to
show  here  right  I  mean  you  definitely
just  have  to  spun  on  rings  and  then  you
just  have  to  show  that  whenever  you  have
a  des  sendable
cover  it  goes  to  such  a  street  cover  but
that's  basically  by
definition  um  and  again  you  could  also
use  this
funny  thing
between  sheeps  and  hypers  sheets  here  in
the  C  topology  here  if  you  wanted
to
um  right  uh
so  ah  right  maybe  I  can  mention  that
serum  that  is  actually  fully
faceful  a  ship  you  view  is  is
a  it's  like  a  full  back  where
you  this  actually  uses  tanak  us
T  do  you  do  you  need  QC
Qs  I  think  he  can  get  rid  of  this
because  he  can  certainly  reduce  to  the
qu  compact  case  and  then  there's  this
argument  of  offer  that  you  can  write  as
a  limit  of  qu  separated  things  you  can
how  how  do  you  reduce  to  the  Quasi
compact  well  you  can  certainly  if  you
want  to  understand  homes  from  X  to  Y  you
can  cover  X  by  but  you  can  X  compact  but
then  comp  wise  they  can  definitely  comp
oh  yeah  okay  there  other  argument  that
you  can  make  the  compact  so
can
okay
all  right  so  uh  you  have  some  algebraic
geometry  sitting  in
there  uh  next  you
have  Tic  space  is  sitting  in
there
which  well  these  These  are  good  from
fromo  one  so  these
are  a  plus
um  like  when  you  talk  about  CH  Ed  spaces
you  always  assume  that  this  is  chy  and
so  what  I'm  saying  now  let  me  also
assume  this  because  otherwise  I  would
run  to  some  longer  discussion  about
exactly  what  I  want  to  do
um  and  so  I  can  just  s  this  to  the  onp
of  the  analytic  ring  a  plus
solid  uh  so
note  that  this  way  we  get  a  different
fun
get  two  uh
different
function  um  so  there  actually  two  ways
to  Ed  sches  into  addic
spaces  so  you  can  take  a  spec
a  you  can  also  match  this  to
SP  a  Comm  Z  AL  Spa  AA
a  and  then  for  there's  a  question  in  the
chat  Peter  it  says  it  says  for  derived
schemes  you're  mapping  the  trivial
analytic  ring  structure  exactly  here  I'm
currently  using  triv  antic  ring
structure
um  right  no
Sol  and  so  you  can  either  now  take  this
furer  and  look  at  a  modules  and  solid  Z
modules  or  you  can  look  at  realtively
solid  a  a
modules  and  so  now  there  and  if  you
wanted  to  you
could  put  it  right  here  suitably
formulated  and  put  it  then  also  put  it
derived  here  um  so  then  there  are  like
three  funs  from  schemes  or  derived
schemes  to  analytic  Stacks  either  with
the  trivial  antic  ring  structure  or  with
the  one  induced  from  the  integers  from
the  solid  integers  or  with  this  gra
of  um  all  of  them  are  like  fully  faceful
embeddings  um
and  all  of  them  are  kind  of  relevant  to
what  we're  doing  so  have  to  kind  of
start  to  differentiate  between  the
different  incarnations  of  the  scheme
um  and  particular  like  the  one  that
actually  the  one  embedding  where  the
Notions  that  we  use  for  analytics  Texs
actually  match  the  usual  Notions  for
schemes  is  actually  this  relatively
solid  one  so  Dustin  introduced  some
Notions  of  like  closed  emerence  open
immersion  and  so  on  and  they  had  a
peculiar  feature
that  like  for  example  any  map  of  Fes
become  a  proper  map  here  and  like  in
particular  like  open  emerg  become  closed
emergence  here  which  sounds  very  weird
um  if  you  use  this  other  embedding
instead  with  a  relatively  solid
structure  then  open  edings  go  to  open
eddings  as  you  would  maybe  rather
expect  and  the  first  one  is  still
another  one  a  still  another  one  so  you
have  three  ways  to  put  skines  well  more
than  three  really  I  mean  the  one  AZ  is
just  you  take  that  one  and  you  base
change  it  to  solid  Z  so  you  could  base
change  to  anything
then  yeah  I'm
saying  I'm  saying
the  the  most  basic  one  is  that  one  and
then
the  the  first  one  up  top  there  with  a  z
solid  is  just  gotten  by  base  changing
that  one  to  from  Z  with  right  let  let  me
just  write  down  what  the  just  explaining
oh  oh  he  can  hear  us
the  green  thing  is  open  and
then  right  so  so  one  thing  to  notice
that  the  first  one  first
of
is
uh  a  fun  from
three  and  then  you  base  change  from  the
spec  of  just
integers  on  spec  of
this  and  I  mean  this  immediately
suggests  the
generalization  uh  for  any  analytic  ring
a  you  can  say  the  RO  schemes  overate
they  can  be  M  antic  Spees  over
a  um  but  just
using  well  maybe  should  have  WR
a  a  triangle  the  underlying
condens  Tri  point  so  you  take  if  you
have  an  antic  ring  then  it  has  an
condens  ring  which  has  anline  this
discrete  ring  and  you  can  look  at
schemes  over  there  so  for  example  I  know
you  might  have  the  complex  numbers  with
the  gases  and  the  ring  structure  and
then  you  can  just  consider  usual  schemes
over  the  usual  complex  numbers  as  a  just
fi  um  and  those  map  totic  Tes  over  there
by  just  using  the  previous  fure  and  then
B
changing  and
so  right  whenever  you  have  ring
over  and  again  this  actually  for
Yeah  you  mentioned  something  slightly
puzzling  about
derived  ad  or  derived  Huber  pairs  yeah
so  what  what  does  it  mean  does  it  mean
that  you  have  a  a  derived  ring  with  a
plus  in  pi  zero  and  it  is  and  all  the  pi
I  just  it  because  I  didn't  want  to  go
into  it  here  right  now  uh  yes  one  can
Define  the  version  like  some  derived
notion  of  like  ubering  but  I  I  really
don't  want  to  do  it  right  now  okay
and  like  for  each  fun  I  wanted  to
indicate  what  extent  we  know  this  is
fully
faceful  um  so  this  F  is  definitely  fully
faceful  on  F  lines  because  then  it
reduces  to  maps  of  rings  and  we  know
that  hubber  rings  and  fully  face  the
analytic  Rings
uh  and  so  here  we  know  that  it's  fully
faceful  under  some  CL  productivity
assumtion
Pro
uh  we  don't  know  a
full  fully  face  result  basically  because
there's  no  we  don't  know  a  good  version
of
tanak  foric
spes  um  I  don't  know  if  this  is  also
like  um  if  it's  worth  warning  that  the
in  contrast  to  the  scheme  case  the  adex
Bas  the  the  funter  doesn't  commute  with
pullbacks
or
right  so  actually  if  if  you  restrict  to
like  Tate  tic  spaces  oric  space  or
however  you  want  to  call  them  then  there
it  behaves  as
expected  as  long  as  you  stay  in  the  shy
case  otherwise  you  might  have  to  make
things  D
um
yeah  but  then  there's  this  PE  feature
that  like  on  the  right  hand  side  uh  if
you  take  a  fire  product  of  f  analytics
St  it's  always  just  the  usual  correspond
to  T  product  of  rings  this  is  not  true
in  adct  spaces  so  in  adct  space  you  can
have  a  fiber  product  of  aoid  space
that's  not  itself  aoid  anymore  um  so
there's  some  subtlety  there  that  can
actually  be  explained  a  very  nice  way  in
our  form  in  a  sense  that  instead  some  f
yeah  let  me  not  do  it
here
um  all  right  so  then  you  can  also  do
this  is  non  geometry  you  can  also  do  not
geometry  like  real  complex  and  for
example  you  can  have  complex  anic  spaces
map  to  Antics
Tex  and  now  this  some  work  over  for
example  the  guest  is  complex
numbers  and
uh  how  can  this  be  described  so  the  one
problem  is  that  usually  like  in  complex
anal  geometry  people  don't  really  tell
you  what  an  fine  space  is  um
um  but  they  could  um
so  any  complex  antic  space  can  always  be
written  as  some  kind  of  Union  of  what  we
call  compax  St
subsets  so  these  are  things  that  really
very  much  like  behave  like  Eon  objects
so  what  is  a  compax
sty  so  for
example  uh  the  vanishing  locus  of  some
ideal
inside  just  a  PO
this  so  these  are  two  of  complex
numbers  all  of  which  absolute  values  are
most
one  so  functions  H  morphic  in  the
neighborhood  and  right  so  what  we  endal
with  with  the  algebra  function  we  put  on
such  a  k  is  over  converted  holomorphic
function
soga  which  is  a  CO  limit  over  all  U
which  strictly  contains  this  complex
subset  of  functions  on
you  and  then  it  is  a  theorem  from  the
70s  or  80s  that  at  least  in  this  case
where  you're  really  close  in  the  poly
disk  uh  this  is  this  uh  seran
algebra  it  is  excellent  and  uh  if  it's
actually  manifold  it's  regular  and  so  on
um  so  it  has  all  the  nicest  properties
you  could  ever  hope  for
um  and  if  you  want  to  talk  about  like
toerent  Chiefs  on  X  then  you  can  really
just  talk  about  like  for  any  c  Stein
subset  you  give  just  a  finally  generated
module  over  this  nerian  ring  and  they
can  space  shap  so  as  far  as  I  remember
for  some  book  on  on  complex  analytic
geometry  so  there  is  a  small  satty  which
is  usually  not  very  important  but  still
for  the  nity  and  so  on  you  there  could
you  could  have  an  ideal  such  that  the
number  of  connected  components  of  the
zero  Locus  is  infinite  and  this  is  bad
for  nity  so  there  is  some  some
fin  that's  why  I  said  that  for  things
which  are  actually  of  this  form  that  a
risky  Clos  point  there  it  is  nrian  what
there  think  on  General  like  com  back
Stein  would  allow  something  like  Pro
finite  subsets  in  a  pus  and  this
wouldn't  be  in  a
serrian  uh  but  if  it's  actually  it's  a
risky  close  subset  of  a  politic  it  is  in
Assyrian  algebra  because  it  is  a
politics  not  arbitary  compact  ah  okay
okay  okay  okay  yeah  I  remember  that  it
was  for  another  other  compact  yeah
okay  um
right  but  but  I  mean  if  you  if  you  like
there  so  there  is  this  way  of  looking  at
complex  geometry  where  there's  really  a
very  close  analog  of  the  notion  of  an  e
aoid  subset  and
everything  really  havean  algebra  and
everything  is  really  very  very  similar
to  how  you  do  in  rigid
geometry  um  and  so  you  would  send  that
similarly  to  like  the  Coit  which  is
actually  against  of
Union  um
over  over
KX  of  the
UNP  of  this  algebra  endowed  with  like
the  Gess  use  compx
noce  I  mean  like  algeb  for  functions
they  have  a  natural  topology  like
uniform  convergence  on  compx  subsets  so
this  also  has  a  natural
topology  you  could  also  Define  direct  in
cond  setting  whatever  it's  a  So-Cal  dual
nuclear  for
SP  and  again  this  is  fully  faceful  on
for
the  the  you  take  this  with
which  kind  of  which  analytic  structure
okay  oh  just  um  so  you  view  it  as  an
algebra  in  the  say  gases  C  theory  yes
and  then  you  just  take  the  induced
analing  structure  so  you  just  check  on
the  underly  so  this  defines  a
condensed  ring  which  is  actually  an
algebra  complex  numbers  obviously  uh  and
it  because  it's  a  du  nuclear  for  it's
actually  gesses  and  so  you  can  just
induce  up  the  Gess  analytical  structure
from  the  complex  numbers  to  here
sorry  not  every  dual  nuclear  should  cret
is  gases  right
but  uh
what  they
are  it's  not  nuclear  as  a  guess  is
certainly  oh  yeah  yeah  yeah  yeah  okay
sorry  yeah  I  mean  these  are  actually
even  nucle  guess  this
here  yeah  what  is  qu
no  but  the  should  before  your  face  for
no
ful  it  should  be  fully  faithful  General
It's  just  tough  to  prove  no  no  but  a
point  it  should  it  give  a  map  I  mean  the
map  of  analytic  it  should  give  a  map  of
underlying  topological  SP  yeah  that's
what  you  need  to  prove  yeah  what  yeah
that  needs  to  be  proved  ah  yeah  and  what
is  projective  it  means  it's  like  a  adits
some  immersion  into  projective  space  so
maybe  like  an  open  subset  of  a  closed  so
what's  the  question  oh  what  is  qu
projective  some  locally  closed  immersion
into  projective
space  so  it's  closed  in  an  open  it's
closed  in  an  open  after
uh
maybe  before  I  go  to  the  next  example  um
there's  actually  nothing  much  special
about  using  like  complex  manold
or  here  I  mean  you  can  do  similar
definitions  work
for  it  works  also  on  the  real  ontic  case
mans  or  actually  also  for  smooth  or  you
could  also  I  don't  know  take  any  kind  of
CK  or  just  c0
topological  you  take  the  Ring  of  CK
first  Well  the  ring  is  again  the
slightly  peculiar  thing  that  you  take
like  still  do  this  Co  limit  of  functions
on
neighborhoods  over  convergence  continues
functions  is  a  kind  of  awkward  notion
that  makes
sense  that's  what  you  should
do  and  each  of  those  cases  you  can
actually
decide  whether  you  want  to  take  real  or
complex  value  functions  and  they  give
you  two  different
funs  where  one  is  just  the  base  change
of  the  other  so  one  you  can  make  make  an
antic  St  over  the  reals  and  you  can  base
change  to  the  complex  if  you  want  to  you
use
you  should  do
something  at  least  as  complete  as
the  all  right
um  and  now
comes  like  a  more  funny  example  so  I
mean  this  is
like  showing  that  all  the  usual  theories
like  of  algebraic  geometry  of  geometry
that  are  known  and  maybe  okay  so  we
should  also  at  some  point  discuss
brokerage  spaces  they  would  also  have
natural  fun
um  uh  they  M  into  our  framework  and  but
in  any  of  those  Frameworks  you  could
also  directly  Define  some  notion  of
Stack  so  there  could  be  you  can  imagine
many  possible  Notions  of  a  complex  anic
stack  by  taking  St  with  complex  and
spaces  and  D  them  with  any  gr  topology
could  think  of  maybe  just  open  covers
and  then  also  stakes  from  that  topology
would  St  just  because  any  open  covers  in
particular  is
cover
all  right  so  go
to
um  but  now  comes  the  SL  M  example  which
is  some  relating  it  back  to  what  we
started  with  Nam  condensed  sets  or
because  everything  has  become  a  St  let
take  condensed
onon  and  as  always  take  the  light
ones  so  those  actually  they  also  M  to
to  uh  how  does  that  work  so  this  takes
any  and  actually  you  could  even  go  to
like  you  don't  really  have  to  go
analytic  for  this
uh  you  can
just
uh  this  what  even  work  for  just  schemes
um  so  in  particular  like  here  there  is
something  condensed  on  the  right  there's
also  something  condensed  namely  like  the
anal  rings  that  were  condensed  rings  and
so  on  but  we're  actually  not  using  this
condensed  structure  here  uh  it  really
comes  from  the  speed  objects  yeah  um  how
does  it  work  so  if  you  have  a  a  live
profile
set  um
then  you  can  just  snap  this  to  the
spectrum  of  its  continuous
function
right  so  if  s  was  a  finite  set  it  was
just  be  a  fin  of  copies  of  spec  Z  in
general  as  is  the  limit  of  finite  set
and  then  similarly  this  is  the  limit  of
These  Fine
schemes  and  so  then  you  can  access
further  to  the
I  and  then  here's  the  reason  that  we
chose  this  really  funny  version  of
uh  between  Chiefs  and  hypers  Chiefs  so
con  anima  they  are  by  definition  hypers
Chiefs
of
an  on  on  life  Prof  steps  for  Theology  of
that  we  always
use  and  so  to  get  this  fun  you  have  to
show  that  if  you  have  any  hyper  cover
here  of  a  light  Prof  set  by  light  profin
set  um  then  it  goes  to  uh  something  for
which  you  enforce  the  S  here  um  it
definitely  goes  to  hyper  cover  and  so
then  there's  a  small  thing  you  have  to
check  that  it  actually  is  the  same
condition  uh  on  the  D  category
holes  basically  the  argument  that
faceful  flat  map  satisfy  the  sent  just
also  proof  the
state  countably  presented  face  yeah  so  I
should  say  that  here  is  the  lightess
condition  is  again  important  because  uh
um  uh  it's  always  true  that  if  you  have
a  stive  map  of  Prof  fin
sets  then  if  you  look  at  the
corresponding  map  of  continuous
functions  it's  always  FL  for  any  Prof
only
space  but  uh  we  need  a  descendible  map
for  our  business  and  so  we  only  know
that  counter  represented  faceful  that
maps  are  descendible  which  is  another
reason  that  we  have  this
Li
she
what  is
question
on
all  right
um  so
uh  yeah  so  here's  a  remark  um
so  so  there  are  different  ways  ofing
schemes  into  antic  stacks  and  now  there
are  also  different  ways  of  embedding
something  like  topological  manifolds
into  antic  Stacks  or  the  complex  or  real
numbers  say  so  either  you  can  do  the
thing  where  you  uh  use  the  algebra  of
continuous  fun  continuous  real  value
functions  to  do
this  or  you  can  treat  the  topological
manifold  purely  as  a  topological  space
or  faceful  into  condens  sets  and  then  go
like  from  condens  sets  to  analytics  STS
this  gives  you  a  different  thing  which
is  actually  defined  over  the  integers
right  because  this  analytics  St
just  integer  value  function  these  are
antic  the
integers  uh  so  these  are  completely
different  incarnations  of  a  topological
manifold
St  but  there's  actually  again  a  map
between
them
so  let's  let's  consider  the  following
example
so  you  can  take  the  two  sphere  as  a
topological
space  and  then  treat  it  as  a  topological
manifold  and  this  gives  you  a  fun
tic  space  which  in  the  compa  case  it
really  is  just  the  UNP  of  the  algebra  of
continuous
functions  and  let  me  work
everywhere  complex
coefficient  which  which  which  category
of  let's  say  see  gashes  what
gashes  uh  the  differ  thing  I  could  do  is
I  could
take
S2  and  treat  as  a  real  analytic
man  and  again
build  build  an  analytic  space  over  the
guess  is  complex
numbers  um  or  inia  you  could  think  about
all  other  possibilities  like  CK  C
Infinity  Real  analytic  and  so  now  you
would  have  here  the  algebra  of  so  what's
good  notation  for  real  antic
functions  no  no  that's  smooth  oh  Omega
yeah
cega  sometimes  it's  called  C  Omega
yeah  the  Omega  I  think  that's
thanks  and  I  don't  know  think  about  all
possible  other  algebra  like  C  Infinity
function  CK  function  between  they  would
all  also  give  an
spaces
um
but  then  uh  you  can  also  like  S2  is  like
one  version
of  P1  of  the  complex  numbers  so  you  can
also  treat  this  as  a  complex  space  and
well  then  you  want  to  get  something  with
the  guess  this  complex
numbers  um  this  is  not  f  f  anymore  right
I  there  are  not  so  many  Global  home
functions  here  it  is  what  it
is  is  not
F  there  but  it's  some
glued  from  two
copies  of  unspe
orphic  functions  on  the  dis  over  conic
functions  on  the  dis  right  so  you  can
cover  this  by  to  this  meeting  along  the
circle  and  similarly  this  would  be  glued
but  from  two  copies  of  over  conic
function  on  it  this  glued  along  the  over
converg  morphic  functions  on  the
circle
topology  he's  gonna  do  it  as  a  scheme
first
okay  uh  so
clearly  like  making  it  more  rigid  you
can  also  now  look  at  Al  functions  so  you
can  also  take1
c  as  a  scheme  and  build  something  over
the  Gus  complex  numbers  uh  so  again  this
is  not  F
fine  evidently  uh  but  it's
glued
uh
from  two  copies
of  the  analytics
spectrum  of  like  a  polom
Al  or  maybe  there  should  call  one  t  and
one  t  inverse  or  whatever
um
and  I  will  make  a  comment  about  this  in
just  a  second  but  let  me  first  finish  um
and  then  I  can
take  again  just  s
two  and  Tre  just  as  a  condens
set  and  go  to  SP  change  to  the  guess
this
complex  so  here  I'm  using  the  fun
from  from
six  so  basically  what  happens  is  that
here  you  want  to  take  the  locally
constant  functions  on
S2  well  there  are  none  on  S2  but  if  you
secretly  think  of  S2  is  being  the  quion
of  a  profite  set  by  Prof  inst  relation
which  you  can  always  do  by  this
condensed  perspective  then  on  those
profite  coverings  you  do  have  locally
constant  functions  and  then  you  can  do
this  bluing  so  secretly  to  evalate  what
this  is  you  have  to  remember  that
secretly  as  two  is  a  qu  of  Prof  set  by
Prof  relation  and  then  you  can  lose
locally  con  functions  here  you  have
algebraic  functions  here  you  have
holomorphic  functions  here  you  have
realic  functions  here  have  continuous
functions  all  of  them  give  you  different
inations  of  what  the  two  sphere  might
be
um  and  now  there's  already  one  really
funny  thing  which  is  something  that  DUS
already  mentioned  a  couple  of  lectures
ago  uh  that  one  intercation  of  the  Gaga
is  that  is  actually  an  iism  of  analytics
Tex
oh  so  this  is
one  kind  of  Gaga
statement  uh  which  is  now  not  just  a
statement  about  some  career  and  Chiefs
or  some  derive  category  of  career  Chiefs
or  whatever
uh  so  not  just  on  the  linear  algebra  but
really  before  you  pass  linear  algebra  on
the  level  of  spaces  there  is  an
ni  particular  one  of  the
r
and  I  mean  this  is  an  instance  where
Upper  ear  the  way  you  build  these  things
is  completely  different  one  is  built
from  just  poomi  one  from  form  functions
and  the  strong  work  topology  that  we
impose  is  precis  is  there  to  ensure  that
you  can  have  the  possibility  of  such
interesting  as
PHS  question  basically  the  way  one
proves  this  is
that  you  first  need  to  prove  it
objective  so  this  is  covered  by  these
two  things  you  need  to  prove  that  those
to  this  together  to  this  steam  guy  here
that  they  are  the  sendable  map  again
then  you  have  to  unrel  what  it  means  to
be  the  and  prove  it  uh  sorry  I  have  a
question  yeah  yeah  uh  are  you  able  to
get  uh  uh  C  Infinity  schemes  or  from  six
you  had  point  six  there  which  it  was  uh
schemes  but  was  that  algebraic  or  could
you  get  other  schemes  like  C  Infinity
schemes  I'm  not  sure  what  C  Infinity
schemes  are  if  I  there  derived  menolds
that's  somewhere  in  the  literature  well
no  that  that's  that's
uh  it  will  it  would  be  generalizing
manifolds  but  it's  not
derived  but  one  can  one  also  has  uh
derived  manifolds  I  guess  but  I  was
asking  the  simpler  C  Infinity  scheme  so
locally  ring  spaces  that  are  with  a
local  model  that's  a  c  infinity  ring
which  is  not  really  an  algebraic  object
but  you  had  a
six  six  right  well  six  was  a  condensing
but  I  think  you  mean  five  um
but
uh  well  for  this  I  guess  you  would  need
to  establish  some  relation  between  this
notion  of  C  infinity  rings  and  analytic
rings  and  this  I'm  not  really  sure
about
um  there  is  however  in  practice  some
kind  of  fun  from  from  some  suitable  CLA
analytic  Rings  towards  C  infinity  rings
and  that  like  C  infinity  rings  that  you
want  that  like  for  any  smooth  function
from  the  real  to  the  real  or  something
like  this  it  gives  you  an  operation  on
your
ring  oh  um  and  after  that's  not  at  all
data  that  we  encode  in  our  strings
because  they  just  appr  some  I  mean  just
have  additional  multiplication  you  don't
ask  that  smooth  functions  like  uh  but
you  can  actually  show  that  in  some
circumstan  where  you  would  expect  it
smooth  functions  act  in  a  unique  way
um  yeah  so  I  mean  so  for  example  this
Theory  here  gives  you  a  completely
different  way  of  approaching  the  subject
of  deriv  manifolds
because  like  if  you  if  you  just  start
with  user  manifolds  and  B  into  our
setting  and  then  you  can  take
intersections  the  intersections  are
derived  because  everything  is  DED  for  us
and  so  you  can  just  if  you  want  to  work
with  Drive  manifolds  you  can  just  use
our  framework  and  just  Works  um  but
usually  there  was  a  reason  that  this  C
Infinity  albas  were  invented  because  you
needed  that  on  this
derived  Tor  product  you  still  want  the
infinity  functions  to  act  which  is  not
so  clear  but  you  can  prove  that  also  in
all  framework  they  they
act  I'm  not  sure  if  this  was  useful
information
uh  yeah  so  so  precise  relation  to  like
steings  I  don't  really  know  and  I  don't
really  know  what  SK  fering  is  also
well  but  for  like  the  purpose  of  having
a  workable  theory  of  like  derived
manifolds  and  so  on  uh  I  think  this  also
sits  fully  Faithfully  in  our
framework  right  um
um  yeah  maybe  that's  all  I
want  these  are  some  examples  what  I  say
next
um  yeah  maybe  one  particular  instance  of
this  is  like  if  you  go  from  the  very  top
to  the  very  bottom  then  those  functions
are  just  have  a  good  take  us  input  any
topologic  manifold  for  example  or  any
compact  House  of  space  or  something  like
this  uh  you  would  basically  always  have
such  a  map  maybe  you  should  assume  it's
five  dimensional  at  some  point  but  other
than  that  you  always  have  such  an  natur
map  from  like  this  Incarnation  using
continuous  functions  to  the  Incarnation
using  constant
functions  but  yeah  the  categories  of
modules  are  very  different  so  up  there
the  C  of  modules  of  course  just  the  it's
fine  so  the  modules  are  just  modules  of
the
algebra  C  Vector  space  was  an  action  of
the  continuous  functions  butas  here
there  just  Chiefs  of  C  Vector  space  on
the  topological
space  one  can
show  if  x  is  a  maybe  locally  comp
sour  the  chat  has  a  question  Peter  which
is  um  why  do  we  consider  gashes  antic
ring  structure  on  the
left
yeah
um  because  otherwise  this  iscl  for
example  if  you  want  the  function  from
complex  spaces  towards  um
towards  analytics  Tex  you  need  that  the
the  intersections  match  up  so  if  you
have  like  an  intersection  of  compact
line  subsets  in  here  then  the
intersections  again  something  compact
Stein  and  you  want  this  to  be  mirrored
on  on  the  side  of  analytics  text  in  our
picture  which  means  that  if  you  compute
the  corresponding  complete  tender
product  of  analytic  rings  of  like
functions  on  D  and  the  other  disc  and
them  together  you  should  get  functions
over  conver  functions  on  S1  and  this  is
a  computation  that  comes  out  correctly
in  the  gas  analytic  ring  structure  but
definitely  doesn't  come  out  for  the
right  for  the  trial  antic  ring  structure
because  then  would  just  take  the  the
usual  tensor  algebraic  tensor  product
can  and  this  is  just  some  nonsense  so
you  need  to  complete  the  tensor  product
and  then  it  comes  out
right  um  so  X  dimensional  sense  uh
then  you  can  look  at  the  category  of
like
Tex  and  a  is
any  and  B  change  it  to  the  spectrum  of
a  and  this  is  just  Chiefs  on
x  CHS  on  just  X  some  consider  toal  space
um  those
vales  so  yeah  so  you  can  think
of  yeah  General  CHS  as  being  some  kind
of  f  she  on  an  Associated
St
um
I  realized  before  that  I  should  have
said  that
uh  the
fun  that  take
any  on  spe  a  towards  the
a  or  also
towards  uh  presentable  St  authenticated
is  linear  over  today
um  satisfy  satisfy  the
send
um  and  hence  induce
funs  on  on  all
entics  so  for  any  St  you  can  Define  the
direct  of  current  on  X  Y  descend
um  and  also  you  can  Define  what  is  like
a  a  chief  of  categories  over
X  so  here  it  is  again  the  intermediate
condition  the  but  for  the  yeah  I  for
this  funny  version
between  the  center
typ  for
any
this  will  actually  both  of  these  things
will  actually  be  part  of  some  general
sixun  formalism  and  at  some  point  we
will  uh  talk  more  about
that  and  basically  all
design  Criterium  for  for  GR  topology  was
actually  precisely  that  that  for  any
analytics  de  we  definitely  want  to  be
able  to  talk  about  the  C  Chiefs  and  at
some  point  we  also  realize  it's  probably
good  to  enforce  r  that  we  can  also  talk
about  Chiefs  of  categories  and  then  we
are  basically  taking  the  strongest
possible  gr  topology  uh  where  this  is
true
what  is
prxl  that  is  the  thing  which  is  gotten
by  descent  from  the  association  and  spec
a  goes  to  pld  da  which  is  Da  modules  in
PRL  okay  which  is  a  tool
yeah  it's  in  two
category  yes  second  thing  two  category
but  we  don't  have  to  care  we  can  just
see  it  as  Infinity  one  by  neglecting  the
non
um
right  okay  so  this  is  some  general
examples  that  maybe  we  want
to  I'm  not  sure  we  cover  any  all  of  them
detail  on  the  remaining  lectures  but
just  want  to  give  a  general  picture  but
I  did  want  to  come  back  to  the  point
where  we  started  discussing  analytics
Tex  which  was  the  construction  of  the  T
of  the
curve  and  discuss  this  in  a  little  more
detail  now  because  now  we  have  the
language  of  talking  about
this
so  let  me  recall  what  we  want  to
do  um  so  we  have  this  and
bring
one
yes
so  this  was
Universal  um
or  and  being  endowed  with  the
topological  unit
2  um  such
that  uh  when  you  take  this
usual  uh  guy  which  is  a  free  guy  on  a
nor
sequence  and  you  tender  it  up  to  the
string
um  then  there  the  operator  that's  one
minus  Q  *
shift  so  shift  is  the  endomorph  of  P
that  just  shifts  all  the  integers  on  S
and  so  this  was  this  ring  structure  we
introduced  some  Lees  ago  and  where  an  on
computation  was  that  you  could  actually
compute  what's  the  underlying  string  of
this  was  and  it  was  this  algebra  of  L
wrong  series  and  Q  which  has  a  certain
funny  to  noral  G  condition  on  their
coefficients
um
so
um  I'm  cond  string
is
the  and  the
existem
a
k  but  maybe  the  underlying  ring  of  the
underlying  condensed  ring  because  now
I'm  just  telling  you  a  set  maybe  uh  I
did  tell  you  the  curent  condensed
structure  on
this  okay  so  this  is  some  funny  ring  of
Lon  series  integral  series  with  a  rather
strong  on  the  growth  of  the
coefficients  and  then  we  recall  that  the
goal
was  to  the  final  litic
curve  over  a  or  really  some  over  a
triangle  star
um  this  would  be  a  schem  uh  such
that  if  I
take  the  Incarnation  of  EQ  as  a  scheme
over
a  Um  this  can  be
written  as  a  quotient  of  a
GM
mm
over
uh  under  multiplication  by  Q  as  an
optim
not
and  I  already  couple  Lees  back  I  gave
the  outline  I  thought  it  should
go
so  the  first  step  to  see  that  there  is
a  certain  kind  of
nor  um  and  I  will  talk  about  this  more
in  just  a  second  um  which  goes
from  uh  the  P1
array  but  really  again  the  P1  Som
incarnated  as  a  scheme  over  a  so  coming
in  space
um  towards
sorry  toward  the  infinity  which
intuitively  Speaking  like  whenever  you
have  a  point
here  tells  you  how  large  it
is  having  a  point  here  is  basically
having  a  point  here  so  maybe  nothing
think  from  a  to  a  field  or  something
like  this  and  then  an  element  of  that
residue  field  and  so  there's  something
like  saying  that  on  any  res  fuel  there
there  would  be  a
nor
Z  to  Z
uh  it's  like  the  absolute  but  now  this
what  this  this  map  here  is  really  meant
to  be  a  map  of  analytic  stats  right  so
the  left  hand  side  is  now  analytic  stack
we  can  take  the  P1  can  make  it  turn  it
into  analytic  St  by  this  General  fun
oper  a  triv  ring  structure  but  then  we
can  Bas  change  it  to
a  and  maybe  here  should  write  I  mean  so
here  you're  incarnating  this  by  treating
it  as  a  condensed
set  and  we  could  or  could  not  base
change  it  to  a  yeah  we  can  you  have  to
choose  some  value  for  the  absolute  value
of  Q  like  say  yeah  so  so  it  should  stand
stand  some  q  q  is  a  section  of
this
um
uh  and  it's
Mullica  in  fact  is  a  unique  such
thing  was  the  properties  I  will  mention
so  it  should  be  multiplicative  in  a
sense  I  will  make  precise  in  just  a
second  so  I  won't  talk  about  these
Norms
second
one  uh  let  say
sense  but  to  make  it  unique  I  should
precisely  specify  where  where  key
goes
um  meaning  like  Q  is  actually  a  section
from  spectrum  of  a  back  into  this  and
this  should  go  to  constant
half
um  and  then  you  can
Define  the  analytic  GM
array  as  a  subset  of  the  P1
array
um  as  a
preim  the  open  subset  um
sorry  the  pimage  of  open  subset  Z
Infinity
in  and
then  you  still  have  q  to  the  acting
there  and  then  can
take  the
quotient
to  the
Z  and
then  uh  by  an  argument  I  already
sketched  last  time  this  this  basically  a
prop
Cur  and  then  you  have  to  prove  the  zero
some  algebraization  theorem  that  would
apply  to  All  Purpose  cures  with  a
section  at  least  that  uh  it's  algebraic
so  it  is  in  the
image  found  different  schemes
over  the  underlying  ring
towards
of
theity  one  I  mean  you  need  to  make  the
line
line  okay  it's  not
clear  even
geomet  it's  but  here  it's  okay  we
discuss  this  you  need  some  Gaga  or
exactly  but
Gaga  and
someing  which  is
again  where  is  a
soft
soft  well  in  our  in  our  approach  to  Gaga
you  don't  really
get  SOA  is  much  more  General  than
sity
but  it  doesn't  really  tell  you  anything
like  reman  rock  for  something  you  don't
aiori  know  is  algebraic  I
mean  so  that's  kind  of  a
separate  okay
um  right  so  one  key  notion  that  some
comes  into  into  this  is  kind  of  a  notion
of  Norm  that  we  want  to  have  here  so  let
me  actually  Define  this  in
general
uh
let
the  and  it's  a  slightly  awkward  notion
of  a  normal  a  that  is  not  really  normal
the  underlying  set  of  a  or  anything  like
that  but  it's  rather  the  thing  that  much
more
geometric  uh
is  can  check  the  P
one  that's
uh  with  which  properties
so  such  that  Z  to  zero  of  course  zero
section  is  constant  zero  the  one  section
is  constant
one
um  and  then  there  is  a  motivity
condition  uh  for  first  Let  me  give  it
one  other  I  want  that  to  near  poent
elements  have  Norm  most
one
um  will  also  follow
something  and  here's  how  I  want  to
phrase
that  so  inside  the  P1  I  well  I  can  first
look  at  the  locus  A1  where  I  didn't
really  have  just  function  but  then  I  can
look  at  the  locus  where  the  function  to
near  potent  and  that  is  some  are  given
by  the
UNP  of  a  and  then  you  join  the  top
variable  so  in  other  words  it's  again
just  this  this  P  treated  as  an
algebra  so  this  maps  to  the
UNP
a  which  is
a11  a  this  is  just
another
ring  I'm  instead  of  taking  the  next  for
S  uh  but  I  want  to  say  that  this  factors
this  composite
vectors  over
Z1  the  Clos  interval  not  the  half
open  yeah  so  things  that  you  actually
want  to  say  it's  less  than
one  it's  turns  out  that  the  better
notion  is  this  one  for  reasons  I  can't
uper  already  explained  but  it's  the  one
that  behaves
well  I  should  may  say  that  for  example
such  a  factoring  here  turns  out  to  be
really  just  a  condition  because  there's
really  a
monomorphism
and  similarly  for  these  other  conditions
that  zero  goes  to  zero  I  want  to  say
that  factors  over  zero  has  a  subset  of
zero  Infinity  again  that's  just
condition  all  right  so  these
are  uh  some  first  properties  but  then
when  you  have  Norm  you  usually
asked  for  some  version
of  uh  multiplicativity  and  some  also
some  Behavior  with  respect  to  addition
turns  out  we  just  forget  about  addition
but  we  keep  the  multiplicativity
condition
um
modli  um
because  uh  I'm  not  working  on  A1  but  P1
and  also  the  image  and  the  reason  I  work
with  P1  is  actually  that  even  on  A1  I
would  want  to  allow  some  functions  that
they  could  have  infinite
nor  and  if  I'm  allowing  eity  on  the
target  of  this  map  I  might  as  well  allow
it  on  the
source  um  but  now  for  multiplicativity  I
have  to  be  slightly  careful  of  tweeting
like  zero  times  infinity  cases  um  one
way  to  do  this  is  as  follows
uh  that  X  inside  of  P1
Cube  closer
of
U  the  locus  where  XY  equal  to
Z  and
similarly  there  are  some  actually  smooth
surface  in  I  did  similarly
Define  some  xar
inside  not  use  x  right  um  inste  of  01
C  so  there  similar  thing  like  on  the
open  part  you  can  look  at  the  for
X  they  have  Incarnation  on  the  level  of
schemes  of  this  clure  and  one
Incarnation  just  on
of  real
numbers  and
then  uh  you  have  X  incarnated  as  a  scham
array  Ms  to  P1
cubed  incarnated  as  a
choay  and  this  match  to  Z  Infinity
Cube  and  I  on  inside  here  you  have  this
xar  and  I
also  Peter  would  it  be  the  same  as  um
asking  first  that  like  the  the  map
giving  the  norm  is  equivariant  for
inversion  on  both  and  then  just  asking
for  multiplicativity  when  you  restrict
to
A1
I  think
so  yeah  that  was  kind  of  the  definition
I  thought  we  were  using  but  maybe
yeah  yeah  but  even  on  A1  you  have  to  be
careful  that  even  A1  can  M  to  zero
Infinity  it  could  M  to  Infinity  oh  sorry
I  should  have  said  that  I  should  have
said  by  i  instead  of  A1  I  should  have
said  the  pre-image  of  of  zero  right  you
could  do  that  yeah  so  it's  yeah  like
what  happens  on  01  and  one  Infinity  it's
just  mirror  images  of  each  other  and
inversion  yeah  and  then  on  the  pr  of  01
you  ask  for  exra  multiplicativity
that's  I  think  that's  correct
uh  yes  I  think  that's  corage  of  01  lies
in  A1  or  not  yeah  lies  inside  A1  yeah  or
maybe  well  I  because  I  said  you're  EO
variant  with  respect  to  inversion  on
both  yes  so  so  that  enforces  that
Infinity  goes  to  zero  yeah  so  up  there  I
could  but  I  didn't  specify  that  infin
should  go  to  Infinity  by  has  to  inity  go
I'm  sorry
infity  if  you  ask  it's  equivariant  for
inversion  on  both  sides  then  Infinity
has  to  go  Z  he  say  now  that  the  requires
inity  going  toity  it  follows  it
follows  why  does  it  follow  because  like
0  time  Infinity  equals  one  is  kind  of  a
point  in  that  closure  I  don't  know  I
think  it
follows  yeah  I  mean  if  Infinity  wouldn't
go  to  Infinity  then  you  would  run  to
some  contradiction
from
but  and  is  it  I  didn't  care  it  let  me
just  specify  it  for
saf  is  it  the  case  that  the  inverse
image
of  of  01  lies  in  A1
yes  yeah  it's  not  entire  I
mean  yeah  have  to  show  that  there  is
no  it's  enough  to  show  that  the  compl
the  complement  of  A1  somehow  which  is
this  where  you  have  to  understand  yeah
you  have  you  have  to  see  that  the
complement  of  this  of  of  A1  is  kind  of
generated  by  just  the  res  SE  the
infinity  section  world  of  analytics  St
okay  so  then  the  inverse  image  of  the
finite  part  of  zero  Infinity  round  in  A1
right  right  okay
yeah  okay  so  let  me  explicit  say  that
there's  no
condition  that's  we  don't  drag
inequality  at  all  no  which  relates  to
this  issue  that  once  came  up  that  two
might  be  unbounded  like  the  nor  of  two
could  so  you  usually  the  two  absolute
value  two  less  to  two  implies  yeah  so  I
don't  ask  any
version
uh  but  if  absolute  value  of  two  is  less
than  equal  to  two  you  get  a  triangle
inequality  or  not  probably  there  is  a
theorem  that
it  does  not  the  absolute  way  of  two  can
be  infinite  in  this  there  example  like
on  this  Gus  analytic
base  like  I
will  I  will  try  to  in  the  remaining
minutes
construct  such  a  norm  on  the  Gus  Bas
ring  and  in  particular  you  can  then  look
what  what  the  norm  of  two  is  and  it's
some  function  like  from  from  the  antic
spectrum  of  a  towards  Z  infinity  and
it's  s
active
so  there  is  some  Locus  where  you  have  to
two
in  all  right  so  the  Ser
method  that  I  already  uh
stated  on  this
boardly  is  that  uh  there  is
unique  Norm  on  uh  on  let's  guess  his
space
uh
taking  and  I  mean  I  could  chosen  any
number  between  zero  and  one  here  and
doesn't  matter  which  one  because  you  can
always  because  I  didn't  use  anything
about  addition  on  the  target  I  can
always  just  rescale  the  Target  by  some
exponential  m  and  still
be  all  right  um
yeah  I  think  I  don't  have  that  much  time
so  let  me  just  give  you  an
idea  um  and  maybe  just  of  the
construction
of
right
so
uh  let  me  start  with
L  let's  say  x  again  some  kind  of  fin
dimensional  locally  compact
h
and  why
some  then  I  claim  that  there's  some  kind
of  tanak  Duality  describing  what  maps
from  y  to  X
are  so  turns  out  that  to  give  such  a
from  y  to  X  it's  just  enough  to  give  a
fun  from  POS  CHS  on  X  towards  POS  CHS  on
y  with  some
property  but  actually  you  don't  really
need  to  specify  the  values  of
all  yeah  let  me  first  see  so  this  is  the
same  thing  as  s  the
fun  F
from
that's  the  r  c  for  B  groups
on  grps  on  X
towards  of  Y  sorry  TOS
X  um  so  this  should
be  uh  temp
functors
Mee  with  all
poits  and  they  should  be  linear  over
DC  uh  such
said  uh  stre  locally  on
y
uh  the  following  happens  so  on
X  you  have  lots  of  it
algebras  namely  for
any  close  subset  of  X  you  can
take  the  constant  she  on  Z  whenever  Z
and  X  is
closed  so  this  becomes  an  unimportant
algebra
the  con  she  on  uh  Chad  has  a  question  so
Chad  has  a  question  it's  about  whether  X
is  supposed  to  be
light
uh  yeah  yeah  that  is  my  yeah  yeah  the
triable
yeah  let's  assume  X  is  really  just
contained  in  some  fin  di
space  um
right  so  so  on  X  any  closed  subset  gives
you  animportant  algebra  which  is  um  the
one
corresponding  on  um  he  wrote  whatever
you  need  to  make  it
work  think  you  wrote
nice
um  this  guy  should  be
connected  this  guy  should
be  uh  can  active  yeah  negative
chological  contains
in  no  higher  CH  just  super
left  the  thing  is
that  I  mean  if  this  actually  mod  Prof
set  then  this  would  some  you  could
basically  check  I  mean  any  close  subset
is  in  the  section
of  both  my  close
subsets  and  so  then  these  would  be  know
just  retracts  of  the  of  the  unit  and  so
why  would  itself  be  fine  then  this
should  of  course  also  go  to  R  of  the
unit  object  but  this  is  a  connective
object
so  uh  so  if  x  was
profinite  uh  and  Y  is  some  fine  then
there
must  images  must  really  be  be  connected
um  in  general  like  giving  such  map  you
can  always  do  locally  so  the  condition
you  have  to  enforce  just  be  some  strect
local  condition  so  it  turns  out  that
this  is  the  condition
sh  locally  on  why  you  mean  the  pullback
or  the  Shri  it's  the  usual  P  usual
pullback  sh  locally  for  the  topology
yeah  and  then  it  remains  connective
after  because  of
okay  and  note  that  the  full  category  of
sheves  on  X  is  actually  generated  by
these  guys  on  the  Cox  so  actually  to
define  the  standard  funter  you  really
only  have  have  to  declare  these  item
poent
algebras  uh  so  so  yeah  so  describing  M
from  y  to  some  uh  such  guy  here  is
completely  determined  by  specifying  for
each  close  subset  of  X  you  have  to
specify  an  IT  algebra  and  if  they  all
already  connected  then  then  you're  good
to  go  and  the  only  thing  you  have  to
check  somehow  is  that  these  it  for
algebras  the  the  kind  of  Tor  prodct
Behavior  exactly  matches  the
interfection  behavior  of  the  close
subsets  of  x
and  that's  what  it  means  to  be  attens
of  maybe  it's  also  nice  to  mention  that
this  is  actually  a  set  even  though  a
priori  it's  an
ana  um
right
okay  so  to
so  uh  we  have  to  find  some  other
important
Al
in  the  Der  category  of
this1  realize  that  space
over
um  and  you  have  to  typically  you  have  to
find  some  that  should  correspond  to  the
pre-image
of  like  some  interval  from  zero  to  some
number
R  where  R
is  so  if  you  know  where  these  PR  should
go  then  because  there's  this
compatibility  on  involution  you  also
know  where  intervals  should  go  that  go
from  somewhere  to  infinity  and  then
everything  else  can  somewh  be  written  as
some  kind  of  and  the  co  limits  and
intersection  some  in  terms
of  of  those  so
really  SC  Norm  you  just  have  to
say  uh  describe  what  is  the  pre-image  of
Zer  R  in  other  words  what's  the  locus  in
here  where  the  the  norm  is  that  most
are
so  disc  of  radius  R  around
zero  which  should  be  some  subset  of  A1
already  which  should  be  the  analytic
spectrum  of  sum
ring  join  variable  T  over
phone  and  this  you  can  just  write  down
um
down  because  and  this  should  be  naely
speaking  this  should  be  certain
sums  n  *  t  to  the
N  um  and  greater  to
zero  which  some  over  convergence  so  that
should  exist  some  R  Prime  greater  than  R
so  that  the  norm  of  a  n  r  Prime  to  the
end  goes  to
zero
or
um  saying  it's  such  differently  uh  you
would  want
that  uh
whenever  what  is  the
nor  inte  okay  okay  okay  well  no  the  are
now  in  a
um  what  is  but  but  what  what  you  do  is
you  can  when  you  have  a  n  sequence  in  a
then  then  you  would  expect  that  the  norm
stays  bounded  because  Bound  by  one  uh  um
or  by  some  constant  um  and  so  to  produce
such  a  thing  you  can  some  stop  with
a  with  a  case  with  the  A  on  L  sequences
and  then  multiply  by  some  poers  of  Q  to
make  it  to  make  it  this  condition  set  F
so  actually  this
is
um  some  Co  liit
off
um  some  t  with  respect  to  some  condition
on  r  that  I'm  uh  not  able  to  write  down
but  basically  you  can  explicitly  write
down  the  corresponding  item  put  algebra
and
is  well  T  is  you  just  have  to  think
about  how  you  can  phrase  this  idea  here
in  terms  of  the  gas  bace  ring  and  the
idea  is  that  the  absolute  value  of  Q
should  be  a  half  so  yeah  starting  from
the  idea  is  that  I  mean  you  can  just
take  any  n  sequence  of  a  and  there's  a
free  algebra  with  a  n  sequence  and  then
multiply  the  a  by  suitable  powers  of  Q
to  uh  to  use  an  algebra  like  that  and
then  then  there's  a  computation  you  have
to  do  is  with  that
uh  if  you  form  certain  tensor  products
of  such  algebra  BR  maybe  centered  at
zero  in  Infinity  then  they  the  tender
products  uh  come  out  in  exactly  the  way
you  would  hope  for  some  of  that  matches
intersection  behavior  of  intervals  from
zero
Infinity  this  is  a  computation  you  have
to  do  uh  which  really  uses  that  you're
working  some  over  the  guess  space  ring
uh  but  then  uh  you  can  produce  such
involment  maybe  I  should  stop  because
I'm  over
time  so  I  mean  here  you  it  looks  like
that  you  have  to  to  consider  the  power
of  Q  such  that  yeah  so  the  T  here
secretly  was  meant  to  be  a  rational
number  um  ah  okay  so  you  have  to  make
sense  of  this  let  I  say  for  certain  for
rational  numbers  which  means  that  when
when  the  power  is  divisible  by  something
you  put  the  qill  to  this  and  then  you
you  do  something  in  between
yes  I  don't  know
yeah  DUS  will  say  is  much  better  next
time
[Laughter]
willing
all
okay  very  nice  question  yeah  you  can  ask
a
question  hi  um  I  was  wondering  is  there
something  like  an  analytic  space  theory
definition  using  your  analytic  stacks
and  maybe  some  axioms  and  I'm  asking
because  I  noticed  like  you  have
different  analytic  spaces  and  they're
theories  and  you  claim  that  they're  all
analytic  Stacks  but  if  we  wanted  to  put
some  say  additional  structure  in  an
analytic  space  maybe  we  could  do  this  by
somehow  having  a  more  general  definition
of  uh  analytic  space  using  your  language
as  opposed  to  going  Case  by  case  so
I  don't  know  if  did  my  question  Mak
sense  it's  not  stacky  right  it's  really
more  like
a  I  mean  you  could  try  to
restrict  the  kinds  of  gluings  you  allow
that  they're  not  too  stacky  but  we  were
never  happy  with  any  of  the  proposed
definition  that  we  could  come  up  with
um  Dustin  we  have  a  better  answer  I  I
didn't  quite  understand  the  question  so
so  like  you're  saying  these  are  there
are  some  examples  so  I  don't  know  I  came
here  a  bit  late  so  someone  said  there
was  tic
spaces  that  someone  showed  though  well
okay  not  someone  Peter  showed  that
they're  analytic  stacks  and  then  complex
analytic  spaces  and  then  somehow  if  you
restrict  to  real  analytic  you  get  real
analytic  spaces  okay  so  it  seems  all
like
uh  uh  I  understand  they  have  different
theories  yeah  so  my  question  is  can  you
Define  analytic  space  Theory  using  your
language  because  if  I  wanted  to  put  an
additional  structure  in  an  analytic
space  perhaps  I  want  a  general
definition  for  all  analytic  spaces  put
the  structure  there  that's  exactly  what
we're  doing  that's  what  analytic  stack
is  so  you  can  say  uh  I  have  an  an
analytic  space  Theory  and  this  reduces
to  each  known  case  so
well  I  are  such  analytic  space  analytic
space  Theory  I  don't  know  what  yeah  I
have  to  know  what  you  mean  by  that  I
don't  know  what  I  mean  I  mean  like
something  that  would  generalize  all  the
yeah  that's  what  we're  that's  that's
what  we're  trying  to  do  with  this
concept  of  analytic  stack  yes  you
shouldn't  be  scared  it  shouldn't  be  put
off  by  the  fact  that  we  change  the  name
from  space  to  stack  that's  that's
basically  a  technicality  but  that  that's
the  goal  so  you  have  a  definition  like
an  analytic  space  is  an  analytic  stack
that  well  you  could  try  yeah  for  example
yeah  that's  my  question  like  do  you  have
a  definition  like  an  analytic  space  is
an  analytic  stack  satisfying  some  axioms
do  you  have
this  well  we  have  too  many  maybe  well
what  I  mean  is  like  do  you  have  the  ult
an  an  ultimate  one  kind  of  like  the
umbrella  one  from  we  we  tried  but  we
could
never  I  mean  you  could  just  say  that  the
the  funter  of  points  takes  values  in
sets
no  F  ones  are  not  okay  sorry
what  if  you  ask  that  it  takes  than  sets
and  even  fim  analytics  STS  are  not  not
fine  well  yeah  but  the  classical  ones
are  so  anyway  the  classic  the  F  line
like  takes  a  ring  to  it's
underlying  like  because  the  test
category  consist  of  derived  things  you
can  yeah  yeah  yeah  that's  that's  true
that's  true  no  but  anyway  you  don't  have
a  definition  of  analytic  but  I  mean  you
could  ask  that  it's  a  CO  of  f  along
monom
that's  the
condition  sorry  what's  the  condition  you
could  ask  that  it's  the  Coit  of  Fes
along  monom
fers
okay  okay
thanks
\end{unfinished}